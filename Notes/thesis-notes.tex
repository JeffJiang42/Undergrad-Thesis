\documentclass[psamsfonts]{amsart}
%
%-------Packages---------
%
\usepackage[h margin=1 in, v margin=1 in]{geometry}
\usepackage{amssymb,amsfonts}
\usepackage[all,arc]{xy}
\usepackage{tikz-cd}
\usepackage{enumerate}
\usepackage{mathrsfs}
\usepackage{amsthm}
\usepackage{mathpazo}
%\usepackage{eulervm}
\usepackage{yfonts}
\usepackage{enumitem}
\usepackage{mathrsfs}
\usepackage{fourier-orns}
\usepackage[all]{xy}
\usepackage{hyperref}
\usepackage{cite}
\usepackage{url}
\usepackage{mathtools}
\usepackage{graphicx}
\usepackage{pdfsync}
\usepackage{mathdots}
\usepackage{calligra}
%
\usepackage{tgpagella}
\usepackage[T1]{fontenc}
%
\usepackage{listings}
\usepackage{color}

\definecolor{dkgreen}{rgb}{0,0.6,0}
\definecolor{gray}{rgb}{0.5,0.5,0.5}
\definecolor{mauve}{rgb}{0.58,0,0.82}

\lstset{frame=tb,
  language=Matlab,
  aboveskip=3mm,
  belowskip=3mm,
  showstringspaces=false,
  columns=flexible,
  basicstyle={\small\ttfamily},
  numbers=none,
  numberstyle=\tiny\color{gray},
  keywordstyle=\color{blue},
  commentstyle=\color{dkgreen},
  stringstyle=\color{mauve},
  breaklines=true,
  breakatwhitespace=true,
  tabsize=3
  }
%
%--------Theorem Environments--------
%
\newtheorem{thm}{Theorem}[section]
\newtheorem*{thm*}{Theorem}
\newtheorem{cor}[thm]{Corollary}
\newtheorem{prop}[thm]{Proposition}
\newtheorem{lem}[thm]{Lemma}
\newtheorem*{lem*}{Lemma}
\newtheorem{conj}[thm]{Conjecture}
\newtheorem*{quest}{Question}
%
\theoremstyle{definition}
\newtheorem{defn}[thm]{Definition}
\newtheorem*{defn*}{Definition}
\newtheorem{defns}[thm]{Definitions}
\newtheorem{con}[thm]{Construction}
\newtheorem{exmp}[thm]{Example}
\newtheorem{exmps}[thm]{Examples}
\newtheorem{notn}[thm]{Notation}
\newtheorem{notns}[thm]{Notations}
\newtheorem{addm}[thm]{Addendum}
\newtheorem{exer}[thm]{Exercise}
\newtheorem*{claim*}{Claim}
\newtheorem{TODO}{\ib{TODO}}
%
\theoremstyle{remark}
\newtheorem{rem}[thm]{Remark}
\newtheorem*{claim}{Claim}
\newtheorem*{aside*}{Aside}
\newtheorem*{rem*}{Remark}
\newtheorem*{hint*}{Hint}
\newtheorem*{note}{Note}
\newtheorem{rems}[thm]{Remarks}
\newtheorem{warn}[thm]{Warning}
\newtheorem{sch}[thm]{Scholium}
%
%--------Macros--------
\renewcommand{\qedsymbol}{$\blacksquare$}
\renewcommand{\hom}{\mathsf{Hom}}
\renewcommand{\emptyset}{\varnothing}
\renewcommand{\O}{\mathscr{O}}
\newcommand{\R}{\mathbb{R}}
\newcommand{\ib}[1]{\textbf{\textit{#1}}}
\newcommand{\Q}{\mathbb{Q}}
\newcommand{\Z}{\mathbb{Z}}
\newcommand{\E}{\mathbb{E}}
\newcommand{\N}{\mathbb{N}}
\renewcommand{\H}{\mathbb{H}}
\newcommand{\C}{\mathbb{C}}
\newcommand{\A}{\mathbb{A}}
\newcommand{\F}{\mathbb{F}}
\newcommand{\M}{\mathcal{M}}
\renewcommand{\S}{\mathbb{S}}
\newcommand{\V}{\vec{v}}
\newcommand{\RP}{\mathbb{RP}}
\newcommand{\CP}{\mathbb{CP}}
\newcommand{\B}{\mathcal{B}}
\newcommand{\GL}{\mathsf{GL}}
\newcommand{\SL}{\mathsf{SL}}
\newcommand{\SP}{\mathsf{SP}}
\newcommand{\SO}{\mathsf{SO}}
\newcommand{\SU}{\mathsf{SU}}
\newcommand{\gl}{\mathfrak{gl}}
\newcommand{\g}{\mathfrak{g}}
\newcommand{\Spin}{\mathrm{Spin}}
\newcommand{\Pin}{\mathrm{Pin}}
\newcommand{\inv}{^{-1}}
\newcommand{\bra}[2]{ \left[ #1, #2 \right] }
\newcommand{\ind}{\lambda \in \Lambda}
\newcommand{\set}[1]{\left\lbrace #1 \right\rbrace}
\newcommand{\abs}[1]{\left\lvert#1\right\rvert}
\newcommand{\norm}[1]{\left\lVert#1\right\rVert}
\newcommand{\transv}{\mathrel{\text{\tpitchfork}}}
\newcommand{\enumbreak}{\ \\ \vspace{-\baselineskip}}
\let\oldexists\exists
\renewcommand\exists{\oldexists~}
\let\oldL\L
\renewcommand\L{\mathfrak{L}}
\makeatletter
\newcommand{\tpitchfork}{%
  \vbox{
    \baselineskip\z@skip
    \lineskip-.52ex
    \lineskiplimit\maxdimen
    \m@th
    \ialign{##\crcr\hidewidth\smash{$-$}\hidewidth\crcr$\pitchfork$\crcr}
  }%
}
\makeatother
\newcommand{\bd}{\partial}
\newcommand{\lang}{\begin{picture}(5,7)
\put(1.1,2.5){\rotatebox{45}{\line(1,0){6.0}}}
\put(1.1,2.5){\rotatebox{315}{\line(1,0){6.0}}}
\end{picture}}
\newcommand{\rang}{\begin{picture}(5,7)
\put(.1,2.5){\rotatebox{135}{\line(1,0){6.0}}}
\put(.1,2.5){\rotatebox{225}{\line(1,0){6.0}}}
\end{picture}}
\DeclareMathOperator{\id}{id}
\DeclareMathOperator{\im}{Im}
\DeclareMathOperator{\grap}{graph}
\DeclareMathOperator{\codim}{codim}
\DeclareMathOperator{\coker}{coker}
\DeclareMathOperator{\supp}{supp}
\DeclareMathOperator{\inter}{Int}
\DeclareMathOperator{\sign}{sign}
\DeclareMathOperator{\sgn}{sgn}
\DeclareMathOperator{\indx}{ind}
\DeclareMathOperator{\alt}{Alt}
\DeclareMathOperator{\Aut}{Aut}
\DeclareMathOperator{\trace}{trace}
\DeclareMathOperator{\ad}{ad}
\DeclareMathOperator{\End}{End}
\DeclareMathOperator{\Ad}{Ad}
\DeclareMathOperator{\Lie}{Lie}
\DeclareMathOperator{\Cliff}{Cliff}
\DeclareMathOperator{\spn}{span}
\DeclareMathOperator{\dv}{div}
\DeclareMathOperator{\grad}{grad}
\DeclareMathOperator{\sheafhom}{\mathscr{H}\text{\kern -3pt {\calligra\large om}}\,}
\newcommand*\myhrulefill{%
   \leavevmode\leaders\hrule depth-2pt height 2.4pt\hfill\kern0pt}
\newcommand\niceending[1]{%
  \begin{center}%
    \LARGE \myhrulefill \hspace{0.2cm} #1 \hspace{0.2cm} \myhrulefill%
  \end{center}}
\newcommand*\sectionend{\niceending{\decofourleft\decofourright}}
\newcommand*\subsectionend{\niceending{\decosix}}
\def\upint{\mathchoice%
    {\mkern13mu\overline{\vphantom{\intop}\mkern7mu}\mkern-20mu}%
    {\mkern7mu\overline{\vphantom{\intop}\mkern7mu}\mkern-14mu}%
    {\mkern7mu\overline{\vphantom{\intop}\mkern7mu}\mkern-14mu}%
    {\mkern7mu\overline{\vphantom{\intop}\mkern7mu}\mkern-14mu}%
  \int}
\def\lowint{\mkern3mu\underline{\vphantom{\intop}\mkern7mu}\mkern-10mu\int}
%
%--------Hypersetup--------
%
\hypersetup{
    colorlinks,
    citecolor=black,
    filecolor=black,
    linkcolor=blue,
    urlcolor=black
}
%
%--------Solution--------
%
\newenvironment{solution}
  {\begin{proof}[Solution]}
  {\end{proof}}
%
%--------Graphics--------
%
%\graphicspath{ {images/} }

\begin{document}
%
\author{Jeffrey Jiang}
%
\title{Undergraduate Thesis Notes}
%
\maketitle
%
\setcounter{section}{1}
%
\section*{Week 1}
%
\begin{exer}
Prove a small lemma
\begin{lem*}
Let $G$ be a group, and $V$ a finite dimensional irreducible complex representation. Then elements of $Z(G)$ act by scalars.
\end{lem*}
\end{exer}
\begin{proof}
Let $z \in Z(G)$, then since $z$ commutes with the action of $G$, it determines a $G$-module isomorphism $\varphi_z : V \to V$. Then $\varphi_z$ admits some eigenvalue $\lambda$, and the $\lambda$ eigenspace of $\varphi_z$ is $G$-invariant, so $\varphi_z$ must be the map $\lambda \id_V$.
\end{proof}
%
%
\begin{exer}
Consider the groups $\Pin_{p,q}$ and $\Spin_{p,q}$, which lie in the Clifford algebra $\Cliff_{p,q}$. A \ib{Pin representation} is a representation $V$ of $\Pin_{p,q}$ that extends to an irreducible Clifford module. Likewise, a \ib{Spin representation} is a representation $V$ of $\Spin_{p,q}$ that extends to an irreducible $\Cliff_{p,q}^0$ module. Find some of these representations.
\end{exer}
%
\iffalse
We'll deal with the easier case first, where we complexify $\Cliff_{p,q}$, giving us the complex Clifford algebra $\Cliff(p+q,\C)$. A reference for this is Fulton and Harris. \\

The complex Clifford algebra $\Cliff(n,\C)$, is the Clifford algebra generated by $\C^n$ with the \emph{bilinear} (not sesquilinear!) form $\langle \cdot, \cdot \rangle$, where
\[
\langle v,w \rangle = \sum_i v^iw^i
\]
We deal with the even dimensional case first.
%
\begin{claim}
There exists a basis $\set{e_1, \ldots, e_n, f_1, \ldots f_n}$ for $\C^{2n}$ such that 
\begin{enumerate}
\item $\langle e_i, e_i \rangle = \langle f_i, f_i \rangle = 0$ 
\item $\langle e_i, f_j \rangle = \delta_{ij}$
\end{enumerate}
\end{claim}
%
\begin{proof}
For convenience, let $\set{a_i}$ denote the first $n$ standard basis vectors for $\C^{2n}$, and let $\set{b_i}$ denote the last $n$ standard basis vectors. Then the basis $\set{a_1, \ldots a_n, ib_1, \ldots ib_n}$ has the property that $\langle a_i, a_i \rangle = 1$ and $\langle ib_j, ib_j \rangle = -1$. Then setting $e_j = a_j + ib_j$ and $f_j = a_j - ib_j$ has the first property, but satisfies $\langle e_i, f_j \rangle = 2\delta_{ij}$. Normalizing these vectors then gives the second property.
\end{proof}
%
We then plan to use this basis to construct the irreducible $\Cliff(2n, \C)$ module. Write 
\[
\C^{2n} = W \oplus W'
\]
where $W = \spn\set{e_i}$. We claim that $\Cliff(2n, \C) \cong \End(\Lambda^\bullet W)$. where $\Lambda^\bullet W$ denotes the exterior algebra of $W$. To construct a map $\Cliff(2n, \C) \to \End(\Lambda^\bullet W)$, we need to provide maps $\varphi : W \to \End(\Lambda^\bullet W)$ and $\varphi' : W' \to \End(\Lambda^\bullet W)$ satisfying the Clifford relations, i.e.
%
\begin{enumerate}
\item $\varphi(w)^2 = \varphi'(w')^2 = 0$
\item $\varphi(w) \circ \varphi'(w') + \varphi'(w') \circ \varphi(w) = 2\langle w,w; \rangle\id $
\item $\varphi(w) \circ \varphi(p) + \varphi(p) \circ \varphi(w) = 0 $
\item $\varphi'(w') \circ \varphi'(p') + \varphi'(p') \circ \varphi(w') = 0$
\end{enumerate}
%
For notational compactness, we let $\varphi_w = \varphi(w)$, and likewise for $\varphi'$. Define $\varphi$ by $\varphi_w(v) = w \wedge v$, and define $\varphi'$ by 
\[
\varphi'_{w'}(v_1 \wedge, \cdot \wedge v_k) = \sum_i (-1)^{i-1} 2\langle w', v_i \rangle v_1 \wedge \cdots \hat{v_i} \cdot \wedge v_k
\]
It is clear that $\varphi_w^2 = 0$, but not immediately so for $\varphi'_{w'}$. We compute
%
\begin{align*}
&(\varphi'_{w'})^2(v_1 \wedge \cdot \wedge v_k) = \sum_i (-1)^{i-1} 2\langle w', v_i \rangle \varphi_{w'}(v_1 \wedge \cdots \hat{v_i} \cdot \wedge v_k) \\
&= \sum_i (-1)^{i-1} \langle w', v_i \rangle \left( \sum_{j < i} (-1)^{j-1} 2\langle w', v_j \rangle v_1 \wedge \cdot \hat{v_j} \cdots \hat{v_i} \cdots \wedge v_k + \sum_{i < j} (-1)^{j} 2\langle w', v_j \rangle v_1 \wedge \cdot \hat{v_i} \cdots \hat{v_j} \cdots \wedge v_k  \right) \\
&= 0
\end{align*}
To see that this is $0$. we note that each term of the $j < i$ and $i < j$ summatations produce the same element up to a sign, and the sign of each term in one summation is reversed in the other one, so the sums cancel, giving us $0$. For the other Clifford relation, we want to show 
\[
w \wedge \varphi'_{w'}(v_1 \wedge \cdots \wedge v_k) + \varphi_{w'}(w \wedge v_1 \wedge \cdot \wedge v_k) = 2\langle w, w' \rangle v_1 \wedge \cdots \cdot v_k
\]
We compute the desired quantity to be
\begin{align*}
&\left( (-1)^{i-1} 2\langle w', v_i \rangle w \wedge v_1 \wedge \cdots \hat{v_i} \cdots \wedge v_k \right) + \left( 2\langle w', w \rangle v_1 \wedge \cdots \wedge v_k \right) + \left( \sum_j (-1)^j 2\langle w', v_j \rangle v_1 \wedge \cdots \hat{v_j} \cdots \wedge v_k  \right) \\
&= 2\langle w' w \rangle v_1 \wedge \cdots \wedge v_k
\end{align*}
Noting that the first and last summations are identical, except that each term has the opposite sign. Finally, we must also show that $\varphi_w \circ \varphi_p = 0$ and $\varphi'_{w'} \circ \varphi'_{p'} = 0$. The first is clear from the skew symmetry of the wedge product. The second follows from a similar argument that $(\varphi'_{w'})^2 = 0$, which involves two summations being the same, but with opposite signs. Therefore, the maps we have given determine a map $\psi : \Cliff(2n, \C) \to \End(\Lambda^\bullet W)$. This map is an isomorphism (still need to check this), and is in fact an isomorphism of graded algebras, where $\Lambda^\bullet W$ has the grading given by parity (also need to check this). We then see that every even complex Clifford algebra has a unique irreducible module, since it is a matrix algebra.
%
\begin{exmp}
The computation for the case $n = 2$ is simple. The set $\set{1, e_1}$ forms a basis for $\Lambda^\bullet W$, and the action of $\Cliff(2,\C)$ is given by
\[
1 = \begin{pmatrix}
1 & 0 \\
0 & 1
\end{pmatrix} \qquad e_1 = \begin{pmatrix}
0 & 0 \\
1 & 0
\end{pmatrix} \qquad f_1 = \begin{pmatrix}
0 & 2 \\
0 & 0 \\
\end{pmatrix} \qquad e_1f_1 = \begin{pmatrix}
0 & 0 \\
0 & 2
\end{pmatrix}
\]
In the case of $n=4$, we pick the ordered basis $(1, e_1 \wedge e_2, e_1, e_2)$ for $\Lambda^\bullet W$, so the even endormorphisms will be block diagonal, and the odd ones will be block antidiagonal. Then the matrices of the generators will be given by
\[
e_1 = \begin{pmatrix}
0 & 0 & 0 & 0 \\
0 & 0 & 0 & 1 \\
1 & 0 & 0 & 0 \\
0 & 0 & 0 & 0
\end{pmatrix} \qquad e_2 = \begin{pmatrix}
0 & 0 & 0 & 0 \\
0 & 0 & -1 & 0 \\
0 & 0 & 0 & 0 \\
1 & 0 & 0 & 0 
\end{pmatrix} \qquad f_1 = \begin{pmatrix}
0 & 0 & 2 & 0 \\
0 & 0 & 0 & 0 \\
0 & 0 & 0 & 0 \\
0 & 2 & 0 & 0
\end{pmatrix} \qquad f_2 = \begin{pmatrix}
0 & 0 & 0 & 2 \\
0 & 0 & 0 & 0 \\
0 & 2 & 0 & 0 \\
0 & 0 & 0 & 0
\end{pmatrix}
\]
\end{exmp}
%
We now consider the odd case of $\Cliff(2n + 1, \C)$. In this case, we have a decomposition 
\[
\C^{2n+1} = W \oplus W' \oplus U
\]
Where we do the same construction as the even case for the subspace $\C^{2n} \subset \C^{2n+1}$, and let $U$ be the span of the last basis vector.
%

%
\fi
Since all Clifford algebras arise (as ungraded algebras) as direct sums of matrix algebras over $\R$, $\C$, or $\H$, it will be useful to characterize the irreducible modules of these matrix algebras. In the case of $\H$, we use the convention that scalar multiplication on a quaternionic vector space acts on the right, so quaternionic matrices can act on the left.
%
\begin{prop}
\item The only irreducible $M_n\R$  module is $\R^n$.
\end{prop}
%
\begin{proof}
We see that there is an increasing chain of left ideals 
\[
0 = I_0 \subset I_1 \subset \ldots \subset I_n = M_n\R
\]
where $I_k$ is the ideal of matrices where all the entries past the $k^{th}$ column are $0$. In addition, we have that this chain of ideals has the property that the quotient space $I_{k+1} / I_k \cong \R^n$ as a left $M_n\R$ module. We note that $\R^n$ is most certainly irreducible, since the orbit of any nonzero vector $v \in \R^n$ is all of $\R^n$. \\

Then let $W$ denote an arbitrary nontrivial irreducible $M_n\R$ module. Fix $w \in W$. Then the orbit of $w$ under the action of $M_n\R$, and since the module is nontrivial, it must be all of $W$. Therefore, the mapping 
\begin{align*}
\varphi : M_n\R &\to W \\
M &\mapsto M \cdot w
\end{align*}
is a surjective map of left $M_n\R$ modules. Since this map is surjective, there exists some $k$ such that $\varphi(I_k) \neq 0$. Let $k$ denote the smallest such $k$. Since $\varphi(I_{k-1}) = 0$, this map factors through the quotient $I_k / I_{k-1}$, which is isomorphic to $R^n$ as a left module. Then since both $\R^n$ and $W$ are irreducible, this implies that the map $I_k/I_{k-1} \to W$ is an isomorphism, so $W \cong \R^n$.
\end{proof}
%
This proof carries over for the matrix algebras $M_n\C$ and $M_n\H$. We then need another lemma to fully classify the real Clifford modules.
\begin{lem}
The algebra $A = M_n\F \oplus M_n\F$ (where $\F = \R$, $\C$, or $\H$) has two irreducible modules, the first being isomorphic to $\F^n$ where the first factor has the standard action and the second factor has the trivial action, and the second is isomorphic to $\F^n$ where the first factor acts trivially, and the second factor has the standard action.
\end{lem}
%
\begin{proof}
We first note that the left ideals $0 \oplus M_n\F$ and $M_n\F \oplus 0$ both admit a chain of left ideals as we did above, which we denote as $I_k$ and $J_k$ respectively, which satisfy $I_{k+1} / I_k \cong \F^n$ and $J_{k+1} / J_k \cong \F^n$ as $A$ modules. where the former has the action of the right factor, and the latter has the action of the right factor. We denote the modules as $R$ and $L$ respectively. We also have an ascending chain of left ideals
\[
0 \subset J_1 \oplus 0 \subset \ldots \subset M_n\F \oplus 0 \subset M_n\F \oplus I_1 \subset \ldots \subset M_n\F \oplus M_n\F
\]
Se note that for the first half of the chain, we have that each successive quotient is isomorphic to $R$. and for the second half, the successive quotients are isomorphic to $L$. Then given a nontrivial irreducible module $W$ and a nonzero $w \in W$, we know that it's orbit under the algebra action must be all of $W$, so the map $A \to W$ given by $(M,T) \mapsto (M,T) \cdot w$ is a surjective map of left $A$ modules. Then we know for some ideal in the chain, the map is nonzero, and then the map will factor through the quotient of that ideal by the previous ideal, giving an isomorphism with either $R$ or $L$.\\

We then need to show that these modules are not isomorphic. Any such isomorphism would be a linear map $\varphi : \F^n \to \F^n$ satisfying
\[
\varphi(Mv) = T\varphi(V)
\]
for all matrices $M$ and $T$. It is clear that no such $\varphi$ exists, so the two modules are not isomorphic.
\end{proof}
%
This gives a classification of all the irreducible (ungraded) Clifford modules. Recall that the (ungraded) classification of Clifford algebras gives us the Clifford ``chessboard" \\\\
%
\resizebox{.8\width}{!}{
\centering
\begin{tabular}{|c|c|c|c|c|c|c|c|}
\hline
$M_8\C$ & $M_8\H$ & $\M_8\H \oplus M_8\H$ & $M_{16}\H$ & $M_{32}\C$ & $M_{64}\R$ & $M_{64}\R \oplus M_{64}\R$ & $M_{128}\R$ \\
\hline 
$M_4\H$ & $M_4\H \oplus M_4\H$ & $M_8\H$ & $M_{16}\C$ & $M_{32}\R$ & $M_{32}\R \oplus M_{32}\R$ & $M_{64}\R$ & $M_{64}\C$ \\
\hline 
$M_2\H \oplus M_2\H$ & $M_4\H$ & $M_8\C$ & $M_{16}\R$ & $M_{16}\R \oplus M_{16}\R$ & $M_{32}\R$ & $M_{32}\C$ & $M_{32}H$ \\
\hline 
$M_2\H$ & $M_4\C$ & $M_8\R$ & $M_8\R \oplus M_8\R$ & $M_{16}\R$ & $M_{16}\C$ & $M_{16}\H$ & $M_{16}\H \oplus M_{16}\H$ \\
\hline 
$M_2\C$ & $M_4\R$ & $M_4\R \oplus M_4\R$ & $M_8\R$ & $M_8\C$ & $M_8\H$ & $M_8\H \oplus M_8\H$ & $M_{16}\H$ \\
\hline
$M_2\R$ & $M_2\R \oplus M_2\R$ & $M_4\R$ & $M_4\C$ & $M_4\H$ & $M_4\H \oplus M_4\H$ & $M_8\H$ & $M_{16}\C$ \\
\hline 
$\R \oplus \R$ & $M_2\R$ & $M_2\C$ & $M_2\H$ & $M_2\H \oplus M_2\H$ & $M_4\H$ & $M_8\C$ & $M_{16}\R$ \\
\hline 
$\R$ & $\C$ & $\H$ & $\H \oplus \H$ & $M_2\H$ & $M_4\C$ & $M_8\R$ & $M_8\R \oplus M_8\R$\\
\hline
\end{tabular}
}
$ $\\
Where the $(p,q)$ element of the table is $\Cliff_{p,q}$ as an ungraded algebra, and the bottom left corner is $\Cliff_{0,0} \cong \R$. All other Clifford algebras can be recovered from this table, since incrementing $p+q$ by $8$ results in tensoring with $M_{16}\R$.
%
\end{document}