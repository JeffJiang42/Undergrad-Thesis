\documentclass[reqno]{amsbook}
%
%-------Packages---------
%
\usepackage[h margin=1 in, v margin=1 in]{geometry}
\usepackage{amssymb,amsfonts}
\usepackage[all,arc]{xy}
\usepackage{tikz-cd}
\usepackage{enumerate}
\usepackage{mathrsfs}
\usepackage{amsthm}
\usepackage{mathpazo}
\usepackage{slashbox}
%\usepackage{eulervm}
\usepackage{yfonts}
\usepackage{enumitem}
\usepackage{mathrsfs}
\usepackage{pifont}
\usepackage{fourier-orns}
\usepackage{adforn}
\usepackage[all]{xy}
\usepackage{hyperref}
\usepackage{cite}
\usepackage{url}
\usepackage{mathtools}
\usepackage{graphicx}
\usepackage{pdfsync}
\usepackage{mathdots}
\usepackage{calligra}
%
\usepackage{tgpagella}
\usepackage[T1]{fontenc}
%
\usepackage{listings}
\usepackage{color}

\definecolor{dkgreen}{rgb}{0,0.6,0}
\definecolor{gray}{rgb}{0.5,0.5,0.5}
\definecolor{mauve}{rgb}{0.58,0,0.82}

\lstset{frame=tb,
  language=Matlab,
  aboveskip=3mm,
  belowskip=3mm,
  showstringspaces=false,
  columns=flexible,
  basicstyle={\small\ttfamily},
  numbers=none,
  numberstyle=\tiny\color{gray},
  keywordstyle=\color{blue},
  commentstyle=\color{dkgreen},
  stringstyle=\color{mauve},
  breaklines=true,
  breakatwhitespace=true,
  tabsize=3
  }

% Markers for end of environments
\newcommand{\exampleQED}{\smash\adfhalfleftarrowhead}
\newcommand{\defnQED}{\ding{118}}
%
%--------Theorem Environments--------
%
\newtheorem{thm}{Theorem}[section]
\newtheorem*{thm*}{Theorem}
\newtheorem{cor}[thm]{Corollary}
\newtheorem{prop}[thm]{Proposition}
\newtheorem{lem}[thm]{Lemma}
\newtheorem*{lem*}{Lemma}
\newtheorem{conj}[thm]{Conjecture}
\newtheorem*{quest}{Question}
%
\newenvironment{exmp}
  {\pushQED{\qed}\renewcommand{\qedsymbol}{\exampleQED}\exampx}
  {\popQED\endexampx}
\newenvironment{defn}
  {\pushQED{\qed}\renewcommand{\qedsymbol}{\defnQED}\defnx}
  {\popQED\enddefnx}
\theoremstyle{definition}
%\newtheorem{defn}[thm]{Definition}
\newtheorem{defnx}[thm]{Definition}
%\newtheorem*{defn*}{Definition}
%\newtheorem{defns}[thm]{Definitions}
\newtheorem{con}[thm]{Construction}
\newtheorem{exampx}[thm]{Example}
%\newtheorem{exmp}[thm]{Example}
%\newtheorem{exmps}[thm]{Examples}
\newtheorem{notn}[thm]{Notation}
\newtheorem{notns}[thm]{Notations}
\newtheorem{addm}[thm]{Addendum}
\newtheorem{exer}[thm]{Exercise}
\newtheorem{TODO}{\ib{TODO}}
%
\theoremstyle{remark}
\newtheorem{rem}[thm]{Remark}
\newtheorem*{claim}{Claim}
\newtheorem*{aside*}{Aside}
\newtheorem*{rem*}{Remark}
\newtheorem*{hint*}{Hint}
\newtheorem*{note}{Note}
\newtheorem{rems}[thm]{Remarks}
\newtheorem{warn}[thm]{Warning}
\newtheorem{sch}[thm]{Scholium}
%
%--------Macros--------
\renewcommand{\qedsymbol}{$\blacksquare$}
\renewcommand{\hom}{\mathsf{Hom}}
\renewcommand{\emptyset}{\varnothing}
\renewcommand{\O}{\mathscr{O}}
\newcommand{\R}{\mathbb{R}}
\newcommand{\ib}[1]{\textbf{\textit{#1}}}
\newcommand{\Q}{\mathbb{Q}}
\newcommand{\Z}{\mathbb{Z}}
\newcommand{\E}{\mathbb{E}}
\newcommand{\N}{\mathbb{N}}
\renewcommand{\S}{\mathbb{S}}
\renewcommand{\H}{\mathbb{H}}
\newcommand{\C}{\mathbb{C}}
\newcommand{\A}{\mathbb{A}}
\newcommand{\F}{\mathbb{F}}
\newcommand{\M}{\mathcal{M}}
\renewcommand{\S}{\mathbb{S}}
\renewcommand{\P}{\mathbb{P}}
\newcommand{\V}{\mathbb{V}}
\newcommand{\RP}{\mathbb{RP}}
\newcommand{\CP}{\mathbb{CP}}
\newcommand{\B}{\mathcal{B}}
\newcommand{\GL}{\mathsf{GL}}
\newcommand{\SL}{\mathsf{SL}}
\newcommand{\SP}{\mathsf{SP}}
\newcommand{\SO}{\mathsf{SO}}
\newcommand{\SU}{\mathsf{SU}}
\newcommand{\Bun}{\mathsf{Bun}}
\newcommand{\gl}{\mathfrak{gl}}
\newcommand{\g}{\mathfrak{g}}
\newcommand{\Spin}{\mathrm{Spin}}
\newcommand{\Pin}{\mathrm{Pin}}
\newcommand{\inv}{^{-1}}
\newcommand{\bra}[2]{ \left[ #1, #2 \right] }
\newcommand{\ind}{\lambda \in \Lambda}
\newcommand{\set}[1]{\left\lbrace #1 \right\rbrace}
\newcommand{\abs}[1]{\left\lvert#1\right\rvert}
\newcommand{\norm}[1]{\left\lVert#1\right\rVert}
\newcommand{\transv}{\mathrel{\text{\tpitchfork}}}
\newcommand{\enumbreak}{\ \\ \vspace{-\baselineskip}}
\let\oldexists\exists
\renewcommand\exists{\oldexists~}
\let\oldL\L
\renewcommand\L{\mathfrak{L}}
\makeatletter
\newcommand{\tpitchfork}{%
  \vbox{
    \baselineskip\z@skip
    \lineskip-.52ex
    \lineskiplimit\maxdimen
    \m@th
    \ialign{##\crcr\hidewidth\smash{$-$}\hidewidth\crcr$\pitchfork$\crcr}
  }%
}
\makeatother
\newcommand{\bd}{\partial}
\newcommand{\lang}{\begin{picture}(5,7)
\put(1.1,2.5){\rotatebox{45}{\line(1,0){6.0}}}
\put(1.1,2.5){\rotatebox{315}{\line(1,0){6.0}}}
\end{picture}}
\newcommand{\rang}{\begin{picture}(5,7)
\put(.1,2.5){\rotatebox{135}{\line(1,0){6.0}}}
\put(.1,2.5){\rotatebox{225}{\line(1,0){6.0}}}
\end{picture}}
\DeclareMathOperator{\id}{id}
\DeclareMathOperator{\im}{Im}
\DeclareMathOperator{\grap}{graph}
\DeclareMathOperator{\codim}{codim}
\DeclareMathOperator{\coker}{coker}
\DeclareMathOperator{\supp}{supp}
\DeclareMathOperator{\inter}{Int}
\DeclareMathOperator{\sign}{sign}
\DeclareMathOperator{\sgn}{sgn}
\DeclareMathOperator{\indx}{ind}
\DeclareMathOperator{\alt}{Alt}
\DeclareMathOperator{\Aut}{Aut}
\DeclareMathOperator{\trace}{trace}
\DeclareMathOperator{\ad}{ad}
\DeclareMathOperator{\End}{End}
\DeclareMathOperator{\Ad}{Ad}
\DeclareMathOperator{\Lie}{Lie}
\DeclareMathOperator{\Cliff}{Cliff}
\DeclareMathOperator{\spn}{span}
\DeclareMathOperator{\dv}{div}
\DeclareMathOperator{\grad}{grad}
\DeclareMathOperator{\sheafhom}{\mathscr{H}\text{\kern -3pt {\calligra\large om}}\,}
\newcommand*\myhrulefill{%
   \leavevmode\leaders\hrule depth-2pt height 2.4pt\hfill\kern0pt}
\newcommand\niceending[1]{%
  \begin{center}%
    \LARGE \myhrulefill \hspace{0.2cm} #1 \hspace{0.2cm} \myhrulefill%
  \end{center}}
\newcommand*\sectionend{\niceending{\decofourleft\decofourright}}
\newcommand*\subsectionend{\niceending{\decosix}}
\def\upint{\mathchoice%
    {\mkern13mu\overline{\vphantom{\intop}\mkern7mu}\mkern-20mu}%
    {\mkern7mu\overline{\vphantom{\intop}\mkern7mu}\mkern-14mu}%
    {\mkern7mu\overline{\vphantom{\intop}\mkern7mu}\mkern-14mu}%
    {\mkern7mu\overline{\vphantom{\intop}\mkern7mu}\mkern-14mu}%
  \int}
\def\lowint{\mkern3mu\underline{\vphantom{\intop}\mkern7mu}\mkern-10mu\int}
%
%--------Hypersetup--------
%
\hypersetup{
    colorlinks,
    citecolor=black,
    filecolor=black,
    linkcolor=blue,
    urlcolor=black
}
%
%--------Solution--------
%
\newenvironment{solution}
  {\begin{proof}[Solution]}
  {\end{proof}}
%
%--------Graphics--------
%
%\graphicspath{ {images/} }
%
\numberwithin{equation}{chapter}
%
\begin{document}
\frontmatter
%
\author{Jeffrey Jiang}
%
\title{Clifford Algebras and Spin Geometry \\
\large Senior Honors Thesis: supervised by Professor Daniel Freed}
%
\maketitle
%
\tableofcontents
%
\mainmatter
%
%
\chapter*{Introduction}
%
\subsectionend
%

%
%
\subsectionend
%
\section*{Acknowledgments}
%

%
\subsectionend
%
\emph{The mathematician does not study pure mathematics because it is useful;
he studies it because he delights in it and he delights in it because it
is beautiful.}
%
\ib{-- ~ Henri Poincar\'e}
%

\chapter{Clifford Algebras and Spin Groups}
%
\subsectionend
$ $\\
\emph{No one fully understands spinors. Their algebra is formally understood,
but their geometrical significance is mysterious. In some sense they describe
the ``square root" of geometry and, just as understanding the concept of
$\sqrt{-1}$ took centuries, the same might be true of spinors.} \\
%
\ib{-- ~ Sir Michael Atiyah}
%
\subsectionend
%
\section{Clifford Algebras}
%
\begin{defn}
Let $V$ be a real finite dimensional vector space with a nondegenerate
symmetric bilinear form $b : V \times V \to \R$. Then the \ib{Clifford Algebra}
of $V$ is the data of a unital associative $\R$-algebra $\Cliff(V,b)$ and a linear
map $i : V \to \Cliff(V,b)$ satisfying the following universal property:
Given any linear map $\varphi : V \to A$ of $V$ into any unital associative
$\R$-algebra $A$ satisfying $\varphi(v)^2 = b(v,v)$, there exists a unique algebra
homomorphism $\tilde{\varphi}: \Cliff(V,b) \to A$ such that the following
diagram commutes:
%
\[\begin{tikzcd}
V \ar[dr, "\varphi"] \ar[d, "i"'] \\
\Cliff(V,b) \ar[r, "\tilde{\varphi}"'] & A
\end{tikzcd}\]
\end{defn}
%
This universal property uniquely characterizes the Clifford algebra $\Cliff(V,b)$
up to unique isomorphism.
%
\begin{thm}
The Clifford algebra is unique up to unique isomorphism, i.e. given
another unital associative algebra $C$ equipped with a linear map $j : V \to C$
satisfying the universal property, there exists a unique algebra isomorphism
$\varphi : \Cliff(V,b) \to C$.
\end{thm}
%
\begin{proof}
Given such an algebra $C$ with a map $j: V \to C$, the map $j$
satisfies the Clifford relation $j(v)^2 = b(v,v)$, so it induces a unique map
$\varphi : \Cliff(V,b) \to C$. The map $i : V \to \Cliff(V,b)$ also
satisfies the Clifford relation, so it induces a map $\psi : C \to \Cliff(V,b)$.
We claim that these maps are inverses. To do so, we show the compositions
$\varphi \circ \psi$ and $\psi \circ \varphi$ are identity. Using the
universal property of the Clifford algebra once more, the map $i$ induces
a unique map $\Cliff(V,b) \to \Cliff(V,b)$ such that
\[\begin{tikzcd}
V \ar[dr, "i"] \ar[d, "i"'] \\
\Cliff(V,b) \ar[r] & \Cliff(V,b)
\end{tikzcd}\]
commutes. The identity map makes this diagram commute, so by uniqueness, this is
the induced map. The map $\psi \circ \varphi$ also makes this diagram commute,
so it must be identity by uniqueness. An identical proof shows that
$\varphi \circ \psi$ is the identity map $\id_C$.
\end{proof}
%
Explicitly, $\Cliff(V,b)$ is realized as a quotient of the tensor algebra
\[
\mathcal{T}(V) = \bigoplus_{n \in \Z^{\geq 0}} V^{\otimes n}
\]
where we quotient by the left ideal generated by elements of the form
$v \otimes v - b(v,v)$, and the linear map
$i: V \to \mathcal{T}(V) / (v \otimes v - b(v,v))$
is given by the inclusion $V \hookrightarrow \mathcal{T}(V)$ followed by the
quotient map. We identify $V$ with its image $i(V)$ as a subspace of $\Cliff(V,b)$.
%
\begin{defn}
Define the bilinear form $b : \R^{p+q} \times \R^{p+q} \to \R$ by
\[
b(v,w) = \sum_{i = 1}^{p} v^iw^i - \sum_{i = p+1}^{p+q} v^iw^i
\]
where the $v^i$ and $w^i$ are the components of $v$ and $w$ with respect to
the standard basis on $\R^{p+q}$. We denote the vector space $\R^{p+q}$
equipped with this bilinear form as $\R^{p,q}$.
\end{defn}
%
Let $V$ be a vector space equipped with a nondegenerate bilinear form $b$ and
fix a basis for $V$. Then the bilinear form $b$ is given by a symmetric
invertible matrix $B$, which is conjugate to a diagonal matrix where all
the diagonal entries are either $1$ or $-1$. If after conjugation, $B$ has
$p$ positive entries and $q$ negative entries, we say $b$ has signature $(p,q)$.
Any bilinear form $b$ of signature $(p,q)$ admits a basis $\set{e_i}$ satisfying
%
\begin{enumerate}
  \item For $1 \leq i \leq p$, we have $b(e_i,e_i) = 1$
  \item For $p+1 \leq j \leq p+q$, we have $b(e_j,e_j) = -1$
  \item For $i \neq j$, we have $b(e_i,e_j) = 0$
\end{enumerate}
%
Any such basis then determines an isomorphism $(V,b) \to \R^{p,q}$.\, and we
call such a basis an \ib{orthogonal basis} for $(V,b)$.
In addition, we get a basis for $\Cliff(V,b)$ given by
\[
\set{e_{i_1}e_{i_2}\cdots e_{i_k}~:~ 0 \leq k \leq n, ~ 1 \leq i_j \leq n}
\]
where we interpret the product of $0$ basis vectors to be the unit element $1$.
This then implies the dimension of $\Cliff(V,b)$ as a vector space is
$2^{\dim V}$. This basis also determines an isomorphism
$\Cliff(V,b) \to \Cliff_{p,q}(\R)$, where $\Cliff_{p,q}(\R)$ is the Clifford
algebra for $\R^{p,q}$. Given $v,w \in V$, we write $v$ and $w$ in these
bases as $v^ie_i$ and $w^ie_i$ (using Einstein summation convention),
and derive the useful relation
%
\begin{align*}
vw + wv &= v^iw^je_ie_j + v^iw^je_je_i \\
&= v^iw^i(e_i^2 + e_i^2) \\
&= 2b(v,w)
\end{align*}
where we use the fact that the $e_i$ are orthgonal to deduce that $e_ie_j = -e_je_i$
if $i \neq j$.
%
\begin{defn}\footnote{It is common in the literature to refer to $\Z/2\Z$
graded vector spaces as super vector spaces. The ``super" prefix often refers
to a $\Z/2\Z$ grading on the relevant object.}
Let $V$ be a vector space. A $\Z /2\Z$ \ib{grading} on $V$ is a direct sum
decomposition $V = V^0 \oplus V^1$. Elements of $V^0$ are said to be \ib{even}
and elements of $V^1$ are said to be \ib{odd}. Elements of the even and odd
subspaces are said to be \ib{homogeneous}. Given an homogeneous element
$v \in V$, define its \ib{parity}, denoted $|v|$ by
\[
|v| = \begin{cases}
0 & v \in V^0 \\
1 & v \in V^1
\end{cases}
\]
Equivalently, it is the data of a linear map
$\varepsilon : V \to V$ such that $\varepsilon$ acts by identity on a subspace $V^0$
of $V$ and negative identity on a complementary subspace $V^1$, which gives the
direct sum decomposition of $V$ as the $\pm 1$ eigenspaces of $\varepsilon$.
The map $\varepsilon$ is called the \ib{grading operator}.

\end{defn}
%
\begin{defn}
A $\Z/2\Z$ \ib{graded algebra} $A$ over $\R$ (often called a superalgebra) is
an $\R$-algebra $A$ equipped with a grading $A = A^0 \oplus A^1$ such that
the multiplication respects the grading, i.e. given homogeneous elements
$a,b \in A$, Their product $ab$ is an element of $A^{|a| + |b|}$ where the
addition is done mod $2$.
\end{defn}
%
\begin{exmp}\enumbreak
\begin{enumerate}
  \item Any $\R$-algebra $A$ can be made into a graded algebra where we let
  $A^0 = A$ and $A^1 = 0$.
  \item The exterior algebra $\Lambda^\bullet V$ of a vector space $V$ is
  a $\Z/2\Z$ graded algebra (in fact, it has a $\Z$ grading as well), where the
  grading is
  $\Lambda^\bullet V = \Lambda^{\text{even}} V \oplus \Lambda^{\text{odd}} V$
  where $\Lambda^{\text{even}} V$ is the subspace spanned by even products of
  vectors and $\Lambda^{\text{odd}} V$ is the subspace spanned by odd products of vectors.
  \item Let $V = V^0 \oplus V^1$ be a $\Z/2\Z$ graded vector space. Then the
  algebra of endomorphisms $\End V$ can be endowed with the structure of a
  graded algebra, where the even subspace consists of linear maps preserving
  the decomposition $V^0 \oplus V^1$, and the odd subspace consists of
  linear maps $T$ reversing the decomposition, i.e. $T(V^i) = V^{i + 1 \mod 2}$.
  In a ordered basis where the first elements are all even and the last
  elements are all odd, the even elements of $\End V$ are block diagonal,
  while the odd elements are block off-diagonal.
\end{enumerate}
\end{exmp}
%
For the most part, the algebras we work with will be $\Z/2\Z$ graded, so
the term ``graded" may be used in lieu of ``$\Z/2\Z$ graded." In the case of
ambiguity, we will specify the grading. \\
The Clifford algebra $\Cliff(V,b)$ is naturally a $\Z/2\Z$ graded algebra. Fix
a basis $\set{e_i}$ for $V$. We then define the grading
\[
\Cliff(V,b) = \Cliff^0(V,b) \oplus \Cliff^1(V,b)
\]
Where $\Cliff^0(V,b)$ is the $\R$-span of all even products of basis vectors,
and $\Cliff^1(V,b)$ is the $\R$-span of all odd products of basis vectors.
Since the product of even elements is again even, the subspace
$\Cliff^0(V,b)$ forms a subalgebra, and is called the \ib{even
subalgebra}. There is an extremely nice relationship between a Clifford algebra
and its even subalgebra.
%
\begin{thm}
The even subalgebra $\Cliff_{p,q}^0(\R)$ is isomorphic to both $\Cliff_{q, p-1}$
and $\Cliff_{p,q-1}$ as ungraded algebras (as long as $p-1 > 0$ or $q-1 > 0$.)
\end{thm}
%
\begin{proof}
Fix a basis $\set{e_1^+, \ldots e_p^+, e_1^- \ldots e_q^-}$ for $\R^{p,q}$, where
$(e_i^+)^2 = 1$ and $(e_i^-)^2 = -1$. We then compute
%
\begin{align*}
(e_i^+e_j^+)^2 &= -(e_i^+)^2(e_j^+)^2 = -1 \\
(e_i^-e_j^-)^2 &= -(e_i^-)^2(e_j^-)^2 = -1 \\
(e_i^+e_j^-)^2 &= -(e_i^+)^2(e_i^-)^2 = 1 \\
(e_i^-e_j^+)^2 &= -(e_i^-)^2(e_j^+)^2 = 1
\end{align*}
%
Assume $q \neq 0$. Then a generating set for $\Cliff_{p,q}^0(\R)$ is
\[
\set{e_1^-e_j^+ ~:~ 1 \leq j \leq p} \cup \set{e_1^-e_k^- ~:~ 2 \leq k \leq q}
\]
All the elements in the first set square to $1$, and all the elements in the
second set square to $-1$. We then get an isomorphism
$\Cliff^0_{p,q}(\R) \to \Cliff_{p,q-1}$ via the mappings
\begin{align*}
e_1^-e_j^+ &\mapsto e_j^+ \\
e_1^-e_k^- &\mapsto e_{k-1}^-
\end{align*}
In the case where $p \neq 0$, we have that an equally good generating set for
$\Cliff_{p,q}^0(\R)$ is
\[
\set{e^+_1e_j^+ ~:~ 2 \leq j \leq p} \cup \set{e_1^+e_i^- ~:~ 1 \leq i \leq q}
\]
Where the elements in the first set square to $-1$ and the elements of the second
set square to $1$. Then the mappings
\begin{align*}
e_1^+e_j^+ &\mapsto e_{j-1}^- \\
e_1^+e_j^- &\mapsto e_j^+
\end{align*}
gives the isomorphism $\Cliff_{p,q}^0(\R) \to \Cliff_{q, p-1}$.
\end{proof}
%
Given two $\R$-algebras $A$ and $B$, we can form their tensor product
$A \otimes B$, which has $A \otimes B$ as the underlying vector space, and the
multiplication is defined as
\[
(a \otimes b)(c \otimes d) = ac \otimes bd
\]
In the case that both $A$ and $B$ are $\Z/2\Z$ graded algebras, we have an alternate
version of the tensor product, where the underlying vector space is also
$A \otimes B$. The grading on the tensor product is the decomposition
\[
A \otimes B = (A^0 \otimes B^0 \oplus A^1 \otimes A^1) \oplus (A^0 \otimes B^1
\oplus A^1 \otimes B^0)
\]
and the multiplication of homogeneous elements is given by
\[
(a \otimes b)(c \otimes d) = (-1)^{|b||c|}(ac \otimes bd)
\]
We see that in the multiplication, we are formally commuting the elements of
$b$ and $c$, and we want to introduce a sign whenever elements are moved past
each other. This is the \ib{Koszul sign rule}. Another concept that needs
a slight modification in the graded case is the opposite algebra. In the
normal case, given an $\R$-algebra $A$, the \ib{opposite algebra} is
the algebra $A^{\text{op}}$ with the same underlying vector space, but
the multiplication in $A^\text{op}$ is given by $a * b = ba$, where $ba$
is the multiplication in $A$. In doing so, we are formally commuting $a$
and $b$, so in the graded situation, we invoke the Koszul sign rule when
defining multiplication in the opposite algebra, and define the multiplication
of homogeneous elements in $A^{\text{op}}$ to be $a * b = (-1)^{|a||b|} ba$.\\

One remarkable fact is that Clifford algebras are closed under the graded
tensor product, i.e. the graded tensor products of two Clifford algebras is
another Clifford algebra. Likewise, the graded opposite algebra of a Clifford
algebra is again a Clifford algebra. For the remainder of this section,
we will let $\otimes$ denote the graded tensor product, and the superscript
$\text{op}$ will denote the graded opposite algebra.
%
\begin{thm}
$\Cliff_{p+t,q+s}(\R) \cong \Cliff_{p,q}(\R) \otimes \Cliff_{t,s}(\R)$
\end{thm}
%
\begin{proof}
To give a map $\varphi : \Cliff_{p+t,q+s}(\R) \to \Cliff_{p,q}(\R) \otimes
\Cliff_{t,s}(\R)$, it is sufficient to specify its action on $\R^{p+t,q+s}$ and
to check that the Clifford relations hold. Let
$\set{b_1^+,\ldots, b_{p+t}^+, b_1^-, \ldots b_{q+s}^-}$ denote the standard
orthogonal basis for $\R^{p+t,q+s}$ where $(b_i^+)^2 = 1$ and $(b_i^-)^2 = -1$.
We then define the bases $\set{e_i^\pm}$ and $\set{f_i^\pm}$ analogously for
$\R^{p,q}$ and $\R^{t,s}$ respectively. Then define $\varphi$ by
%
\begin{align*}
\varphi(b_i^+) &= \begin{cases}
e_i^+ \otimes 1 & 1 \leq i \leq p \\
1 \otimes f_i^+ & p+1 \leq i \leq p+t
\end{cases} \\
\varphi(b_i^-) &= \begin{cases}
e_i^- \otimes 1 & 1 \leq i \leq q \\
1 \otimes f_i^- & q+1 \leq i \leq q+s
\end{cases}
\end{align*}
%
This map is injective on generators, so if we show that this satisfies the
Clifford relations, then the map given by extending the map to all of
$\Cliff_{p+t,q+s}(\R)$ will be an isomorphism by a dimension count. Showing
the Clifford relations amounts to showing
%
\begin{enumerate}
  \item $\varphi(b_i^+)^2 = 1$
  \item $\varphi(b_i^-)^2 = -1$
  \item The images of any pair of distinct basis vectors anticommute.
\end{enumerate}
%
The first two are relations are clear from how we defined $\varphi$. To show
that the images of distinct basis vectors anticommute, there are serveral
cases to consider. Given $b_i^+$ and $b_j^+$ where $1 \leq i,j \leq p$, they
anticommute, because $e_i^+$ and $e_j^+$ anticommute. In the case where
$1 \leq i \leq p$ and $p+1 \leq j \leq p+t$, we compute
%
\begin{align*}
\varphi(b_i^+)\varphi(b_j^+) + \varphi(b_j^+)\varphi(b_i+) &=
(e_i^+ \otimes 1)(1 \otimes f_j^+) + (1\otimes f_j^+)(e_i^+ \otimes 1) \\
&= e_i^+ \otimes f_j^+ - e_i^+ \otimes f_j^+
\end{align*}
where we use the Koszul sign rule for the second term, noting that $f_j+$ and
$e_i^+$ are both odd elements. The proof that the images of the $b_i^-$ anti commute with
each other, as well as the proof that the images of the $b_i^+$ and $b_i^-$
anticommute are exactly the same.
%
\end{proof}
%
\begin{thm}
The graded opposite algebra $\Cliff_{p,q}^{\text{op}}(\R)$ is isomorphic to
$\Cliff_{q,p}(\R)$.
\end{thm}
%
\begin{proof}
Fix an orthgonal basis $\set{e_i^\pm}$ for $\R^{p,q}$, where
$(e_i^\pm)^2 = \pm 1$. We note that since the $e_i^\pm$ are odd elements,
they square to $\mp 1$ in the opposite algebra. Indeed, the mapping
$e_i^\pm \to e_i^\mp$ defines the isomorphism
$\Cliff_{p,q}^\text{op} \to \Cliff_{q,p}$.
\end{proof}
%
Because of these theorems, once we compute a few of the lower dimensional
Clifford algebras, we will have enough data to fully classify all Clifford
algebras over $\R$.
%
\begin{exmp}[\ib{Some low dimensional examples}]\enumbreak
\begin{enumerate}
  \item The Clifford algebra $\Cliff_{0,0}(\R)$ is isomorphic to $\R$.
  \item As ungraded algebras, the Clifford algebra $\Cliff_{0,1}(\R)$ is isomorphic
  to $\C$, where the isomorphism is given by $e_1 \mapsto i$.
  \item As ungraded algebras, $\Cliff_{0,2}(\R)$ is isomorphic to the quaternions
  $\H$, where the isomorphism is given by $e_1 \mapsto i$ and $e_2 \mapsto j$.
  \item As graded algebras, $\Cliff_{1,1}(\R)$ is isomorphic to $\End(\R^{1|1})$,
  where $\R^{1|1}$ denotes the $\Z/2\Z$ graded vector space $\R \oplus \R$.
  The isomorphism is given by
  \[
  e_1^+ \mapsto \begin{pmatrix}
  0 & 1 \\
  1 & 0
  \end{pmatrix} \qquad e_1^- \mapsto \begin{pmatrix}
  0 & 1 \\
  -1 & 0
  \end{pmatrix}
  \]
  \item As ungraded algebras $\Cliff_{1,0}(\R)$ is isomorphic to the product
  algebra $\R \times \R$, where $e_1 \mapsto (1,-1)$.
  \item As ungraded algebras, $\Cliff_{2,0}(\R)$ is isomorphic to the algebra $M_2\R$
  of $2 \times 2$ matrices with coefficients in $\R$. The isomorphism is given by
  \[
  e_1 \mapsto \begin{pmatrix}
  0 & 1 \\
  1 & 0
  \end{pmatrix} \qquad e_2 \mapsto \begin{pmatrix}
  1 & 0 \\
  0 & -1
  \end{pmatrix}
  \]
\end{enumerate}
\end{exmp}
%
To classify all Clifford algebras as ungraded algebras, it suffices to know
the following table: \\\\
\resizebox{.72\width}{!}{
\centering
\begin{tabular}{c |c|c|c|c|c|c|c|c|}
\hline
7 & $M_8\C$ & $M_8\H$ & $M_8\H \times M_8\H$ & $M_{16}\H$ & $M_{32}\C$ & $M_{64}\R$ & $M_{64}\R \times M_{64}\R$ & $M_{128}\R$ \\
\hline
6 & $M_4\H$ & $M_4\H \times M_4\H$ & $M_8\H$ & $M_{16}\C$ & $M_{32}\R$ & $M_{32}\R \times M_{32}\R$ & $M_{64}\R$ & $M_{64}\C$ \\
\hline
5 & $M_2\H \times M_2\H$ & $M_4\H$ & $M_8\C$ & $M_{16}\R$ & $M_{16}\R \times M_{16}\R$ & $M_{32}\R$ & $M_{32}\C$ & $M_{32}\H$ \\
\hline
4 & $M_2\H$ & $M_4\C$ & $M_8\R$ & $M_8\R \times M_8\R$ & $M_{16}\R$ & $M_{16}\C$ & $M_{16}\H$ & $M_{16}\H \times M_{16}\H$ \\
\hline
3 & $M_2\C$ & $M_4\R$ & $M_4\R \times M_4\R$ & $M_8\R$ & $M_8\C$ & $M_8\H$ & $M_8\H \times M_8\H$ & $M_{16}\H$ \\
\hline
2 & $M_2\R$ & $M_2\R \times M_2\R$ & $M_4\R$ & $M_4\C$ & $M_4\H$ & $M_4\H \times M_4\H$ & $M_8\H$ & $M_{16}\C$ \\
\hline
1 & $\R \times \R$ & $M_2\R$ & $M_2\C$ & $M_2\H$ & $M_2\H \times M_2\H$ & $M_4\H$ & $M_8\C$ & $M_{16}\R$ \\
\hline
0 &  $\R$ & $\C$ & $\H$ & $\H \times \H$ & $M_2\H$ & $M_4\C$ & $M_8\R$ & $M_8\R \times M_8\R$\\
\hline
\slashbox{p}{q} & 0 & 1 & 2 & 3 & 4 & 5 & 6 & 7
\end{tabular}
}
$ $\\\\\\
To read the table, the bottom left entry is $\Cliff_{0,0} \cong \R$, and moving to
the right increments the signature from $(p,q)$ to $(p,q+1)$, and moving up
increments the signature $(p,q)$ to $(p+1,q)$. Any other Clifford algebra
can be obtained from an algebra on this table by tensoring with $M_{16}\R$, since
incremeting the signature by $8$ (by adding to either $p$ or $q$) results in
tensoring with $M_{16}\R$.
%

%
%
\section{Clifford Modules}
%
\begin{defn}
A (left) \ib{Clifford module} for the Clifford algebra $\Cliff_{p,q}(\R)$ is a module
for $\Cliff_{p,q}(\R)$ in the usual sense i.e. a real vector space $V$ equipped
with an algebra action $\bullet: \Cliff_{p,q} \times V \to V$ satisfying
\begin{enumerate}
  \item Every element of $\Cliff_{p,q}(\R)$ acts linearly on $V$.
  \item $(AB) \cdot v = A\cdot(B \cdot v)$ for all $v \in V$.
  \item $(A + B) \cdot v = A\cdot v + B\cdot V$ for all $v \in V$.
\end{enumerate}
Equivalently, it is the data of a real vector space $V$ and a homomorphism
$\Cliff_{p,q}(\R) \to \End(V)$.
\end{defn}
%
\begin{defn}
A Clifford module is \ib{irreducible} if there exist no proper nontrivial submodules.
\end{defn}
%
From the classification of Clifford algebras, all the Clifford algebras are either
matrix algebras $M_n\F$, where $\F = \R$, $\C$, or $\H$, or products
$M_n\F \times M_n\F$ of two copies of the same matrix algebra. This is sufficient
to conclude that Clifford algebras are semisimple, so all Clifford modules
will be direct sums of irreducible modules. Therefore, classifying all Clifford
modules reduces to classifying the irreducible Clifford modules.
%
\begin{thm}
Let $\F = \R$, $\C$, or $\H$. Then any nontrivial irreducible module for
$M_n\F$ is isomorphic to $\F^n$ with the standard action.
\end{thm}
%
\begin{proof}
We first note that $M_n\F$ acts transitively on $\F^n$, which implies that
it is irreducible. We then must show that $\F^n$ is, up to isomorphism, the only
irreducible $M_n\F$ module. The matrix algebra $M_n\F$ admits an increasing
chain of left ideals
\[
0 = I_0 \subset I_1 \subset \ldots \subset I_n = M_n\F
\]
where $I_k$ is the set of matrices where only the first $k$ columns are nonzero.
These ideals have the property that the quotient $I_k / I_{k-1}$ is isomorphic
to $\F^n$ as a left $M_n\F$ module. Then let $M$ be some nontrivial irreducible
$M_n\F$ module, and fix $m \in M$. Then the orbit $M_n\F \cdot m$ of $m$
under the algebra action is a nonzero submodule, so it must be all of $M$.
Then the map $\varphi : M_n\F \to M$ given by $A \mapsto A \cdot m$ is
a surjective map of left $M_n\F$ modules. Then there must exist some smallest
$k$ such that $\varphi(I_k)$ is nonzero, and by construction,
$\varphi\vert_{I_k}$ factors through the quotient $I_k / I_{k-1}$, which
is isomorphic to $\F^n$ with the standard action. Then since $\F^n$ is irreducible,
this gives us a nontrivial map between irreducible modules, which is an isomorphism
by Schur's Lemma.
\end{proof}
%
\begin{thm}
Any nontrivial irreducible module for $M_n\F \times M_n\F$ is isomorphic to
either $\F^n$ where the left factor acts in the usual way, and the right factor
acts by $0$, or $\F^n$ where the left factor acts by $0$ and the right factor
acts in the usual way.
\end{thm}
%
\begin{proof}
Let $R$ denote $\F^n$ where the right factor acts nontrivially, and let $L$
denote $\F^n$ where the left factor acts nontrivially. Both $L$ and $R$ are
irreducible since $M_n\F \times M_n\F$ acts transitively on them. To show
that they are the only irreducible modules up to isomorphism, we use a similar
techinique as above. Let $I_k$ denote the chain of increasing ideals in $M_n\F$,
as we used above. Then $M_n\F \times M_n\F$ admits a chain of increasing left
ideals $J_k$
\[
0 = J_0 \subset I_1 \times \set{0} \subset \ldots \subset M_n\F \times \set{0}
\subset M_n\F \times I_1 \subset \ldots \subset M_n\F \times M_n\F = J_{2n}
\]
We note that for $1 \leq k \leq n$, we have that $J_k / J_{k-1}$ is isomorphic to
$L$, and for $n+1 \leq k \leq 2n$, we have that $J_k / J_{k-1}$ is isomorphic to
$R$. Then given a nontrivial irreducible module $M$ and a nonzero element $m$,
we get a surjective map $\varphi : M_n\F \times M_n\F$ where $A \mapsto A\cdot m$.
Like before, there exists some smallest $k$ such that $\varphi(J_k)$ is nonzero,
which then factors through to an isomorphism $J_k / J_{k-1} \to M$, so $M$
is either isomorphic to $R$ or $L$.
\end{proof}
%
This then gives a full classification of the irreducible ungraded Clifford
modules.
%

%
%\section{Complex Clifford Algebras}
%
Much of the discussion regarding Clifford algebras can be reconstructed
using complex vector spaces. However, there is an important distinction
to be made. Over $\C$, the notion of signature no longer makes sense
when discussing bilinear forms. Given a bilinear form $B : V \times V \to \C$
and a vector $v \in V$ with $B(v,v) = 1$, we have that $B(iv,iv) = -1$. Therefore,
the complex Clifford algebras generated by $V$ with any nondegenerate bilinear
form are entirely determined by their dimension. In the case of $\C^n$ with
the standard bilinear form
\[
\langle v,w \rangle = \sum_i v^iw^i
\]
we denote the Clifford algebra by $\Cliff_n(\C)$. In the complex case, the
classification is much simpler, and is determined by the parity of the dimension. \\

In the even case of $\C^{2n}$, we first prove a lemma.
%
\begin{lem}
There exists a basis $\set{e_1,\ldots e_n, f_1, \ldots, f_n}$ for $\C^{2n}$
satisfying
\begin{enumerate}
  \item $\langle e_i, e_i \rangle = \langle f_i, f_i \rangle = 0$
  \item $\langle e_i, f_j \rangle = \delta_{ij}$
\end{enumerate}
where $\delta_{ij} = 0$ if $i \neq j$ and $\delta_{ij} = 1$ if $i = j$.
\end{lem}
%
\begin{proof}
Let $\set{a_i}$ denote the first $n$ standard basis vectors for $\C^{2n}$, and let
$\set{b_i}$ denote the last $n$ standard basis vectors. Then setting
$e_i = a_i + ib_i$ and $f_i = a_i - ib_i$, we get a basis
$\set{e_1, \ldots e_n, f_1, \ldots, f_n}$ for $\C^{2n}$. We then compute
%
\begin{align*}
\langle e_i, e_i \rangle &=\langle a_i + ib_i, a_i + ib_i \rangle \\
&= \langle a_i, a_i \rangle +2\langle a_i, ib_i \rangle + \langle ib_i, ib_i \rangle \\
&= 1 + 0 + -1 \\
&= 0 \\
\langle f_i, f_i \rangle &= \langle a_i - ib_i, a_i - ib_i \rangle \\
&= \langle a_i,a_i \rangle -2\rangle a_i, ib_i \rangle + \langle ib_i,ib_i\rangle \\
&= 0 \\
\langle e_i, f_j \rangle &= \langle a_i + ib_i, a_j - ib_j \rangle \\
&= \langle a_i,a_j \rangle - \langle a_i,ib_j \rangle + \langle a_j, ib_i \rangle
- \langle ib_i, ib_j \rangle \\
&= \delta_{ij} + 0 + 0 + \delta{ij} \\
&= 2\delta_{ij}
\end{align*}
So normalizing the $e_i$ and $f_i$ by dividing by $\sqrt{2}$ gives the desired basis.
%
\end{proof}
%
This basis gives a direct sum decomposition $\C^{2n} = W \oplus W'$ where
$W$ is the span of the $e_i$ and $W'$ is the span of the $f_i$. We then claim
that $\Cliff_{2n}(\C)$ is isomorphic to the endomorphism algebra
$\End(\Lambda^\bullet W)$. To give a map $\Cliff_{2n}(\C) \to
\End(\Lambda^\bullet W)$, we need to specify two maps
$\varphi : W \to \End(\Lambda^\bullet W)$ and
$\varphi' : W' \to \End(\Lambda^\bullet W)$ satisfying
%
\begin{enumerate}
  \item $\varphi^2 = (\varphi')^2 = 0$.
  \item
\end{enumerate}

%
%
\section{The Pin and Spin Groups}
%
The group of invertible elements in $\Cliff_{p,q}(\R)$, denoted
$\Cliff_{p,q}^\times(\R)$ contains a group $\Pin_{p,q}$, which is a double
cover of the group $O_{p,q}$ of matrices preserving the standard bilinear
form $\langle \cdot,\cdot \rangle$ on $\R^{p,q}$. Inside of $\Pin_{p,q}$,
there exists a subgroup $\Spin_{p,q} \subset \Pin_{p,q}$, which double covers
the group $SO_{p,q}$, which consists of the subgroup of $O_{p,q}$ where
all the elements have determinant equal to $1$.
%
\begin{defn}
The \ib{Pin group} $\Pin_{p,q}$ is the subgroup of $\Cliff_{p,q}^\times(\R)$
generated by the set
\[
\set{v \in \R^{p,q} ~:~ v^2 = \pm 1}
\]
The \ib{Spin group} $\Spin_{p,q}$ is the subgroup of $\Pin_{p,q}$ generated by even
products of basis vectors, i.e.
\[
\Spin_{p,q} = \Pin_{p,q} \cap \Cliff_{p,q}^0(\R)
\]
In indefinite signatures. $SO_{p,q}$ has multiple components. Some conventions
let $\Spin_{p,q}$ denote the double cover of the identity component $SO_{p,q}^+$,
which is the identity component of $\Spin_{p,q}$, denoted $\Spin^0_{p,q}$.
In the case that the bilinear form is definite, we let $\Pin_n^+ = \Pin_{n,0}$
and $\Pin_n^- = \Pin_{0,n}$. There is no such distinction for the Spin groups
in definite signatures.
\end{defn}
%
\begin{thm}
$\Spin_{p,q} \cong \Spin_{q,p}$.
\end{thm}
%
\begin{proof}
Recall that we have an isomorphism $\Cliff_{p,q}^{\text{op}} \to \Cliff_{q,p}$
where $e_i^\pm \mapsto e_i^\mp$. In addition, the even subalgebra $\Cliff_{q,p}^0$
is isomorphic to the (ungraded) opposite algebra of the even subalgebra
$\Cliff_{p,q}^0$. Therefore, the Spin group $\Spin_{q,p} \subset \Cliff_{q,p}$
is isomorphic to the opposite group $\Spin_{p,q}^\text{op}$. We then know
that every group is isomorphic to its opposite group via the map $g \mapsto g\inv$,
giving us the desired isomorphism.
\end{proof}
%
%
In particular, this implies that the Spin groups in definite signatures are
isomorphic, so we will henceforth denote them as $\Spin_n$.
%
To show that the Pin and Spin groups cover $O_{p,q}$ and $\Spin_{p,q}$, we make
a short digression. Given a vector $v \in \R^{p,q}$, we can define a reflection
map $R_v : \R^{p,q} \to \R^{p,q}$ given by $R_v(w) = w - 2\langle v,w \rangle v$,
which will reflect across the hyperplane $v^\perp$.
%
\begin{thm}[\ib{Cartan-Dieudonn\'e}]
Any orthogonal transformation $A \in O_{p,q}$ is the composition
of at most $p+q$ hyperplane reflections, where we interpret the identity map as the
composition of $0$ reflections.
\end{thm}
%
\begin{proof}
We prove this by induction on $n = p+q$. The case $n=1$ is trivial, since
$O_1 = \set{\pm 1}$. Then given $A \in O_{p,q}$, fix some nonzero $v \in \R^{p,q}$.
Then define $R : \R^{p,q} \to \R^{p,q}$ by
\[
R(w) = w - 2 \frac{\langle Av - v,w \rangle}{\langle Av -v, Av - v\rangle}(Av - v)
\]
Then $R$ is a reflection about the hyperplane orthogonal to $Av - v$,
and will interchange $v$ and $Av$. Therefore, $RA$ is an orthogonal transformation
fixing $v$. Since $RA$ is orthogonal, it will also fix the orthogonal complement
$v^\perp$, so it will restrict to an orthgonal transformation on $v^\perp$. The
orthogonal complement $v^\perp$ is $1$ dimension lower than $\R^{p,q}$, and restricting
the bilinear form to $v^\perp$, we know by the inductive hypotehsis that
$RA\vert_{v^\perp}$ can be written as at most $n-1$ hyperplane reflections in
$v^\perp$. Since $RA$ fixes $v$, we can extend all of these transformations to
a hyperplane reflection on all of $\R^{p,q}$, by taking the span of each hyperplane
with $v$, giving us that $RA$ is a composition of at most $n-1$ reflections. Finally,
composing $RA$ with $R$ gives us that $A$ can be written as a composition of at most
$n$ hyperplane reflections.
\end{proof}
%TODO Twisted Adjoint action
The Cartan-Dieudonn\'e theorem will be the central piece for showing that the
Pin and Spin groups are double covers of the orthogonal groups. The Clifford
algebra has an automorphism
$\alpha : \Cliff_{p,q}(\R) \to \Cliff_{p,q}(\R)$ that extends the mapping
$v \mapsto v-$. The action of $\alpha$ on a product $v_1\cdots v_k$ of vectors
$v_i \in \R^{p,q}$ is
\[
\alpha(v_1\cdots v_k) = (-1)^k v_1\cdots v_k
\]
the automorphism $\alpha$ gives another way to realize the grading on
$\Cliff_{p,q}$ -- it is the grading operator. The $+1$-eigenspace of $\alpha$
is exactly $\Cliff_{p,q}^0(\R)$, and the $-1$-eigenspace is the odd subspace
$\Cliff_{p,q}^0(\R)$.
%
\begin{thm}
There exist $2$-to-$1$ group homomorphisms $\Pin_{p,q} \to O_{p,q}$ and
$\Spin_{p,q} \to SO_{p,q}$, i.e. there exist short exact sequences of groups
\[\begin{tikzcd}
0 \ar[r] & \set{\pm 1} \ar[r] & \Pin_{p,q} \ar[r] & O_{p,q} \ar[r] & 0 \\
0 \ar[r] & \set{\pm 1} \ar[r] &\Spin_{p,q} \ar[r] & SO_{p.q} \ar[r] & 0
\end{tikzcd}\]
\end{thm}
%
\begin{proof}
We first consider the case of $\Pin_{p,q}$. To do this, we need to construct
a group action where $\Pin_{p,q}$ acts on $\R^{p,q}$ by orthogonal transformations.
\iffalse
There exists an involution $T : \Cliff_{p,q}(\R) \to \Cliff_{p,q}(\R)$, where given
the standard orthogonal basis $\set{e_1, \ldots, e_{p+1}}$, we define
\[
T(e_{i_1}\cdots e_{i_k}) = e_{i_k}\cdots e_{i_1}
\]
and extending linearly to the remainder of $\Cliff_{p,q}(\R)$. Given $a \in
\Cliff_{p,q}(\R)$, we denote $T(a)$ by $a^T$.
\fi
We note that for a vector
$v \in \R^{p,q}$ (identifying $\R^{p,q}$ as a subspace of $\Cliff_{p,q}(\R)$),
satisfying $\langle v,v \rangle = \pm 1$, we have that
%$v^T = v$ and
$v\inv = \pm v$. Then given $g \in \Pin_{p,q}$, and $v \in \R^{p,q}$, we claim
that the left action
\[
g \cdot v = \alpha(g)vg\inv
\]
defines the group action we desire. To show this, we must show that this indeed
maps $\R^{p,q}$ back into itself, and that the group elements act by orthogonal
transformations. We first compute this actions on vectors $v \in \R^{p,q}$ with
$\langle v,v \rangle = \pm 1$. In either case, since $v \in \R^{p,q}$, we have
that $\alpha(v) = -v$. First assume that $\langle v,v \rangle = 1$. Then given
$w \in \R^{p,q}$, we compute
%
\begin{align*}
-vwv\inv &= -vwv \\
&= (wv - 2\langle v,w \rangle)v \\
&= w - 2\langle v,w \rangle v
\end{align*}
%
Which is hyperplane reflection about the orthogonal complement of $v$. In the
case that $\langle v,v \rangle = -1$, we compute
%
\begin{align*}
-vwv\inv &= -vw(-v) \\
&= (2\langle -v,w\rangle + wv)(-v) \\
&= w - 2\langle -v,w \rangle(-v)
\end{align*}
which is hyperplane reflection about the orthogonal complement of $-v$,
which is the same as the orthogonal complement of $v$. Then given two
vectors $v_1,v_2 \in \R^{p,q}$, we have that $\alpha(v_1v_2) = v_1v_2$,
so given $w \in \R^{p,q}$,
\[
\alpha(v_1v_2)w(v_1v_2)\inv = (-v_1)(-v_2)wv_2\inv v_1\inv
\]
which is exactly the composition of hyperplane reflection about $v_2^\perp$,
with hyperplane reflection about $v_1^\perp$. Therefore,
$\Pin_{p,q}$ acts by orthogonal transformations, giving us a homomorphism
$\Pin_{p,q} \to O_{p,q}$. This map is surjective by the Cartan-Dieudonn\'e
theorem, and it can be verified that the kernel is $\set{\pm 1}$. In addition,
an even number of hyperplane reflections is orientation preserving, which gives
a surjection $\Spin_{p,q} \to SO_{p,q}$, by restricting the map
$\Pin_{p,q} \to O_{p,q}$. In addition, the kernel $\set{\pm 1}$ is contained in $\Spin_{p,q}$, so this map is also a double covering.
%
\end{proof}
%
We also have the complex Pin and Spin groups, denoted $\Pin_n\C$ and $\Spin_n\C$,
which double cover the complex orthogonal groups $O_n\C$ and $SO_n\C$
respectively.\\

Two simple examples of spin groups occur in dimensions $2$ and $3$.
Since $SO_2 \cong \mathbb{T}$, where
\[
\mathbb{T} = \set{z \in \C ~:~ |z| = 1}
\]
we have that $\Spin_2 \cong \mathbb{T}$, where the covering map is given
by $z \mapsto z^2$. In the case of $SO_3$, we consider the unit quaternions,
which form a Lie group isomorphic to the group $SU_2$. Then given $q \in SU_2$, we define
the map $\varphi_q : \R^3 \to \R^3$ where $\varphi_q(v) = qv\bar{q}$, where
$\bar{q}$ is the quaternionic conjugate of $q$, i.e.
\[
\overline{a + bi + cj +dk} = a -bi -cj -dk
\]
and we identify $v = v^ie_i$ is  with $v^1 i + v^2j + v^3k \in \H$.
The mapping $q \mapsto \varphi_q$ then gives a double cover $SU_2 \to SO_3$.
In particular, $SU_2$ is diffeomorphic to the sphere $S^3$, so the double
covering realizes $SO_3$ as the quotient of $S^3$ by the antipodal map,
giving us that $SO_3$ is diffeomorphic to $\RP^3$. \\

Many examples of low dimensional Spin groups arise from investigating the
relationship between a $4$ dimensional complex vector space $V$ and its second
exterior power $\Lambda^2V$. Fix a volume form $\mu \in \Lambda^4V^*$. This then
induces a symmetric, nondegenerate bilinear form $\langle \cdot,\cdot \rangle$
on $\Lambda^2V$ by
\[
\langle \alpha,\beta \rangle = \langle \alpha \wedge \beta, \mu \rangle
\]
where $\langle \alpha \wedge \beta, \,u \rangle$ denotes the natural pairing of
the vector space $\Lambda^4V$ with its dual $\Lambda^4V^*$. Fix a basis
$\set{e_i}$ for $V$ where $\mu(e_1 \wedge e_2 \wedge e_3 \wedge e_4) = 1$. In this basis,
we see that the group of transformations $\Aut(V, \mu)$ preserving $\mu$ is
isomorphic to the group $SL_4\C$. In addition, each map $T \in \Aut(V, \mu)$
induces a map $\Lambda^2 T : \Lambda^2V \to \Lambda^2V$, which is determined
by the formula $\Lambda^2 T(v \wedge w) = Tv \wedge Tw$. For any $T \in \Aut(V, \mu)$,
the induced map $\Lambda^2 T$ preserves the bilinear form on $\Lambda^2V$, so the
mapping $T \mapsto \Lambda^2V$ determines a group homomorphism
$\Aut(V, \mu) \to \Aut(\Lambda^2V, \langle \cdot,\cdot \rangle)$, where
$\Aut(\Lambda^2V,\langle\cdot,\cdot\rangle)$ denotes the group of linear automorphisms
preserving the bilinear form. The kernel of this map is $\set{\pm \id_V}$, and fixing
an orthogonal basis for $\langle\cdot,\cdot\rangle$ gives us that this map is
a double cover $SL_4\C \to SO_6\C$, so $SL_4\C$ is isomorphic to the
complex spin group $\Spin_4\C$ \\

If we then fix a hermitian inner product $h : V \times V \to \C$, we can
consider the automorphisms $\Aut(V, \mu, h)$ preserving $h$ and $\mu$, which
is isomorphic to the group $SU_4$. The bilinear form $h$ induces a hermitian
inner product (which we also denote $h$) on $\Lambda^2V$ defined by
\[
h(v_1 \wedge v_2, v_3 \wedge v_4) = \det \begin{pmatrix}
h(v_1,v_3) & h(v_1, v_4) \\
h(v_2,v_3) & h(v_2,v_4)
\end{pmatrix}
\]
Then if $T \in \Aut(V,\mu,h)$, $\Lambda^2T$ preserves the bilinear form
$\langle\cdot,\cdot\rangle$ induced by $\mu$ as well as the hermitian inner
product induced by $h$. The group that preserves both of these structures
is isomorphic to $SO_6\C \cap U_6$, which is $SO_6\R$. This gives us that
$SU_4 \cong \Spin_6$. \\

In general, one can play the game of fixing additional structure on $V$
(e.g. a real structure, quaternionic structure, symplectic form) and look
for the induced structure on $\Lambda^2V$. This then gives a map from
automorphisms of $V$ preserving this additional structure to automorphisms
of $\Lambda^2V$ preserving the induced structure. Playing this game then
determines several other low dimensional Spin groups.
%
\begin{align*}
&\Spin_5\C \cong Sp_4\C \qquad\Spin_4 \cong Sp(4)
\qquad\:\:\:\Spin_4\C \cong SL_2\C \times
SL_2\C \\
&\Spin_{1,3}^0 \cong SL_2\C \qquad\: \Spin_{1,2}^0 \cong SL_2\R
\qquad\Spin_{1,5}^0 \cong SL_2\H
\end{align*}
%
Where $Sp_4\C$ denotes the group of $4\times4$ matrices preserving a symplectic
form, $Sp(4) = Sp_4\C \cap U_4$, and $SL_2\H$ denotes the automorphisms of a
$2$ dimensional quaternionic vector space with determinant $1$ when regarded
as $4 \times 4$ complex matrices.
%
\begin{defn}
Given a Pin group $\Pin_{p,q}$, the \ib{Pinor representations} are representations
of $\Pin_{p,q}$ that arise from an irreducible Clifford module $M$ (i.e. the
action of $\Pin_{p,q}$ can be extended to an action of $\Cliff_{p,q}(\R)$). The
\ib{Spinor representations} are defined analogously for the group $\Spin_{p,q}$.
\end{defn}
%
From the classification of Clifford modules, we get a classification of
all the Pinor representations. From the relationship between a Clifford
algebra and its even subalgebra, we also get a complete classification of all
the Spinor representations.
%

%
%
\section{Projective Spin Representations}
%
Using the isomorphisms of the Clifford algebras with matrix algebras
or products of matrix algebras along with the identification of the
even subalgebra with another Clifford algebra, we have a complete classification
of the irreducible modules over the even subalgebras $\Cliff^0_{n}(\R)$.
Restricting to the Spin group $\Spin_n \subset \Cliff^0_n(\R)$, this gives us
the Spin representations. These Spin representations define projective
representations of the group $SO_n$.
%
\begin{prop}
Let $G$ be a group, and $V$ a finite dimensional irreducible representation of $G$.
Then every element of the center acts by scalars, i.e. for $g \in Z(G)$, there
exists a scalar $\lambda_g$ such that for all $v$ in $V$, we have
\[
g \cdot v = \lambda_g v
\]
\end{prop}
%
\begin{proof}
Let $g \in Z(G)$. Then for every $h \in G$ and $v \in V$, we have that
\[
h \cdot (g\cdot v) = g \cdot (h\cdot v)
\]
Therefore, the action of $g$ defines a $G$-equivariant map $V \to V$, which
by Schur's lemma must necessarilly be a scalar multiple of the identity.
\end{proof}
%
Given a Spin representation $\S$, we have that that the elements
$\pm 1$ act by scalars. Then since $SO_n$ is the quotient of $\Spin_n$ by the
subgroup $\set{\pm 1}$ of the center, we get a projective representation of
$SO_n$ on the projectivization $\P\S$ in the following way:
Given an element $A \in SO_n$, we have that $A$ lifts to two elements
$\set{\pm \tilde{A}} \subset \Spin_n$ which differ by $-1$. Since $-1$
acts by a scalar on $\S$, these elements determine the same action of
$\P\S$, giving us a well defined action of $SO_n$ on $\P\S$.

In some sense, the Spin representation $\S$ is not realized canonically,
while the projective Spin representation $\P\S$ is canonical. Given
an isomorphism $\varphi : V \to W$, this induces a unique algebra isomorphism
$\End V \to \End W$ where $A \in \End V$ is mapped to
$\varphi \circ A \circ \varphi\inv$. However, the converse is not true.
%
\begin{prop}
Let $\varphi : \End V \to \End W$ be an algebra isomorphism. Then there does
not exist a unique isomorphism $\psi : V \to W$ inducing $\varphi$.
\end{prop}
%
To prove this, we need a lemma.
%
\begin{lem}
The group of algebra automorphisms $\Aut(M_n\F)$ is isomorphic to the
projective general linear group $PGL_n\F = GL_n\F / Z(GL_n\F)$.
\end{lem}
%
\begin{proof}
Let $\alpha : M_n\F \to M_n\F$ be an algebra automorphism. We know that
$M_n\F$ admits a single irreducible module $M$, which is $\F^n$ with the
standard action. Then $\alpha$ defines another module $M^\alpha$, which
is the same underlying vector space as $M$, with the algebra action given
by $M \cdot v = \alpha(M)v$, where the right hand side is the action of
$\alpha(M)$ on the module $M$. Since $\alpha$ is an algebra automorphism,
$M_n\F$ acts transitively on $M^\alpha$, so it is also an irreducible module,
which must be isomorphic to $M$. Therefore, there exists a module isomorphism
$A : M \to M^\alpha$. Since $M$ and $M^\alpha$ are the same underlying vector
space, $A$ is also a linear isomorphism $A : M \to M$, thought of as a vector space
instead of a module. Then since $A$ is a module homomorphism. we have that
for any $M \in M_n\F$,
\[
A \circ M = \alpha(M) \circ A \implies A \circ M \circ A\inv = \alpha(M)
\]
so $\alpha$ is given by conjugation by $A \in GL(M)$. In a basis, this
tells us that the map $GL_n\F \to M_n\F$ given by conjugation is surjective,
and the kernel of this map is the center of $GL_n\F$, so by the first isomorphism
theorem, we have that $\Aut(M_n\F) \cong PGL_n\F$.
\end{proof}
%
\begin{proof}[Proof of Proposition]
Fix bases $\F^n \to V$ and $\F^n \to W$. This induces isomorphisms
$M_n\F \to V$ and $M_n\F \to W$. Then in these bases, the algebra
isomorphism $\varphi$ is given by an automorphism $M_n\F \to M_n\F$, and
the question now translates to asking whether this automorphism induces
an isomorphism $\F^n \to \F^n$. From the lemma, we know this is false -- an
automorphism of $M_n\F$ only determines an element of $PGL_n\F$, so it
is induced by an entire family of automorphisms differing by $Z(GL_n\F)$.
\end{proof}
%
However, an isomorphism $\varphi : \End V \to \End W$ does induce an isomorphism
$\P V \to \P W$ of projective spaces. To see, this we make an identification
between $1$ dimensional subspaces of $V$ with maximal left ideals of $\End V$.
%
\begin{prop}
There is a bijection
\[
\set{\text{Maximal left ideals of } \End V} \longleftrightarrow \P V
\]
\end{prop}
%
\begin{proof}
Given a line $L \in \P V$, the \ib{annihilator} of $L$ is the set
\[
\mathrm{Ann}(L) = \set{M \in \End V ~:~ M(L) = 0}
\]
In fact, $\mathrm{Ann}(L)$
is a left ideal in $\End V$, since given $A \in \End V$ and $M \in \mathrm{Ann}(L)$,
$L$ lies in the kernel of $A \circ M$. We claim that $\mathrm{Ann}(L)$ is maximal.
Suppose $\mathrm{Ann}(L) \subset I$ is properly contained in a left ideal $I$.
Fix an ordered basis for $V$ in which the first basis element is a
nonzero element of $L$, then elements of $\mathrm{Ann}(L)$ are represented by
matrices with all zeroes in the first column. Then since $\mathrm{Ann}(L)$ is
properly contained in $I$, there exists some $M \in I$ such that
$M \notin \mathrm{Ann}(L)$, which implies that as a matrix, the first column
of $M$ is nonzero. Then pick a matrix $A \in \mathrm{Ann}(L)$ in which
the nonzero columns complete the first column into a basis for $\R^n$. Then
$A + M$ is an invertible element of $\End V$, so $I$ must be all of $\End V$.
Therefore $\mathrm{Ann}(L)$ is maximal. To show that the mapping
$L \mapsto \mathrm{Ann}(L)$ is a bijection, we produce an inverse. Let
$I \subset \End V$ be a maximal ideal. Then we claim that the subspace
\[
\V(I) = \bigcap_{M \in I}\ker M
\]
is a $1$ dimensional subspace of $V$. We note that $\V(I)$ cannot be trivial,
since this would imply that $I$ would contain an invertible element, contradicting
that it is a proper ideal. We also see that it cannot be higher than $2$ dimensional,
since otherwise, $I$ woud be contained in the annihilator of a proper nontrivial
subspace of $\V(I)$, contradicting maximality of $I$. We then claim that
these two mappings are inverses. We certainly have that
$\mathrm{Ann}(\V(I)) \supset I$, so by maximality, this must be $I$. In addition
it is clear that $\V(\mathrm{Ann}(L)) = L$ by the definition of $\V(I)$ and
the annihilator. Therefore, these mappings are inverses.
\end{proof}
%
Therefore, given an algebra isomorphism $\varphi : \End V \to \End W$, this
induces a map $\P V \to \P W$ since the image of a maximal ideal under an
isomorphism is a maximal ideal. In addition, the induced map is a bijection,
since it has an inverse given by the induced map of $\varphi\inv$. In
addition, the group of units $GL(V)$ acts on $\P V$ by right multiplication --
given a maximal ideal $I$ and $A \in GL(V)$, the ideal $I \cdot A$ is also
a maximal left ideal. \\

This gives us a canonical realization of the Spin representations. In the
case that the even subalgebra is isomorphic to a matrix algebra $M_n\F$,
the projective Spin representation is restriction of the action of
$\Cliff_n^0(\R)$ on maximal left ideals of $\Cliff_n^0(\F)$. In the case
that the even subalgebra is isomorphic to a product $M_n\F \times M_n$,
the irreducible modules identify the subalgebras $L$ and $R$
isomorphic to $M_n\F \times \set{0}$ and $\set{0} \times M-n\F$ by finding the
maximal subalgebra that acts nontrivially. Looking at the maximal left ideals
of these subalgebras then identifies the two projective Spin representations.
In addition, since $-1$ acts trivially on left ideals, these projective
Spin representions descend to the quotient $\Spin_n / \set{\pm 1} \cong SO_n$,
giving us the projective representations of $SO_n$.
%

%
\chapter{Spin Structures on Manifolds}
%
\subsectionend $ $\\
%
\emph{Not all the geometrical structures are``equal". It would seem that the
Riemannian and complex structures, with their contacts with other fields of
mathematics and with their richness in results, should occupy a central position
in differential geometry. A unifying idea is the notion of a $G$-structure, which
is the modern version of an equivalence problem first emphasized and exploited
in its various special cases by \'Elie Cartan.} \\
%
\ib{-- ~ Shiing-Shen Chern}
%
\subsectionend
%
\section{Fiber Bundles}
%
\begin{defn}
Let $M$ and $F$ be smooth manifolds. Then a \ib{fiber bundle} over $M$ with
model fiber $F$ is a the data of a smooth manifold $E$ with a smooth map
$\pi : E \to M$ such that for every point $p \in M$, there is a neighborhood
$U \subset M$ containing $p$ such that there exists an diffeomorphism
$\varphi : \pi\inv(U) \to U \times F$ such that the diagram
\[\begin{tikzcd}
\pi\inv(U) \ar[rr, "\varphi"] \ar[dr, "\pi"']&& U \times F \ar[dl, "p_U"] \\
& U
\end{tikzcd}\]
commutes, where $p_U$ denotes projection onto the first factor. The map $\varphi$ is
called a \ib{local trivialization} of the fiber bundle $\pi : E \to M$.
\end{defn}
%
Given a fiber bundle $\pi : E \to M$ we often denote the fiber $\pi\inv(p)$
by $E_p$.
%
\begin{defn}
let $\pi : E \to M$ and $p : L \to M$ be fiber bundles with model fiber $F$
over $M$. A \ib{bundle homomorphism} is a smooth map $\varphi : E \to L$ such
that the following diagram commutes
\[\begin{tikzcd}
E \ar[dr, "\pi"']\ar[rr,"\varphi"] & & L \ar[dl, "p"] \\
& M
\end{tikzcd}\]
\end{defn}
%
An important property of fiber bundles is that they pull back.
\begin{defn}
Let $\pi : E \to M$ be a fiber bundle with model fiber $F$
and let $f : X \to M$ be a smooth map. Then the \ib{pullback} of $E$ by $f$ is
the data of a smooth manifold
\[
f^*E = \set{(x, p)~:~ x \in X, p \in \pi\inv(f(x)) }
\]
along with the projection $p : f^*E \to M$ given by $(x,p) \mapsto x$, giving
$f^*E \to X$ the structure of a fiber bundle over $X$ with model fiber $F$.
The bundle $f^*E$ also comes equipped with a natural map $\alpha : f^*E \to E$
where $\alpha(x,p) = p$. Pullbacks give rise to the commutative diagram
\[\begin{tikzcd}
f^*E \ar[d, "p"']\ar[r, "\alpha"] & E \ar[d, "\pi"] \\
X \ar[r, "f"']& M
\end{tikzcd}\]
and are an instance of a general construction called the \ib{fibered product}.
\end{defn}
%
\begin{defn}
Let $\pi : E \to M$ be a fiber bundle. A \ib{local section} of $\pi : E \to M$
is a smooth map $\sigma : U \to E$ of an open set $U \subset M$ such that
$\pi \circ \sigma = \id_U$. If $U = M$, the section is called a \ib{global section}.
Equivalently, it is the smooth assignment of an element in $E_p$ to each point
$p \in U$. We denote the set of sections of $\pi : E \to M$ over a set $U$
as $\Gamma_U(E)$.
\end{defn}
%
A fiber bundle is a very general construction in which the fibers $F$ do
not necessarily admit extra structure. An important special case of a fiber bundle
is a vector bundle, where the fibers are vector spaces.
%
\begin{defn}
Let $M$ be a smooth manifold. A \ib{vector bundle} over $M$ is fiber bundle
$\pi : E \to M$ with model fiber $\R^n$ (or $\C^n)$ such that the local
trivializations $\varphi : \pi\inv(U) \to U \times \R^n$ (or $\C^n$)
restrict to linear isomorphisms on the fibers, i.e. for all $p \in U$, the restriction
$\varphi\vert_{\pi\inv(p)} : \pi\inv(p) \to \set{p} \times \R^n$
(or $\C^n$) is an isomorphism. The dimension $n$ of the fibers is called the
\ib{rank} of the vector bundle. A \ib{vector bundle homomorphism}
$\varphi : E \to L$ is a bundle homomorphism with the added stipulation that
the restrictions to the fibers $\varphi\vert_{E_x}$ are linear maps.
\end{defn}
%
\begin{exmp} \enumbreak
\begin{enumerate}
  \item Given a smooth manifold $M$, the tangent bundle $TM = \coprod_{p \in M} T_pM$
  is a vector bundle, where the rank is the dimension of $M$.
  \item The \ib{tautological bundle} over $\RP^n$ is the vector bundle that
  assigns to each subspace $\ell \in \RP^n$ itself as its fiber. An analgous
  construction defines the tautological bundle over the Grassmannian
  $\mathrm{Gr}_k(\R^n)$.
\end{enumerate}
\end{exmp}
%
\begin{defn}
A \ib{Lie group} is a smooth manifold $G$ with a group structure such that the
multiplication map $(g,h) \mapsto gh$ and the inversion map $g \mapsto g\inv$
are smooth.
\end{defn}
%
\begin{exmp} \enumbreak
\begin{enumerate}
  \item The group $GL_n\R$ of invertible $n \times n$ matrices is an open
  subset of $M_n\R$, and therefore a smooth $n^2$ dimensional manifold.
  \item The group $SL_n\R$ of $n\times n$ matrices with determinant $1$ is a
  closed submanifold of $GL_\R$.
  \item The orthogonal groups $O_n$ and special orthogonal groups $SO_n$ are Lie groups.
  \item The unitary groups $U_n$ and special unitary groups $SU_n$ are Lie groups.
\end{enumerate}
\end{exmp}
%
Another important class of fiber bundles are principal bundles, in which
the fibers have the structure of $G$-torsors.
%
\begin{defn}
Let $G$ be a Lie group, and $M$ a smooth manifold. A \ib{Principal } $G$\ib{-bundle}
over $M$ is the data of
%
\begin{enumerate}
  \item A smooth manifold $P$ with a map $\pi : P \to M$.
  \item A smooth right $G$-action on $P$ that is free and transitive on the
  fibers of $\pi$.
  \item For every point $p \in M$, a neighborhood $U \subset M$ containing $p$ and
  a $G$-equivariant diffeomorphism $\varphi: \pi\inv(U) \to U \times G$ (where
  the right action on $U \times G$ is right multiplication on the second factor)
  such that we get the commutative diagram
  \[\begin{tikzcd}
  \pi\inv(U) \ar[dr, "\pi"']\ar[rr, "\varphi"]&& U \times G \ar[dl, "p_U"]\\
  & U
  \end{tikzcd}\]
  where $p_U$ denotes projection onto the first factor.
\end{enumerate}
A \ib{principal bundle homomorphism} is a bundle homomorphism $\varphi : P \to Q$
that is $G$-equivariant.
\end{defn}
%
\begin{exmp}
Given a smooth manifold $M$ and a point $p \in M$, a basis of the
tangent space is a linear isomorphism $b : \R^n \to T_pM$. The group
$GL_n\R$ acts freely and transitively on the set of bases $\mathcal{B}_p$ on the right by
$b \cdot g = b \circ g$. Then the \ib{frame bundle} of $M$, denoted
$\mathcal{B}(M)$ is the disjoint union
\[
\mathcal{B}(M) = \coprod_{p \in M}\mathcal{B}_p
\]
where $\pi$ is the projection map $(p,b) \mapsto p$. Then $\mathcal{B}(M)$
is a principal $GL_n\R$ bundle over $M$.
\end{exmp}
%
\begin{exmp}
Given a smooth manifold $M$, a Riemannian metric $g$ induces an inner product
$g_p$ on each tangent space $T_pM$, where $g_p$ denotes the metric $g$ evaluated
at $p$. Then the set of orthonormal bases of $T_pM$ is the
set of all linear isometries $(\R^n, \langle\cdot,\cdot\rangle) \to (M, g_p)$
where $\langle\cdot,\cdot\rangle$ is the standard inner product on $\R^n$.
Then taking the disjoint union over all points $p\in M$ of orthonormal bases
for the tangent spaces $T_pM$ forms the \ib{orthonormal frame bundle}
$\mathcal{B}_O(M)$, which is a principal $O_n$ bundle.
\end{exmp}
%
\begin{defn}
Let $\pi : P \to M$ be a principal $G$-bundle over $M$, and let $F$ be a
smooth manifold with a smooth left $G$ action. Then the \ib{associated fiber
bundle}, denoted $P \times_G F$, is the set
\[
P \times_G F = P \times G / (p,g) \sim (p\cdot h, h\inv g)
\]
Since the group action on $P$ preserves the fibers, the projection
$p_1 : P \times F \to P$ composed with the projection $\pi : P \to M$
descends to the quotient, givng us a projection map
$\Phi : P \times_G F \to M$.
\end{defn}
%
The first thing to check is that $P \times_G F$ is a fiber bundle, justifying
the name.
%
\begin{prop}
Let $\pi : P \to M$ be a principal $G$-bundle and $F$ a manifold with a left $G$ action.
then the associated bundle $ \Phi : P \times_G F \to M$ is a fiber bundle with
model fiber $F$.
\end{prop}
%
\begin{proof}
We wish to provide local trivializations  $\Phi\inv(U) \to U \times F$ for the
associated bundle. Fix a local trivialization  $\psi : \pi\inv(U) \to U \times G$.
Then $\psi$ is of the form $\psi(p) = (\pi(p), \tilde{\psi}(p))$ for some
$\tilde{\psi} : \pi\inv(U) \to G$, satisfying $\tilde{\psi}(p\cdot g) = \tilde{\psi}(p)g$.
Define
\begin{align*}
\varphi : \Phi\inv(U) &\to U \times F \\
[p,f] &\mapsto (\pi(p), \tilde{\psi}(p)\cdot f)
\end{align*}
We first note that this is well defined on equivalence classes, since
\[
[p\cdot g, g\inv \cdot f] \mapsto (\pi(p \cdot g), \tilde{\psi}(p \cdot g)
\cdot g\inv \cdot f)
= (\pi(p), \psi(p)\cdot f)
\]
So $\varphi$ is well defined. We also note that $\Phi(\varphi[p,f]) = \pi(p)$
by how $\Phi$ was defined, so $\varphi$ is a local trivialization provided
it is a homeomorphism. To show this, we construct an inverse.
Define $\alpha : U \times F \to \Phi\inv(U)$
by $\alpha(u,f) = [\psi\inv(u,e), f]$, where $e$ denotes the identity element
of $G$. We then claim that $\alpha$ is the inverse. We compute
%
\begin{align*}
(\varphi \circ \alpha)(u,f) &= \varphi[\psi\inv(u,e),f] \\
&= (u, e\cdot f)\\
&= (u,f)
\end{align*}
%
In the other direction, we compute
%
\begin{align*}
(\alpha \circ \varphi)[p,f] &= \alpha(\pi(p), \tilde{\psi}(p)) \\
&= [\psi\inv(\pi(p), e), f] \\
&= [p, f]
\end{align*}
%
Therefore, $\varphi$ is a local trivialization, giving us that
$\Phi: P \times_G F \to M$ is a fiber bundle with model fiber $F$.
%
\end{proof}
%
There is a correspondence between sections of an associated bundle and
$G$-equivariant maps $P \to F$.
%
\begin{prop}
Let $\pi : P \to M$ be a principal $G$-bundle. Then there is a bijection
\[
\set{G\text{-equivariant maps } P \to F} \longleftrightarrow \Gamma_M(P \times_G F)
\]
\end{prop}
%
\begin{proof}
Since $F$ has a left $G$-action, we first convert this to a right $G$-action
by $f \cdot g = g\inv \cdot f$. Then what we mean by a $G$-equivariant map is
a map $\varphi : P \to F$ such that
\[
\varphi(p\cdot g) = g\inv\cdot \varphi(p)
\]
We then wish to use $G$-equivariant map $\varphi$ to produce a section
$\tilde{\varphi} : M \to P \times_G F$ of the associated bundle. For a
point $x \in M$, pick any $p \in \pi\inv(x)$ in the fiber. Then define
\[
\tilde{\varphi}(x) = [p, \varphi(p)]
\]
where $[p, \varphi(p)]$ denotes the equivalence class of $(p, \varphi(p))$ in
$P \times_G F$. We first claim that this map is well defined, i.e. it is
independent of our choice of point $p \in \pi\inv(x)$. We know that $G$ acts
freely and transitively on $\pi\inv(x)$, so all points in the fiber are of
the form $p \cdot g$ for a unique $g \in G$. Then we have that for any $g \in G$
\[
(p\cdot g, \varphi(p \cdot g)) = (p \cdot g, g\inv \cdot \varphi(p)) \sim (p, \varphi(p))
\]
where we use the $G$-equivariance of $\varphi$ and the definition of the equivalence
relation on the associated bundle. In addition, this induced map is
a section of $P \times_G F \to M$, since the image is represented by an
element of $P \times F$ with an element of the fiber $\pi\inv(x)$ in the first
factor. \\

Conversely, given a section $\sigma : M \to P \times_G F$, we wish to produce
a $G$-equivariant map $P \to F$. Given such a section, and a point $x \in M$,
we have that $\sigma(x) = [p, f]$ for some $p \in P$ and $f \in F$. Then
define $\tilde{\sigma} : P \to F$ such that $\tilde{\sigma}(p) = f$, and
$\tilde{\sigma}(p \cdot g) = g\inv \cdot f$. Since $G$ acts freely and transitively
on the fibers, this is well defined, and specifies the map on every
point of $P$. In addition, $\tilde{\sigma}$ is $G$-equivariant by construction.
Then the maps
\[
\set{G\text{-equivariant maps } P \to F} \longleftrightarrow \Gamma_M(P \times_G F)
\]
we provided are easily verified to be inverses to each other, giving us the
correspondence.
\end{proof}
%
In some sense, the geometry of the associated fiber bundle $P \times_G F$
is controlled by the group $G$, as $G$ determines a disinguished
group of symmetries of the fiber $F$.
%
\begin{exmp}[The tangent bundle as an associated bundle]
Given a manifold $M$, we can take the frame bundle $\pi : \B(M) \to M$, which is a
principal $GL_n\R$ bundle. The group $GL_n\R$ acts linearly on $\R^n$ in the
standard way, giving us an associated vector bundle $\B(M) \times_{GL_n\R} \R^n$.
We claim that this bundle is isomorphic to the tangent bundle $TM$, i.e. there
exists an diffeomorphism $\varphi : \B(M) \times_{GL_n\R} \R^n \to TM$ that
restricts to linear isomorphisms on the fibers and the diagram
\[\begin{tikzcd}
\B(M) \times_{GL_n\R} \R^n \ar[rr, "\varphi"] \ar[dr] && TM \ar[dl] \\
& M
\end{tikzcd}\]
commutes, where the maps to $M$ are the bundle projections. Recall that
elements of $\B(M) \times_{GL_n\R} R^n$ are represented by pairs $(b, v)$,
where $b: \R^n \to T_{\pi(b)}M$ is a linear isomorphism, and $v$ is a vector in $\R^n$.
Then define $\varphi$ by
\[
\varphi[b,v] = (\pi(b), b(v))
\]
This is well defined, since $\varphi[b\circ g, g\inv(v)] = (\pi(b \circ g),
(b\circ g)(g\inv(v))) = (\pi(b), b(v))$. This is an isomorphism, where the
inverse mapping maps $(p,v) \in TM$ to $(b, \tilde{v})$ where $b$ is any
basis of $T_pM$ and $\tilde{v}$ is the coordinate representation of $v$ in the
basis $b$. In this example, we see that the associated bundle identifies the same vector
under different coordinate transformations, which defines the symmetries of
$TM$.
\end{exmp}
%
In general, given a rank $n$ vector bundle $E \to M$, we can construct the
frame bundle  $\B(E)$ for $E$ and recover $E$ by taking the associated bundle
$\B(E) \times_{GL_n\R} \R^n$, so the process of taking frames and constructing
associated bundles are inverses of each other.
%
\begin{defn}
Let $\pi : P \to M$ be a principal $G$-bundle, and $\rho : H \to G$ a group
homomorphism. The map $\rho$ gives $G$ a left $H$ action where
$h \cdot p = \rho(h) \cdot p$. A \ib{reduction of structure group} is the
data of a principal $H$ bundle $\varphi : Q \to M$ and an $H$-equivariant bundle
homomorphism $F : Q \to P$.
\end{defn}
%
The map $F : Q \to P$ induces a map $\tilde{F} : Q \times_H G \to P$, where we
map the equivalence class $[q,g]$ to $F(q) g$. This is well defined
on equivalences classes since
%TODO this next line might be unnecesary, since it is the same map
%constructed using the correspondence with sections -- double check this.
\[
(q \cdot h, \rho(h)\inv g) \mapsto F(q \cdot h)\rho(h)\inv g = F(q)\rho(h)\rho(h)\inv g
= F(q)g
\]
%
\begin{exmp}[Reduction from $GL_n\R$ to $O_n$]
Let $M$ be a smooth manifold, and $\pi : \B(M) \to M$ its bundle of frames.
The inclusion map $\iota: O_n \hookrightarrow GL_n\R$ gives an action of $O_n$ on
$\B(M)$, where given $b \in \B(M)$ and $T \in O_n$, $b \cdot T = b \cdot \iota(T)$.
We then take the quotient by this $O_n$ action, giving us a quotient map
$q : \B(M) / O_n$. Since the inclusion is injective, the $O_n$ action
is free on $\B(M)$, so this gives $q : \B(M) \to \B(M) / O_n$ the structure of a
$O_n$ bundle. In addition, the action of $O_n$ preserves the fibers of
$\pi : \B(M) \to M$, so $\pi$ descends to the quotient, so $\B(M) / O_n \to M$
is a fiber bundle with model fiber $GL_n\R / O_n$. Since $GL_n\R$ deformation
retracts onto $O_n$ via the Gram-Schmidit algorithm, $GL_n\R / O_n$ is
contractible, so the fiber bundle $\B(M)/ O_n \to M$ admits global sections.
Then given a section $\sigma : M \to \B(M) / O_n$, this gives an $O_n$ bundle
over $M$ via the pullback $\sigma^*\B(M)$. In addition, we get a $O_n$-equivariant
map $\sigma^*\B(M) \to \B(M)$ given by $(p, b) \mapsto (p, \iota(b))$. The
bundle $\sigma^*\B(M)$ can be thought of as the bundle of orthonormal frames
with respect to some Riemannian metric on $M$, and the fact that $\B(M) / O_n$
admits sections corresponds to the fact that every manifold admits a Riemannian
metric.
\end{exmp}
%
Principal bundles can be thought of as a generalization of covering spaces.
For discrete groups $G$, the data of a principal $G$-bundle $P \to M$ is the
data of a (possibly disconnected) $|G|$-fold cover of $M$ with deck transformation
group isomorphic to $G$. Discreteness of $G$ allows for a very complete description
of all principal $G$-bundles over a given manifold $M$.
%
\begin{defn}
Let $(M,x_0)$ be a pointed topological space (in our case, usually a manifold),
i.e. a space $M$ with a choice of distinguished basepoint $x_0 \in M$. Then
a \ib{pointed principal $G$-bundle} over $M$ is the data of a principal $G$-bundle
$\pi : P \to M$, along with a choice of basepoint $p_0 \in \pi\inv(x_0)$. We
denote this as $\pi : (P,p_0) \to (M, x_0)$. We denote the set of pointed principal
$G$-bundles over $(M,x_0)$ as $\Bun_G(M ,x_0)$.
\end{defn}
%
We now recall some properties of covering spaces. A pointed covering space
$\pi : (\tilde{M}, \tilde{x_0}) \to (M, x_0)$ has the \ib{path lifting} property,
i.e. given a path $\gamma : I \to M$ with $\gamma(0) = x_0$, there exists a
unique path $\tilde{\gamma} : I \to \tilde{M}$ with $\gamma(0) = \tilde{x_0}$
such that $\pi \circ \tilde{\gamma} = \gamma$. In addition, the covering
space has the \ib{homotopy lifting property}. Given paths
$\gamma_1, \gamma_2 : I \to M$ with $\gamma_i(0) = x_0$ and a homotopy
$F : I \times I \to M$ from $\gamma_1$ to $\gamma_2$, there exists a unique
homotopy $\tilde{F} : I \times I \to \tilde{M}$ between $\tilde{\gamma}_1$
and $\tilde{\gamma_2}$. These two facts define an action of $\pi_1(M, x_0)$
on the fiber $\pi\inv(x_0)$ where given a homotopy class $[\gamma]$ of
loops, the action of $[\gamma]$ on $p \in \pi\inv(x_0)$ is the endpoint
of $\tilde{\gamma}(1)$ of the lifted loop $\tilde{\gamma}$. The unique
path lifting and homotopy lifting properties guarantees that this defines
a group action, called the \ib{holonomy} of the covering space. In the
special case that our covering space is a principal $G$-bundle with a discrete $G$,
the point $\tilde{\gamma}(1)$ is equal to $\tilde{x_0} \cdot g$ for a unique
group element $g \in G$, so the holonomy action determines a group homomorphism
$\pi_1(M, x_0) \to G$. In fact, the holonomy completely determines the bundle.
%
\begin{thm}
For a discrete group $G$ and a connected manifold $M$, there is a bijection
\[
\Bun_G(M,x_0) \longleftrightarrow \hom(\pi_1(M,x_0), G)
\]
\end{thm}
%
\begin{proof}
We once more provide maps in both directions. Every pointed principal $G$-bundle
$(P,p_0) \to (M, x_0)$ determines a group homomorphism $\pi_1(M,x_0) \to G$
by the holonomy action. In the other direction, given a group homomorphism
$\varphi : \pi_1(M, x_0) \to G$, we want to construct a pointed principal $G$
bundle $(P, p_0) \to (M, x_0)$ with holonomy $\varphi$. Every pointed space
$(M, x_0)$ has a \ib{universal cover} $(\tilde{M}, \tilde{x}_0) \to (M,x_0)$,
which has a free and transitive action by $\pi_1(M,x_0)$, making it a pointed
principal $\pi_1(M,x_0)$ bundle over $(M,x_0)$. In addition, the homomorphism
$\varphi : \pi_1(M, x_0) \to G$ determines a left action of $\pi_1(M,x_0)$
on $G$ by $[\gamma] \cdot g = \varphi[\gamma] g$. This allows us to define
the associated bundle
\[
P = \tilde{M} \times_{\pi_1(M, x_0)} G = \tilde{X} \times G
/ (x,g) \sim (x \cdot [\gamma], \varphi[\gamma]\inv g)
\]
where our distinguished basepoint $p_0 \in P$ is $p_0 = [\tilde{x}_0, e]$.
This is a fiber bundle with model fiber $G$, and the right action of $G$
on $\tilde{X} \times G$ descends to a free an transitive action on the quotient,
making it a principal $G$-bundle. We then check the holonomy action of a
loop $[\gamma] \in \pi_1(M, x_0)$. We compute for $p$ in the fiber over $x_0$
\[
p \cdot [\gamma] = [\tilde{x_0} \cdot [\gamma], h] \sim [x_0, \varphi[\gamma]\cdot h]
\]
So the holonomy is $\varphi$, giving us the desired correspondence.
\end{proof}
%
The set $\Bun_G(M, x_0)$ comes equipped with an action of $G$, where the action
of $g \in G$ on a pointed bundle $(P, p_0) \to (M, x_0)$  is by permuting basepoints,
i.e. $(P, p_0) \cdot g = (P, p_0 \cdot g)$, where the covering map is left untouched.
Quotienting by this group action gives us the identification of $\Bun_G(M,x_0) / G$
with isomorphism classes of principal $G$-bundles $P \to M$. where the
isomorphisms are without basepoints. From the perspective of pointed bundles
$(P,p_0) \to (M, x_0)$ as homomorphisms $\varphi : \pi_1(M, x_0) \to G$,
the group action of $G$ on $\Bun_G(M,x_0)$ is an action by conjugation.
Given $(P, p_0) \to (M, x_0)$ with holonomy $\varphi : \pi_1(M, x_0) \to G$,
we want to know the holonomy $\varphi_g$ of the bundle
$(P, p_0 \cdot g) \to (M, x_0)$. Given a loop $\gamma$ based at $x_0$, let
$\tilde{\gamma}$ denote its lift to $(P, p_0)$ by uniqueness of path lifting,
the lift of $\gamma$ to $(P, p_0 \cdot g)$ is the path
$\tilde{\gamma}_g(t) = \tilde{\gamma}(t) \cdot g$. Then
%
\begin{align*}
\tilde{\gamma_g(1)} &= \tilde{\gamma}(1) \cdot g \\
&= p_0 \cdot \varphi[\gamma] \cdot g \\
&= p_0 \cdot g \cdot g\inv \cdot \varphi[\gamma] \cdot g
\end{align*}
%
So $\varphi_g = g \inv\cdot\varphi\cdot g$.
%

%
%
\section{Dirac Operators in $\R^n$}
%
One important application of Clifford algebras and Spin groups comes from
physics. In the process of developing a relativistic equation for the electron,
Paul Dirac saw the need for a first order differential operator $D$ such
that $D$ squared to the Laplace operator.
\[
\Delta = -\sum_i \frac{\partial^2}{\partial(x^i)^2}
\footnote{Our reasoning for choosing this sign convention for $\Delta$ is twofold
-- one reason is that the spectrum of $\Delta$ is positive with this choice
of sign, and the second is that this definition coincides with a generalized
Laplacian on a Riemannian manifold.}
\]
If $D$ were the be first order, it would have to be written as
\[
D = a^i \frac{\partial}{\partial x^i}
\]
for some coefficients $a^i$. However, it is clear that choosing scalar coefficients
for the $a^i$ will not suffice. For example, in $\R^2$ with the standard coordinates
$x$ and $y$, we see that given any first order operator
$D = a^1\partial_x + a^2\partial_y$ satisfies
\[
D^2 = \left(a^1\frac{\partial}{\partial x} + a^2\frac{\partial}{\partial y}\right)^2
= (a^1)^2\frac{\partial^2}{\partial x^2} + a^1a^2\frac{\partial^2}{\partial x \partial y}
+ a^2a^1 \frac{\partial^2}{\partial y \partial x} + (a^2)^2 \frac{\partial^2}{\partial y^2}
\]
From this equation, we see that the $a^i$ must square to $-1$, and we must also
have that $a^1a^2 + a^2a^1 = 0$ in order for the mixed partial terms to vanish.
This is not possible if the $a^i$ are scalars (in either $\R$ or $\C$).
However, the required relations are exactly the relations between orthogonal
basis vectors in the Clifford algebra $\Cliff_{0,n}(\R)$!
%
\begin{defn}
Let $\set{e_i}$ be the standard basis for $\R^n$, $\set{e^i}$ its dual basis, and
$x^1, \ldots ,x^n$ the standard coordinates on $\R^n$.
The \ib{Dirac operator} on $\R^n$ is the first order differential operator
\[
D = e^i\frac{\partial}{\partial x^i}
\]
\end{defn}
%
It is not clear from the definition what function space $D$ should act on.
The partial derivative operators make sense for any vector valued function, but
multiplcation by $e^i$ does not make sense an arbitrary vector space -- it must
be a Clifford module. Therefore, $D$ is an operator
$D: C^\infty(\R^n, M) \to C^\infty(\R^n, M)$ on smooth functions from $\R^n$
to a Clifford module $M$.
%
%TODO More stuff later
%
%
%
\section{Spin Structures}
%
Every manifold $M$ admits a Riemannian metric, so there always exists a
reduction of structure group from $GL_n\R$ to $O_n$. If $M$ is orientable,
we can reduce the structure group further to $SO_n$, which corresponds to a
choice of orientation on $M$. A \ib{Spin structure} on $M$
is a further reduction of structure group to $\Spin_n$, with respect to the
double cover $\Spin_n \to SO_n$, i.e. the data of a principal $\Spin_n$ bundle
$P \to M$, along with a $\Spin_n$-equivariant map $P \to \B_{SO}(M)$, where
$\B_{SO}(M)$ denotes the bundle of positively oriented orthonomal frames. In
particular, this map is a double cover, and fiber-wise is the double covering
$\Spin_n \to SO_n$.
Depending on the value $n$, there exist either one or two Spin representations
for $\Spin_n$. Let $\S$ denote the direct sum all of all the Spin representations
for $\Spin_n$, i.e. $\S$ is equal to the single Spin representation in the
case that $\Spin_n$ only admits one representation, and $\S = \S^+ \oplus \S^-$
in the case that $\Spin_n$ admits two inequivalent Spin representations $\S^+$
and $\S^-$. Then given a Spin structure on $M$ with a principal $\Spin_n$ bundle
$P$, the \ib{spinor bundle} over $M$ is the associated vector bundle
$P \times_{\Spin_n} \S$.
%
\begin{exmp}[Spin structures on $S^1$]
The group $\Spin_1$ is equal to $\set{\pm 1}$, so a principal $\Spin_1$-bundle
over $\S^1$ is a double cover $\pi : P \to \S^1$ along with a $\Spin_1$-equivariant
map $P \to \B_{SO_1}(S^1)$. Since $SO_1 = 1$, $\B_{SO}(S^1)$ is the trivial
bundle $S^1 \times \set{1}$. Therefore, specifying a $\Spin_1$-equivariant map
$P \to \S^1 \times \set{1}$ is no additional data, since we are forced to map
all of the fiber $\pi\inv(x)$ to $(x,1)$ for any $x \in S^1$. Consequently,
all double covers give rise to a reduction of structure group to $\Spin_1$.
There are only two double covers of $\Spin_1$. One of them is the disconnected
double cover, which is the disjoint union $S^1 \coprod S^1$, which we denote
as $\pi_1 : P_1 \to S^1$. The other is the connected double cover, which the
circle covering itself via the map $z \mapsto z^2$, which we denote as
$\pi_2 : P_2 \to S^1$. For convenience, we parameterize $P_2$ with angles
$\theta \in [0, 4\pi)$, so the covering map is given by $\theta \mapsto e^{i\theta}$.
The Spin representation is the sign representation on $\R$, where $-1$
acts by multiplication by $-1$, and the complexifying gives us an action on
$\C$ where $-1$ acts by multiplication by $-1$, giving us two spinor bundles
$P_1 \times_{\Spin_1} \C$ and $P_2 \times_{\Spin_1} \C$.

In the first case,the associated bundle is a trivial bundle. Using the identification
with $\Spin_1$-equivariant maps $P_1 \to \C$ with sections of the associated
bundle, it suffices to find such a map to produce a global section of
$\pi_1 : P_1 \to S^1$. Write $P_1$ as the disjoint union $S_1 \coprod S_1$,
with the circles parameterized by angles $\theta,\varphi \in [0,2\pi)$.
The $\Spin_1$ action is then given by $\theta \mapsto -\varphi$ and
$\varphi \mapsto -\theta$. Then the mappings $\theta \mapsto e^{i\theta}$
and $\varphi \mapsto e^{-i\varphi}$ define a $\Spin_1$-equivariant map
$P_1 \to \C$, giving us a trivialization of the associated bundle
$P_1 \times_{\Spin_1} \C$. In addition, we see that sections of $P_1\times_{\Spin_1}\C$
are equivalent data to maps $S^1 \to \C$, since once we map one of the components
of $P_1$ into $\C$, this entirely determines how we need to map the other component
in order to remain $\Spin_1$-equivariant. This further allows us to identify
sections of the spinor bundle with $2\pi$-periodic functions $\R \to \C$.

In the second case, the bundle is also trivial! We again contruct a trivialization
for the associated bundle by providing a $\Spin_1$-equivariant map
$\sigma : P_2 \to \C$. Using the parameterization of $P_2$ with angles
$\theta \in [0,4\pi)$, the $\Spin_1$ action on $P_2$ is given by
$-1 \cdot \theta = \theta + 2\pi \mod 4\pi$. Then define $\sigma$ by
$\sigma(\theta) = e^{i\theta/2}$. This map is $\Spin_1$-equivariant, so
it produces a trivialization of the spinor bundle $P_2 \times_{\Spin_1} \C$.
In addition, we see that sections of the spinor bundle correspond
with $2\pi$-antiperiodic functions, i.e. functions $\psi : \R \to \C$ such that
$f(\theta) = -f(\theta + 2\pi)$, using the fact that we parameterized $P_2$
with angles from $[0, 4\pi)$. Another way to write a $2\pi$-antiperiodic map
$\psi$ is as a product $\psi(\theta) = e^{i\theta/2}f(\theta)$ for a $2\pi$-periodic
function $f$, which is the representation of the section $\psi$ with respect to the
trivialization $\sigma$ defined above.

The two different spin structures produced two isomorphic vector bundles,
but there is still a way to distinguish between the two -- their Dirac operators.
The Clifford algebra $\Cliff_{0,1}(\R)$ is isomorphic to $\C$ as an $\R$-algebra
via the mappings $1 \mapsto 1$, and $e_1 \mapsto i$, so the Dirac operator on
$\R$ can also be written as $i \frac{d}{dt}$. We can then use our identifications
of sections of the spinor bundles with functions $\R \to \C$ to investigate the
Dirac operators on each bundle. In the case of the disconnected double cover $P_1$,
we have the identifications of sections of $P_1 \times{\Spin_1} \C$ with
$2\pi$-periodic functions $\R \to \C$. Then given a section $\psi$, we
can identify it as a function $\R \to \C$, and use the Dirac operator in $\R$,
which will again be a $2\pi$-periodic function, giving us another section,
giving us that the Dirac operator $D_1$ on sections of $P_1 \times_{\Spin_1} \C$
is just $i\frac{d}{d\theta}$. For the connected double cover, we used the
global section associated to $\sigma(\theta) = e^{i\theta /2}$ to identify
sections of $P_2 \times_{\Spin_1} \C$ as products
$\psi(\theta) = e^{i\theta/2}f(\theta)$ for a $2\pi$-periodic function $f : \R \to \C$.
Then applying the Dirac operator from $\R$ to this function, we get
%
\begin{align*}
D\psi(\theta) &= i\frac{d}{d\theta}\left( e^{i\theta/2}f(\theta) \right) \\
&= e^{i\theta/2}\frac{\partial f}{\partial \theta} - \frac{1}{2}e^{i\theta/2}
\end{align*}
%
So in the local trivialization $\sigma(\theta) = e^{i\theta/2}$, the operator $D_2$
operates on $2\pi$-periodic functions, just like $D_1$, and is given by
$D_2 = i\partial_\theta - \frac{1}{2}$. In particular, the first operator $D_1$ has
integer spectrum, and the spectrum of $D_2$ is the spectrum of $D_1$ shifted by
$\frac{1}{2}$, which allows us to distinguish to two Spin structures on $S^1$
by their Dirac operators.
%
\end{exmp}
%
%TODO Orientation on M induces an orientation on the boundary of M.
Given an orientable manifold $M$, we can reduce the structure group
of the frame bundle $\B(M)$ to $SO_n$, giving us a principal $SO_n$ bundle
$\B_{SO}(M)$. If $M$ has a nonempty boundary, this induces an orientation on
$\partial M$ in the following way: by first reducing the structure group to $O_n$,
we get a Riemannian metric $g$ on $M$. Then given a point $p \in \partial M$,
the tangent space of the boundary $T_p\partial M$ is a codimension $1$ subspace
of $T_pM$. The Riemannian metric $g$ allows us to pick a distinguished complementary
subspace to $T_p\partial M$ -- the orthogonal complement $(T_p \partial M)^\perp$.
From this subspace, we have a distinguished choice of vector -- the outward
normal vector. In appropriate coordinates, the inclusion of the tangent space
$T_p\partial M$ is given locally by the inclusion
%
\begin{align*}
\R^{n-1} &\hookrightarrow \R^n \\
(x^1, \ldots, x^{n-1}) &\mapsto (0, x^1, \ldots, x^{n-1]})
\end{align*}
%
and the outward normal is the unit length vector with a positive first component.
This defines a vector field $N$ along $\partial M$, where the value $N_p$
of $N$ at the point $p$ is the outward normal vector in $T_pM$.
On the boundary $\partial M$, we restrict the frame bundle $\B_{SO}(M)$
to $\partial M$ by pulling back by the inclusion $\partial M \hookrightarrow M$,
giving us the restricted bundle $\B_{SO}(M)\vert_{\partial M}$, which is a principal
$SO_n$-bundle over $\partial M$. An element $b \in \B_{SO}(M)$ is an orientation
preserving linear isometry $(\R, \langle \cdot,\cdot\rangle) \to (T_pM, g_p)$, and
using the normal vector, we can define a subbundle
$\B_{SO}(\partial M) \subset \B_{SO}(M)\vert_{\partial M}$ by
\[
\B_{SO}(\partial M) = \set{b \in \B_{SO}(M)\vert_{\partial M} ~:~
b(e_1) = N_{\pi(e_1)}}
\]
where $e_1$ denotes the first standard basis vector of $\R^n$, and $\pi$ is
the bundle projection. We then get an $SO_{n-1}$ action on $\B_{SO}(\partial M)$,
where we include $SO_{n-1} \hookrightarrow SO_n$ as matrices of the form
\[
\begin{pmatrix}
1 & 0 \\
0 & A
\end{pmatrix}
\]
Where $A \in SO_{n-1}$. This then acts on each fiber of $\B_{SO}(\partial M)$
by precomposition. This action is free and transitive, so this gives
$\B_{SO}(\partial M)$ the stucture of a principal $SO_{n-1}$-bundle over
$\partial M$. In this way, we see that an orientation on $M$ determines
an orientation on the boundary. The preceding discussion can be summarized in
the diagram
\[\begin{tikzcd}
\B_{SO}(\partial M) \ar[r, hookrightarrow] \ar[dr] &
\B_{SO}(M)\vert_{\partial M} \ar[d] \ar[r, hookrightarrow] & \B_{SO}(M) \ar[d] \\
& \partial M \ar[r, hookrightarrow] & M
\end{tikzcd}\]
%TODO Spin structure on M induces a Spin structure on the boundary of M.

In a similar fashion, a Spin structure on $M$ will also induce a Spin structure
on $\partial M$, though the process is slightly more involved. Given a Spin
manifold $M$, it comes equipped with a principal $\Spin_n$-bundle $\B_{\Spin}(M)$
along with a $\Spin_n$-equivariant map $\B_{\Spin}(M) \to \B_{SO}(M)$, where
the Spin action on $\B_{SO}(M)$ is induced by the double cover $\Spin_n \to SO_n$.
Just as before, we pullback both $\B_{SO}(M)$ and $\B_{\Spin}(M)$ along
the inclusion $\partial M \hookrightarrow M$. In addition, we construct the
$SO_{n-1}$-bundle $\B_{SO}(\partial M)$, which gives us the diagram
\[\begin{tikzcd}
&\B_{\Spin}(M)\vert_{\partial M} \ar[r, hookrightarrow]\ar[d] & \B_{\Spin}(M) \ar[d] \\
\B_{SO}(\partial M) \ar[r, hookrightarrow] \ar[dr] &\B_{SO}(M)\vert_{\partial M}
\ar[d] \ar[r, hookrightarrow] & \B_{SO}(M) \ar[d] \\
&\partial M \ar[r, hookrightarrow] & M
\end{tikzcd}\]
This diagram tells us the exact ingredients we need to construct the
$\Spin_{n-1}$-bundle over $\partial M$ -- it must be the pullback of
$\B_{\Spin}(M)\vert_{\partial M}$ along the inclusion
$\B_{SO}(\partial M) \hookrightarrow B_{SO}(M)\vert_{\partial M}$, so it fits
into the commutative square
\[\begin{tikzcd}
\B_{\Spin}(\partial M) \ar[d] \ar[r, hookrightarrow] &
\B_{\Spin}(M)\vert_{\partial M}\ar[d] \\
\B_{SO}(\partial M) \ar[r, hookrightarrow] & \B_{\Spin}(M)\vert_{\partial M}
\end{tikzcd}\]
In addition, it comes equipped with a map $p : \B_{\Spin}(\partial M) \to \partial M$
by composing the map $\B_{\Spin}(\partial M) \to \B_{SO}(\partial M)$ with the projection
$\pi : \B_{SO}(\partial M)$. However, it is not immediately clear that the pullback
bundle, (which we will suggestively denote by $\B_{\Spin}(\partial M)$) is
indeed a principal $\Spin_{n-1}$-bundle over $\partial M$. As a principal
$\Spin_n$-bundle, $\B_{\Spin}(M)\vert_{\partial M}$ comes equipped with an
action of $\Spin_n$. We then take the subgroup $G \subset \Spin_n$ of elements
that preserve the fibers of the map $p$. Since the $\Spin_n$ action on
$\B_{\Spin}(M)$ is free and transtive on the fibers of
$\B_{\Spin}(M)\vert_{\partial M} \to \partial M$, the action of $G$ on
$\B_{\Spin}(\partial M)$ will be as well, giving it the structure of a principal
$G$-bundle over $\partial M$. We then claim that $G \cong \Spin_{n-1}$. Since
the fibers are $G$-torsors, it suffices to check on a single fiber of the map
$\B_{\Spin}(M) \to \partial M$. Fix point $x \in \partial M$, and an
oriented orthonormal frame $b \in \pi\inv(x)$. This then determines a diffeomorphism
$\varphi : SO_{n-1} \to \pi\inv(x)$. where $\varphi(g) = g \cdot x$. This
diffeomorphism induces a group stucture on the fiber $\pi\inv(x)$ such
that $\varphi$ is a group isomorphism. Then the restriction of the double cover
$\B_{\Spin}(M)\vert_{\partial M} \to \B_{SO}(M)\vert_{\partial M}$ to
$\B_{\Spin}(\partial M)$ is also a double cover. Fixing a point $b'$ in the
preimage of $b$ under $\B_{\Spin}(\partial M) \to B_{SO}(\partial M)$ then
determines an isomorphism of the fiber with $G$, which shows that $G$ double
covers $SO_{n-1}$. Therefore, $G \cong \Spin_{n-1}$. and $\B_{Spin}(\partial M)$
determines a Spin structure on $\partial M$. All in all, the construction
is summarized by the diagram
\[\begin{tikzcd}
\B_{\Spin}(\partial M) \ar[d] \ar[r, hookrightarrow] & \B_{\Spin}(M)\vert_{\partial M}
\ar[r, hookrightarrow]\ar[d] & \B_{\Spin}(M) \ar[d] \\
\B_{SO}(\partial M) \ar[r, hookrightarrow] \ar[dr] &\B_{SO}(M)\vert_{\partial M}
\ar[d] \ar[r, hookrightarrow] & \B_{SO}(M) \ar[d] \\
&\partial M \ar[r, hookrightarrow] & M
\end{tikzcd}\]
%

%
%
\section{$\Pin^\pm_n$ Structures}
%
The same general construction for Spin structures works for $\Pin^\pm_n$ as well.
%
\begin{defn}
Let $M$ be a smooth manifold. Then a \ib{$\Pin_n^\pm$ structure} on $M$ is the
data of a principal $\Pin_n^\pm$-bundle $P \to M$, along with a
$\Pin^\pm_n$-equivariant map $P \to \B_O(M)$, where the $\Pin^\pm_n$ action on
$\B_O(M)$ is given by the double cover $\Pin^\pm_n \to O_n$.
\end{defn}
%
Just like with Spin, a $\Pin^\pm_n$ structure on $M$ induces a $\Pin^\pm_{n-1}$
on the boundary using the outward unit normal, which gives the analogous diagram
\[\begin{tikzcd}
\B_{\Pin^\pm}(\partial M) \ar[d]\ar[r, hookrightarrow] & \B_{\Pin^\pm}(M)\vert_{\partial M}
\ar[d]\ar[r, hookrightarrow] & \B_{\Pin^\pm}(M) \ar[d]\\
\B_O(\partial M) \ar[dr]\ar[r, hookrightarrow] & \B_O(M)\vert_{\partial M}
\ar[d]\ar[r, hookrightarrow] & \B_O(M) \ar[d]\\
& \partial M \ar[r, hookrightarrow]& M
\end{tikzcd}\]
%
\begin{exmp}[Pin structures on $S^1$] %TODO FIX THIS
There are two problems to discuss here, since $\Pin_1^+$ and $\Pin_1^-$ are
different groups, namely
\begin{align*}
\Pin_1^+ &\cong \Z/2\Z \times \Z / 2\Z \\
\Pin_1^- &\cong \Z/4\Z
\end{align*}
We do the $\Pin_1^+$ case first. In this case, the Clifford algebra
$\Cliff_{1,0}(\R)$ is isomorphic to $\R \times \R$, where the isomorphism is
determined by $e_1 \mapsto (-1,1)$. There are two irreducible modules corresponding
to projection onto one of the factors, so the Pinor representations are the one
dimensional real representations $\P^+$ and $\P^-$, where the action of $e_1$ on
$\P^\pm$ is by $\pm 1$. Let $\P = \P^+ \oplus \P^-$. In addition, we have two
principal $\Pin^+_1$-bundles over $S^1$. One has $4$ components that are permuted
by the action of $\Pin_1^+$, and the other has two components, which
are both the double cover $z \mapsto z^2$.

Then for each $\Pin_1^+$-bundle,
we take the associated bundle. For $P_1$, we have that sections of
$P_1 \times_{\Pin_1^+} \P$ are equivalent to $\Pin_1^+$-equivariant maps
$P_1 \to P$. Since the action of $\Pin_1^+$ simply permutes the $4$ components
of $P_1$ and all the components are diffeomorphic to $S^1$, we can map
a single component to to $\P$, and this determines the map on the other $4$
components. Therefore, sections are equivalent data to maps $S^1 \to \P$, i.e.
$2\pi$-periodic maps $\R \to \P$. The Dirac operator $D$ for this bundle then
given by $D = e_1\partial_\theta$. For the other bundle, there are two components,
which are each the connected double cover of $S^1$. The element $-1$ acts in the same
way as it did for the connected double cover, and the element $e_1$ permutes
the two components. Then sections of the associated bundle are again equivalent
data to $2\pi$-antiperiodic maps $S^1 \to \P$. The Dirac operator is then given
by $i\partial_\theta - \frac{1}{2}$. \\

For $\Pin_1^-$, we also have two principal bundles, which we again denote
$P_1$ and $P_2$. The first is $P_1$, which has $4$ connected components that are cyclically
permuted by the action of $\Pin_1^-$, and $P_2$ is the connected $4$-fold
cover $z \mapsto z^4$. There is only a single irreducible module for $\Cliff_{0,1}(\R)$,
which is $\C$, which is then the single Pinor representation $\P$ where
$e_1$ acts by $i$. The associated bundle to $P_1$ is the trivial bundle with the
Dirac operator being given by $D = i\partial_\theta$. For the second case,
the sections of the associated bundle are determined by functions
$\psi : \R \to \C$ where $f(\theta + 2\pi) = if(\theta)$. We can then write these
as $\psi = e^{i\theta/4}f$ for a $2\pi$-periodic function $f$, so under this
trivialization of the associated bundle, the Dirac operator is given by
$i\partial_\theta - \frac{1}{4}$.
\end{exmp}
%
%
\end{document}
