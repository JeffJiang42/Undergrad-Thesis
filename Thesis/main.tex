\documentclass[reqno]{amsbook}
%
%-------Packages---------
%
\usepackage[h margin=1 in, v margin=1 in]{geometry}
\usepackage{amssymb,amsfonts}
\usepackage[all,arc]{xy}
\usepackage{tikz-cd}
\usepackage{enumerate}
\usepackage{mathrsfs}
\usepackage{amsthm}
\usepackage{mathpazo}
%\usepackage{eulervm}
\usepackage{yfonts}
\usepackage{enumitem}
\usepackage{mathrsfs}
\usepackage{fourier-orns}
\usepackage[all]{xy}
\usepackage{hyperref}
\usepackage{cite}
\usepackage{url}
\usepackage{mathtools}
\usepackage{graphicx}
\usepackage{pdfsync}
\usepackage{mathdots}
\usepackage{calligra}
%
\usepackage{tgpagella}
\usepackage[T1]{fontenc}
%
\usepackage{listings}
\usepackage{color}

\definecolor{dkgreen}{rgb}{0,0.6,0}
\definecolor{gray}{rgb}{0.5,0.5,0.5}
\definecolor{mauve}{rgb}{0.58,0,0.82}

\lstset{frame=tb,
  language=Matlab,
  aboveskip=3mm,
  belowskip=3mm,
  showstringspaces=false,
  columns=flexible,
  basicstyle={\small\ttfamily},
  numbers=none,
  numberstyle=\tiny\color{gray},
  keywordstyle=\color{blue},
  commentstyle=\color{dkgreen},
  stringstyle=\color{mauve},
  breaklines=true,
  breakatwhitespace=true,
  tabsize=3
  }
%
%--------Theorem Environments--------
%
\newtheorem{thm}{Theorem}[section]
\newtheorem*{thm*}{Theorem}
\newtheorem{cor}[thm]{Corollary}
\newtheorem{prop}[thm]{Proposition}
\newtheorem{lem}[thm]{Lemma}
\newtheorem*{lem*}{Lemma}
\newtheorem{conj}[thm]{Conjecture}
\newtheorem*{quest}{Question}
%
\theoremstyle{definition}
\newtheorem{defn}[thm]{Definition}
\newtheorem*{defn*}{Definition}
\newtheorem{defns}[thm]{Definitions}
\newtheorem{con}[thm]{Construction}
\newtheorem{exmp}[thm]{Example}
\newtheorem{exmps}[thm]{Examples}
\newtheorem{notn}[thm]{Notation}
\newtheorem{notns}[thm]{Notations}
\newtheorem{addm}[thm]{Addendum}
\newtheorem{exer}[thm]{Exercise}
\newtheorem{TODO}{\ib{TODO}}
%
\theoremstyle{remark}
\newtheorem{rem}[thm]{Remark}
\newtheorem*{claim}{Claim}
\newtheorem*{aside*}{Aside}
\newtheorem*{rem*}{Remark}
\newtheorem*{hint*}{Hint}
\newtheorem*{note}{Note}
\newtheorem{rems}[thm]{Remarks}
\newtheorem{warn}[thm]{Warning}
\newtheorem{sch}[thm]{Scholium}
%
%--------Macros--------
\renewcommand{\qedsymbol}{$\blacksquare$}
\renewcommand{\hom}{\mathsf{Hom}}
\renewcommand{\emptyset}{\varnothing}
\renewcommand{\O}{\mathscr{O}}
\newcommand{\R}{\mathbb{R}}
\newcommand{\ib}[1]{\textbf{\textit{#1}}}
\newcommand{\Q}{\mathbb{Q}}
\newcommand{\Z}{\mathbb{Z}}
\newcommand{\E}{\mathbb{E}}
\newcommand{\N}{\mathbb{N}}
\renewcommand{\H}{\mathbb{H}}
\newcommand{\C}{\mathbb{C}}
\newcommand{\A}{\mathbb{A}}
\newcommand{\F}{\mathbb{F}}
\newcommand{\M}{\mathcal{M}}
\renewcommand{\S}{\mathbb{S}}
\newcommand{\V}{\vec{v}}
\newcommand{\RP}{\mathbb{RP}}
\newcommand{\CP}{\mathbb{CP}}
\newcommand{\B}{\mathcal{B}}
\newcommand{\GL}{\mathsf{GL}}
\newcommand{\SL}{\mathsf{SL}}
\newcommand{\SP}{\mathsf{SP}}
\newcommand{\SO}{\mathsf{SO}}
\newcommand{\SU}{\mathsf{SU}}
\newcommand{\gl}{\mathfrak{gl}}
\newcommand{\g}{\mathfrak{g}}
\newcommand{\Spin}{\mathrm{Spin}}
\newcommand{\Pin}{\mathrm{Pin}}
\newcommand{\inv}{^{-1}}
\newcommand{\bra}[2]{ \left[ #1, #2 \right] }
\newcommand{\ind}{\lambda \in \Lambda}
\newcommand{\set}[1]{\left\lbrace #1 \right\rbrace}
\newcommand{\abs}[1]{\left\lvert#1\right\rvert}
\newcommand{\norm}[1]{\left\lVert#1\right\rVert}
\newcommand{\transv}{\mathrel{\text{\tpitchfork}}}
\newcommand{\enumbreak}{\ \\ \vspace{-\baselineskip}}
\let\oldexists\exists
\renewcommand\exists{\oldexists~}
\let\oldL\L
\renewcommand\L{\mathfrak{L}}
\makeatletter
\newcommand{\tpitchfork}{%
  \vbox{
    \baselineskip\z@skip
    \lineskip-.52ex
    \lineskiplimit\maxdimen
    \m@th
    \ialign{##\crcr\hidewidth\smash{$-$}\hidewidth\crcr$\pitchfork$\crcr}
  }%
}
\makeatother
\newcommand{\bd}{\partial}
\newcommand{\lang}{\begin{picture}(5,7)
\put(1.1,2.5){\rotatebox{45}{\line(1,0){6.0}}}
\put(1.1,2.5){\rotatebox{315}{\line(1,0){6.0}}}
\end{picture}}
\newcommand{\rang}{\begin{picture}(5,7)
\put(.1,2.5){\rotatebox{135}{\line(1,0){6.0}}}
\put(.1,2.5){\rotatebox{225}{\line(1,0){6.0}}}
\end{picture}}
\DeclareMathOperator{\id}{id}
\DeclareMathOperator{\im}{Im}
\DeclareMathOperator{\grap}{graph}
\DeclareMathOperator{\codim}{codim}
\DeclareMathOperator{\coker}{coker}
\DeclareMathOperator{\supp}{supp}
\DeclareMathOperator{\inter}{Int}
\DeclareMathOperator{\sign}{sign}
\DeclareMathOperator{\sgn}{sgn}
\DeclareMathOperator{\indx}{ind}
\DeclareMathOperator{\alt}{Alt}
\DeclareMathOperator{\Aut}{Aut}
\DeclareMathOperator{\trace}{trace}
\DeclareMathOperator{\ad}{ad}
\DeclareMathOperator{\End}{End}
\DeclareMathOperator{\Ad}{Ad}
\DeclareMathOperator{\Lie}{Lie}
\DeclareMathOperator{\Cliff}{Cliff}
\DeclareMathOperator{\spn}{span}
\DeclareMathOperator{\dv}{div}
\DeclareMathOperator{\grad}{grad}
\DeclareMathOperator{\sheafhom}{\mathscr{H}\text{\kern -3pt {\calligra\large om}}\,}
\newcommand*\myhrulefill{%
   \leavevmode\leaders\hrule depth-2pt height 2.4pt\hfill\kern0pt}
\newcommand\niceending[1]{%
  \begin{center}%
    \LARGE \myhrulefill \hspace{0.2cm} #1 \hspace{0.2cm} \myhrulefill%
  \end{center}}
\newcommand*\sectionend{\niceending{\decofourleft\decofourright}}
\newcommand*\subsectionend{\niceending{\decosix}}
\def\upint{\mathchoice%
    {\mkern13mu\overline{\vphantom{\intop}\mkern7mu}\mkern-20mu}%
    {\mkern7mu\overline{\vphantom{\intop}\mkern7mu}\mkern-14mu}%
    {\mkern7mu\overline{\vphantom{\intop}\mkern7mu}\mkern-14mu}%
    {\mkern7mu\overline{\vphantom{\intop}\mkern7mu}\mkern-14mu}%
  \int}
\def\lowint{\mkern3mu\underline{\vphantom{\intop}\mkern7mu}\mkern-10mu\int}
%
%--------Hypersetup--------
%
\hypersetup{
    colorlinks,
    citecolor=black,
    filecolor=black,
    linkcolor=blue,
    urlcolor=black
}
%
%--------Solution--------
%
\newenvironment{solution}
  {\begin{proof}[Solution]}
  {\end{proof}}
%
%--------Graphics--------
%
%\graphicspath{ {images/} }
%
\numberwithin{equation}{chapter}
%
\begin{document}
\frontmatter
%
\author{Jeffrey Jiang}
%
\title{Thesis}
%
\maketitle
%
\tableofcontents
%
\mainmatter
%

\chapter{Clifford Algebras and Spin Groups}
%
\subsectionend
$ $\\
\emph{No one fully understands spinors. Their algebra is formally understood,
but their geometrical significance is mysterious. In some sense they describe
the ``square root" of geometry and, just as understanding the concept of
$\sqrt{-1}$ took centuries, the same might be true of spinors.} \\
%
\ib{-- ~ Sir Michael Atiyah}
%
\subsectionend
%
\section{Clifford Algebras}
%
\begin{defn}
Let $V$ be a real finite dimensional vector space with a nondegenerate
symmetric bilinear form $b : V \times V \to \R$. Then the \ib{Clifford Algebra}
of $V$ is the data of a unital associative $\R$-algebra $\Cliff(V,b)$ and a linear
map $i : V \to \Cliff(V,b)$ satisfying the following universal property:
Given any linear map $\varphi : V \to A$ of $V$ into any unital associative
$\R$-algebra $A$ satisfying $\varphi(v)^2 = b(v,v)$, there exists a unique algebra
homomorphism $\tilde{\varphi}: \Cliff(V,b) \to A$ such that the following
diagram commutes:
%
\[\begin{tikzcd}
V \ar[dr, "\varphi"] \ar[d, "i"'] \\
\Cliff(V,b) \ar[r, "\tilde{\varphi}"'] & A
\end{tikzcd}\]
\end{defn}
%
This universal property uniquely characterizes the Clifford algebra $\Cliff(V,b)$
up to unique isomorphism.
%
\begin{thm}
The Clifford algebra is unique up to unique isomorphism, i.e. given
another unital associative algebra $C$ equipped with a linear map $j : V \to C$
satisfying the universal property, there exists a unique algebra isomorphism
$\varphi : \Cliff(V,b) \to C$.
\end{thm}
%
\begin{proof}
Given such an algebra $C$ with a map $j: V \to C$, the map $j$
satisfies the Clifford relation $j(v)^2 = b(v,v)$, so it induces a unique map
$\varphi : \Cliff(V,b) \to C$. The map $i : V \to \Cliff(V,b)$ also
satisfies the Clifford relation, so it induces a map $\psi : C \to \Cliff(V,b)$.
We claim that these maps are inverses. To do so, we show the compositions
$\varphi \circ \psi$ and $\psi \circ \varphi$ are identity. Using the
universal property of the Clifford algebra once more, the map $i$ induces
a unique map $\Cliff(V,b) \to \Cliff(V,b)$ such that
\[\begin{tikzcd}
V \ar[dr, "i"] \ar[d, "i"'] \\
\Cliff(V,b) \ar[r] & \Cliff(V,b)
\end{tikzcd}\]
commutes. The identity map makes this diagram commute, so by uniqueness, this is
the induced map. The map $\psi \circ \varphi$ also makes this diagram commute,
so it must be identity by uniqueness. An identical proof shows that
$\varphi \circ \psi$ is the identity map $\id_C$.
\end{proof}
%
Explicitly, $\Cliff(V,b)$ is realized as a quotient of the tensor algebra
\[
\mathcal{T}(V) = \bigoplus_{n \in \Z^{\geq 0}} V^{\otimes n}
\]
where we quotient by the left ideal generated by elements of the form
$v \otimes v - b(v,v)$, and the linear map
$i: V \to \mathcal{T}(V) / (v \otimes v - b(v,v))$
is given by the inclusion $V \hookrightarrow \mathcal{T}(V)$ followed by the
quotient map. We identify $V$ with its image $i(V)$ as a subspace of $\Cliff(V,b)$.
%
\begin{defn}
Define the bilinear form $b : \R^{p+q} \times \R^{p+q} \to \R$ by
\[
b(v,w) = \sum_{i = 1}^{p} v^iw^i - \sum_{i = p+1}^{p+q} v^iw^i
\]
where the $v^i$ and $w^i$ are the components of $v$ and $w$ with respect to
the standard basis on $\R^{p+q}$. We denote the vector space $\R^{p+q}$
equipped with this bilinear form as $\R^{p,q}$.
\end{defn}
%
Let $V$ be a vector space equipped with a nondegenerate bilinear form $b$ and
fix a basis for $V$. Then the bilinear form $b$ is given by a symmetric
invertible matrix $B$, which is conjugate to a diagonal matrix where all
the diagonal entries are either $1$ or $-1$. If after conjugation, $B$ has
$p$ positive entries and $q$ negative entries, we say $b$ has signature $(p,q)$.
Any bilinear form $b$ of signature $(p,q)$ admits a basis $\set{e_i}$ satisfying
%
\begin{enumerate}
  \item For $1 \leq i \leq p$, we have $b(e_i,e_i) = 1$
  \item For $p+1 \leq j \leq p+q$, we have $b(e_j,e_j) = -1$
  \item For $i \neq j$, we have $b(e_i,e_j) = 0$
\end{enumerate}
%
Any such basis then determines an isomorphism $(V,b) \to \R^{p,q}$.\, and we
call such a basis an \ib{orthogonal basis} for $(V,b)$.
In addition, we get a basis for $\Cliff(V,b)$ given by
\[
\set{e_{i_1}e_{i_2}\cdots e_{i_k}~:~ 0 \leq k \leq n, ~ 1 \leq i_j \leq n}
\]
where we interpret the product of $0$ basis vectors to be the unit element $1$.
This then implies the dimension of $\Cliff(V,b)$ as a vector space is
$2^{\dim V}$. This basis also determines an isomorphism
$\Cliff(V,b) \to \Cliff_{p,q}(\R)$, where $\Cliff_{p,q}(\R)$ is the Clifford
algebra for $\R^{p,q}$. Given $v,w \in V$, we write $v$ and $w$ in these
bases as $v^ie_i$ and $w^ie_i$ (using Einstein summation convention),
and derive the useful relation
%
\begin{align*}
vw + wv &= v^iw^je_ie_j + v^iw^je_je_i \\
&= v^iw^i(e_i^2 + e_i^2) \\
&= 2b(v,w)
\end{align*}
where we use the fact that the $e_i$ are orthgonal to deduce that $e_ie_j = -e_je_i$
if $i \neq j$.
%
\begin{defn}\footnote{It is common in the literature to refer to $\Z/2\Z$
graded vector spaces as super vector spaces. The ``super" prefix often refers
to a $\Z/2\Z$ grading on the relevant object.}
Let $V$ be a vector space. A $\Z /2\Z$ \ib{grading} on $V$ is a direct sum
decomposition $V = V^0 \oplus V^1$. Elements of $V^0$ are said to be \ib{even}
and elements of $V^1$ are said to be \ib{odd}. Elements of the even and odd
subspaces are said to be \ib{homogeneous}. Given an homogeneous element
$v \in V$, define its \ib{parity}, denoted $|v|$ by
\[
|v| = \begin{cases}
0 & v \in V^0 \\
1 & v \in V^1
\end{cases}
\]
Equivalently, it is the data of a linear map
$\varepsilon : V \to V$ such that $\varepsilon$ acts by identity on a subspace $V^0$
of $V$ and negative identity on a complementary subspace $V^1$, which gives the
direct sum decomposition of $V$ as the $\pm 1$ eigenspaces of $\varepsilon$.
The map $\varepsilon$ is called the \ib{grading operator}.

\end{defn}
%
\begin{defn}
A $\Z/2\Z$ \ib{graded algebra} $A$ over $\R$ (often called a superalgebra) is
an $\R$-algebra $A$ equipped with a grading $A = A^0 \oplus A^1$ such that
the multiplication respects the grading, i.e. given homogeneous elements
$a,b \in A$, Their product $ab$ is an element of $A^{|a| + |b|}$ where the
addition is done mod $2$.
\end{defn}
%
\begin{exmp}\enumbreak
\begin{enumerate}
  \item Any $\R$-algebra $A$ can be made into a graded algebra where we let
  $A^0 = A$ and $A^1 = 0$.
  \item The exterior algebra $\Lambda^\bullet V$ of a vector space $V$ is
  a $\Z/2\Z$ graded algebra (in fact, it has a $\Z$ grading as well), where the
  grading is
  $\Lambda^\bullet V = \Lambda^{\text{even}} V \oplus \Lambda^{\text{odd}} V$
  where $\Lambda^{\text{even}} V$ is the subspace spanned by even products of
  vectors and $\Lambda^{\text{odd}} V$ is the subspace spanned by odd products of vectors.
  \item Let $V = V^0 \oplus V^1$ be a $\Z/2\Z$ graded vector space. Then the
  algebra of endomorphisms $\End V$ can be endowed with the structure of a
  graded algebra, where the even subspace consists of linear maps preserving
  the decomposition $V^0 \oplus V^1$, and the odd subspace consists of
  linear maps $T$ reversing the decomposition, i.e. $T(V^i) = V^{i + 1 \mod 2}$.
  In a ordered basis where the first elements are all even and the last
  elements are all odd, the even elements of $\End V$ are block diagonal,
  while the odd elements are block off-diagonal.
\end{enumerate}
\end{exmp}
%
For the most part, the algebras we work with will be $\Z/2\Z$ graded, so
the term ``graded" may be used in lieu of ``$\Z/2\Z$ graded." In the case of
ambiguity, we will specify the grading. \\
The Clifford algebra $\Cliff(V,b)$ is naturally a $\Z/2\Z$ graded algebra. Fix
a basis $\set{e_i}$ for $V$. We then define the grading
\[
\Cliff(V,b) = \Cliff^0(V,b) \oplus \Cliff^1(V,b)
\]
Where $\Cliff^0(V,b)$ is the $\R$-span of all even products of basis vectors,
and $\Cliff^1(V,b)$ is the $\R$-span of all odd products of basis vectors.
Since the product of even elements is again even, the subspace
$\Cliff^0(V,b)$ forms a subalgebra, and is called the \ib{even
subalgebra}. There is an extremely nice relationship between a Clifford algebra
and its even subalgebra.
%
\begin{thm}
The even subalgebra $\Cliff_{p,q}^0(\R)$ is isomorphic to both $\Cliff_{q, p-1}$
and $\Cliff_{p,q-1}$ as ungraded algebras (as long as $p-1 > 0$ or $q-1 > 0$.)
\end{thm}
%
\begin{proof}
Fix a basis $\set{e_1^+, \ldots e_p^+, e_1^- \ldots e_q^-}$ for $\R^{p,q}$, where
$(e_i^+)^2 = 1$ and $(e_i^-)^2 = -1$. We then compute
%
\begin{align*}
(e_i^+e_j^+)^2 &= -(e_i^+)^2(e_j^+)^2 = -1 \\
(e_i^-e_j^-)^2 &= -(e_i^-)^2(e_j^-)^2 = -1 \\
(e_i^+e_j^-)^2 &= -(e_i^+)^2(e_i^-)^2 = 1 \\
(e_i^-e_j^+)^2 &= -(e_i^-)^2(e_j^+)^2 = 1
\end{align*}
%
Assume $q \neq 0$. Then a generating set for $\Cliff_{p,q}^0(\R)$ is
\[
\set{e_1^-e_j^+ ~:~ 1 \leq j \leq p} \cup \set{e_1^-e_k^- ~:~ 2 \leq k \leq q}
\]
All the elements in the first set square to $1$, and all the elements in the
second set square to $-1$. We then get an isomorphism
$\Cliff^0_{p,q}(\R) \to \Cliff_{p,q-1}$ via the mappings
\begin{align*}
e_1^-e_j^+ &\mapsto e_j^+ \\
e_1^-e_k^- &\mapsto e_{k-1}^-
\end{align*}
In the case where $p \neq 0$, we have that an equally good generating set for
$\Cliff_{p,q}^0(\R)$ is
\[
\set{e^+_1e_j^+ ~:~ 2 \leq j \leq p} \cup \set{e_1^+e_i^- ~:~ 1 \leq i \leq q}
\]
Where the elements in the first set square to $-1$ and the elements of the second
set square to $1$. Then the mappings
\begin{align*}
e_1^+e_j^+ &\mapsto e_{j-1}^- \\
e_1^+e_j^- &\mapsto e_j^+
\end{align*}
gives the isomorphism $\Cliff_{p,q}^0(\R) \to \Cliff_{q, p-1}$.
\end{proof}
%
Given two $\R$-algebras $A$ and $B$, we can form their tensor product
$A \otimes B$, which has $A \otimes B$ as the underlying vector space, and the
multiplication is defined as
\[
(a \otimes b)(c \otimes d) = ac \otimes bd
\]
In the case that both $A$ and $B$ are $\Z/2\Z$ graded algebras, we have an alternate
version of the tensor product, where the underlying vector space is also
$A \otimes B$. The grading on the tensor product is the decomposition
\[
A \otimes B = (A^0 \otimes B^0 \oplus A^1 \otimes A^1) \oplus (A^0 \otimes B^1
\oplus A^1 \otimes B^0)
\]
and the multiplication of homogeneous elements is given by
\[
(a \otimes b)(c \otimes d) = (-1)^{|b||c|}(ac \otimes bd)
\]
We see that in the multiplication, we are formally commuting the elements of
$b$ and $c$, and we want to introduce a sign whenever elements are moved past
each other. This is the \ib{Koszul sign rule}. Another concept that needs
a slight modification in the graded case is the opposite algebra. In the
normal case, given an $\R$-algebra $A$, the \ib{opposite algebra} is
the algebra $A^{\text{op}}$ with the same underlying vector space, but
the multiplication in $A^\text{op}$ is given by $a * b = ba$, where $ba$
is the multiplication in $A$. In doing so, we are formally commuting $a$
and $b$, so in the graded situation, we invoke the Koszul sign rule when
defining multiplication in the opposite algebra, and define the multiplication
of homogeneous elements in $A^{\text{op}}$ to be $a * b = (-1)^{|a||b|} ba$.\\

One remarkable fact is that Clifford algebras are closed under the graded
tensor product, i.e. the graded tensor products of two Clifford algebras is
another Clifford algebra. Likewise, the graded opposite algebra of a Clifford
algebra is again a Clifford algebra. For the remainder of this section,
we will let $\otimes$ denote the graded tensor product, and the superscript
$\text{op}$ will denote the graded opposite algebra.
%
\begin{thm}
$\Cliff_{p+t,q+s}(\R) \cong \Cliff_{p,q}(\R) \otimes \Cliff_{t,s}(\R)$
\end{thm}
%
\begin{proof}
To give a map $\varphi : \Cliff_{p+t,q+s}(\R) \to \Cliff_{p,q}(\R) \otimes
\Cliff_{t,s}(\R)$, it is sufficient to specify its action on $\R^{p+t,q+s}$ and
to check that the Clifford relations hold. Let
$\set{b_1^+,\ldots, b_{p+t}^+, b_1^-, \ldots b_{q+s}^-}$ denote the standard
orthogonal basis for $\R^{p+t,q+s}$ where $(b_i^+)^2 = 1$ and $(b_i^-)^2 = -1$.
We then define the bases $\set{e_i^\pm}$ and $\set{f_i^\pm}$ analogously for
$\R^{p,q}$ and $\R^{t,s}$ respectively. Then define $\varphi$ by
%
\begin{align*}
\varphi(b_i^+) &= \begin{cases}
e_i^+ \otimes 1 & 1 \leq i \leq p \\
1 \otimes f_i^+ & p+1 \leq i \leq p+t
\end{cases} \\
\varphi(b_i^-) &= \begin{cases}
e_i^- \otimes 1 & 1 \leq i \leq q \\
1 \otimes f_i^- & q+1 \leq i \leq q+s
\end{cases}
\end{align*}
%
This map is injective on generators, so if we show that this satisfies the
Clifford relations, then the map given by extending the map to all of
$\Cliff_{p+t,q+s}(\R)$ will be an isomorphism by a dimension count. Showing
the Clifford relations amounts to showing
%
\begin{enumerate}
  \item $\varphi(b_i^+)^2 = 1$
  \item $\varphi(b_i^-)^2 = -1$
  \item The images of any pair of distinct basis vectors anticommute.
\end{enumerate}
%
The first two are relations are clear from how we defined $\varphi$. To show
that the images of distinct basis vectors anticommute, there are serveral
cases to consider. Given $b_i^+$ and $b_j^+$ where $1 \leq i,j \leq p$, they
anticommute, because $e_i^+$ and $e_j^+$ anticommute. In the case where
$1 \leq i \leq p$ and $p+1 \leq j \leq p+t$, we compute
%
\begin{align*}
\varphi(b_i^+)\varphi(b_j^+) + \varphi(b_j^+)\varphi(b_i+) &=
(e_i^+ \otimes 1)(1 \otimes f_j^+) + (1\otimes f_j^+)(e_i^+ \otimes 1) \\
&= e_i^+ \otimes f_j^+ - e_i^+ \otimes f_j^+
\end{align*}
where we use the Koszul sign rule for the second term, noting that $f_j+$ and
$e_i^+$ are both odd elements. The proof that the images of the $b_i^-$ anti commute with
each other, as well as the proof that the images of the $b_i^+$ and $b_i^-$
anticommute are exactly the same.
%
\end{proof}
%
\begin{thm}
The graded opposite algebra $\Cliff_{p,q}^{\text{op}}(\R)$ is isomorphic to
$\Cliff_{q,p}(\R)$.
\end{thm}
%
\begin{proof}
Fix an orthgonal basis $\set{e_i^\pm}$ for $\R^{p,q}$, where
$(e_i^\pm)^2 = \pm 1$. We note that since the $e_i^\pm$ are odd elements,
they square to $\mp 1$ in the opposite algebra. Indeed, the mapping
$e_i^\pm \to e_i^\mp$ defines the isomorphism
$\Cliff_{p,q}^\text{op} \to \Cliff_{q,p}$.
\end{proof}
%
Because of these theorems, once we compute a few of the lower dimensional
Clifford algebras, we will have enough data to fully classify all Clifford
algebras over $\R$.
%
\begin{exmp}[\ib{Some low dimensional examples}]\enumbreak
\begin{enumerate}
  \item The Clifford algebra $\Cliff_{0,0}(\R)$ is isomorphic to $\R$.
  \item As ungraded algebras, the Clifford algebra $\Cliff_{0,1}(\R)$ is isomorphic
  to $\C$, where the isomorphism is given by $e_1 \mapsto i$.
  \item As ungraded algebras, $\Cliff_{0,2}(\R)$ is isomorphic to the quaternions
  $\H$, where the isomorphism is given by $e_1 \mapsto i$ and $e_2 \mapsto j$.
  \item As graded algebras, $\Cliff_{1,1}(\R)$ is isomorphic to $\End(\R^{1|1})$,
  where $\R^{1|1}$ denotes the $\Z/2\Z$ graded vector space $\R \oplus \R$.
  The isomorphism is given by
  \[
  e_1^+ \mapsto \begin{pmatrix}
  0 & 1 \\
  1 & 0
  \end{pmatrix} \qquad e_1^- \mapsto \begin{pmatrix}
  0 & 1 \\
  -1 & 0
  \end{pmatrix}
  \]
  \item As ungraded algebras $\Cliff_{1,0}(\R)$ is isomorphic to the product
  algebra $\R \times \R$, where $e_1 \mapsto (1,-1)$.
  \item As ungraded algebras, $\Cliff_{2,0}(\R)$ is isomorphic to the algebra $M_2\R$
  of $2 \times 2$ matrices with coefficients in $\R$. The isomorphism is given by
  \[
  e_1 \mapsto \begin{pmatrix}
  0 & 1 \\
  1 & 0
  \end{pmatrix} \qquad e_2 \mapsto \begin{pmatrix}
  1 & 0 \\
  0 & -1
  \end{pmatrix}
  \]
\end{enumerate}
\end{exmp}
%
To classify all Clifford algebras as ungraded algebras, it suffices to know
the following table: \\\\
\resizebox{.72\width}{!}{
\centering
\begin{tabular}{c |c|c|c|c|c|c|c|c|}
\hline
7 & $M_8\C$ & $M_8\H$ & $M_8\H \times M_8\H$ & $M_{16}\H$ & $M_{32}\C$ & $M_{64}\R$ & $M_{64}\R \times M_{64}\R$ & $M_{128}\R$ \\
\hline
6 & $M_4\H$ & $M_4\H \times M_4\H$ & $M_8\H$ & $M_{16}\C$ & $M_{32}\R$ & $M_{32}\R \times M_{32}\R$ & $M_{64}\R$ & $M_{64}\C$ \\
\hline
5 & $M_2\H \times M_2\H$ & $M_4\H$ & $M_8\C$ & $M_{16}\R$ & $M_{16}\R \times M_{16}\R$ & $M_{32}\R$ & $M_{32}\C$ & $M_{32}\H$ \\
\hline
4 & $M_2\H$ & $M_4\C$ & $M_8\R$ & $M_8\R \times M_8\R$ & $M_{16}\R$ & $M_{16}\C$ & $M_{16}\H$ & $M_{16}\H \times M_{16}\H$ \\
\hline
3 & $M_2\C$ & $M_4\R$ & $M_4\R \times M_4\R$ & $M_8\R$ & $M_8\C$ & $M_8\H$ & $M_8\H \times M_8\H$ & $M_{16}\H$ \\
\hline
2 & $M_2\R$ & $M_2\R \times M_2\R$ & $M_4\R$ & $M_4\C$ & $M_4\H$ & $M_4\H \times M_4\H$ & $M_8\H$ & $M_{16}\C$ \\
\hline
1 & $\R \times \R$ & $M_2\R$ & $M_2\C$ & $M_2\H$ & $M_2\H \times M_2\H$ & $M_4\H$ & $M_8\C$ & $M_{16}\R$ \\
\hline
0 &  $\R$ & $\C$ & $\H$ & $\H \times \H$ & $M_2\H$ & $M_4\C$ & $M_8\R$ & $M_8\R \times M_8\R$\\
\hline
\slashbox{p}{q} & 0 & 1 & 2 & 3 & 4 & 5 & 6 & 7
\end{tabular}
}
$ $\\\\\\
To read the table, the bottom left entry is $\Cliff_{0,0} \cong \R$, and moving to
the right increments the signature from $(p,q)$ to $(p,q+1)$, and moving up
increments the signature $(p,q)$ to $(p+1,q)$. Any other Clifford algebra
can be obtained from an algebra on this table by tensoring with $M_{16}\R$, since
incremeting the signature by $8$ (by adding to either $p$ or $q$) results in
tensoring with $M_{16}\R$.
%

%
%
\section{The Pin and Spin Groups}
%
The group of invertible elements in $\Cliff_{p,q}(\R)$, denoted
$\Cliff_{p,q}^\times(\R)$ contains a group $\Pin_{p,q}$, which is a double
cover of the group $O_{p,q}$ of matrices preserving the standard bilinear
form $\langle \cdot,\cdot \rangle$ on $\R^{p,q}$. Inside of $\Pin_{p,q}$,
there exists a subgroup $\Spin_{p,q} \subset \Pin_{p,q}$, which double covers
the group $SO_{p,q}$, which consists of the subgroup of $O_{p,q}$ where
all the elements have determinant equal to $1$.
%
\begin{defn}
The \ib{Pin group} $\Pin_{p,q}$ is the subgroup of $\Cliff_{p,q}^\times(\R)$
generated by the set
\[
\set{v \in \R^{p,q} ~:~ v^2 = \pm 1}
\]
The \ib{Spin group} $\Spin_{p,q}$ is the subgroup of $\Pin_{p,q}$ generated by even
products of basis vectors, i.e.
\[
\Spin_{p,q} = \Pin_{p,q} \cap \Cliff_{p,q}^0(\R)
\]
In indefinite signatures. $SO_{p,q}$ has multiple components. Some conventions
let $\Spin_{p,q}$ denote the double cover of the identity component $SO_{p,q}^+$,
which is the identity component of $\Spin_{p,q}$, denoted $\Spin^0_{p,q}$.
In the case that the bilinear form is definite, we let $\Pin_n^+ = \Pin_{n,0}$
and $\Pin_n^- = \Pin_{0,n}$. There is no such distinction for the Spin groups
in definite signatures.
\end{defn}
%
\begin{thm}
$\Spin_{p,q} \cong \Spin_{q,p}$.
\end{thm}
%
\begin{proof}
Recall that we have an isomorphism $\Cliff_{p,q}^{\text{op}} \to \Cliff_{q,p}$
where $e_i^\pm \mapsto e_i^\mp$. In addition, the even subalgebra $\Cliff_{q,p}^0$
is isomorphic to the (ungraded) opposite algebra of the even subalgebra
$\Cliff_{p,q}^0$. Therefore, the Spin group $\Spin_{q,p} \subset \Cliff_{q,p}$
is isomorphic to the opposite group $\Spin_{p,q}^\text{op}$. We then know
that every group is isomorphic to its opposite group via the map $g \mapsto g\inv$,
giving us the desired isomorphism.
\end{proof}
%
%
In particular, this implies that the Spin groups in definite signatures are
isomorphic, so we will henceforth denote them as $\Spin_n$.
%
To show that the Pin and Spin groups cover $O_{p,q}$ and $\Spin_{p,q}$, we make
a short digression. Given a vector $v \in \R^{p,q}$, we can define a reflection
map $R_v : \R^{p,q} \to \R^{p,q}$ given by $R_v(w) = w - 2\langle v,w \rangle v$,
which will reflect across the hyperplane $v^\perp$.
%
\begin{thm}[\ib{Cartan-Dieudonn\'e}]
Any orthogonal transformation $A \in O_{p,q}$ is the composition
of at most $p+q$ hyperplane reflections, where we interpret the identity map as the
composition of $0$ reflections.
\end{thm}
%
\begin{proof}
We prove this by induction on $n = p+q$. The case $n=1$ is trivial, since
$O_1 = \set{\pm 1}$. Then given $A \in O_{p,q}$, fix some nonzero $v \in \R^{p,q}$.
Then define $R : \R^{p,q} \to \R^{p,q}$ by
\[
R(w) = w - 2 \frac{\langle Av - v,w \rangle}{\langle Av -v, Av - v\rangle}(Av - v)
\]
Then $R$ is a reflection about the hyperplane orthogonal to $Av - v$,
and will interchange $v$ and $Av$. Therefore, $RA$ is an orthogonal transformation
fixing $v$. Since $RA$ is orthogonal, it will also fix the orthogonal complement
$v^\perp$, so it will restrict to an orthgonal transformation on $v^\perp$. The
orthogonal complement $v^\perp$ is $1$ dimension lower than $\R^{p,q}$, and restricting
the bilinear form to $v^\perp$, we know by the inductive hypotehsis that
$RA\vert_{v^\perp}$ can be written as at most $n-1$ hyperplane reflections in
$v^\perp$. Since $RA$ fixes $v$, we can extend all of these transformations to
a hyperplane reflection on all of $\R^{p,q}$, by taking the span of each hyperplane
with $v$, giving us that $RA$ is a composition of at most $n-1$ reflections. Finally,
composing $RA$ with $R$ gives us that $A$ can be written as a composition of at most
$n$ hyperplane reflections.
\end{proof}
%TODO Twisted Adjoint action
The Cartan-Dieudonn\'e theorem will be the central piece for showing that the
Pin and Spin groups are double covers of the orthogonal groups. The Clifford
algebra has an automorphism
$\alpha : \Cliff_{p,q}(\R) \to \Cliff_{p,q}(\R)$ that extends the mapping
$v \mapsto v-$. The action of $\alpha$ on a product $v_1\cdots v_k$ of vectors
$v_i \in \R^{p,q}$ is
\[
\alpha(v_1\cdots v_k) = (-1)^k v_1\cdots v_k
\]
the automorphism $\alpha$ gives another way to realize the grading on
$\Cliff_{p,q}$ -- it is the grading operator. The $+1$-eigenspace of $\alpha$
is exactly $\Cliff_{p,q}^0(\R)$, and the $-1$-eigenspace is the odd subspace
$\Cliff_{p,q}^0(\R)$.
%
\begin{thm}
There exist $2$-to-$1$ group homomorphisms $\Pin_{p,q} \to O_{p,q}$ and
$\Spin_{p,q} \to SO_{p,q}$, i.e. there exist short exact sequences of groups
\[\begin{tikzcd}
0 \ar[r] & \set{\pm 1} \ar[r] & \Pin_{p,q} \ar[r] & O_{p,q} \ar[r] & 0 \\
0 \ar[r] & \set{\pm 1} \ar[r] &\Spin_{p,q} \ar[r] & SO_{p.q} \ar[r] & 0
\end{tikzcd}\]
\end{thm}
%
\begin{proof}
We first consider the case of $\Pin_{p,q}$. To do this, we need to construct
a group action where $\Pin_{p,q}$ acts on $\R^{p,q}$ by orthogonal transformations.
\iffalse
There exists an involution $T : \Cliff_{p,q}(\R) \to \Cliff_{p,q}(\R)$, where given
the standard orthogonal basis $\set{e_1, \ldots, e_{p+1}}$, we define
\[
T(e_{i_1}\cdots e_{i_k}) = e_{i_k}\cdots e_{i_1}
\]
and extending linearly to the remainder of $\Cliff_{p,q}(\R)$. Given $a \in
\Cliff_{p,q}(\R)$, we denote $T(a)$ by $a^T$.
\fi
We note that for a vector
$v \in \R^{p,q}$ (identifying $\R^{p,q}$ as a subspace of $\Cliff_{p,q}(\R)$),
satisfying $\langle v,v \rangle = \pm 1$, we have that
%$v^T = v$ and
$v\inv = \pm v$. Then given $g \in \Pin_{p,q}$, and $v \in \R^{p,q}$, we claim
that the left action
\[
g \cdot v = \alpha(g)vg\inv
\]
defines the group action we desire. To show this, we must show that this indeed
maps $\R^{p,q}$ back into itself, and that the group elements act by orthogonal
transformations. We first compute this actions on vectors $v \in \R^{p,q}$ with
$\langle v,v \rangle = \pm 1$. In either case, since $v \in \R^{p,q}$, we have
that $\alpha(v) = -v$. First assume that $\langle v,v \rangle = 1$. Then given
$w \in \R^{p,q}$, we compute
%
\begin{align*}
-vwv\inv &= -vwv \\
&= (wv - 2\langle v,w \rangle)v \\
&= w - 2\langle v,w \rangle v
\end{align*}
%
Which is hyperplane reflection about the orthogonal complement of $v$. In the
case that $\langle v,v \rangle = -1$, we compute
%
\begin{align*}
-vwv\inv &= -vw(-v) \\
&= (2\langle -v,w\rangle + wv)(-v) \\
&= w - 2\langle -v,w \rangle(-v)
\end{align*}
which is hyperplane reflection about the orthogonal complement of $-v$,
which is the same as the orthogonal complement of $v$. Then given two
vectors $v_1,v_2 \in \R^{p,q}$, we have that $\alpha(v_1v_2) = v_1v_2$,
so given $w \in \R^{p,q}$,
\[
\alpha(v_1v_2)w(v_1v_2)\inv = (-v_1)(-v_2)wv_2\inv v_1\inv
\]
which is exactly the composition of hyperplane reflection about $v_2^\perp$,
with hyperplane reflection about $v_1^\perp$. Therefore,
$\Pin_{p,q}$ acts by orthogonal transformations, giving us a homomorphism
$\Pin_{p,q} \to O_{p,q}$. This map is surjective by the Cartan-Dieudonn\'e
theorem, and it can be verified that the kernel is $\set{\pm 1}$. In addition,
an even number of hyperplane reflections is orientation preserving, which gives
a surjection $\Spin_{p,q} \to SO_{p,q}$, by restricting the map
$\Pin_{p,q} \to O_{p,q}$. In addition, the kernel $\set{\pm 1}$ is contained in $\Spin_{p,q}$, so this map is also a double covering.
%
\end{proof}
%
We also have the complex Pin and Spin groups, denoted $\Pin_n\C$ and $\Spin_n\C$,
which double cover the complex orthogonal groups $O_n\C$ and $SO_n\C$
respectively.\\

Two simple examples of spin groups occur in dimensions $2$ and $3$.
Since $SO_2 \cong \mathbb{T}$, where
\[
\mathbb{T} = \set{z \in \C ~:~ |z| = 1}
\]
we have that $\Spin_2 \cong \mathbb{T}$, where the covering map is given
by $z \mapsto z^2$. In the case of $SO_3$, we consider the unit quaternions,
which form a Lie group isomorphic to the group $SU_2$. Then given $q \in SU_2$, we define
the map $\varphi_q : \R^3 \to \R^3$ where $\varphi_q(v) = qv\bar{q}$, where
$\bar{q}$ is the quaternionic conjugate of $q$, i.e.
\[
\overline{a + bi + cj +dk} = a -bi -cj -dk
\]
and we identify $v = v^ie_i$ is  with $v^1 i + v^2j + v^3k \in \H$.
The mapping $q \mapsto \varphi_q$ then gives a double cover $SU_2 \to SO_3$.
In particular, $SU_2$ is diffeomorphic to the sphere $S^3$, so the double
covering realizes $SO_3$ as the quotient of $S^3$ by the antipodal map,
giving us that $SO_3$ is diffeomorphic to $\RP^3$. \\

Many examples of low dimensional Spin groups arise from investigating the
relationship between a $4$ dimensional complex vector space $V$ and its second
exterior power $\Lambda^2V$. Fix a volume form $\mu \in \Lambda^4V^*$. This then
induces a symmetric, nondegenerate bilinear form $\langle \cdot,\cdot \rangle$
on $\Lambda^2V$ by
\[
\langle \alpha,\beta \rangle = \langle \alpha \wedge \beta, \mu \rangle
\]
where $\langle \alpha \wedge \beta, \,u \rangle$ denotes the natural pairing of
the vector space $\Lambda^4V$ with its dual $\Lambda^4V^*$. Fix a basis
$\set{e_i}$ for $V$ where $\mu(e_1 \wedge e_2 \wedge e_3 \wedge e_4) = 1$. In this basis,
we see that the group of transformations $\Aut(V, \mu)$ preserving $\mu$ is
isomorphic to the group $SL_4\C$. In addition, each map $T \in \Aut(V, \mu)$
induces a map $\Lambda^2 T : \Lambda^2V \to \Lambda^2V$, which is determined
by the formula $\Lambda^2 T(v \wedge w) = Tv \wedge Tw$. For any $T \in \Aut(V, \mu)$,
the induced map $\Lambda^2 T$ preserves the bilinear form on $\Lambda^2V$, so the
mapping $T \mapsto \Lambda^2V$ determines a group homomorphism
$\Aut(V, \mu) \to \Aut(\Lambda^2V, \langle \cdot,\cdot \rangle)$, where
$\Aut(\Lambda^2V,\langle\cdot,\cdot\rangle)$ denotes the group of linear automorphisms
preserving the bilinear form. The kernel of this map is $\set{\pm \id_V}$, and fixing
an orthogonal basis for $\langle\cdot,\cdot\rangle$ gives us that this map is
a double cover $SL_4\C \to SO_6\C$, so $SL_4\C$ is isomorphic to the
complex spin group $\Spin_4\C$ \\

If we then fix a hermitian inner product $h : V \times V \to \C$, we can
consider the automorphisms $\Aut(V, \mu, h)$ preserving $h$ and $\mu$, which
is isomorphic to the group $SU_4$. The bilinear form $h$ induces a hermitian
inner product (which we also denote $h$) on $\Lambda^2V$ defined by
\[
h(v_1 \wedge v_2, v_3 \wedge v_4) = \det \begin{pmatrix}
h(v_1,v_3) & h(v_1, v_4) \\
h(v_2,v_3) & h(v_2,v_4)
\end{pmatrix}
\]
Then if $T \in \Aut(V,\mu,h)$, $\Lambda^2T$ preserves the bilinear form
$\langle\cdot,\cdot\rangle$ induced by $\mu$ as well as the hermitian inner
product induced by $h$. The group that preserves both of these structures
is isomorphic to $SO_6\C \cap U_6$, which is $SO_6\R$. This gives us that
$SU_4 \cong \Spin_6$. \\

In general, one can play the game of fixing additional structure on $V$
(e.g. a real structure, quaternionic structure, symplectic form) and look
for the induced structure on $\Lambda^2V$. This then gives a map from
automorphisms of $V$ preserving this additional structure to automorphisms
of $\Lambda^2V$ preserving the induced structure. Playing this game then
determines several other low dimensional Spin groups.
%
\begin{align*}
&\Spin_5\C \cong Sp_4\C \qquad\Spin_4 \cong Sp(4)
\qquad\:\:\:\Spin_4\C \cong SL_2\C \times
SL_2\C \\
&\Spin_{1,3}^0 \cong SL_2\C \qquad\: \Spin_{1,2}^0 \cong SL_2\R
\qquad\Spin_{1,5}^0 \cong SL_2\H
\end{align*}
%
Where $Sp_4\C$ denotes the group of $4\times4$ matrices preserving a symplectic
form, $Sp(4) = Sp_4\C \cap U_4$, and $SL_2\H$ denotes the automorphisms of a
$2$ dimensional quaternionic vector space with determinant $1$ when regarded
as $4 \times 4$ complex matrices.
%
\begin{defn}
Given a Pin group $\Pin_{p,q}$, the \ib{Pinor representations} are representations
of $\Pin_{p,q}$ that arise from an irreducible Clifford module $M$ (i.e. the
action of $\Pin_{p,q}$ can be extended to an action of $\Cliff_{p,q}(\R)$). The
\ib{Spinor representations} are defined analogously for the group $\Spin_{p,q}$.
\end{defn}
%
From the classification of Clifford modules, we get a classification of
all the Pinor representations. From the relationship between a Clifford
algebra and its even subalgebra, we also get a complete classification of all
the Spinor representations.
%

%
\end{document}
