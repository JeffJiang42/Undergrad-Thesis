%
\section{Projective Spin Representations}
%
Using the isomorphisms of the Clifford algebras with matrix algebras
or products of matrix algebras along with the identification of the
even subalgebra with another Clifford algebra, we have a complete classification
of the irreducible modules over the even subalgebras $\Cliff^0_{n}(\R)$.
Restricting to the Spin group $\Spin_n \subset \Cliff^0_n(\R)$, this gives us
the Spin representations. These Spin representations define projective
representations of the group $SO_n$.
%
\iffalse
\begin{prop}
 Let $G$ be a group, and $V$ a finite dimensional irreducible representation of $G$.
 Then every element of the center acts by scalars, i.e. for $g \in Z(G)$, there
 exists a scalar $\lambda_g$ such that for all $v$ in $V$, we have
 \[
  g \cdot v = \lambda_g v
 \]
\end{prop}
%
\begin{proof}
 Let $g \in Z(G)$. Then for every $h \in G$ and $v \in V$, we have that
 \[
  h \cdot (g\cdot v) = g \cdot (h\cdot v)
 \]
 Therefore, the action of $g$ defines a $G$-equivariant map $V \to V$, which
 by Schur's Lemma must necessarily be a scalar multiple of the identity.
\end{proof}
\fi
%
Given a Spin representation $\S$, we have that the elements
$\pm 1$ act by scalars since they arise by restricting the action of a Clifford
algebra on an irreducible module. Then since $SO_n$ is the quotient of $\Spin_n$
by the subgroup $\set{\pm 1}$ of the center, we get a projective representation of
$SO_n$ on the projectivization $\P\S$ in the following way:
Given an element $A \in SO_n$, we have that $A$ lifts to two elements
$\set{\pm \tilde{A}} \subset \Spin_n$ which differ by $-1$. Since $-1$
acts by a scalar on $\S$, these elements determine the same action on
$\P\S$, giving us a well defined action of $SO_n$ on $\P\S$.

The Spin representation $\S$ is not realized canonically,
while the projective Spin representation $\P\S$ is canonical. Given
an isomorphism $\varphi : V \to W$, this induces a unique algebra isomorphism
$\End V \to \End W$ where $A \in \End V$ is mapped to
$\varphi \circ A \circ \varphi\inv$. However, the converse is not true.
%
\begin{prop}
 An algebra isomorphism $\varphi : \End V \to \End W$ induces a $Z(\F)$-torsor
 of maps $V \to W$, where $Z(\F)$ is the center of $\F$.
\end{prop}
%
To prove this, we need a lemma.
%
\begin{lem}
 The group of algebra automorphisms $\Aut(M_n\F)$ is isomorphic to the
 projective general linear group $PGL_n\F = GL_n\F / Z(GL_n\F)$.
\end{lem}
%
\begin{proof}
 Let $\alpha : M_n\F \to M_n\F$ be an algebra automorphism. We know that
 $M_n\F$ admits a single irreducible module $M$ up to isomorphism,
 which is $\F^n$ with the standard action. Then $\alpha$ defines another module
 $M^\alpha$, which is the same underlying vector space as $M$ with the algebra
 action given by $T \cdot v = \alpha(T)v$, where the right hand side is the action of
 $\alpha(T)$ on the module $M$. Since $\alpha$ is an algebra automorphism,
 $M_n\F$ acts transitively on $M^\alpha$, so it is also an irreducible module,
 which must be isomorphic to $M$. Therefore, there exists a module isomorphism
 $A : M \to M^\alpha$. Since $M$ and $M^\alpha$ are the same underlying vector
 space, $A$ is also a linear isomorphism $A : M \to M$, thought of as a vector space
 instead of a module. Then since $A$ is a module homomorphism, we have that
 for any $T \in M_n\F$,
 \[
  A \circ T = \alpha(T) \circ A \implies A \circ T \circ A\inv = \alpha(T)
 \]
 so $\alpha$ is given by conjugation by $A \in GL(M)$. In a basis, this
 tells us that the map $GL_n\F \to \Aut(M_n\F)$ given by conjugation is surjective,
 and the kernel of this map is the center of $GL_n\F$, so by the first isomorphism
 theorem, we have that $\Aut(M_n\F) \cong PGL_n\F$.
\end{proof}
%
\begin{proof}[Proof of Proposition]
 Fix bases for $V$ and $W$. These bases then induce isomorphisms
 $M_n\F \to V$ and $M_n\F \to W$. In these bases, the algebra
 isomorphism $\varphi$ is given by an automorphism $M_n\F \to M_n\F$. From the
 lemma, we know that this automorphism is determined by an element of
 $PGL_n\F = GL_n\F / Z(GL_n\F)$. The center $Z(\GL_n\F)$ consists of
 scalar matrices $\lambda I$ with $\lambda \in Z(\F)$, which is acted on
 freely and transitively by $Z(\F)$ by multiplication, giving it the structure
 of a $Z(\F)$-torsor.
\end{proof}
%
However, an isomorphism $\varphi : \End V \to \End W$ does induce a unique isomorphism
$\P V \to \P W$ of projective spaces. To see, this we make an identification
between $1$ dimensional subspaces of $V$ with maximal left ideals of $\End V$.
%
\begin{prop}
 There is a bijection
 \[
  \set{\text{Maximal left ideals of } \End V} \longleftrightarrow \P V
 \]
\end{prop}
%
\begin{proof}
 Given a line $L \in \P V$, the \ib{annihilator} of $L$ is the set
 \[
  \mathrm{Ann}(L) = \set{M \in \End V ~:~ M(L) = 0}
 \]
 In fact, $\mathrm{Ann}(L)$
 is a left ideal in $\End V$, since given $A \in \End V$ and $M \in \mathrm{Ann}(L)$,
 $L$ lies in the kernel of $A \circ M$. We claim that $\mathrm{Ann}(L)$ is maximal.
 Suppose $\mathrm{Ann}(L) \subset I$ is properly contained in a left ideal $I$.
 Fix an ordered basis for $V$ in which the first basis element is a
 nonzero element of $L$, then elements of $\mathrm{Ann}(L)$ are represented in this basis
 by matrices with all zeroes in the first column. Then since $\mathrm{Ann}(L)$ is
 properly contained in $I$, there exists some $M \in I$ such that
 $M \notin \mathrm{Ann}(L)$, which implies that as a matrix, the first column
 of $M$ is nonzero. Then pick a matrix $A \in \mathrm{Ann}(L)$ in which
 the nonzero columns complete the first column into a basis for $\R^n$. Then
 $A + M$ is an invertible element of $\End V$, so $I$ must be all of $\End V$.
 Therefore $\mathrm{Ann}(L)$ is maximal. To show that the mapping
 $L \mapsto \mathrm{Ann}(L)$ is a bijection, we produce an inverse. Let
 $I \subset \End V$ be a maximal ideal. Then we claim that the subspace
 \[
  \V(I) = \bigcap_{M \in I}\ker M
 \]
 is a $1$ dimensional subspace of $V$. We note that $\V(I)$ cannot be trivial,
 since this would imply that $I$ would contain an invertible element, contradicting
 that it is a proper ideal. We also see that it cannot be higher than $2$ dimensional,
 since otherwise, $I$ woud be contained in the annihilator of a proper nontrivial
 subspace of $\V(I)$, contradicting maximality of $I$. We then claim that
 these two mappings are inverses. We certainly have that
 $\mathrm{Ann}(\V(I)) \supset I$, so by maximality, this must be $I$. In addition
 it is clear that $\V(\mathrm{Ann}(L)) = L$ by the definition of $\V(I)$ and
 the annihilator. Therefore, these mappings are inverses.
\end{proof}
%
Therefore, given an algebra isomorphism $\varphi : \End V \to \End W$, this
induces a map $\P V \to \P W$ since the image of a maximal left ideal under an
isomorphism is again a maximal left ideal. In addition, the induced map is a bijection,
since it has an inverse given by the induced map of $\varphi\inv$. In
addition, the group of units $GL(V)$ acts on $\P V$ by right multiplication --
given a maximal ideal $I$ and $A \in GL(V)$, the ideal $I \cdot A$ is also
a maximal left ideal. \\

This gives us a canonical realization of the Spin representations. In the
case that the even subalgebra is isomorphic to a matrix algebra $M_n\F$,
the projective Spin representation is restriction of the action of
$\Cliff_n^0(\R)$ on maximal left ideals of $\Cliff_n^0(\F)$. In the case
that the even subalgebra is isomorphic to a product $M_n\F \times M_n\F$,
the irreducible modules identify the subalgebras $L$ and $R$
isomorphic to $M_n\F \times \set{0}$ and $\set{0} \times M_n\F$ by singling out
the maximal subalgebra that acts nontrivially. Looking at the maximal left ideals
of these subalgebras then identifies the two projective Spin representations.
In addition, since $-1$ acts trivially on left ideals, these projective
Spin representions descend to the quotient $\Spin_n / \set{\pm 1} \cong SO_n$,
giving us the projective representations of $SO_n$.
%
