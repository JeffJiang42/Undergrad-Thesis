\chapter{Dirac Operators on Manifolds}
%
\subsectionend $ $\\
%
\emph{Mathematics is a part of physics. Physics is an experimental science, a part of
natural science. Mathematics is the part of physics where experiments are cheap.} \\
%
\ib{-- ~ V.I. Arnold}
%
\subsectionend
%
\section{Connections}
%
We previously defined the Dirac operator $D$ on $\R^n$ and explored some
instances of $D$ in low dimensions. The constructions implicitly used the
Riemannian geometry on $\R^n$, along with its associated Spin structure.
To explore the Dirac operator in more generality on manifolds, we need to
develop some of the Riemannian geometry in the nonlinear world of manifolds.
%TODO Motivate connections, distributions, generalization of vector fields
%
\begin{defn}
Let $M$ be a smooth manifold. A \ib{distribution} on $X$ is a vector subbundle
$E \subset TX$.
\end{defn}
%
\begin{defn}
let $\pi : E \to M$ be a fiber bundle. The map $\pi$ determines a distinguished
subspace $\ker d\pi_e \subset T_eE$, called the \ib{vertical subspace}
which defines a distribution $V$ over $E$ called the \ib{vertical distribution}.
A \ib{connection} on $E$ is another distribution $H \subset TE$ such that
at every point $e \in E$, we have $V_e \oplus H_e = T_eE$. We also refer to the
distribution $H$ as a \ib{horizontal distribution}. In other words a connection
is a choice of splitting of the short exact sequence of vector bundles
\[\begin{tikzcd}
0 \ar[r] & V \ar[r] & TE \ar[r] & \pi^*TM \ar[r] & 0
\end{tikzcd}\]
If $E$ is a principal $G$-bundle, we ask for the horizontal distribution $H$
to be $G$-invariant, i.e. $H_{p \cdot g} = d(R_g)_p(H_p)$, where $R_g : E \to E$
denotes the right action of $g \in G$.
\end{defn}
%
Note that for a point $e \in E$, the projection map $d\pi_e$ restricted to
the horizontal subspace $H_e$ is a linear isomorphism $H_e \to T_{\pi(e)}X$.
%
\begin{defn}
Let $E \to M$ be a vector bundle. A \ib{$E$-valued $k$-form} is a section of
the vector bundle $\Lambda^kT^*M \otimes E \to M$. We denote the space of
$E$-valued $k$-forms by $\Omega_M^k(E)$. In the case that $E \to M$ is a
trivial bundle $E = M \times V$ for some vector space $V$, we abbreviate
this as a $V$-valued $k$ form, and the space of $V$-valued $k$-forms is denoted
$\Omega_M^k(V)$.
\end{defn}
%
We are mostly concerned with connections on principal bundles, which have a
richer structure than connections on a general fiber bundle. Let $\pi : P \to M$
be a principal $G$-bundle, and let $\mathfrak{g}$ denote the Lie algebra of $G$.
Let $\exp : \g \to G$ denote the exponential map. Fix $x \in M$, a point
$p \in \pi\inv(x)$, and an element $X \in \g$. This determines a curve
$\gamma_X : I \to \pi\inv(x)$ with $\gamma_X(0) = p$ given by
\[
\gamma_X(t) = p \cdot \exp(tX)
\]
Since the action of $G$ preserves the fiber $\pi\inv(x)$, this curve lies
entirely in the fiber $\pi\inv(x)$, so the tangent vector $\dt{\gamma}_X$
at $t = 0$ is an element of the vertical space. In addition, since the
action of $G$ is free, the tangent vector $\dt{\gamma}_X = 0$ if and only if
$X = 0$, so the linear mapping $X \mapsto \dt{\gamma}_X$ is injective, and
consequently, an isomorphism $\g \to V_p$. Doing this over all points of the
fiber, given $X \in \g$, we get a vector field $\tilde{X}$
on the fiber $\pi\inv(x)$ by differentiating the right action of $\exp(tX)$.

Defining these vector fields fiberwise  then gives an isomorphism $
\underline{\g} \to V$, where $\underline{\g}$ denotes the trivial bundle
$P \times \g$. Under this isomorphism, the exact sequence of vector bundles becomes
\[\begin{tikzcd}
0 \ar[r] & \underline{\g} \ar[r] & TP \ar[r] & \pi^*TM \ar[r] & 0
\end{tikzcd}\]
One important thing to note is how these vector fields transform under the
right action of $G$.
%
\begin{prop}
Let $X \in \g$, and let $\tilde{X}$ be the vector field on $P$ induced by
$X$. Let $g \in G$, and let $R_g : P \to P$ denote the diffeomorphism given
by right multiplication by $g$. Then
\[
(R_g)_*(\tilde{X}) = \widetilde{\Ad_{g\inv} X}
\]
Where $\Ad_{g\inv} : \g \to \g$ is the linear map obtained by differentiating the
conjugation action $h \mapsto g\inv h g$ at the identity element of $G$.
\end{prop}
%
\begin{proof}
Let $p \in P$. By definition, we have that
\[
\tilde{X}_p = \frac{d}{dt}\bigg\vert_{t=0} p \cdot \exp(tX)
\]
Therefore, we have that
\begin{align*}
[(R_g)_*(\tilde{X})]_{p\cdot g} &=
d(R_g)_p\left( \frac{d}{dt}\bigg\vert_{t = 0} p\cdot \exp(tX) \right) \\
&= \frac{d}{dt}\bigg\vert_{t = 0} p \cdot \exp(tX)g \\
&= \frac{d}{dt}\bigg\vert_{t = 0}p \cdot g (g\inv \exp(tX)g) \\[5pt]
&= \left( \widetilde{\Ad_g\inv X} \right)_{p \cdot g}
\end{align*}
\end{proof}
%
Over a point $p \in P$, we have a direct sum decomposition $T_pP = V_p \oplus H_p$,
giving us a projection map $T_pP \to V_p$. Using the identification of $V_p$
with $\g$, this then determines a linear map $ \omega_p : T_pP \to \g$, where
$\ker\omega_p = H_p$. Doing this over all fibers $\pi\inv(p)$, this determines
a $\g$-valued $1$-form $\omega \in \Omega^1_P(\g)$ called the
\ib{connection $1$-form}. From the connection $1$-form $\omega$, we can recover
$H$ as the kernel of $\omega$. In addition, it is clear from the definition that
given $X \in \g$, we have $\omega_p(\tilde{X}_p) = X$.
%
\begin{prop}
Let $\pi : P \to M$ be a principal $G$-bundle and $\omega \in \Omega^1_P(\g)$ a
connection $1$-form on $P$. Then for $g \in G$, $R_g^*\omega = \Ad_{g\inv}\omega$.
where $R_g^*$ denotes pullback by the right action of $g$ on $P$.
\end{prop}
%
\begin{proof}
Fix $p \in P$, and let $v \in T_pP$. Since $T_pP = V_p \oplus H_p$, we have
that $v = \tilde{X}_p + h$ for $X \in \g$ and $h \in H_p$. We then compute
\begin{align*}
(R_g^*\omega)_p(v) &= (R_g^*\omega)_p(\tilde{X}_p + h) \\
&= \omega_{p \cdot g}((R_g)_*(\tilde{X}_p ) + (R_g)_*h)
\end{align*}
Then since $(R_g)_*\tilde{X} = \widetilde{\Ad_{g\inv}X}$ and
$(R_g)_*h \in H_{p \cdot g}$, we get that
\begin{align*}
\omega_{p \cdot g}((R_g)_*(\tilde{X}_p ) + (R_g)_*h) &=
\omega_{p \cdot g}((\widetilde{\Ad_{g\inv}X})_{p\cdot g} + 0 \\
&= \Ad_{g\inv} X \\
&= \Ad_{g\inv}\omega(\tilde{X}_p + h)
\end{align*}
\end{proof}
%
We mentioned before that principal $G$-bundles can be thought of as
generalizations of covering spaces, since for discrete $G$, a principal $G$-bundle
is a covering space. One notion we want to generalize is the notion of path
lifting. Given a covering space $\pi : P \to M$, a point $x \in M$, a point
$p \in \pi\inv(x)$ in the fiber, and a curve $\gamma : I \to M$ with
$\gamma(0) = x$, we use the fact that $p$ lies in a unique sheet
$U_\alpha \subset \pi\inv(U)$ for some evenly covered neighborhood $U$ of $x$
to lift $\gamma$ to a path $\tilde{\gamma} : I \to P$ starting at $p$. This relies
on the fact that $\pi$ restricted to $U_\alpha$ is a diffeomorphism onto $U$.
If $G$ is not discrete, we have no hope of this working without additional data
since $P$ has a larger dimension than $M$. The connection $H \subset TP$ is
exactly the data we need for path lifting. Using the same setup as above,
the curve $\gamma$ determines a vector field $V$ along its image given by
its velocity $V_{\gamma(t)} = \dt{\gamma}(t)$. Then since
$d\pi_p\vert_{H_p} : H_p \to T_{\pi(p)}M$ is an isomorphism, there exists
a unique horizontal vector $h_p \in T_pP$ at each point $p \in \pi\inv(\gamma(t))$
satisfying $d\pi_p(h_p) = V_{\gamma(t)}$, which determines a vector field along
the pullback bundle $\gamma^*TP$, which can naturally be thought of as a subset
of $TP$. Integrating these vector fields gives a flow along the pullback bundle,
and following the flow line starting at our favorite point in the fiber
$\pi\inv(x)$ gives us a path $\tilde{\gamma} : I \to P$ such that
$\pi \circ \tilde{\gamma} = \gamma$, i.e. a lift of $\gamma$. One important
thing to note is how path lifting interacts with the group action.
%
\begin{prop}
Let $\pi : P \to M$ be a principal $G$-bundle equipped with a connection
$H \subset TP$, and let $\gamma : I \to M$ be a path in $M$ with $\gamma(0) = x$.
Then let $\tilde{\gamma} : I \to P$ be the lift of $\gamma$ starting at $p$.
Then the lift of $\gamma$ starting at $p \cdot g$ is $\gamma \circ R_g$,
where $R_g$ is the right action of $g \in G$.
\end{prop}
%
\begin{proof}
Since the action of $R_g$ preserves the fibers of $\pi$,
$\tilde{\gamma} \circ R_g$ satisfies $\pi \circ \gamma \circ R_g = \gamma$.
By uniqueness of integral curves of the horizontal vector field,
$\gamma \circ R_g$ must be the lift of $\gamma$ based at $p \cdot g$.
\end{proof}
%
Path lifting gives us a notion of \ib{parallel transport} -- an identification of
the fibers along a path on $M$.
%
\begin{defn}
Let $\pi : P \to M$ be a principal $G$-bundle with $H \subset TP$ a connection.
Let $\gamma : I \to M$ be a path starting at $x \in M$. The velocity vector
field of $\gamma$ induces a horizontal vector field $V$ on $P$. Let
$\gamma_p : I \to P$ denote the lift of $\gamma$ starting at $p \in \pi\inv(x)$.
Define the \ib{parallel transport} maps $\tau_t : P_x \to P_{\gamma(t)}$ by
$\tau_t(p) = \gamma_p(t)$
\end{defn}
%
We note that the maps $\tau_t$ are isomorphisms, since the inverse map is given
by parallel transport along the curve $\overline{\gamma}(t) = \gamma(1-t)$.
The parallel transport maps justify the naming of a connection -- a connection
gives the necessary to connect nearby fibers along a curve $\gamma$.
This notion of path lifting carries over to any associated bundle $P \times_G F$,
which is one way to see that the connection on $P$ induces a connection on
every associated bundle. As before, let $\gamma : I \to M$ be a path
with $\gamma(0) = x$, and let $[p,f] \in P \times_G F$ in the fiber over $x$,
i.e. $p \in \pi\inv(x)$. Let $\tilde{\gamma} : I \to P$ denote the lift
of $\gamma$ to $P$ starting at $p$. Then define
$\widehat{\gamma} : I \to P \times_G F$ by
$\widehat{\gamma}(t) = [\tilde{\gamma}(t), f]$, which is visibly a lift of
$\gamma$ to $P \times_G F$ starting at $[p,f]$. However, we should check that
this is independent of our choice of representative $[p,f]$. Let
$[p\cdot g, g\inv \cdot f]$ be another representative of the point $[p,f]$.
Then the lift of $\gamma$ to $P$ starting at $p \cdot g$ is the curve
$\tilde{\gamma} \cdot g$, so using this representative, we would define
$\widehat{\gamma}(t) = [\tilde{\gamma} \cdot g, g\inv \cdot f] =
[\tilde{\gamma}(t), f]$, so this definition does not rely on our choice of
representative. \\

Given a principal $G$-bundle $\pi : P \to M$ and a vector space $V$ with
a linear representation $\rho : G \to GL(V)$, let $E = P \times_G V$ be the
associated bundle. We proved earlier that there is a bijective correspondence
\[
\Gamma_M(E) \longleftrightarrow
\set{\alpha : P \to V ~:~ \alpha(p \cdot g) = \rho(g)\inv \cdot \alpha(p)}
\]
Thinking of functions $P \to V$ as $0$-forms on $P$ valued in $V$ and sections
of $E$ as $0$-forms on $M$ valued in $E$, we can write this correspondence more
suggestively as
\[
\Omega_M^0(E) \longleftrightarrow
\set{\alpha \in \Omega_P^0(V) ~:~ R_g^*\psi = \rho(g)\inv\alpha} \subset \Omega_P^0(V)
\]
Where $R_g^*$ denotes the pullback of $\alpha$ along right multiplication by
$g \in G$. This notation suggests that this correspondence should generalize
to arbitrary $k$-forms.
%
\begin{prop}
There is a bijective correspondence
\[
\Omega_M^k(E) \longleftrightarrow
\set{\alpha \in \Omega_P^k(V) ~:~ R_g^*\alpha = \rho(g)\inv\alpha ~~\forall g \in G,~
\iota_{\widehat{\xi}}(\alpha) = 0 ~~\forall \xi \in \g} \subset \Omega_P^k(V)
\]
Where $\iota_{\widehat{\xi}}(\alpha)$  denotes the interior multiplication
of $\alpha$ with the vector field $\widehat{\xi}$ on $P$ induced by $\xi$, i.e.
\[
\iota_{\widehat{\xi}}(\alpha)(V_1, \ldots, V_{k-1}) =
\alpha(\widehat{\xi}, V_1, \ldots, V_{k-1})
\]
For convenience, we denote this subset of $k$-forms on $P$ by
$\Omega^{k,G}_P(V)$, where the superscript $G$ reminds us that these $k$-forms
are $G$-invariant in the appropriate sense.
\end{prop}
%
\begin{proof}%TODO Finish this
We provide maps in both directions. Let $\pi_E : E \to M$ denote the bundle
projection, and let $\omega \in \omega_M^k(E)$. We want to define
$\widetilde{\omega} \in \Omega_P^{k,G}(V)$. Using $\pi$, we pullback
$\omega$ to obtain a $k$-form $\pi^*\omega \in \Omega_P^k(E)$, and define
$\tilde{\omega}$ as follows. Given a point $p \in P$ and $v \in T_pP$,
$(\pi^*\omega)_p(v) = [p, a]$, where $[p,a]$ is the unique representative in
it's class with $p$ in the first component. We then define
$\tilde{\omega}_p(v) = a$.
\end{proof}
%

%







\iffalse
Having defined a connection, we should check that such an object actually exists.
To show this, we first need a few lemmas.
%
\begin{defn}
Let
\[\begin{tikzcd}
0 \ar[r] & A \ar[r, "\varphi"] & B \ar[r, "\psi"] & C \ar[r] & 0
\end{tikzcd}\]
be a short exact sequence of vector spaces. A \ib{splitting} of this
exact sequence is a map $i : C \to B$ such that $\varphi \circ i = \id_C$.
If a splitting exists, we say that the short exact sequence splits.
\end{defn}
%
%
\begin{lem}[\ib{The splitting lemma}]
Let
\[\begin{tikzcd}
0 \ar[r] & A \ar[r, "\varphi"] & B \ar[r, "\psi"] & C \ar[r] & 0
\end{tikzcd}\]
be a short exact sequence of vector spaces. Then the following are equivalent:
\begin{enumerate}
  \item The short exact sequence splits.
  \item There is an isomorphism $\rho : B \to A \oplus C$ such that the diagram
  \[\begin{tikzcd}
  0 \ar[r] & A \ar[d, "\id_A"']\ar[r, "\varphi"] & B \ar[d, "\rho"] \ar[r, "\psi"]
  & C \ar[r] \ar[d, "\id_C"]& 0 \\
  0 \ar[r] & A \ar[r, hookrightarrow] & A \oplus C \ar[r] & C \ar[r] & 0
  \end{tikzcd}\]
  commutes, where the map $A \hookrightarrow A \oplus C$ is the inclusion
  $a \mapsto a + 0$, and the map $A \oplus C \to C$ is the mapping
  $a + c \mapsto c$.
  \item There exists a linear map $j : B \to A$ such that $j \circ \varphi = \id_A$.
\end{enumerate}
\end{lem}
%
\begin{proof}
We prove $(1) \implies (2) \implies (3) \implies (1)$. For $(1) \implies (2)$,
suppose we have a splitting $i : C \to B$. Since $i$ is injective and
$\psi \circ i = \id_C$, the image $i(C)$ must intersect $\ker\psi$ only
at $0$. Then since the sequence is exact, $\ker\psi = \varphi(A)$, so
$i(C) \cap \varphi(A) = 0$. Therefore, any element $b \in B$ can be written
uniquely as $a + c$ with $a \in \varphi(A)$ and $c \in i(C)$. Then define
$\rho : B \to A \oplus C$ by $\rho(b) = a + c$, where $a$ and $c$ are as above.\\

For $(2) \implies (3)$, define $j : B \to A$ by $\tilde{j} \circ \rho$, where
$\tilde{j} : A \oplus C \to A$ is the mapping $a + c \mapsto a$. \\

For $(3) \implies (1)$, Fix a basis $c_1, \ldots c_n$ for $c$. We know that
for each $c_i$ that $\psi\inv(c_i) = k_i + \varphi(A)$ for some $k_i \in B$.
Then define $i : C \to B$ by $c_i \mapsto k_i$.
\end{proof}
%
\begin{lem}
Every short exact sequence of vector spaces
\[\begin{tikzcd}
0 \ar[r] & A \ar[r, "\varphi"] & B \ar[r, "\psi"] & C \ar[r] & 0
\end{tikzcd}\]
splits.
\end{lem}
%
\begin{proof}
We want to define a map $i : C \to B$ such that $\psi \circ i = \id_C$. Let
$\set{c_1, \ldots, c_n}$ be a basis for $C$. Since the sequence is exact,
the map $\psi : B \to C$ is surjective, so for each $c_i$, we can pick an element
$b_i \in B$ such that $\psi(b_i) = c_i$. Then define $i(c_i) = b_i$ and
extend $i$ to all of $i$, giving the desired splitting.
\end{proof}
%
\begin{lem}
The space of splittings of a short exact sequence
\[\begin{tikzcd}
0 \ar[r] & A \ar[r, "\varphi"] & B \ar[r, "\psi"] & C \ar[r] & 0
\end{tikzcd}\]
is an affine space over $\hom(C,B)$,
i.e. the difference $i - j$ can be canonically identified with an element of
$\hom(C,A)$.
\end{lem}
%
\begin{proof} % Finish the proof that it is affine?
Let $i,j : C \to B$ be splittings of the exact sequence, and let
$c \in C$. Then since $i$ and $j$ are splittings,
\[
\psi((i-j)(c)) = \psi(i(c)) - \psi(j(c)) = c - c = 0
\]
Therefore, $(i-j)(c) \in \ker\psi$, so by exactness, there exists an $a \in A$
such that $\varphi(a) = (i-j)(c)$. By exactness, we know $\varphi$ is injective,
so this $a$ is unique. Therefore, the mapping $c \mapsto \varphi\inv((i-j)(c))$
defines the desired linear map $C \to A$.
\end{proof}
%
\begin{prop}
Given a fiber bundle $\pi : E \to M$ with model fiber $F$, a connection
$H \subset TE$ exists.
\end{prop}
%
\begin{proof}
It suffices to show that the short exact sequence of vector bundles
\[\begin{tikzcd}
0 \ar[r] & V \ar[r] & TE \ar[r] & \pi^*TM \ar[r] & 0
\end{tikzcd}\]
splits. Over any point $e \in E$, the space of splittings of
$0 \to V_e \to T_eE \to (\pi^*TM)_e$ is affine over the vector space
$\hom((\pi^*TM))_e \to V_e)$. Doing this fiberwise, we get a fiber bundle
$S \to E$ where the fiber over each $e \in E$ is the space of splittings
of the exact sequence, which can be seen by working in local trivializations.
Then let $\set{U_\alpha}$ be an open cover of $E$ equipped with local
trivializations $\varphi_\alpha$ of $S \to E$. The exact sequence
\[
0 \to V\vert_{\pi\inv(U_\alpha)} \to TE\vert_{\pi\inv(U_\alpha)} \to
\pi^*TM\vert_{\pi\inv(U_\alpha)}
\]
is modeled by the exact sequence $0 \to TF \to TU_\alpha \oplus TF \to TU_\alpha \to 0$,
which has a natural splitting given by the inclusion
$i_\alpha : TU_\alpha \hookrightarrow TU_\alpha \oplus TF$. The local trivialization
$\varphi_\alpha$ then allows us to transport this to a splitting over $U_\alpha$.
Then let $\set{\theta}$ be a partition of unity subordinate to $\set{U_\alpha}$.
Since the space of splittings is an affine space, it is closed under affine
combinations of splittings, so we can define a splitting
$i : \pi^*TM \to TE$ by $i = \sum_\alpha i_\alpha$.
\end{proof}
%TODO identification of vertical spaces with the Lie algebra.
%Discuss Lie algebra/exponential?
We are mostly concerned with connections on principal bundles, which have a
richer structure than connections on a general fiber bundle. Let $\pi : P \to M$
be a principal $G$-bundle, and let $\mathfrak{g}$ denote the Lie algebra of $G$.
Let $\exp : \g \to G$ denote the exponential map. Fix $x \in M$, a point
$p \in \pi\inv(x)$, and an element $X \in \g$. This determines a curve
$\gamma_X : I \to \pi\inv(x)$ with $\gamma_X(0) = p$ given by
\[
\gamma_X(t) = p \cdot \exp(tX)
\]
Since the action of $G$ preserves the fiber $\pi\inv(x)$, this curve lies
entirely in the fiber $\pi\inv(x)$, so the tangent vector $\dt{\gamma}_X$
at $t = 0$ is an element of the vertical space. In addition, since the
action of $G$ is free, the tangent vector $\dt{\gamma}_X = 0$ if and only if
$X = 0$, so the linear mapping $X \mapsto \dt{\gamma}_X$ is injective, and
consequently, an isomorphism $\g \to V_p$. Doing this on all points of $P$
then gives an isomorphism $\underline{\g} \to V$, where $\underline{\g}$ denotes
the trivial bundle $M \times \g$. Under this isomorphism, the exact
sequence of vector bundles becomes
\[\begin{tikzcd}
0 \ar[r] & \underline{\g} \ar[r] & TE \ar[r] & \pi^*TM \ar[r] & 0
\end{tikzcd}\]
%TODO Connection 1-form
By the splitting lemma, a connection is equivalent to a bundle homomorphism
$\theta : TE \to \underline{\g}$ such that the composition $\underline{\g} \to TE$
followed by $\theta$ is the identity map on $\underline{\g}$.
\fi
%
