%
\section{The Pin and Spin Groups}
%
The group of invertible elements in $\Cliff_{p,q}(\R)$, denoted
$\Cliff_{p,q}^\times(\R)$ contains a group $\Pin_{p,q}$, which a double
cover of the group $O_{p,q}$ of matrices preserving the standard bilinear
form $\langle \cdot,\cdot \rangle$ on $\R^{p,q}$. Inside of $\Pin_{p,q}$,
there exists a subgroup $\Spin_{p,q} \subset \Pin_{p,q}$, which double covers
the group $SO_{p,q}$, which consists of the subgroup of $O_{p,q}$ where
all the elements have determinant equal to $1$.
%
\begin{defn}
The \ib{Pin group} $\Pin_{p,q}$ is the subgroup of $\Cliff_{p,q}^\times(\R)$
generated by the set
\[
\set{v \in \R^{p,q} ~:~ v^2 = \pm 1}
\]
The \ib{Spin group} $\Spin_{p,q}$ is the subgroup of $\Pin_{p,q}$ generated by even
products of basis vectors, i.e. $\Spin_{p,q} = \Pin_{p,q} \cap \Cliff_{p,q}^0(\R)$. I
n the case that the bilinear form is definite, we let $\Pin_n^+ = \Pin_{n,0}$
and $\Pin_n^- = \Pin_{0,n}$. There is no such distinction for the Spin groups
in definite signatures.
\end{defn}
%
\begin{thm}
$\Spin_{0,n} \cong \Spin_{n,0}$.
\end{thm}
%
\begin{proof}
Recall that we have an isomorphism $\Cliff_{p,q}^{\text{op}} \to \Cliff_{q,p}$
where $e_i^\pm \mapsto e_i^\mp$. In addition, the even subalgebra $\Cliff_{q,p}^0$
is isomorphic to the (ungraded) opposite algebra of the even subalgebra
$\Cliff_{p,q}^0$. Therefore, the Spin group $\Spin_{q,p} \subset \Cliff_{q,p}$
is isomorphic to the opposite group $\Spin_{p,q}^\text{op}$. We then know
that every group is isomorphic to its opposite group via the map $g \mapsto g\inv$,
giving us the desired isomorphism.
\end{proof}
%
In particular, this implies that the Spin groups in definite signatures are
isomorphic, so we will henceforth denote them as $\Spin_n$.
%
To show that the Pin and Spin groups cover $O_{p,q}$ and $\Spin_{p,q}$, we make
a short digression. Given a vector $v \in \R^{p,q}$, we can define a reflection
map $R_v : \R^{p,q} \to \R^{p,q}$ given by $R_v(w) = w - 2\langle v,w \rangle v$,
which will reflect across the hyperplane $v^\perp$.
%
\begin{thm}[\ib{Cartan-Dieudonn\'e}]
Any orthogonal transformation $A \in O_{p,q}$ is the composition
of at most $p+q$ hyperplane reflections, where we interpret the identity map as the
composition of $0$ reflections.
\end{thm}
%
\begin{proof}
We prove this by induction on $n = p+q$. The case $n=1$ is trivial, since
$O_1 = \set{\pm 1}$. Then given $A \in O_{p,q}$, fix some nonzero $v \in \R^{p,q}$.
Then define $R : \R^{p,q} \to \R^{p,q}$ by
\[
R(w) = w - 2 \frac{\langle Av - v,w \rangle}{\langle Av -v, Av - v\rangle}(Av - v)
\]
Then $R$ is a reflection about the hyperplane orthogonal to $Av - v$,
and will interchange $v$ and $Av$. Therefore, $RA$ is an orthogonal transformation
fixing $v$. Since $RA$ is orthogonal, it will also fix the orthogonal complement
$v^\perp$, so it will restrict to an orthgonal transformation on $v^\perp$. The
orthogonal complement $v^\perp$ is $1$ dimension lower than $\R^{p,q}$, and restricting
the bilinear form to $v^\perp$, we know by the inductive hypotehsis that
$RA\vert_{v^\perp}$ can be written as at most $n-1$ hyperplane reflections in
$v^\perp$. Since $RA$ fixes $v$, we can extend all of these transformations to
a hyperplane reflection on all of $\R^{p,q}$, by taking the span of each hyperplane
with $v$, giving us that $RA$ is a composition of at most $n-1$ reflections. Finally,
composing $RA$ with $R$ gives us that $A$ can be written as a composition of at most
$n$ hyperplane reflections.
\end{proof}
%
The Cartan-Dieudonn\'e theorem will be the central piece for showing that the
Pin and Spin groups are double covers of the orthogonal groups.
%
\begin{thm}
There exist $2$-to-$1$ group homomorphisms $\Pin_{p,q} \to O_{p,q}$ and
$\Spin_{p,q} \to SO_{p,q}$, i.e. there exist short exact sequences of groups
\[\begin{tikzcd}
0 \ar[r] & \set{\pm 1} \ar[r] & \Pin_{p,q} \ar[r] & O_{p,q} \ar[r] & 0 \\
0 \ar[r] & \set{\pm 1} \ar[r] &\Spin_{p,q} \ar[r] & SO_{p.q} \ar[r] & 0
\end{tikzcd}\]
\end{thm}
%
\begin{proof}
We first consider the case of $\Pin_{p,q}$. To do this, we need to construct
a group action where $\Pin_{p,q}$ acts on $\R^{p,q}$ by orthogonal transformations.
\iffalse
There exists an involution $T : \Cliff_{p,q}(\R) \to \Cliff_{p,q}(\R)$, where given
the standard orthogonal basis $\set{e_1, \ldots, e_{p+1}}$, we define
\[
T(e_{i_1}\cdots e_{i_k}) = e_{i_k}\cdots e_{i_1}
\]
and extending linearly to the remainder of $\Cliff_{p,q}(\R)$. Given $a \in
\Cliff_{p,q}(\R)$, we denote $T(a)$ by $a^T$.
\fi
We note that for a vector
$v \in \R^{p,q}$, identifying $\R^{p,q}$ as a subspace of $\Cliff_{p,q}(\R)$,
satisfying $\langle v,v \rangle = \pm 1$, we have that
%$v^T = v$ and
$v\inv = \pm v$. Then given $g \in \Pin_{p,q}$, and $v \in \R^{p,q}$, we claim
that the left action
\[
g \cdot v = -gvg\inv
\]
defines the group action we desire. To show this, we must show that this indeed
maps $\R^{p,q}$ back into itself, and that the group elements act by orthogonal
trasnformations. It suffices to check this on the generating set of elements
$v$ with $\langle v,v \rangle = \pm 1$. First assume that $\langle v,v \rangle = 1$.
Then given $w \in \R^{p,q}$, we compute
%
\begin{align*}
-vwv\inv &= -vwv \\
&= (wv - 2\langle v,w \rangle)v \\
&= w - 2\langle v,w \rangle v
\end{align*}
%
Which is hyperplane reflection about the orthogonal complement of $v$. In the
case that $\langle v,v \rangle = -1$, we compute
%
\begin{align*}
-vwv\inv &= -vw(-v) \\
&= (2\langle -v,w\rangle + wv)(-v) \\
&= w - 2\langle -v,w \rangle(-v)
\end{align*}
which is hyperplane reflection about the orthogonal complement of $-v^\perp$,
which is the same as the orthogonal complement of $v^\perp$. Therefore,
$\Pin_{p,q}$ acts by orthogonal transformations, giving us a homomorphism
$\Pin_{p,q} \to O_{p,q}$. This map is surjective by the Cartan-Dieudonn\'e
theorem, and it can be verified that the kernel is $\set{\pm 1}$.
%
\end{proof}
%
We also have the complex Pin and Spin groups, denoted $\Pin_n\C$ and $\Spin_n\C$,
which double cover the complex orthogonal groups $O_n\C$ and $SO_n\C$
respectively.\\

Two simple examples of spin groups occur in dimensions $2$ and $3$.
Since $SO_2 \cong \mathbb{T}$, where
\[
\mathbb{T} = \set{z \in \C ~:~ |z| = 1}
\]
we have that $\Spin_2 \cong \mathbb{T}$, where the covering map is given
by $z \mapsto z^2$. In the case of $SO_3$, we consider the unit quaternions,
which form a Lie group isomorphic to the group $SU_2$. Then given $q \in SU_2$, we define
the map $\varphi_q : \R^3 \to \R^3$ where $\varphi_q(v) = qv\bar{q}$, where
$\bar{q}$ is the quaternionic conjugate of $q$ (e.g. $\overline{a + bi + cj +dk}
= a -bi -cj -dk$), and $v = v^ie_i$ is idenitified with $v^1 i + v^2j + v^3k \in \H$.
The mapping $q \mapsto \varphi_q$ then gives a double cover $SU_2 \to SO_3$.
In particular, $SU_2$ is diffeomorphic to the sphere $S^3$, so the double
covering realizes $SO_3$ as the quotient of $S^3$ by the antipodal map,
giving us that $SO_3$ is diffeomorphic to $\RP^3$ \\

Many examples of low dimensional Spin groups arise from investigating the
relationship between a $4$ dimensional complex vector space $V$ and its second
exterior power $\Lambda^2V$. Fix a volume form $\mu \in \Lambda^4V^*$. This then
induces a symmetric, nondegenerate bilinear form $\langle \cdot,\cdot \rangle$
on $\Lambda^2V$ by
\[
\langle \alpha,\beta \rangle = \langle \alpha \wedge \beta, \mu \rangle
\]
where $\langle \alpha \wedge \beta, \,u \rangle$ denotes the natural pairing of
the vector space $\Lambda^4V$ with its dual $\Lambda^4V^*$. Fix a basis
$\set{e_i}$ for $V$ where $\mu(e_1 \wedge e_2 \wedge e_3 \wedge e_4) = 1$. In this basis,
we see that the group of transformations $\Aut(V, \mu)$ perserving $\mu$ is
isomorphic to the group $SL_4\C$. In addition, each map $T \in \Aut(V, \mu)$
induces a map $\Lambda^2 T : \Lambda^2V \to \Lambda^2V$, which is determined
by the formula $\Lambda^2 T(v \wedge w) = Tv \wedge Tw$. For any $T \in \Aut(V, \mu)$,
the induced map $\Lambda^2 T$ preserves the bilinear form on $\Lambda^2V$, so the
mapping $T \mapsto \Lambda^2V$ determines a group homomorphism
$\Aut(V, \mu) \to \Aut(\Lambda^2V, \langle \cdot,\cdot \rangle)$, where
$\Aut(\Lambda^2V,\langle\cdot,\cdot\rangle)$ denotes the group of linear automorphisms
perserving the bilinear form. The kernel of this map is $\set{\pm \id_V}$, and fixing
an orthogonal basis for $\langle\cdot,\cdot\rangle$ gives us that this map is
a double cover $SL_4\C \to SO_6\C$, so $SL_4\C$ is isomorphic to the
complex spin group $\Spin_4\C$ \\

If we then fix a hermitian inner product $h : V \times V \to \C$, we can
consider the automorphisms $\Aut(V, \mu, h)$ preserving $h$ and $\mu$, which
is isomorphic to the group $SU_4$. The bilinear form $h$ induces a hermitian
inner product (which we also denote $h$) on $\Lambda^2V$ defined by
\[
h(v_1 \wedge v_2, v_3 \wedge v_4) = \det \begin{pmatrix}
h(v_1,v_3) & h(v_1, v_4) \\
h(v_2,v_3) & h(v_2,v_4)
\end{pmatrix}
\]
Then if $T \in \Aut(V,\mu,h)$, $\Lambda^2T$ preserves the bilinear form
$\langle\cdot,\cdot\rangle$ induced by $\mu$ as well as the hermitian inner
product induced by $h$. The group that preserves both of these structures
is isomorphic to $SO_6\C \cap U_6$, which is $SO_6\R$. This gives us that
$SU_4 \cong \Spin_6$. \\

In general, one can play the game of fixing additional structure on $V$
(e.g. a real stucture, quaternionic structure, symplectic form) and look
for the induced structure on $\Lambda^2V$. This then gives a map from
automorphisms of $V$ preserving this additional structure to automorphisms
of $\Lambda^2V$ preserving the induced structure. Playing this game then
determines several other low dimensional Spin groups.
%
\begin{align*}
&\Spin_5\C \cong Sp_4\C \qquad\Spin_4 \cong Sp(4) \qquad\:\:\:\Spin_4\C \cong SL_2\C \times
SL_2\C \\
&\Spin_{1,3} \cong SL_2\C \qquad\: \Spin_{1,2} \cong SL_2\R \qquad\Spin_{1,5} \cong SL_2\H
\end{align*}
%
Where $Sp_4\C$ denotes the group of $4\times4$ matrices preserving a symplectic
form, $Sp(4) = Sp_4\C \cap U_4$, and $SL_2\H$ denotes the automorphisms of a
$2$ dimensional quaternionic vector space with determinant $1$ when regarded
as $4 \times 4$ complex matrices.
%
\begin{defn}
Given a Pin group $\Pin_{p,q}$, the \ib{Pinor representations} are representations
of $\Pin_{p,q}$ that arise from an irreducible Clifford module $M$ (i.e. the
action of $\Pin_{p,q}$ can be extended to an action of $\Cliff_{p,q}(\R)$). The
\ib{Spinor representations} are defined analogously for the group $\Spin_{p,q}$.
\end{defn}
%
From the classification of Clifford modules, we get a classification of
all the Pinor representations. From the relationship between a Clifford
algebra and its even subalgebra, we also get a complete classification of all
the Spinor representations.
%
