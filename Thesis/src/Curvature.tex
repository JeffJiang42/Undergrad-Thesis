%
\section{Curvature}
%
An important quantity associated to a connection is it's \ib{curvature}, which
heuristically measures its deviation from our standard notions of a directional
derivative. We first discuss curvature on principal bundles, and will then
explore its relationship with associated bundles.
%
\begin{defn}
Let $G$ be a Lie group with Lie algebra $\g$, and $\pi : P \to M$ a principal $G$-bundle.
Let $\Theta \in \Omega_P^1(\g)$ be a connection on $P$. The \ib{curvature tensor} is
a $2$-form $\Omega \in \Omega_P^2(\g)$ defined by the equation
\[
\Omega = d\Theta + \frac{1}{2}[\Theta \wedge \Theta]
\]
where the action of $[\Theta \wedge \Theta]$ on $X$ is defined to be
\[
[\Theta \wedge \Theta](X) = [\Theta(X), \Theta(X)]
\]
\end{defn}
%
The first thing to note is that it descends to the base manifold $M$, i.e. it defines
an element of $\Omega_M^2(\g_P)$, where $\g_P$ is the \ib{adjoint bundle} of $P$,
which is the associated vector bundle $P \times_G \g$, where the action of $G$ on
$\g$ is the adjoint action. This is due to the following properties of $\Omega$.
%
\begin{prop} \enumbreak
\begin{enumerate}
  \item $R_g^*\Omega = \Ad_{g\inv}\Omega$
  \item $\iota_{\widehat{\xi}}(\Omega) = 0$ for all $\xi \in \g$.
\end{enumerate}
\end{prop}
%
\begin{proof}
For the first property, we compute
\begin{align*}
R_g^*\Omega &= R_g^*d\Theta + \frac{1}{2} R_g^*[\Theta \wedge \Theta] \\
&= d(R_g^*\Theta)+ \frac{1}{2}[R_g^*\Theta \wedge R_g^*\Theta]\\
&= \Ad_{g\inv}d\Theta + \frac{1}{2}[\Ad_{g\inv}\Theta \wedge \Ad_{g\inv}\Theta] \\
&= \Ad_g\inv\Omega
\end{align*}
where we use the fact that pullback commutes with exterior derivative and
the wedge product, and the fact that the adjoint action of $G$ on $\g$ commutes with
brackets. For the second property, let $\xi \in \g$. We then compute
\[
\iota_{\widehat{\xi}}(\Omega) = \iota_{\widehat{\xi}}(d\Theta)
+ \iota_{\widehat{\xi}}([\Theta \wedge \Theta])
\]
We compute the value of the terms separately. To compute $\iota_{\widehat{\xi}}(d\Theta)$,
we use Cartan's magic formula again to get
\[
\L_{\widehat{\xi}}\Theta = d(\iota_{\widehat{\xi}}(\Theta))
+ \iota_{\widehat{\xi}}(d\Theta) = 0 + \iota_{\widehat{\xi}}(\Theta)
\]
since $\iota_{\widehat{\xi}}(\Theta) = \xi$ for any $\xi \in \g$, and $d$ of a
constant form is $0$. Therefore,
$\L_{\widehat{\xi}}\Theta = \iota_{\widehat{\xi}}(d\Theta)$. We then note that
\begin{align*}
\L_{\widehat{\xi}}\Theta &= \frac{d}{dt}\bigg\vert_{t = 0} R_{\exp(t\xi)}^*\Theta \\
&= \frac{d}{dt}\bigg\vert_{t=0} \Ad_{\exp(t\xi)}\Theta \\
&= -[\xi, \Theta]
\end{align*}
where we use the fact that the derivative of $\Ad$ is the Lie bracket.
\end{proof}
%