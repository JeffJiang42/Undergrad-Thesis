%
\section{Curvature}
%
An important quantity associated to a connection is it's \ib{curvature}, which
heuristically measures its deviation from our standard notions of a directional
derivative. We first discuss curvature on principal bundles, and will then
explore its relationship with associated bundles.
%
\begin{defn}
Let $G$ be a Lie group with Lie algebra $\g$, and $\pi : P \to M$ a principal $G$-bundle.
Let $\Theta \in \Omega_P^1(\g)$ be a connection on $P$. The \ib{curvature tensor} is
a $2$-form $\Omega \in \Omega_P^2(\g)$ defined by the equation
\[
\Omega = d\Theta + \frac{1}{2}[\Theta \wedge \Theta]
\]
where the action of $[\Theta \wedge \Theta]$ on $X,Y$ is defined to be
\[
[\Theta \wedge \Theta](X,Y) = [\Theta(X), \Theta(Y)]
\]
\end{defn}
%
The first thing to note is that it descends to the base manifold $M$, i.e. it defines
an element of $\Omega_M^2(\g_P)$, where $\g_P$ is the \ib{adjoint bundle} of $P$,
which is the associated vector bundle $P \times_G \g$, where the action of $G$ on
$\g$ is the adjoint action. This is due to the following properties of $\Omega$.
%
\begin{prop} \enumbreak
\begin{enumerate}
  \item $R_g^*\Omega = \Ad_{g\inv}\Omega$
  \item $\iota_{\widehat{\xi}}(\Omega) = 0$ for all $\xi \in \g$.
\end{enumerate}
\end{prop}
%
\begin{proof}
For the first property, we compute
\begin{align*}
R_g^*\Omega &= R_g^*d\Theta + \frac{1}{2} R_g^*[\Theta \wedge \Theta] \\
&= d(R_g^*\Theta)+ \frac{1}{2}[R_g^*\Theta \wedge R_g^*\Theta]\\
&= \Ad_{g\inv}d\Theta + \frac{1}{2}[\Ad_{g\inv}\Theta \wedge \Ad_{g\inv}\Theta] \\
&= \Ad_g\inv\Omega
\end{align*}
where we use the fact that pullback commutes with exterior derivative and
the wedge product, and the fact that the adjoint action of $G$ on $\g$ commutes with
brackets. For the second property, let $\xi \in \g$. We then have that
\[
\iota_{\widehat{\xi}}(\Omega) = \iota_{\widehat{\xi}}(d\Theta)
+ \iota_{\widehat{\xi}}[\Theta \wedge \Theta]
\]
We compute the value of the terms separately. To compute $\iota_{\widehat{\xi}}(d\Theta)$,
we use Cartan's magic formula again to get
\[
\L_{\widehat{\xi}}\Theta = d(\iota_{\widehat{\xi}}(\Theta))
+ \iota_{\widehat{\xi}}(d\Theta) = 0 + \iota_{\widehat{\xi}}(\Theta)
\]
since $\iota_{\widehat{\xi}}(\Theta) = \xi$ for any $\xi \in \g$, and $d$ of a
constant form is $0$. Therefore,
$\L_{\widehat{\xi}}\Theta = \iota_{\widehat{\xi}}(d\Theta)$. We then note that
\begin{align*}
\L_{\widehat{\xi}}\Theta &= \frac{d}{dt}\bigg\vert_{t = 0} R_{\exp(t\xi)}^*\Theta \\
&= \frac{d}{dt}\bigg\vert_{t=0} \Ad_{\exp(t\xi)\inv}\Theta \\
&= -[\xi, \Theta]
\end{align*}
where the action of $[\xi, \Theta]$ on $X$ is $[\xi, \Theta](X) = [\xi,\Theta(X)]$,
and we use the fact that the derivative of $\Ad$ is the Lie bracket. We then compute
\[
\iota_{\widehat{\xi}}[\Theta \wedge \Theta] =
[\iota_{\widehat{\xi}}(\Theta) \wedge \Theta] = [\xi, \Theta]
\]
Therefore, $\iota_{\widehat{\xi}}(\Omega) = 0$, so the curvature form descends to a
form in $\Omega_M^2(\g_P)$.
\end{proof}
%TODO Biachi Identity?
%
Having defined the curvature form, we explore its relationship with associated
vector bundles of a principal bundle $\pi : P \to M$ equipped with a connection $\Theta$.
Let $W$ be a vector space with a homomorphism $\rho : G \to GL(W)$, and
$\dt{\rho} : \g \to \End W$ its associated Lie algebra map. Then let $E \to M$ be
the associated vector bundle $E = P \times_G W$. Recall that the connection $\Theta$ on
$P$ induces an exterior covariant derivative
%
\begin{align*}
d_\Theta : \Omega^k_M(E) &\to \Omega^{k+1}_M(E) \\
\psi &\mapsto d\psi + \dt{\rho}(\Theta) \wedge \psi
\end{align*}
%
which gives us a sequence of maps
\[\begin{tikzcd}
\Omega^0_M(E) \ar[r, "d_\Theta"] & \Omega^1_M(E) \ar[r, "d_\Theta"]
& \Omega^2_M(E) \ar[r, "d_\Theta"]  & \cdots
\end{tikzcd}\]
However, this does \emph{not} form a complex, i.e. $d_\Theta^2 \neq 0$. For
$\psi \in \Omega^0_M(E)$, we compute
%
\begin{align*}
d_\Theta^2\psi &= d_\Theta(d\psi + \dt{\rho}(\Theta)\psi) \\
&= d^2\psi + \dt{\rho}(\Theta) \wedge d\psi + d(\dt{\rho}(\Theta)\psi)
+ \dt{\rho}(\Theta) \wedge \dt{\rho}(\Theta)\psi \\
&= \dt{\rho}(\Theta) \wedge d\psi + d(\dt{\rho}(\Theta)\psi)
+ \dt{\rho}(\Theta) \wedge \dt{\rho}(\Theta)\psi
\end{align*}
%
We wan to relate this to the curvature form $\Omega$ on $P$. To do this, we use a
useful lemma.
%
\begin{lem}
\[
\dt{\rho}(\Theta) \wedge \dt{\rho}(\Theta) = \dt{\rho}([\Theta \wedge \Theta])
\]
\end{lem}
%
% TODO Figure out what is happening with the 1/2
\begin{proof}

\end{proof}
%
Therefore, the curvature form $\Omega$ on $P$ determines an $\End E$ valued
$2-$form on any associated vector bundle $E$.
%
\begin{defn}
Let $P$ be a principal bundle with connection, and $\Omega$ the curvature form.
$\rho : G \to GL(W)$ a representation, with corresponding Lie algebra representation
$\dt{\rho} : \g \to \End W$. $E = P \times_G W$. The \ib{curvature transformation} is a
$2$-form $R \in \Omega^2(\End E)$ defined by $R = \dt{\rho}(\Omega)$.
\end{defn}
%
Given tangent vectors $V,W$, we often denote the endomorphism $R(V,W)$ by $R_{V,W}$.
%