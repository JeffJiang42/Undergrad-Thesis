\section{Complex Clifford Algebras}
%
Much of the discussion regarding Clifford algebras can be reconstructed
using complex vector spaces. However, there is an important distinction
to be made. Over $\C$, the notion of signature no longer makes sense
when discussing bilinear forms. Given a bilinear form $B : V \times V \to \C$
and a vector $v \in V$ with $B(v,v) = 1$, we have that $B(iv,iv) = -1$. Therefore,
the complex Clifford algebras generated by $V$ with any nondegenerate bilinear
form are entirely determined by their dimension. In the case of $\C^n$ with
the standard bilinear form
\[
\langle v,w \rangle = \sum_i v^iw^i
\]
we denote the Clifford algebra by $\Cliff_n(\C)$. In the complex case, the
classification is much simpler, and is determined by the parity of the dimension. \\

In the even case of $\C^{2n}$, we first prove a lemma.
%
\begin{lem}
There exists a basis $\set{e_1,\ldots e_n, f_1, \ldots, f_n}$ for $\C^{2n}$
satisfying
\begin{enumerate}
  \item $\langle e_i, e_i \rangle = \langle f_i, f_i \rangle = 0$
  \item $\langle e_i, f_j \rangle = \delta_{ij}$
\end{enumerate}
where $\delta_{ij} = 0$ if $i \neq j$ and $\delta_{ij} = 1$ if $i = j$.
\end{lem}
%
\begin{proof}
Let $\set{a_i}$ denote the first $n$ standard basis vectors for $\C^{2n}$, and let
$\set{b_i}$ denote the last $n$ standard basis vectors. Then setting
$e_i = a_i + ib_i$ and $f_i = a_i - ib_i$, we get a basis
$\set{e_1, \ldots e_n, f_1, \ldots, f_n}$ for $\C^{2n}$. We then compute
%
\begin{align*}
\langle e_i, e_i \rangle &=\langle a_i + ib_i, a_i + ib_i \rangle \\
&= \langle a_i, a_i \rangle +2\langle a_i, ib_i \rangle + \langle ib_i, ib_i \rangle \\
&= 1 + 0 + -1 \\
&= 0 \\
\langle f_i, f_i \rangle &= \langle a_i - ib_i, a_i - ib_i \rangle \\
&= \langle a_i,a_i \rangle -2\rangle a_i, ib_i \rangle + \langle ib_i,ib_i\rangle \\
&= 0 \\
\langle e_i, f_j \rangle &= \langle a_i + ib_i, a_j - ib_j \rangle \\
&= \langle a_i,a_j \rangle - \langle a_i,ib_j \rangle + \langle a_j, ib_i \rangle
- \langle ib_i, ib_j \rangle \\
&= \delta_{ij} + 0 + 0 + \delta{ij} \\
&= 2\delta_{ij}
\end{align*}
So normalizing the $e_i$ and $f_i$ by dividing by $\sqrt{2}$ gives the desired basis.
%
\end{proof}
%
This basis gives a direct sum decomposition $\C^{2n} = W \oplus W'$ where
$W$ is the span of the $e_i$ and $W'$ is the span of the $f_i$. We then claim
that $\Cliff_{2n}(\C)$ is isomorphic to the endomorphism algebra
$\End(\Lambda^\bullet W)$. To give a map $\Cliff_{2n}(\C) \to
\End(\Lambda^\bullet W)$, we need to specify two maps
$\varphi : W \to \End(\Lambda^\bullet W)$ and
$\varphi' : W' \to \End(\Lambda^\bullet W)$ such that for all $w,p \in W$ and
$w',p' \in W'$ we have
%
\begin{enumerate}
  \item $\varphi(w) \circ \varphi(p) + \varphi(p) \circ \varphi(w) = 0$
  \item $\varphi'(w') \circ \varphi(p') + \varphi(p') \circ \varphi(w') = 0$
  \item $\varphi(w) \circ \varphi'(w') + \varphi'(w') \circ \varphi(w) =
  2\langle w,w' \rangle$
\end{enumerate}
where we use the fact that $\langle \cdot,\cdot \rangle$ vanishes on $W$ and
$W'$. For notational convenience, we will denote $\varphi(w)$ as $\varphi_w$, and
will do the same with $\varphi'$. Define
$\varphi_w : \Lambda^\bullet W \to \Lambda^\bullet W$ by
$\varphi_w(\alpha) = w \wedge \alpha$ and $\varphi'(w')$ by
\[
\varphi'_{w'}(v_1 \wedge \cdots \wedge v_k) = 2\sum_i(-1)^{i-1}\langle w',v_i\rangle
v_1 \wedge \cdots \widehat{v_i} \cdots \wedge v_k
\]
which will satisfy these relations, and these two maps define the desired
isomorphism $\Cliff_{2n}(\C) \to \End(\Lambda^\bullet W)$. \\

In the odd dimensional case of $\C^{2n + 1}$, we decompose $\C^{2n + 1}$ as
$\C^{2n + 1} = W \oplus W' \oplus U$, where $W$ and $W'$ are the same as in the
decomposition in the even case, and $U$ is the orthogonal complement to
$W \oplus W'$. We then define the maps from $W$ and $W'$ into
$\End(\Lambda^\bullet W)$, and then define a map $U \to \End(\Lambda^\bullet W)$
where a unit vector in $U$ acts by identity on the odd elements of $\Lambda^\bullet W$
and by negative identity on the even elements. This then defines a map
$\varphi : \Cliff_{2n + 1}(\C) \to \End(\Lambda^\bullet W)$. Repeating this with
$\End(\Lambda^\bullet W')$ gives another map
$\psi : \Cliff_{2n+1}(\C) \to \End(\Lambda^\bullet W')$. Then the product map
$\varphi \times \psi : \Cliff_{2n+1}(\C) \to \End(\Lambda^\bullet W) \times
\End(\Lambda^\bullet W')$ is the desired isomorphism. \\




\iffalse
We then claim that these two maps satisfy the relations specified above. From
the skew symmetry of the wedge product,
$\varphi(w) \circ \varphi(p) + \varphi(p) \circ \varphi(w) = 0$. For
the second relation, we compute
%
\begin{align*}
&\varphi'(w') \circ \varphi(p')  =
2\sum_i (-1)^{i-1} \langle p',v_i\rangle\varphi'_{w'}(v_1 \wedge \cdots
\widehat{v_i} \cdots \wedge v_k) \\
= 2 \sum_i (-1)^{i-1} \langle p',v_i \rangle &\left( \left(2\sum_{j < i}(-1)^{j-1}
\langle w',v_j\rangle v_1 \wedge \cdots \widehat{v_j} \cdots \widehat{v_i}
\cdots \wedge v_k \right) + \left( 2\sum_{j > i} (-1)^{j} \langle w',v_j\rangle
v_1 \wedge \cdots \widehat{v_i} \cdots \widehat{v_j} \cdots \wedge v_k\right)\right)
\end{align*}
%
computing $\varphi'(p') \circ \varphi'(w')$ then gives a similar result,
with the roles of $p'$ and $w'$ reversed, which gives the same result with a
reversed sign.
\fi
