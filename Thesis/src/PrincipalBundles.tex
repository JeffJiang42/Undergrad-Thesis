\section{Principal Bundles}
%
\begin{defn}
Let $M$ and $F$ be smooth manifolds. Then a \ib{fiber bundle} over $M$ with
model fiber $F$ is a the data of a smooth manifold $E$ with a smooth map
$\pi : E \to M$ such that for every point $p \in M$, there is a neighborhood
$U \subset M$ containing $p$ such that there exists an diffeomorphism
$\varphi : \pi\inv(U) \to U \times F$ such that the diagram
\[\begin{tikzcd}
\pi\inv(U) \ar[rr, "\varphi"] \ar[dr, "\pi"']&& U \times F \ar[dl, "p_U"] \\
& U
\end{tikzcd}\]
Where $P_U$ denotes projection onto the first factor. The map $\varphi$ is
called a \ib{local trivialization} of the fiber bundle $\pi : E \to M$.
\end{defn}
%
\begin{defn}
Let $G$ be a Lie group, and $M$ a smooth manifold. A \ib{Principal } $G$\ib{-bundle}
over $M$ is the data of
%
\begin{enumerate}
  \item A smooth manifold $P$ with a map $\pi : P \to M$.
  \item A smooth right $G$-action on $P$ that is free and transitive on the
  fibers of $\pi$.
  \item For every point $p \in M$, a neighborhood $U \subset M$ containing $p$ and
  a $G$-equivariant diffeomorphism $\varphi: \pi\inv(U) \to U \times G$ (where
  the right action on $U \times G$ is right multiplication on the second factor)
  such that we get the commutative diagram
  \[\begin{tikzcd}
  \pi\inv(U) \ar[dr, "\pi"']\ar[rr, "\varphi"]&& U \times G \ar[dl, "p_U"]\\
  & U
  \end{tikzcd}\]
  where $p_U$ denotes projection onto the first factor.
\end{enumerate}
\end{defn}
%
\begin{exmp}
Given a smooth manifold $M$ and a point $p \in M$, a basis of the
tangent space is a linear isomorphism $b : \R^n \to T_pM$. The group
$GL_n\R$ acts on the set of bases $\mathcal{B}_p$  on the right by
$b \cdot g = b \circ g$. Then the \ib{frame bundle} of $M$, denoted
$\mathcal{B}(M)$ is the disjoint union
\[
\mathcal{B}(M) = \coprod_{p \in M}\mathcal{B}_p
\]
where $\pi$ is the projection map $(p,b) \mapsto p$. Then $\mathcal{B}(M)$
is a principal $GL_n\R$ bundle over $M$.
\end{exmp}
%
\begin{exmp}
Given a smooth manifold $M$ with a Riemannian metric $g$, this induces an inner
product on each tangent space. Then the set of orthonormal bases of $T_pM$ is the
set of all linear isometries $T_pM \to \R^n$, where $\R^n$ is equipped with the
standard inner product. Then the disjoint union over all points $p$ of
orthonormal bases for the tangent spaces forms the \ib{orthonormal frame bundle}
$\mathcal{B}_O(M)$, which is a principal $O_n$ bundle.
\end{exmp}
%
\begin{defn}
Let $\pi : P \to m$ be a principal $G$-bundle over $M$, and let $F$ be a
smooth manifold with a smooth left $G$ action. Then the \ib{associated fiber
bundle}, denoted $P \times_G F$, is the set
\[
P \times_G F = P \times G / (p,g) \sim (p\cdot h, h\inv g)
\]
\end{defn}
%
\begin{defn}
Let $\pi : P \to M$ be a principal $G$-bundle, and $\rho : H \to G$ a group
homomorphism. The map $\rho$ gives $P$ a left $H$ action where
$h \cdot p = \rho(h) \cdot p$. A \ib{reduction of structure group} is the
data of a principal $H$ bundle $\varphi : Q \to M$ and an $H$ equivariant map
$F : Q \to P$.
\end{defn}
%
The map $F : Q \to P$ induces a map $\tilde{F} : Q \times_H G$, where we
map the equivalence class $[q,g]$ to $F(q) g$. This is well defined
on equivalences classes since
\[
(q \cdot h, \rho(h)\inv g) \mapsto F(q \cdot h)\rho(h)\inv g = F(q)\rho(h)\rho(h)\inv g
= F(q)g
\]
