
\chapter{Clifford Algebras and Spin Groups}
%
\subsectionend
$ $\\
\emph{No one fully understands spinors. Their algebra is formally understood,
but their geometrical significance is mysterious. In some sense they describe
the ``square root" of geometry and, just as understanding the concept of
$\sqrt{-1}$ took centuries, the same might be true of spinors.} \\
%
\ib{-- ~ Sir Michael Atiyah}
%
\subsectionend
%
\section{Clifford Algebras}
%
\begin{defn}
Let $V$ be a real finite dimensional vector space with a nondegenerate
symmetric bilinear form $b : V \times V \to \R$. Then the \ib{Clifford Algebra}
of $V$ is the data of a unital associative $\R$-algebra $\Cliff(V,b)$ and a linear
map $i : V \to \Cliff(V,b)$ satisfying the following universal property:
Given any linear map $\varphi : V \to A$ of $V$ into any unital associative
$\R$-algebra $A$ satisfying $\varphi(v)^2 = b(v,v)$, there exists a unique algebra
homomorphism $\tilde{\varphi}: \Cliff(V,b) \to A$ such that the following
diagram commutes:
%
\[\begin{tikzcd}
V \ar[dr, "\varphi"] \ar[d, "i"'] \\
\Cliff(V,b) \ar[r, "\tilde{\varphi}"'] & A
\end{tikzcd}\]
\end{defn}
%
This universal property uniquely characterizes the Clifford algebra $\Cliff(V,b)$
up to unique isomorphism.
%
\begin{thm}
The Clifford algebra is unique up to unique isomorphism, i.e. given
another unital associative algebra $C$ equipped with a linear map $j : V \to C$
satisfying the universal property, there exists a unique algebra isomorphism
$\varphi : \Cliff(V,b) \to C$.
\end{thm}
%
\begin{proof}
Given such an algebra $C$ with a linear map $j: V \to C$, the map $j$
satisfies the Clifford relation $j(v)^2 = b(v,v)$, so it induces a unique algebra
homomorphism $\varphi : \Cliff(V,b) \to C$. The linear map $i : V \to \Cliff(V,b)$ also
satisfies the Clifford relation, so it induces an algebra homomorphism
$\psi : C \to \Cliff(V,b)$.
We claim that these maps are inverses. To do so, we show the compositions
$\varphi \circ \psi$ and $\psi \circ \varphi$ are identity. Using the
universal property of the Clifford algebra once more, the map $i$ induces
a unique map $\Cliff(V,b) \to \Cliff(V,b)$ such that
\[\begin{tikzcd}
V \ar[dr, "i"] \ar[d, "i"'] \\
\Cliff(V,b) \ar[r] & \Cliff(V,b)
\end{tikzcd}\]
commutes. The identity map makes this diagram commute, so by uniqueness, this is
the induced map. The map $\psi \circ \varphi$ also makes this diagram commute,
so it must be identity by uniqueness. An identical proof shows that
$\varphi \circ \psi$ is the identity map $\id_C$.
\end{proof}
%
Explicitly, $\Cliff(V,b)$ is realized as a quotient of the tensor algebra
\[
\mathcal{T}(V) = \bigoplus_{n \in \Z^{\geq 0}} V^{\otimes n}
\]
where we quotient by the two-sided ideal generated by elements of the form
$v \otimes v - b(v,v)$, and the linear map
$i: V \to \mathcal{T}(V) / (v \otimes v - b(v,v))$
is given by the inclusion $V \hookrightarrow \mathcal{T}(V)$ followed by the
quotient map. We identify $V$ with its image $i(V)$ as a subspace of $\Cliff(V,b)$.
%
\begin{thm}
The Clifford algebra is functorial in the following way: Given vector spaces
$V$ and $V'$ equipped with nondegenerate symmetric bilinear forms $b$ and $b'$
respectively and a linear map $T: V \to V'$ such that $b(v,w) = b'(Tv,Tw)$
for all $v,w \in V$, there exists a unique algebra homomorphism
$T_* : \Cliff(V,b) \to \Cliff(V',b')$
such that
\[\begin{tikzcd}
V \ar[r, "T"] \ar[d, "i"']& V' \ar[d, "i'"]\\
\Cliff(V,b) \ar[r, "T_*"'] & \Cliff(V',b')
\end{tikzcd}\]
commutes, where $i$ and $i'$ are the inclusions of $V$ and $V'$ into their
respective Clifford algebras.
\end{thm}
%
\begin{proof}
We use the realization of $\Cliff(V,b)$ as the quotient
\[
\Cliff(V,b) = \mathcal{T}(V) / (v \otimes v - b(v,v))
\]
The map $T$ induces a map $\tilde{T} : \mathcal{T}(V) \to \mathcal{T}(V')$
where the action on homogeneous elements is given by
\[
\tilde{T}(v_1 \otimes \cdots \otimes v_k) = Tv_1 \otimes \cdots \otimes Tv_k
\]
The fact that $b(v,w) = b'(Tv, Tw)$ implies that $\tilde{T}$ maps the
ideal $(v \otimes v - b(v,v))$ to the ideal $(v' \otimes v' - b'(v',b'))$, so
the map $\tilde{T}$ descends to the quotients, giving a map
$T_* : \Cliff(V,b) \to \Cliff(V',b')$. The fact that the diagram commutes
follows from the fact that the maps $i$ and $i'$ are the inclusions of
$V$ and $V'$ into their respective tensor algebras, followed by the quotient
maps into $\Cliff(V,b)$ and $\Cliff(V', b')$ respectively.
\end{proof}
%
\begin{defn}
Define the bilinear form $b : \R^{p+q} \times \R^{p+q} \to \R$ by
\[
b(v,w) = \sum_{i = 1}^{p} v^iw^i - \sum_{i = p+1}^{p+q} v^iw^i
\]
where the $v^i$ and $w^i$ are the components of $v$ and $w$ with respect to
the standard basis on $\R^{p+q}$. We denote the vector space $\R^{p+q}$
equipped with this bilinear form as $\R^{p,q}$.
\end{defn}
%
Let $V$ be a vector space equipped with a nondegenerate bilinear form $b$ and
fix a basis for $V$. Then the bilinear form $b$ is given by a symmetric
invertible matrix $B$, which is conjugate to a diagonal matrix where all
the diagonal entries are either $1$ or $-1$. If after conjugation, $B$ has
$p$ positive entries and $q$ negative entries, we say $b$ has signature $(p,q)$.
Any bilinear form $b$ of signature $(p,q)$ admits a basis $\set{e_i}$ satisfying
%
\begin{enumerate}
\item For $1 \leq i \leq p$, we have $b(e_i,e_i) = 1$
\item For $p+1 \leq j \leq p+q$, we have $b(e_j,e_j) = -1$
\item For $i \neq j$, we have $b(e_i,e_j) = 0$
\end{enumerate}
%
Any such basis then determines an isomorphism $(V,b) \to \R^{p,q}$, and we
call such a basis an \ib{orthogonal basis} for $(V,b)$.
In addition, we get a basis for $\Cliff(V,b)$ given by
\[
\set{e_{i_1}e_{i_2}\cdots e_{i_k}~:~ 0 \leq k \leq n, ~ 1 \leq i_j \leq n}
\]
where we interpret the product of $0$ basis vectors to be the unit element $1$.
This then implies the dimension of $\Cliff(V,b)$ as a vector space is
$2^{\dim V}$. This basis also determines an isomorphism
$\Cliff(V,b) \to \Cliff_{p,q}(\R)$, where $\Cliff_{p,q}(\R)$ is the Clifford
algebra for $\R^{p,q}$. Given $v,w \in V$, we write $v$ and $w$ in these
bases as $v^ie_i$ and $w^ie_i$ (using Einstein summation convention),
and derive the useful relation
%
\begin{align*}
vw + wv & = v^iw^je_ie_j + v^iw^je_je_i \\
     & = v^iw^i(e_i^2 + e_i^2)       \\
     & = 2b(v,w)
\end{align*}
where we use the fact that the $e_i$ are orthogonal to deduce that
$e_ie_j = -e_je_i$ if $i \neq j$.
%
\begin{defn}\footnote{It is common in the literature to refer to $\Z/2\Z$
graded vector spaces as super vector spaces. The ``super" prefix often refers
to a $\Z/2\Z$ grading on the relevant object.}
Let $V$ be a vector space. A $\Z /2\Z$ \ib{grading} on $V$ is a direct sum
decomposition $V = V^0 \oplus V^1$. Elements of $V^0$ are said to be \ib{even}
and elements of $V^1$ are said to be \ib{odd}. Elements of the even and odd
subspaces are said to be \ib{homogeneous}. Given an homogeneous element
$v \in V$, define its \ib{parity}, denoted $|v|$ by
\[
|v| = \begin{cases}
0 & v \in V^0 \\
1 & v \in V^1
\end{cases}
\]
Equivalently, it is the data of a linear map
$\varepsilon : V \to V$ such that $\varepsilon$ acts by identity on a subspace $V^0$
of $V$ and negative identity on a complementary subspace $V^1$, which gives the
direct sum decomposition of $V$ as the $\pm 1$ eigenspaces of $\varepsilon$.
The map $\varepsilon$ is called the \ib{grading operator}.

\end{defn}
%
\begin{defn}
A $\Z/2\Z$ \ib{graded algebra} $A$ over $\R$ (often called a superalgebra) is
an $\R$-algebra $A$ equipped with a grading $A = A^0 \oplus A^1$ such that
the multiplication respects the grading, i.e. given homogeneous elements
$a,b \in A$, their product $ab$ is an element of $A^{|a| + |b|}$ where the
addition is done mod $2$.
\end{defn}
%
\begin{exmp}\enumbreak
\begin{enumerate}
\item Any $\R$-algebra $A$ can be made into a graded algebra where we let
    $A^0 = A$ and $A^1 = 0$.
\item The exterior algebra $\Lambda^\bullet V$ of a vector space $V$ is
    a $\Z/2\Z$ graded algebra (in fact, it has a $\Z$ grading as well), where the
    grading is
    $\Lambda^\bullet V = \Lambda^{\text{even}} V \oplus \Lambda^{\text{odd}} V$
    where $\Lambda^{\text{even}} V$ is the subspace spanned by even products of
    vectors and $\Lambda^{\text{odd}} V$ is the subspace spanned by odd products of vectors.
\item Let $V = V^0 \oplus V^1$ be a $\Z/2\Z$ graded vector space. Then the
    algebra of endomorphisms $\End V$ can be endowed with the structure of a
    graded algebra, where the even subspace consists of linear maps preserving
    the decomposition $V^0 \oplus V^1$, and the odd subspace consists of
    linear maps $T$ reversing the decomposition, i.e. $T(V^i) = V^{i + 1 \mod 2}$.
    In a ordered basis where the first elements are all even and the last
    elements are all odd, the even elements of $\End V$ are block diagonal,
    while the odd elements are block off-diagonal.
    \qedhere\end{enumerate}
\end{exmp}
%
For the most part, the algebras we work with will be $\Z/2\Z$ graded, so
the term ``graded" may be used in lieu of ``$\Z/2\Z$ graded." In the case of
ambiguity, we will specify the grading. \\
The Clifford algebra $\Cliff(V,b)$ is naturally a $\Z/2\Z$ graded algebra. Fix
a basis $\set{e_i}$ for $V$. We then define the grading
\[
\Cliff(V,b) = \Cliff^0(V,b) \oplus \Cliff^1(V,b)
\]
where $\Cliff^0(V,b)$ is the $\R$-span of all even products of basis vectors,
and $\Cliff^1(V,b)$ is the $\R$-span of all odd products of basis vectors.
Since the product of even elements is again even, the subspace
$\Cliff^0(V,b)$ forms a subalgebra, and is called the \ib{even
subalgebra}. \\

As $\Z/2\Z$-graded \emph{vector spaces}, $\Cliff(V,b)$ is \emph{naturally}
isomorphic to the exterior algebra $\Lambda^\bullet(V)$. To construct the isomorphism,
we first construct an algebra homomorphism $\Cliff(V,b) \to \End(\Lambda^\bullet(V))$.
where we map $v \in V$
to $\varepsilon(v) + 1/2 \iota(v)$, where
\begin{align*}
&\varepsilon(v)(\omega) = v \wedge \omega \\
&\iota(v)(v_1 \wedge\cdots\wedge v_k)
= \sum_{i = 1}^k (-1)^{k+1}
\langle v,v_i\rangle v_1 \wedge \cdots \widehat{v_i} \cdots \wedge v_k
\end{align*}
where $\widehat{v_i}$ indicates that the $i^{th}$ term in the wedge product is missing.
This map satisfies the Clifford relations, so it extends to a unique map
$\varphi : \Cliff(V,b) \to \End(\Lambda^\bullet(V))$. Then the mapping
$v \mapsto \varphi(v)(1)$ is the desired isomorphism
$\Cliff(V,b) \to \Lambda^\bullet(V)$. In particular, if $e_i$ is an orthogonal basis
for $V$,
\[
\varphi(e_{i_1}e_{i_2}\cdots e_{i_k}) = e_{i_1}\wedge e_{i_2}\wedge\cdots\wedge e_{i_k}
\]
Of course, $\Cliff(V,b)$ and
$\Lambda^\bullet(V)$ are noncanonically isomorphic, since they are the same dimension
and the even and odd subspaces are the same dimensions, but the key point here is that
the isomorphism is canonical. Since them map is not an algebra homomorphism, a useful
thing to note is how multiplication in the Clifford algebra looks like in the exterior
algebra under the isomorphism $\varphi$.
%
\begin{prop}
For $v \in V$ and $\eta \in \Cliff(V,b)$
\[
\varphi(v\eta) = v \wedge \eta + \iota(v)(\eta)
\]
\end{prop}
%
\begin{proof}
Fix an orthogonal basis $\set{e_i^\pm}$ where $(e_i^+)^2 = 1$,
$(e_i^-)^2 = -1$. It suffices to check that the identity holds when $v$ is a
basis vector and $\eta$ is a product of basis vectors. We compute
\begin{align*}
e_i^\pm e_{i_1}^\pm\cdots e_{i_k}^\pm &= \begin{cases}
\pm e_{i_2}\cdots e_{i_k} & e_{i_1}^\pm = e_i^\pm \\
e_1e_{i_1}\cdots e_{i_k} & e_i^\pm \neq e_{i_1}^\pm
\end{cases}
\end{align*}
In addition, we have that
\begin{align*}
e_i^\pm \wedge (e_{i_1}^\pm\wedge\cdots\wedge e_{i_k}^\pm)
+ \iota(e_i^\pm)(e_{i_1}^\pm\wedge\cdots\wedge e_{i_k}^\pm) = \begin{cases}
0 \pm e_{i_2}^\pm \wedge \cdots \wedge e_{i_k}^\pm & e_1^\pm = e_{i_1}^\pm \\
e_i^\pm \wedge e_{i_1}^\pm \wedge\cdots\wedge e_{i_k}^\pm, & e_i^\pm \neq e_{i_1}^\pm
\end{cases}
\end{align*}
Then using the fact that $\varphi$ sends products of orthogonal vectors to
wedges of orthogonal vectors, we see that the desired identity holds on a basis,
completing the proof.
\end{proof}
%
There is an extremely nice relationship between a Clifford algebra
and its even subalgebra.
%
\begin{thm}
The even subalgebra $\Cliff_{p,q}^0(\R)$ is isomorphic to both $\Cliff_{q, p-1}$
and $\Cliff_{p,q-1}$ as ungraded algebras (as long as $p-1 > 0$ or $q-1 > 0$.)
\end{thm}
%
\begin{proof}
Fix a basis $\set{e_1^+, \ldots e_p^+, e_1^- \ldots e_q^-}$ for $\R^{p,q}$, where
$(e_i^+)^2 = 1$ and $(e_i^-)^2 = -1$. We then compute
%
\begin{align*}
(e_i^+e_j^+)^2 & = -(e_i^+)^2(e_j^+)^2 = -1 \\
(e_i^-e_j^-)^2 & = -(e_i^-)^2(e_j^-)^2 = -1 \\
(e_i^+e_j^-)^2 & = -(e_i^+)^2(e_i^-)^2 = 1  \\
(e_i^-e_j^+)^2 & = -(e_i^-)^2(e_j^+)^2 = 1
\end{align*}
%
Assume $q \neq 0$. Then a generating set for $\Cliff_{p,q}^0(\R)$ is
\[
\set{e_1^-e_j^+ ~:~ 1 \leq j \leq p} \cup \set{e_1^-e_k^- ~:~ 2 \leq k \leq q}
\]
All the elements in the first set square to $1$, and all the elements in the
second set square to $-1$. We then get an isomorphism
$\Cliff^0_{p,q}(\R) \to \Cliff_{p,q-1}$ via the mappings
\begin{align*}
e_1^-e_j^+ & \mapsto e_j^+     \\
e_1^-e_k^- & \mapsto e_{k-1}^-
\end{align*}
In the case where $p \neq 0$, we have that an equally good generating set for
$\Cliff_{p,q}^0(\R)$ is
\[
\set{e^+_1e_j^+ ~:~ 2 \leq j \leq p} \cup \set{e_1^+e_i^- ~:~ 1 \leq i \leq q}
\]
Where the elements in the first set square to $-1$ and the elements of the second
set square to $1$. Then the mappings
\begin{align*}
e_1^+e_j^+ & \mapsto e_{j-1}^- \\
e_1^+e_j^- & \mapsto e_j^+
\end{align*}
gives the isomorphism $\Cliff_{p,q}^0(\R) \to \Cliff_{q, p-1}$.
\end{proof}
%
Given two $\R$-algebras $A$ and $B$, we can form their tensor product
$A \otimes B$, which has $A \otimes B$ as the underlying vector space, and the
multiplication is defined as
\[
(a \otimes b)(c \otimes d) = ac \otimes bd
\]
In the case that both $A$ and $B$ are $\Z/2\Z$ graded algebras, we have an alternate
version of the tensor product, where the underlying vector space is also
$A \otimes B$. The grading on the tensor product is the decomposition
\[
A \otimes B = (A^0 \otimes B^0 \oplus A^1 \otimes A^1) \oplus (A^0 \otimes B^1
\oplus A^1 \otimes B^0)
\]
and the multiplication of homogeneous elements is given by
\[
(a \otimes b)(c \otimes d) = (-1)^{|b||c|}(ac \otimes bd)
\]
We see that in the multiplication, we are formally commuting the elements of
$b$ and $c$, and we want to introduce a sign whenever elements are moved past
each other. This is the \ib{Koszul sign rule}. Another concept that needs
a slight modification in the graded case is the opposite algebra. In the
ungraded case, given an $\R$-algebra $A$, the \ib{opposite algebra} is
the algebra $A^{\text{op}}$ with the same underlying vector space, but
the multiplication in $A^\text{op}$ is given by $a * b = ba$, where $ba$
is the multiplication in $A$. In doing so, we are formally commuting $a$
and $b$, so in the graded situation, we invoke the Koszul sign rule when
defining multiplication in the opposite algebra, and define the multiplication
of homogeneous elements in $A^{\text{op}}$ to be $a * b = (-1)^{|a||b|} ba$.\\

One remarkable fact is that Clifford algebras are closed under the graded
tensor product, i.e. the graded tensor products of two Clifford algebras is
another Clifford algebra. Likewise, the graded opposite algebra of a Clifford
algebra is again a Clifford algebra. For the remainder of this section,
we will let $\otimes$ denote the graded tensor product, and the superscript
$\text{op}$ will denote the graded opposite algebra.
%
\begin{thm}
$\Cliff_{p+t,q+s}(\R) \cong \Cliff_{p,q}(\R) \otimes \Cliff_{t,s}(\R)$
\end{thm}
%
\begin{proof}
To give a map $\varphi : \Cliff_{p+t,q+s}(\R) \to \Cliff_{p,q}(\R) \otimes
\Cliff_{t,s}(\R)$, it is sufficient to specify its action on $\R^{p+t,q+s}$ and
to check that the Clifford relations hold. Let
$\set{b_1^+,\ldots, b_{p+t}^+, b_1^-, \ldots b_{q+s}^-}$ denote the standard
orthogonal basis for $\R^{p+t,q+s}$ where $(b_i^+)^2 = 1$ and $(b_i^-)^2 = -1$.
We then define the bases $\set{e_i^\pm}$ and $\set{f_i^\pm}$ analogously for
$\R^{p,q}$ and $\R^{t,s}$ respectively. Then define $\varphi$ by
%
\begin{align*}
\varphi(b_i^+)  & = \begin{cases}
e_i^+ \otimes 1 & 1 \leq i \leq p     \\
1 \otimes f_i^+ & p+1 \leq i \leq p+t
\end{cases} \\
\varphi(b_i^-)  & = \begin{cases}
e_i^- \otimes 1 & 1 \leq i \leq q     \\
1 \otimes f_i^- & q+1 \leq i \leq q+s
\end{cases}
\end{align*}
%
This map is injective on generators, so if we show that this satisfies the
Clifford relations, the map given by extending the map to all of
$\Cliff_{p+t,q+s}(\R)$ will be an isomorphism by a dimension count. Showing
the Clifford relations amounts to showing
%
\begin{enumerate}
\item $\varphi(b_i^+)^2 = 1$
\item $\varphi(b_i^-)^2 = -1$
\item The images of any pair of distinct basis vectors anticommute.
\end{enumerate}
%
The first two are relations are clear from how we defined $\varphi$. To show
that the images of distinct basis vectors anticommute, there are serveral
cases to consider. Given $b_i^+$ and $b_j^+$ where $1 \leq i,j \leq p$, they
anticommute, because $e_i^+$ and $e_j^+$ anticommute. In the case where
$1 \leq i \leq p$ and $p+1 \leq j \leq p+t$, we compute
%
\begin{align*}
\varphi(b_i^+)\varphi(b_j^+) + \varphi(b_j^+)\varphi(b_i+) & =
(e_i^+ \otimes 1)(1 \otimes f_j^+) + (1\otimes f_j^+)(e_i^+ \otimes 1) \\
                                                           & = e_i^+ \otimes f_j^+ - e_i^+ \otimes f_j^+
\end{align*}
where we use the Koszul sign rule for the second term, noting that $f_j+$ and
$e_i^+$ are both odd elements. The proof that the images of the $b_i^-$ anti commute with
each other, as well as the proof that the images of the $b_i^+$ and $b_i^-$
anticommute are exactly the same.
%
\end{proof}
%
\begin{thm}
The graded opposite algebra $\Cliff_{p,q}^{\text{op}}(\R)$ is isomorphic to
$\Cliff_{q,p}(\R)$.
\end{thm}
%
\begin{proof}
Fix an orthogonal basis $\set{e_i^\pm}$ for $\R^{p,q}$, where
$(e_i^\pm)^2 = \pm 1$. We note that since the $e_i^\pm$ are odd elements,
they square to $\mp 1$ in the opposite algebra. Indeed, the mapping
$e_i^\pm \to e_i^\mp$ defines the isomorphism
$\Cliff_{p,q}^\text{op} \to \Cliff_{q,p}$.
\end{proof}
%
Because of these theorems, once we compute a few of the lower dimensional
Clifford algebras, we will have enough data to fully classify all Clifford
algebras over $\R$.
%
\begin{exmp}[\ib{Some low dimensional examples}]\enumbreak
\begin{enumerate}
\item The Clifford algebra $\Cliff_{0,0}(\R)$ is isomorphic to $\R$.
\item As ungraded algebras, the Clifford algebra $\Cliff_{0,1}(\R)$ is isomorphic
      to $\C$, where the isomorphism is given by $e_1 \mapsto i$.
\item As ungraded algebras, $\Cliff_{0,2}(\R)$ is isomorphic to the quaternions
      $\H$, where the isomorphism is given by $e_1 \mapsto i$ and $e_2 \mapsto j$.
\item As graded algebras, $\Cliff_{1,1}(\R)$ is isomorphic to $\End(\R^{1|1})$,
      where $\R^{1|1}$ denotes the $\Z/2\Z$ graded vector space $\R \oplus \R$.
      The isomorphism is given by
      \[
       e_1^+ \mapsto \begin{pmatrix}
       0 & 1 \\
       1 & 0
       \end{pmatrix} \qquad e_1^- \mapsto \begin{pmatrix}
       0 & 1 \\
       -1 & 0
       \end{pmatrix}
      \]
\item As ungraded algebras $\Cliff_{1,0}(\R)$ is isomorphic to the product
      algebra $\R \times \R$, where $e_1 \mapsto (1,-1)$.
\item As ungraded algebras, $\Cliff_{2,0}(\R)$ is isomorphic to the algebra $M_2\R$
      of $2 \times 2$ matrices with coefficients in $\R$. The isomorphism is given by
      \[
       e_1 \mapsto \begin{pmatrix}
       0 & 1 \\
       1 & 0
       \end{pmatrix} \qquad e_2 \mapsto \begin{pmatrix}
       1 & 0 \\
       0 & -1
       \end{pmatrix}
      \]
      \qedhere\end{enumerate}
\end{exmp}
%
To classify all Clifford algebras as ungraded algebras, it suffices to know
the following table, which is derived by identifying a few low dimensional
Clifford algebras by hand and using the fact that the graded tensor product of
Clifford algebras is another Clifford algebra. \\\\
\resizebox{.72\width}{!}{
 \centering
 \begin{tabular}{c |c|c|c|c|c|c|c|c|}
  \hline
  7               & $M_8\C$              & $M_8\H$              & $M_8\H \times M_8\H$ & $M_{16}\H$           & $M_{32}\C$                 & $M_{64}\R$                 & $M_{64}\R \times M_{64}\R$ & $M_{128}\R$                \\
  \hline
  6               & $M_4\H$              & $M_4\H \times M_4\H$ & $M_8\H$              & $M_{16}\C$           & $M_{32}\R$                 & $M_{32}\R \times M_{32}\R$ & $M_{64}\R$                 & $M_{64}\C$                 \\
  \hline
  5               & $M_2\H \times M_2\H$ & $M_4\H$              & $M_8\C$              & $M_{16}\R$           & $M_{16}\R \times M_{16}\R$ & $M_{32}\R$                 & $M_{32}\C$                 & $M_{32}\H$                 \\
  \hline
  4               & $M_2\H$              & $M_4\C$              & $M_8\R$              & $M_8\R \times M_8\R$ & $M_{16}\R$                 & $M_{16}\C$                 & $M_{16}\H$                 & $M_{16}\H \times M_{16}\H$ \\
  \hline
  3               & $M_2\C$              & $M_4\R$              & $M_4\R \times M_4\R$ & $M_8\R$              & $M_8\C$                    & $M_8\H$                    & $M_8\H \times M_8\H$       & $M_{16}\H$                 \\
  \hline
  2               & $M_2\R$              & $M_2\R \times M_2\R$ & $M_4\R$              & $M_4\C$              & $M_4\H$                    & $M_4\H \times M_4\H$       & $M_8\H$                    & $M_{16}\C$                 \\
  \hline
  1               & $\R \times \R$       & $M_2\R$              & $M_2\C$              & $M_2\H$              & $M_2\H \times M_2\H$       & $M_4\H$                    & $M_8\C$                    & $M_{16}\R$                 \\
  \hline
  0               & $\R$                 & $\C$                 & $\H$                 & $\H \times \H$       & $M_2\H$                    & $M_4\C$                    & $M_8\R$                    & $M_8\R \times M_8\R$       \\
  \hline
  \slashbox{p}{q} & 0                    & 1                    & 2                    & 3                    & 4                          & 5                          & 6                          & 7
 \end{tabular}
}
$ $\\\\\\
To read the table, the bottom left entry is $\Cliff_{0,0} \cong \R$, and moving to
the right increments the signature from $(p,q)$ to $(p,q+1)$, and moving up
increments the signature $(p,q)$ to $(p+1,q)$. Any other Clifford algebra
can be obtained from an algebra on this table by tensoring with $M_{16}\R$, since
incrementing the signature by $8$ (by adding to either $p$ or $q$) results in
tensoring with $M_{16}\R$.
%
