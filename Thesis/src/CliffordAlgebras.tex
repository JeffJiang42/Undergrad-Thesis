\chapter{Preliminaries}
%
\section{Clifford Algebras}
%
\begin{defn}
Let $V$ be a real finite dimensional vector space with a nondegenerate
symmetric bilinear form $b : V \times V \to \R$. Then the \ib{Clifford Algebra}
of $V$ is the data of a unital associative $\R$-algebra $\Cliff(V,b)$ and a linear
map $i : V \to \Cliff(V,b)$ satisfying the following universal property:
Given any linear map $\varphi : V \to A$ of $V$ into any unital associative
$\R$-algebra $A$ satisfying $\varphi(v)^2 = b(v,v)$, there exists a unique map
$\tilde{\varphi}: \Cliff(V,b) \to A$ such that the following diagram commutes:
%
\[\begin{tikzcd}
V \ar[dr, "\varphi"] \ar[d, "i"'] \\
\Cliff(V,b) \ar[r, "\tilde{\varphi}"'] & A
\end{tikzcd}\]
\end{defn}
%
Explicitly, we can construct $\Cliff(V,b)$ as a quotient of the tensor algebra
\[
\mathcal{T}(V) = \bigoplus_{n \in \Z^{\geq 0}} V^{\otimes n}
\]
by the left ideal generated by elements of the form $v \otimes v - b(v,v)$, and
the map $i: V \to \mathcal{T}(V) / (v \otimes v - b(v,v))$ is given by the
inclusion $V \hookrightarrow \mathcal{T}(V)$ followed by the quotient map.
We identify $V$ with its image $i(V)$ as a subspace of $\Cliff(V,b)$.\\

Fix a basis for $V$. Then the bilinear form $b$ is given by a symmetric
invertible matrix $B$, which is conjugate to a diagonal matrix where all
the diagonal entries are either $1$ or $-1$. If after conjugation $B$ has
$p$ positive entries and $q$ negative entries, we say $b$ has signature $(p,q)$.
Any bilinear form $b$ of signature $(p,q)$ admits a basis $\set{e_i}$ satisfying
%
\begin{enumerate}
  \item For $1 \leq i \leq p$, we have $b(e_i,e_i) = 1$
  \item For $p+1 \leq j \leq p+q$, we have $b(e_j,e_j) = -1$
  \item For $i \neq j$, we have $b(e_i,e_j) = 0$
\end{enumerate}
%
Any such basis then determines an isomorphism $(V,b) \to \R^{p|q}$, where
$\R^{p|q}$ denotes $\R^{p+q}$ with the bilinear form given by the matrix
\[
\begin{pmatrix}
\id_{\R^p} & 0 \\
0 & -\id_{\R^q}
\end{pmatrix}
\]
In addition, we get a basis for $\Cliff(V,b)$, given by
\[
\set{e_{i_1}e_{i_2}\cdots e_{i_k}~:~ 0 \leq k \leq n, ~ 1 \leq i_j \leq n}
\]
where we interpret the product of $0$ basis vectors to be the unit element $1$.
which then implies the dimension of $\Cliff(V,b)$ as a vector space is
$2^{\dim V}$. This basis also determines an isomorphism
$\Cliff(V,b) \to \Cliff_{p,q}(\R)$, where $\Cliff_{p,q}(\R)$ is the Clifford
algebra for $\R^{p|q}$. Given $v,w \in V$, we write $v$ and $w$ in these
bases as $v^ie_i$ and $w^ie_i$ (using Einstein summation convention),
and derive the useful relation
%
\begin{align*}
vw + wv &= v^iw^je_ie_j + v^iw^je_je_i \\
&= v^iw^i(e_i^2 + e_i^2 \\
&= 2b(v,w)
\end{align*}
where we use the fact that the $e_i$ are orthgonal to deduce that $e_ie_j = -e_je_i$
if $i \neq j$.
%
Finally, this basis allows us to see that $\Cliff(V,b)$
has a $\Z/2\Z$ grading
\[
\Cliff(V,b) = \Cliff^0(V,b) \oplus \Cliff^1(V,b)
\]
Where $\Cliff^0(V,b)$ is the $\R$-span of all even products of basis vectors,
and $\Cliff^1(V,b)$ is the $\R$-span of all odd products of basis vectors. In
particular, $\Cliff^0(V,b)$ forms a subalgebra, and is called the \ib{even
subalgebra}, and we say that the elements of $\Cliff^0(V,b)$ are even. We then
call $\Cliff^1(V,b)$ the odd subspace, and say that its elements are odd.
Elements that are contained in the odd or even subspace are called
\ib{homogeneous}, and given a homogeneous element $a \in \Cliff(V,b)$,
we define it's \ib{parity} $|a|$ as
\[
|a| = \begin{cases}
0 & a \in \Cliff^0(V,b) \\
1 & a \in \Cliff^1(V,b)
\end{cases}
\]\\
There is an extremely nice relationship between a Clifford algebra and it's
even subalgebra.
%
\begin{thm}
The even subalgebra $\Cliff_{p,q}^0(\R)$ is isomorphic to both $\Cliff_{q, p-1}$
and $\Cliff_{p,q-1}$ as ungraded algebras (as long as $p-1 > 0$ or $q-1 > 0$.)
\end{thm}
%
\begin{proof}
Fix a basis $\set{e_1^+, \ldots e_p^+, e_1^- \ldots e_q^-}$ for $\R^{p|q}$, where
$(e_i^+)^2 = 1$ and $(e_i^-)^2 = -1$. Then a quick computation shows that
%
\begin{align*}
(e_i^+e_j^+)^2 &= -(e_i^+)^2(e_j^+)^2 = -1 \\
(e_i^-e_j^-)^2 &= -(e_i^-)^2(e_j^-)^2 = -1 \\
(e_i^+e_j^-)^2 &= -(e_i^+)^2(e_i^-)^2 = 1 \\
(e_i^-e_j^+)^2 &= -(e_i^-)^2(e_j^+)^2 = 1
\end{align*}
%
Assume $q \neq 0$. Then a generating set for $\Cliff_{p,q}^0(\R)$ is
\[
\set{e_1^-e_j^+ ~:~ 1 \leq j \leq p} \cup \set{e_1^-e_k^- ~:~ 2 \leq k \leq q}
\]
All the elements in the first set square to $1$, and all the elements in the
second set square to $-1$. We then get an isomorphism
$\Cliff^0_{p,q}(\R) \to \Cliff_{p,q-1}$ via the mappings
\begin{align*}
e_1^-e_j^+ &\mapsto e_j^+ \\
e_1^-e_k^- &\mapsto e_{k-1}^-
\end{align*}
In the case where $p \neq 0$, we have that an equally good generating set for
$\Cliff_{p,q}^0(\R)$ is
\[
\set{e^+_1e_j^+ ~:~ 2 \leq j \leq p} \cup \set{e_1^+e_i^- ~:~ 1 \leq i \leq q}
\]
Where the elements in the first set square to $-1$ and the elements of the second
set square to $1$. Then the mappings
\begin{align*}
e_1^+e_j^+ \mapsto e_{j-1}^- \\
e_1^+e_j^- \mapsto e_j^+
\end{align*}
gives the isomorphism $\Cliff_{p,q}^0(\R) \to \Cliff_{q, p-1}$.
\end{proof}
%
Given two $\R$-algebras $A$ and $B$, we can form their tensor product
$A \otimes B$, which has $A \otimes B$ as the underlying vector space, and the
multiplication is defined as
\[
(a \otimes b)(c \otimes d) = ac \otimes bd
\]
In the case that both $A$ and $B$ are $\Z/2\Z$ graded algebras, we have an alternate
version of the tensor product, where the underlying vector space is also
$A \otimes B$, but the multiplication is given by
\[
(a \otimes b)(c \otimes d) = (-1)^{|b||c|}(ac \otimes bd)
\]
We see that in the multiplication, we are formally commuting the elements of
$b$ and $c$, and we want to introduce a sign whenever elements are moved past
each other. This is the \ib{Koszul sign rule}. Another concept that needs
a slight modification in the graded case is the opposite algebra. In the
normal case, given an $\R$-algebra $A$, the \ib{opposite algebra} is
the algebra $A^{\text{op}}$ with the same underlying vector space, but
the multiplication in $A^\text{op}$ is given by $a * b = ba$, where $ba$
is the multiplication in $A$. In doing so, we are formally commuting $a$
and $b$, so in the graded situation, we invoke the Koszul sign rule when
defining the opposite algebra, and define the multiplication in $A^{\text{op}}$
to be $a * b = (-1)^{|a||b|} ba$.\\

One remarkable fact is that Clifford algebras are closed under the graded
tensor product, i.e. the graded tensor products of two Clifford algebras is
another Clifford algebra. Likewise, the graded opposite algebra of a Clifford
algebra is again a Clifford algebra. For the remainder of this section,
we will let $\otimes$ denote the graded tensor product, and the superscript
$\text{op}$ will denote the graded opposite algebra.
%
\begin{thm}
$\Cliff_{p+t,q+s}(\R) \cong \Cliff_{p,q}(\R) \otimes \Cliff_{t,s}(\R)$
\end{thm}
%
\begin{proof}
To give a map $\varphi : \Cliff_{p+t,q+s}(\R) \to \Cliff_{p,q}(\R) \otimes
\Cliff_{t,s}(\R)$, it is sufficient to specify it's action on $\R^{p+t|q+s}$, and
checking that the Clifford relations hold. Let
$\set{b_1^+,\ldots, b_{p+t}^+, b_1^-, \ldots b_{q+s}^-}$ denote the standard
orthogonal basis for $\R^{p+t|q+s}$ where $(b_i^+)^2 = 1$ and $(b_i^-)^2 = -1$.
We then define the bases $\set{e_i^\pm}$ and $\set{f_i^\pm}$ analogously for
$\R^{p|q}$ and $\R^{t|s}$ respectively. Then define $\varphi$ by
%
\begin{align*}
\varphi(b_i^+) &= \begin{cases}
e_i^+ \otimes 1 & 1 \leq i \leq p \\
1 \otimes f_i^+ & p+1 \leq i \leq p+t
\end{cases} \\
\varphi(b_i^-) &= \begin{cases}
e_i^- \otimes 1 & 1 \leq i \leq q \\
1 \otimes f_i^- & q+1 \leq i \leq q+s
\end{cases}
\end{align*}
%
This map is injective on generators, so if we show that this satisfies the
Clifford relations, then the map given by extending the map to all of
$\Cliff_{p+t,q+s}(\R)$ will be an isomorphism by dimension reasons. Showing
the Clifford relations amounts to showing
%
\begin{enumerate}
  \item $\varphi(b_i^+)^2 = 1$
  \item $\varphi(b_i^-)^2 = -1$
  \item The images of any pair of distinct basis vectors anticommute.
\end{enumerate}
%
The first two are relations are clear from how we defined $\varphi$. To show
that the images of distinct basis vectors anticommute, there are serveral
cases to consider. Given $b_i^+$ and $b_j^+$ where $1 \leq i,j \leq p$, they
anticommute, because $e_i^+$ and $e_j^+$ anticommute. In the case where
$1 \leq i \leq p$ and $p+1 \leq j \leq p+t$, we compute
%
\begin{align*}
\varphi(b_i^+)\varphi(b_j^+) + \varphi(b_j^+)\varphi(b_i+) &=
(e_i^+ \otimes 1)(1 \otimes f_j^+) + (1\otimes f_j^+)(e_i^+ \otimes 1) \\
&= e_i^+ \otimes f_j^+ - e_i^+ \otimes f_j^+
\end{align*}
where we use the Koszul sign rule for the second term, noting that $f_j+$ and
$e_i^+$ are both odd. The proof that the images of the $b_i^-$ anti commute with
each other, as well as the proof that the images of the $b_i^+$ and $b_i^-$
anticommute are exactly the same.
%
\end{proof}
%
\begin{thm}
The graded opposite algebra $\Cliff_{p,q}^{\text{op}}$ is isomorphic to
$\Cliff_{q,p}$.
\end{thm}
%
\begin{proof}
TODO
\end{proof}
%
Because of these theorems, once we compute a few of the lower dimensional
Clifford algebras, we will have enough data to fully classify all Clifford
algebras over $\R$. In the case $p = q = 0$, we let $\Cliff_{0,0}(\R) = \R$.
%
\begin{exmp}[\ib{Some low dimensional examples}]\enumbreak
\begin{enumerate}
  \item As ungraded algebras, the Clifford algebra $\Cliff_{0,1}(\R)$ is isomorphic
  to $\C$, where the isomorphism is given by $e_1 \mapsto i$.
  \item As ungraded algebras, $\Cliff_{0,2}(\R)$ is isomorphic to the quaternions
  $\H$, where the isomorphism is given by $e_1 \mapsto i$ and $e_2 \mapsto j$.
  \item As a graded algebra, $\Cliff_{1,1}(\R)$ is isomorphic to $\End(\R^{1|1})$.
  The isomorphism is given by
  \[
  e_1^+ \mapsto \begin{pmatrix}
  0 & 1 \\
  1 & 0
  \end{pmatrix} \qquad e_1^- \mapsto \begin{pmatrix}
  0 & 1 \\
  -1 & 0
  \end{pmatrix}
  \]
  \item As ungraded algebras $\Cliff_{1,0}(\R)$ is isomorphic to the product
  algebra $\R \times \R$, where $e_1 \mapsto (1,-1)$.
  \item As ungraded algebras, $\Cliff_{2,0}(\R)$ is isomorphic to $M_2\R$. The
  isomorphism is given by
  \[
  e_1 \mapsto \begin{pmatrix}
  0 & 1 \\
  1 & 0
  \end{pmatrix} \qquad e_2 \mapsto \begin{pmatrix}
  1 & 0 \\
  0 & -1
  \end{pmatrix}
  \]
\end{enumerate}
\end{exmp}
%
To classify all Clifford algebras as ungraded algebras, it suffices to know
the following table: \\\\
\resizebox{.8\width}{!}{
\centering
\begin{tabular}{|c|c|c|c|c|c|c|c|}
\hline
$M_8\C$ & $M_8\H$ & $M_8\H \times M_8\H$ & $M_{16}\H$ & $M_{32}\C$ & $M_{64}\R$ & $M_{64}\R \times M_{64}\R$ & $M_{128}\R$ \\
\hline
$M_4\H$ & $M_4\H \times M_4\H$ & $M_8\H$ & $M_{16}\C$ & $M_{32}\R$ & $M_{32}\R \times M_{32}\R$ & $M_{64}\R$ & $M_{64}\C$ \\
\hline
$M_2\H \times M_2\H$ & $M_4\H$ & $M_8\C$ & $M_{16}\R$ & $M_{16}\R \times M_{16}\R$ & $M_{32}\R$ & $M_{32}\C$ & $M_{32}H$ \\
\hline
$M_2\H$ & $M_4\C$ & $M_8\R$ & $M_8\R \times M_8\R$ & $M_{16}\R$ & $M_{16}\C$ & $M_{16}\H$ & $M_{16}\H \times M_{16}\H$ \\
\hline
$M_2\C$ & $M_4\R$ & $M_4\R \times M_4\R$ & $M_8\R$ & $M_8\C$ & $M_8\H$ & $M_8\H \times M_8\H$ & $M_{16}\H$ \\
\hline
$M_2\R$ & $M_2\R \times M_2\R$ & $M_4\R$ & $M_4\C$ & $M_4\H$ & $M_4\H \times M_4\H$ & $M_8\H$ & $M_{16}\C$ \\
\hline
$\R \times \R$ & $M_2\R$ & $M_2\C$ & $M_2\H$ & $M_2\H \times M_2\H$ & $M_4\H$ & $M_8\C$ & $M_{16}\R$ \\
\hline
$\R$ & $\C$ & $\H$ & $\H \times \H$ & $M_2\H$ & $M_4\C$ & $M_8\R$ & $M_8\R \times M_8\R$\\
\hline
\end{tabular}
}
$ $\\\\\\
To read the table, the bottom right entry is $\Cliff_{0,0} \cong \R$, and moving to
the right increments the signature from $(p,q)$ to $(p,q+1)$, and moving up
increments the signature $(p,q)$ to $(p+1,q)$. Any other Clifford algebra
can be obtained from an algebra on this table by tensoring with $M_{16}\R$, since
incremeting the signature by $8$ (by adding to either $p$ or $q$) results in
tensoring with $M_{16}\R$.
%
\begin{defn}
A (left) \ib{Clifford module} for the Clifford algebra $\Cliff_{p,q}(\R)$ is a module
for $\Cliff_{p,q}(\R)$ in the usual sense i.e. a real vector space $V$ equipped
with an algebra action $\bullet: \Cliff_{p,q} \times V \to V$ satisfying
\begin{enumerate}
  \item Every element of $\Cliff_{p,q}(\R)$ acts linearly on $V$.
  \item $(AB) \cdot v = A\cdot(B \cdot v)$ for all $v \in V$.
  \item $(A + B) \cdot v = A\cdot v + B\cdot V$ for all $v \in V$.
\end{enumerate}
Equivalently, it is the data of a real vector space $V$ and a homomorphism
$\Cliff_{p,q}(\R) \to \End(V)$.
\end{defn}
%
\begin{defn}
A Clifford module is \ib{irreducible} if there exist no proper nontrivial submodules.
\end{defn}
%
From the classification of Clifford algebras, all the Clifford algebras are either
matrix algebras $M_n\F$, where $\F = \R$, $\C$, or $\H$, or products
$M_n\F \times M_n\F$ of two copies of the same matrix algebra. This is sufficient
to conclude that Clifford algebras are semisimple, so all Clifford modules
will be direct sums of irreducible modules. Therefore, classifying all Clifford
modules reduces to classifying the irreducible Clifford modules.
%
\begin{thm}
Let $\F = \R$, $\C$, or $\H$. Then any nontrivial irreducible module for
$M_n\F$ is isomorphic to $\F^n$ with the standard action.
\end{thm}
%
\begin{proof}
We first note that $M_n\F$ acts transitively on $\F^n$, which implies that
it is irreducible. We then must show that $\F^n$ is, up to isomorphism, the only
irreducible $M_n\F$ module. The matrix algebra $M_n\F$ admits an increasing
chain of left ideals
\[
0 = I_0 \subset I_1 \subset \ldots \subset I_n = M_n\F
\]
where $I_k$ is the set of matrices where only the first $k$ columns are nonzero.
These ideals have the property that the quotient $I_k / I_{k-1}$ is isomorphic
to $\F^n$ as a left $M_n\F$ module. Then let $M$ be some nontrivial irreducible
$M_n\F$ module, and fix $m \in M$. Then the orbit $M_n\F \cdot m$ of $m$
under the algebra action is a nonzero submodule, so it must be all of $M$.
Then the map $\varphi : M_n\F \to M$ given by $A \mapsto A \cdot m$ is
a surjective map of left $M_n\F$ modules. Then there must exist some smallest
$k$ such that $\varphi(I_k)$ is nonzero, and by construction,
$\varphi\vert_{I_k}$ factors through the quotient $I_k / I_{k-1}$, which
is isomorphic to $\F^n$ with the standard action. Then since $\F^n$ is irreducible,
this gives us a nontrivial map between irreducible modules, which is an isomorphism
by Schur's Lemma.
\end{proof}
%
\begin{thm}
Any nontrivial irreducible module for $M_n\F \times M_n\F$ is isomorphic to
either $\F^n$ where the left factor acts in the usual way, and the right factor
acts by $0$, or $\F^n$ where the left factor acts by $0$ and the right factor
acts in the usual way.
\end{thm}
%
\begin{proof}
Let $R$ denote $\F^n$ where the right factor acts nontrivially, and let $L$
denote $\F^n$ where the left factor acts nontrivially. Both $L$ and $R$ are
irreducible since $M_n\F \times M_n\F$ acts transitively on them. To show
that they are the only irreducible modules up to isomorphism, we use a similar
techinique as above. Let $I_k$ denote the chain of increasing ideals in $M_n\F$,
as we used above. Then $M_n\F \times M_n\F$ admits a chain of increasing left
ideals $J_k$
\[
0 = J_0 \subset I_1 \times \set{0} \subset \ldots \subset M_n\F \times \set{0}
\subset M_n\F \times I_1 \subset \ldots \subset M_n\F \times M_n\F = J_{2n}
\]
We note that for $1 \leq k \leq n$, we have that $J_k / J_{k-1}$ is isomorphic to
$L$, and for $n+1 \leq k \leq 2n$, we have that $J_k / J_{k-1}$ is isomorphic to
$R$. Then given a nontrivial irreducible module $M$ and a nonzero element $m$,
we get a surjective map $\varphi : M_n\F \times M_n\F$ where $A \mapsto A\cdot m$.
Like before, there exists some smallest $k$ such that $\varphi(J_k)$ is nonzero,
which then factors through to an isomorphism $J_k / J_{k-1} \to M$, so $M$
is either isomorphic to $R$ or $L$.
\end{proof}
%
This then gives a full classification of the irreducible ungraded Clifford
modules.
%
