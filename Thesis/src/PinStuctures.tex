%
\section{$\Pin^\pm_n$ Structures}
%
The same general construction for Spin structures works for $\Pin^\pm_n$ as well.
%
\begin{defn}
Let $M$ be a smooth manifold. Then a \ib{$\Pin_n^\pm$ structure} on $M$ is the
data of a principal $\Pin_n^\pm$-bundle $P \to M$, along with a
$\Pin^\pm_n$-equivariant map $P \to \B_O(M)$, where the $\Pin^\pm_n$ action on
$\B_O(M)$ is given by the double cover $\Pin^\pm_n \to O_n$.
\end{defn}
%
Just like with Spin, a $\Pin^\pm_n$ structure on $M$ induces a $\Pin^\pm_{n-1}$
on the boundary using the outward unit normal, which gives the analogous diagram
\[\begin{tikzcd}
\B_{\Pin^\pm}(\partial M) \ar[d]\ar[r, hookrightarrow] & \B_{\Pin^\pm}(M)\vert_{\partial M}
\ar[d]\ar[r, hookrightarrow] & \B_{\Pin^\pm}(M) \ar[d]\\
\B_O(\partial M) \ar[dr]\ar[r, hookrightarrow] & \B_O(M)\vert_{\partial M}
\ar[d]\ar[r, hookrightarrow] & \B_O(M) \ar[d]\\
& \partial M \ar[r, hookrightarrow]& M
\end{tikzcd}\]
%
\begin{exmp}[Pin structures on $S^1$]
There are two problems to discuss here, since $\Pin_1^+$ and $\Pin_1^-$ are
different groups, namely
\begin{align*}
\Pin_1^+ &\cong \Z/2\Z \times \Z / 2\Z \\
\Pin_1^- &\cong \Z/4\Z
\end{align*}
We do the $\Pin_1^+$ case first. In this case, the Clifford algebra
$\Cliff_{1,0}(\R)$ is isomorphic to $\R \times \R$, where the isomorphism is
determined by $e_1 \mapsto (-1,1)$. There are two irreducible modules corresponding
to projection onto one of the factors, so the Pinor representations are the one
dimensional real representations $\P^+$ and $\P^-$, where the action of $e_1$ on
$\P^\pm$ is by $\pm 1$. Let $\P = \P^+ \oplus \P^-$. In addition, we have two
principal $\Pin^+_1$-bundles over $S^1$. One has $4$ components that are permuted
by the action of $\Pin_1^+$, and the other has two components, which
are both the double cover $z \mapsto z^2$.

Then for each $\Pin_1^+$-bundle,
we take the associated bundle. For $P_1$, we have that sections of
$P_1 \times_{\Pin_1^+} \P$ are equivalent to $\Pin_1^+$-equivariant maps
$P_1 \to P$. Since the action of $\Pin_1^+$ simply permutes the $4$ components
of $P_1$ and all the components are diffeomorphic to $S^1$, we can map
a single component to to $\P$, and this determines the map on the other $4$
components. Therefore, sections are equivalent data to maps $S^1 \to \P$, i.e.
$2\pi$-periodic maps $\R \to \P$. The Dirac operator $D$ for this bundle then
given by $D = e_1\partial_\theta$. For the other bundle, there are two components,
which are each the connected double cover of $S^1$. The element $-1$ acts in the same
way as it did for the connected double cover, and the element $e_1$ permutes
the two components. Then sections of the associated bundle are again equivalent
data to $2\pi$-antiperiodic maps $S^1 \to \P$. The Dirac operator is then given
by $i\partial_\theta - \frac{1}{2}$. \\

For $\Pin_1^-$, we also have two principal bundles, which we again denote
$P_1$ and $P_2$. The first is $P_1$, which has $4$ connected components that are cyclically
permuted by the action of $\Pin_1^-$, and $P_2$ is the connected $4$-fold
cover $z \mapsto z^4$. There is only a single irreducible module for $\Cliff_{0,1}(\R)$,
which is $\C$, which is then the single Pinor representation $\P$ where
$e_1$ acts by $i$. The associated bundle to $P_1$ is the trivial bundle with the
Dirac operator being given by $D = i\partial_\theta$. For the second case,
the sections of the associated bundle are determined by functions
$\psi : \R \to \C$ where $f(\theta + 2\pi) = if(\theta)$. We can then write these
as $\psi = e^{i\theta/4}f$ for a $2\pi$-periodic function $f$, so under this
trivialization of the associated bundle, the Dirac operator is given by
$i\partial_\theta - \frac{1}{4}$.
\end{exmp}
%