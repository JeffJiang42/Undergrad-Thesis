%
\section{$\Pin^\pm_n$ Structures}
%
The same general construction for Spin structures works for $\Pin^\pm_n$ as well.
When we refer to $\Pin^\pm_n$, we refer to a fixed choice of $\Pin^+_n$ or
$\Pin^-_n$.
%
\begin{defn}
Let $M$ be a smooth manifold. Then a \ib{$\Pin_n^\pm$ structure} on $M$ is
a reduction of structure group for $O_n$ to $\Pin^\pm_n$.
\end{defn}
%
Just like with Spin, a $\Pin^\pm_n$ structure on $M$ induces a $\Pin^\pm_{n-1}$
on the boundary using the outward unit normal, which gives the analogous diagram
\[\begin{tikzcd}
\B_{\Pin^\pm}(\partial M) \ar[d]\ar[r, hookrightarrow] & \B_{\Pin^\pm}(M)\vert_{\partial M}
\ar[d]\ar[r, hookrightarrow] & \B_{\Pin^\pm}(M) \ar[d]\\
\B_O(\partial M) \ar[dr]\ar[r, hookrightarrow] & \B_O(M)\vert_{\partial M}
\ar[d]\ar[r, hookrightarrow] & \B_O(M) \ar[d]\\
& \partial M \ar[r, hookrightarrow]& M
\end{tikzcd}\]
%
\begin{exmp}[Pin structures on $S^1$]
There are two problems to discuss here, since $\Pin_1^+$ and $\Pin_1^-$ are
different groups, namely
%
\begin{align*}
\Pin_1^+ &\cong \Z/2\Z \times \Z / 2\Z \\
\Pin_1^- &\cong \Z/4\Z
\end{align*}
The isomorphism $\Pin^+_1 \to \Z/2\Z \times \Z/2\Z$ maps $-1 \mapsto (1,0)$ and
$e_1 \mapsto (0,1)$, and the isomorphism $\Pin^-_1 \to \Z/4\Z$ maps $e_1 \mapsto 1$.
%

We do the $\Pin_1^+$ case first. In this case, the Clifford algebra
$\Cliff_{1,0}(\R)$ is isomorphic to $\R \times \R$, where on the basis $\set{1,e_1}$,
we map $1\mapsto 1$ and $e_1 \mapsto (-1,1)$. There are two irreducible
modules corresponding
to projection onto one of the factors, so the Pinor representations are the one
dimensional real representations $\P^+$ and $\P^-$, where the action of $e_1$ on
$\P^\pm$ is by $\pm 1$. Let $\P = \P^+ \oplus \P^-$. We then take inventory
of all the principal $\Pin_1^+$-bundles over $S^1$. From before, we know that these
are classified by homomorphisms $\pi_1(S^1, \theta_0) \to \Pin_1^+$ up to
conjugation. Since $\Pin_1^+$ is abelian, the conjugation action is trivial,
so principal $\Pin_1^+$-bundles are classified by homomorphisms
$\pi_1(S^1,\theta_0) \to \Pin_1^+$. There are $4$ such homomorphisms,
which are all determined by the image of a generator $\alpha$ of $
\pi_1(S^1,\theta_0)$, since the fundamental group is infinite cyclic.
We denote all the bundles $P_g \to S^1$, where $g$ is the image of $\alpha$ under
the corresponding homomorphism.
%
\begin{enumerate}
  \item The bundle $P_1 \to S^1$ has $4$ connected components, each diffeomorphic
  to a circle. The map $P-1 \to S^1$ maps each component diffeomorphically onto
  $S^1$, and the action of $\Pin_1^+$ permutes these $4$ components.
  \item The bundle $P_{-1} \to S^1$ has two components, each diffeomorphic to
  a circle and double covering $S^1$. The element $-1$ acts by rotating both
  circles by $\pi$, and the element $e_1$ exchanges the two circles.
  \item The bundle $P_{e_1} \to S^1$ has two components, each diffeomorphic to a
  circle and double covering $S^1$. The action of $e_1$ rotates both circles by
  $\pi$, and $-1$ exchanges the two circles.
  \item The bundle $P_{-e_1} \to S^1$, has two components, each diffeomorphic
  to a circle and double covering $S^1$. The element $-e_1$ acts by rotation by
  $\pi$, and the element $-1$ exchanges the two circles.
\end{enumerate}
%
We then must determine which bundles admit
$\Pin_1^+$-equivariant maps $P_g \to \B_O(S^1)$, which is a pair of disjoint
circles permuted by $O_1 = \set{\pm 1}$. Under the double covering, the
elements $\pm 1 \in \Pin_1^+$ act trivially, while the elements $\pm e_1$ act by $-1$.
%
\begin{enumerate}
  \item The bundle $P_1 \to S^1$ admits a $\Pin_1^+$-equivariant map
  $P_1 \to \B_O(S^1)$. After mapping one of the components diffeomorphically
  onto one of the components of $\B_O(S^1)$, we know how to map
  the other $3$ components to make the map $\Pin_1^+$-equivariant.
  \item The bundle $P_{-1} \to S^1$ can also define a $\Pin_1^+$ structure on
  $S^1$. Since $-1$ rotates both circles by $\pi$, we map each component to
  a component of $\B_O(S^1)$ via the double cover $z \mapsto z^2$.
  \item The bundle $P_{e_1} \to S^1$ cannot determine a $\Pin^+_1$ structure.
  The action of $e_1$ rotates both components of $P_{e_1}$ by $\pi$, but
  permutes the components of $\B_O(S^1)$. Any $\Pin_1^+$-equivariant map
  $P_{e_1} \to \B_O(S^1)$ must map one component of $P_{e_1}$ onto two
  components of $\B_O(S^1)$, which is impossible
  \item The bundle $P_{-e_1} \to S^1$ also cannot determine a $\Pin_1^+$
  structure by the same reason that $P_{e_1}$ cannot.
\end{enumerate}
%
One thing to note is that an orientation of $S^1$ and a $\Pin_1^+$ structure
on $S^1$ determine a Spin structure. A choice of component of $\B_O(S^1)$
determines an orientation, and taking the preimage of that component
under the maps $P_{\pm 1} \to \B_O(S^1)$, we recover a Spin structure. In
particular, the bundle $P_1$ corresponds to the Spin structure with Dirac operator
$i\partial_\theta$ and the bundle $P_{-1}$ corresponds to the Spin structure
with Dirac operator $i\partial_\theta - \frac{1}{2}$.  Indeed, the associated
bundles $P_{\pm 1} \times_{\Pin_1^+} \P$ have the same Dirac operators as
their corresponding Spin structures.\\

For the case of $\Pin_1^-$, there are again $4$ different principal $\Pin_1^-$
bundles, classified by homomorphisms $\pi_1(S^1, \theta_0) \to \Pin_1^-$.
As before, we let $\alpha$ denote a generator for $\pi_1(S^1, \theta_0)$, and
let $P_g \to S^1$ denote the principal $\Pin_1^-$ bundle corresponding to
the homomorphism mapping $\alpha \mapsto g$.
%
\begin{enumerate}
  \item The bundle $P_1 \to S^1$ has $4$ components cyclically permuted
  by the action of $e_1$.
  \item The bundle $P_{-1} \to S^1$ has $2$ components, both double covering $S^1$.
  the element $-1$ acts by rotation by $\pi$ on both factors, and the action
  of $e_1$ permutes the two components.
  \item The bundle $P_{e_1} \to S^1$ corresponds to the $4$-fold covering
  $S^1 \to S^1$ given by $z \mapsto z^4$. The action of $e_1$ acts by
  rotation by $\pi / 2$.
  \item The bundle $P_{-e_1} \to S^1$ also corresponds to the $4$-fold covering,
  $S^1 \to S^1$, where the action of $-e_1$ is given by rotation by $\pi / 2$.
\end{enumerate}
%
Again, we need to figure out which bundles can determine $\Pin_1^-$ structures
on $S^1$.
%
\begin{enumerate}
  \item The bundle $P_1$ admits a map $P_1 \to \B_O(S^1)$. After mapping any
  component diffeomorphically onto a component of $\B_O(S^1)$, the rest of the
  map is determined.
  \item The bundle $P_{-1}$ admits a map $P_{-1} \to \B_O(S^1)$, where
  each component maps onto a component of $\B_O(S^1)$ via the double cover
  $z \mapsto z^2$.
  \item This bundle cannot determine a $\Pin_1^-$ structure, since there
  cannot exist a continuous surjective map from a connected space to a
  disconnected space.
  \item This bundle cannot determine a $\Pin_1^-$ structure by the same
  reasoning as above.
\end{enumerate}
Again, we see that the $\Pin_1^-$ structures on $S^1$ correspond to $\Spin$
structures once we fix an orientation, and the associated bundles to these
$\Pin_1^-$-bundles will have the same Dirac operator as their corresponding
spin structures.
\end{exmp}
%
On the circle, we observed that a $\Pin^\pm$ structure and and orientation
are equivalent data to a Spin structure. This is true in general via the same
construction. A $\Pin^\pm$ structure is the data of a principal
$\Pin^\pm_n$-bundle $\B_{\Pin^\pm}(M) \to M$ along with a
$\Pin^\pm_n$-equivariant map $\B_{\Pin^\pm}(M) \to \B_O(M)$. An orientation
determines a subset $\B_{SO}(M) \subset \B_O(M)$ that is a principal
$SO_n$-bundle over $M$, and taking the preimage of $\B_{SO}(M)$ under the
map $\B_{\Pin^\pm}(M) \to \B_O(M)$ is a principal $\Spin_n$-bundle over $M$,
since the preimage of $SO_n \subset O_n$ under the double covers
$\Pin^\pm_n \to O_n$ is a subgroup isomorphic to $\Spin_n$.
The restriction of $\B_{\Pin^\pm}(M) \to \B_O(M)$ then determines the
desired Spin structure.\\

Unlike Spin structures, $\Pin^\pm$ structures do not require an orientation,
and can be defined on nonorientable manifolds.
%
\begin{exmp}[$\Pin^\pm$ structures on the M\"obius band]
Let $M$ denote the M\"obius band, which is the quotient space
\[
M = [-1,1] \times [-1,1] / (-1, x) \sim (1, -x)
\]
The square $S = [-1,1] \times [-1,1]$ inherits a Riemannian metric from $\R^2$,
and the orthonormal frame bundle $\B_O(S)$ is isomorphic to
the product bundle $S \times O_2$. This then descends to $M$ after
we specify gluing data for the identified edges of the square. The identification
of the edges reverses the direction in the second factor, which corresponds to
a transformation by the matrix
\[
A = \begin{pmatrix}
1 & 0 \\
0 & -1
\end{pmatrix}
\]
giving us that the orthonormal frame bundle $\B_O(M)$ of $M$ induced by the
metric on $S$ is isomorphic to
\[
S\times O_2 / (-1,x, b) \sim (1,-x, A \cdot b)
\]
We do the $\Pin^+_2$ case first. Inside of $\Cliff_{2,0} \cong M_2\R$,
the group $\Pin_2^+$ is topologically two disjoint circles. The identity component
is \[
\Spin_2 = \set{\cos\theta + \sin\theta e_1e_2 ~:~ \theta \in \R}
\]
and the second component is the subset $e_1 \cdot \Spin_2$. The square
$S$ admits a single principal $\Pin_2^+$-bundle, which is the trivial bundle
$S \times \Pin_2^+$, and any principal $\Pin_2^+$-bundle over $M$ will come
by specifying gluing data of $S \times \Pin_2^+$ on the identified edges of $S$.
More explicitly, any principal $\Pin_2^+$-bundle over $M$ will arise as a
quotient
\[
S \times \Pin_2^+ / (-1,x,g) \sim (1, -x, \psi(g))
\]
where $\psi : \Pin_2^+ \to \Pin_2^+$ is a $\Pin_2^+$-equivariant diffeomorphism.
From this characterization, we can determine which maps $\psi$ determine
a principal $\Pin^+_2$-bundle $P \to M$ that admit a $\Pin_2^+$-equivariant
map $P \to \B_O(M)$. For points that that are not identified, i.e. points
on $S$ with first component not equal to $\pm 1$, we can specify
$P \to \B_O(M)$ fiberwise by the standard double cover $\Pin_2^+ \to O_2$. On
the identified edges, in order for the map to be $\Pin_2^+$-equivariant, we
require that the image of the equivalence class $[-1,x,g] \in P$ to be
$[-1, x, \rho(g)]$, where $\rho : \Pin_2^+ \to O_2$ is the double cover. However,
we have that $[-1,x,g] = [1, -x, \psi(g)]$ and
$[-1,x,\rho(g)] = [1,-x, A \cdot \rho(g)]$, so we need $\psi(g)$ to be one
of the preimages of $A \cdot \rho(g)$ under the double cover in order for this
to be well defined. Since $A$ is an orthogonal transformation, it has two
preimages under the double cover, which are $\rho\inv(A) = \pm e_1$. Therefore,
$\psi$ must be left multiplication by either $e_1$ or $-e_1$, giving us two
$\Pin^+$ structures on $M$. \\

We then want to determine the induced $\Pin_1^+$ structure on the boundary.
Using the standard coordinates on $S$ inherited from $\R^2$, the outward normal
on the top edge of the square points upwards, while the outward normal points
downwards on the bottom edge of the square. Therefore, the fiber of a point
on the top edge under $\B_O(M) \to M$ is the two point set
\[
\set{\begin{pmatrix}
0 & 1 \\
1 & 0
\end{pmatrix}, \begin{pmatrix}
0 & -1 \\
1 & 0
\end{pmatrix}}
\]
and the fiber over a point on the bottom edge is
\[
\set{\begin{pmatrix}
0 & 1 \\
-1 & 0
\end{pmatrix}, \begin{pmatrix}
0 & -1 \\
-1 & 0
\end{pmatrix}}
\]
and on the endpoints, we have the identifications
\[
\begin{pmatrix}
0 & 1 \\
1 & 0
\end{pmatrix} \sim \begin{pmatrix}
0 & 1 \\
-1 & 0
\end{pmatrix} \qquad \begin{pmatrix}
0 & -1 \\
1 & 0
\end{pmatrix} \sim \begin{pmatrix}
0 & -1 \\
-1 & 0
\end{pmatrix}
\]
We then need to compute the preimages of the matrices under the double cover
$\rho : \Pin_2^+ \to O_2$ to find the fiber under $\B_{\Pin^+}(M) \to M$. We
compute
%
\begin{align*}
\rho\inv\begin{pmatrix}
0 & 1 \\
1 & 0
\end{pmatrix} &= \set{\pm \frac{\sqrt{2}}{2}(e_1 + e_2)} \\[5pt]
\rho\inv\begin{pmatrix}
0 & -1 \\
1 & 0
\end{pmatrix} &= \set{\pm\frac{\sqrt{2}}{2}(-1 + e_1e_2)} \\[5pt]
\rho\inv\begin{pmatrix}
0 & 1 \\
-1 & 0
\end{pmatrix} &= \set{\pm\frac{\sqrt{2}}{2}(1 + e_1e_2)} \\[5pt]
\rho\inv\begin{pmatrix}
0 & -1 \\
-1 & 0
\end{pmatrix} &= \set{\pm\frac{\sqrt{2}}{2}(-e_1 + e_2)}
\end{align*}
%
Depending on our choice of $\pm e_1$ for the gluing map at the endpoints, we
get different identifications on the endpoints of the top edge and the
endpoints on the bottom edge, but for either we choose, the resulting bundle
over $\partial M \cong S^1$ has four connected components, where each of the
$4$ elements in the fibers of the top edge of the square pair with one of the
elements in the fibers of the bottom edge, giving $4$ circles. If we orient
the boundary of $M$, this gives us the nonbounding Spin structure corresponding
to the disconnected double cover of $S^1$. \\

We now do the case of $\Pin^-_2$. Most of the discussion with $\Pin^+_2$ carries
over to $\Pin^-_2$, up until the point where we compute the preimages of
the matrices under double cover. Let $\varphi : \Pin_2^- \to O_2$ be the
double cover. We compute the preimages
%
\begin{align*}
\varphi\inv\begin{pmatrix}
0 & 1 \\
1 & 0
\end{pmatrix} &= \set{\pm \frac{\sqrt{2}}{2}(e_1 + e_2)} \\[5pt]
\varphi\inv\begin{pmatrix}
0 & -1 \\
1 & 0
\end{pmatrix} &= \set{\pm\frac{\sqrt{2}}{2}(1 + e_1e_2)} \\[5pt]
\varphi\inv\begin{pmatrix}
0 & 1 \\
-1 & 0
\end{pmatrix} &= \set{\pm\frac{\sqrt{2}}{2}(-1 + e_1e_2)} \\[5pt]
\varphi\inv\begin{pmatrix}
0 & -1 \\
-1 & 0
\end{pmatrix} &= \set{\pm\frac{\sqrt{2}}{2}(-e_1 + e_2)}
\end{align*}
%
This defines a $\Pin^-_2$ structure on $\partial M$ that has two connected
components. To see this, we glue the fiber over top right corner
with the fiber over the bottom left corner via left multiplication by $e_1$,
giving the identifications
%
\begin{align*}
\pm \frac{\sqrt{2}}{2}(e_1 + e_2) &\sim \pm\frac{\sqrt{2}}{2}(-1 + e_1e_2) \\
\pm\frac{\sqrt{2}}{2}(1 + e_1e_2) &\sim \mp\frac{\sqrt{2}}{2}(-e_1 + e_2)
\end{align*}
%
The identifications of the fiber over the bottom right corner with the fiber
over the top left corner are
%
\begin{align*}
\pm\frac{\sqrt{2}}{2}(-1 + e_1e_2) &\sim \mp\frac{\sqrt{2}}{2}(e_1 + e_2)\\
\pm\frac{\sqrt{2}}{2}(-e_1 + e_2) &\sim \pm\frac{\sqrt{2}}{2}(1 + e_1e_2)
\end{align*}
The two flipped signs connects two of the components of the fiber together,
giving two components instead of the four components in the case with
$\Pin^+_2$, so after fixing an orientation of $\partial M$, we get the Spin
structure corresponding to the bounding Spin structure, i.e. the Spin structure
associated to the connected double cover of $S^1$.
%
\end{exmp}
%
To every smooth manifold $M$, there is an associated oriented manifold called the
orientation double cover, which is a principal $\Z/2\Z$-bundle encoding
information about the orientability of $M$.
%
\begin{defn}
Let $M$ be a smooth manifold. Then the \ib{orientation double cover} is the set
\[
\tilde{M} = \set{(p, o) ~:~ p \in M, o \text{ is an orientation of } T_pM}
\]
This comes with a natural map $\tilde{M} \to M$ mapping $(p,o) \mapsto p$,
and a natural $\Z/2\Z$ action where $(p,o) \mapsto (p,-o)$. Local coordinates
on $M$ induce local coordinates on $\tilde{M}$, which we can use to define the
topology and smooth structure. Under this topology and smooth structure, the
map $\tilde{M} \to M$ is a smooth double covering.
\end{defn}
%
The double cover is a local diffeomorphism, so its differential induces
identifications of tangent spaces $T_{(p,o)}\tilde{M} \to T_pM$, which allows
us to define a canonical orientation on $\tilde{M}$ where we orient
$T_{(p,o)}\tilde{M}$ with the orientation $o$. The orientation double cover
detects orientability of $M$ in the following way.
%
\begin{thm} \enumbreak
\begin{enumerate}
  \item If $M$ is orientable, $\tilde{M}$ is to diffeomorphic the disconnected
  double cover $M \coprod M$.
  \item If $M$ is not orientable, $\tilde{M}$ is connected. Furthermore,
  $\tilde{M}$ is unique in the following way: Given another double cover
  $\hat{M} \to M$ of an oriented manifold $\hat{M}$ onto $M$, there is a unique
  orientation preserving diffeomorphism $\varphi : \hat{M} \to \tilde{M}$ such
  that
  \[\begin{tikzcd}
  \hat{M} \ar[dr]\ar[rr, "\varphi"] && \tilde{M}\ar[dl] \\
   & M
  \end{tikzcd}\]
  commutes.
\end{enumerate}
\end{thm}
%
In particular, given an unorientable manifold $M$, any oriented double
cover of $M$ is isomorphic to the orientation double cover.
%
\begin{exmp}
The cylinder $S^1 \times I$ can be realized as the orientation double
cover of the M\"obius band $M$. Let $f : S^1 \times I \to S^1 \times I$ be the
map flipping the cylinder. Then the quotient space $S^1 \times I / (x \sim f(x))$
is diffeomorphic to the M\"obius band, and the quotient map is the double cover.
\end{exmp}
%
\begin{thm}
A $\Pin^\pm$ structure on a manifold induces a Spin structure on the orientation
double cover.
\end{thm}
%
\begin{proof}
Let $ \pi : \tilde{M} \to M$ denote the orientation double cover. Fix a Riemannian
metric $g$ on $M$. This induces a Riemannian metric on $\tilde{M}$, which
is the pullback metric $\pi^*g$. In addition, the orthonormal frame bundle
$\B_O(\tilde{M})$ with respect to the metric $\pi^*g$ is isomorphic to the
pullback bundle $\pi^*\B_O(M)$. Pulling back $\B_{\Pin^\pm}(M)$ along the map
$\pi^*\B_O(\tilde{M})$, we get the a $\Pin^\pm_n$-bundle $\B_{\Pin^\pm}$ over
$\tilde{M}$ with a map to $\pi^*\B_{O}(M) \cong \B_O(\tilde{M})$, so it defines
a $\Pin^\pm$ structure on $\tilde{M}$, giving us the diagram
\[\begin{tikzcd}
\B_{\Pin^\pm}(\tilde{M}) \ar[d]\ar[r] & \B_{\Pin^\pm}(M) \ar[d]\\
\pi^*\B_O(M) \ar[d]\ar[r] & \B_O(M) \ar[d]\\
\tilde{M} \ar[r, "\pi"'] & M
\end{tikzcd}\]
Then since $\tilde{M}$ is oriented, and an orientation along with a $\Pin^\pm$
structure determines a Spin structure, we get a Spin structure on $\tilde{M}$.
\end{proof}
%