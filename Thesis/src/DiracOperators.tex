%
\section{Dirac Operators in $\R^n$}
%
One important application of Clifford algebras and Spin groups comes from
physics. In the process of developing a relativistic equation for the electron,
Paul Dirac saw the need for a first order differential operator $D$ such
that $D$ squared to the Laplace operator.\footnote{One reason for choosing this sign
 convention for $\Delta$ is that the spectrum of $\Delta$
is positive with this choice of sign}
\[
 \Delta = -\sum_i \frac{\partial^2}{\partial(x^i)^2}
\]
If $D$ were the be first order, it would have to be written as
\[
 D = a^i \frac{\partial}{\partial x^i}
\]
for some coefficients $a^i$. However, it is clear that choosing scalar coefficients
for the $a^i$ will not suffice. For example, in $\R^2$ with the standard coordinates
$x$ and $y$, any first order operator $D = a^1\partial_x + a^2\partial_y$ satisfies
\[
 D^2 = \left(a^1\frac{\partial}{\partial x} + a^2\frac{\partial}{\partial y}\right)^2
 = (a^1)^2\frac{\partial^2}{\partial x^2} + a^1a^2\frac{\partial^2}{\partial x \partial y}
 + a^2a^1 \frac{\partial^2}{\partial y \partial x} + (a^2)^2 \frac{\partial^2}{\partial y^2}
\]
From this equation, we see that the $a^i$ must square to $-1$, and we must also
have that $a^1a^2 + a^2a^1 = 0$ in order for the mixed partial terms to vanish.
This is not possible if the $a^i$ are scalars (in either $\R$ or $\C$).
However, the required relations are exactly the relations between orthogonal
basis vectors in the Clifford algebra $\Cliff_{0,n}(\R)$!
%
\begin{defn}
 Let $\set{e_i}$ be the standard basis for $\R^n$, $\set{e^i}$
 \footnote{The choice of using the dual basis elements $e^i$ instead of the
 $e_i$ is so that $D$ behaves tensorially with respect to coordinate change.}
 its dual basis, and $x^1, \ldots ,x^n$ the standard coordinates on $\R^n$.
 The \ib{Dirac operator} on $\R^n$ is the first order differential operator
 \[
  D = e^i\frac{\partial}{\partial x^i}
 \]
\end{defn}
%
It is not clear from the definition what function space $D$ should act on.
The partial derivative operators make sense for any vector valued function, but
multiplication by $e^i$ does not make sense an arbitrary vector space -- it must
be a Clifford module. Therefore, $D$ is an operator
$D: C^\infty(\R^n, M) \to C^\infty(\R^n, M)$ on smooth functions from $\R^n$
to a Clifford module $M$. In fuller generality, functions from $\R^n \to M$
are equivalent to smooth sections of $\R^n \times M \to \R^n$, which is a bundle
of Clifford modules over $\R^n$. This viewpoint generalizes more naturally,
and the Dirac operator defined on a general Riemannian manifold is an operator
on sections of some bundle. \\

The Dirac operators in dimensions $1$, $2$, and $4$ exhibit some extremely
interesting behavior, corresponding to the appearances of $\C$ and $\H$
when $\R^n$ is given a negative definite bilinear form. In $1$ dimension,
the Clifford algebra $\Cliff_{0,1}(\R)$ admits the ordered basis $(1,e^1)$, which
then gives a basis for $C^\infty(\R, \Cliff_{0,1}(\R))$ as a
$C^\infty(\R)$ module where for any $f : \R \to \Cliff_{0,1}(\R)$, we have the
decomposition $f(x) = u(x) + e^1 v(x)$ , giving a column vector representation
\[
 f = \begin{pmatrix}
 u \\
 v
 \end{pmatrix}
\]
The Dirac operator is $e^1\partial_x$, which is represented in matrix form as
\[
 D = \begin{pmatrix}
 0 & -\partial_x \\
 \partial_x & 0
 \end{pmatrix}
\]
Note that the matrix is block off-diagonal. The vector space
$C^\infty(\R, \Cliff_{0,1}(\R))$ has a natural grading where the even elements
are maps $\R \to \Cliff_{0,1}^0(\R)$ and the odd elements are maps
$\R \to \Cliff_{0,1}^1(\R)$. Since we picked the ordered basis $(1,e^1)$
for $\Cliff_{0,1}(\R)$, whose elements are even and odd respectively, the bottom
left block of the matrix is action of $D$ as a map from the even subspace to
the odd subspace, and the top right block of the matrix is the action of $D$
as a map from the odd subspace to the even subspace.
This shows that $D$ is an odd operator, as it reverses the grading on
$C^\infty(\R, \Cliff_{0,1})$. \\

In $2$ dimensions, $\Cliff_{0,2}(\R)$ has the ordered basis $(1, e^1e^2, e_1,e_2)$,
where we choose the ordering by the parity of the elements. Again, we can
write any function $f : \R^2 \to \Cliff_{0,2}(\R)$ in terms of this basis
as \[
f(x,y) = f_0(x,y) + e^1e^2f_{12}(x,y) + e^1f_1(x,y) + e^2f_2(x,y)
\]
where the component functions are all elements of $C^\infty(\R^2)$. In this
basis, the action of the Dirac operator $D = e^1\partial_x + e^2\partial_y$
is represented by the matrix of differential operators
\[
 D = \begin{pmatrix}
 0 & 0 & -\partial_x & -\partial_y \\
 0 & 0 & -\partial_y & \partial_x \\
 \partial_x & \partial_y & 0 & 0 \\
 \partial_y & -\partial_x & 0 & 0
 \end{pmatrix}
\]
which is actually a familiar differential operator in disguise. We make the
identification of $\Cliff_{0,2}(\R)$ with the quaternions $\H$, which
have a direct sum decomposition $\H = \C \oplus \C$, giving a $\Z/2\Z$-grading.
The mappings $e_1e_2 \mapsto i$ and $e_2 \mapsto i$ determine vector space
isomorphisms of the even and odd subspaces of $\Cliff_{0,2}(\R)$ with $\C$,
giving an isomorphism of $\Cliff_{0,2}(\R)$ with $\H = \C \oplus \C$. The
operator in the top right block is then exactly twice the differential operator
\[
 -\frac{\partial}{\partial z} = -\frac{1}{2}\left( \frac{\partial}{\partial x}
 - i \frac{\partial}{\partial y}\right)
\]
on functions $\R^2 \to \C$, and the operator in the bottom block is the twice
the operator
\[
 \frac{\partial}{\partial \overline{z}} = \frac{1}{2}\left( \frac{\partial}{\partial x}
 + i \frac{\partial}{\partial y}\right)
\]
So $D$ is more compactly represented under this identification as
\[
 D = \begin{pmatrix}
 0 & 2\partial_z \\
 2\partial_{\overline{z}} & 0
 \end{pmatrix}
\]
In $4$ dimensions, the irreducible module over $\Cliff_{0,3} \cong M_2\H$ is
$\H^2$. This has a direct sum decomposition as $\H^2 = \H \oplus \H$, allowing
us to represent functions $f : \R^4 \to \H^2$ as pairs of functions $\R^4 \to \H$,
which are the maps into the even and odd subspaces of $\H^2 = \H \oplus \H$.
The quaternions admit operators similar to the operators $\partial_z$ and
$\partial_{\overline{z}}$, which are
%
\begin{align*}
 \frac{\partial}{\partial q}            & = \frac{1}{4}\left( \frac{\partial}{\partial x^1}
 - i \frac{\partial}{\partial x^2} - j\frac{\partial}{\partial x^3}
 - k \frac{\partial}{\partial x^4}\right) \\[5pt]
 \frac{\partial}{\partial \overline{q}} & = \frac{1}{4}\left( \frac{\partial}{\partial x^1}
 + i \frac{\partial}{\partial x^2} + j\frac{\partial}{\partial x^3}
 + k \frac{\partial}{\partial x^4}\right)
\end{align*}
%
Using these identifications, the Dirac operator is compactly represented as
\[
 D = \begin{pmatrix}
 0 & -4\partial_q \\
 4\partial_{\overline{q}} & 0
 \end{pmatrix}
\]
which encodes quaternionic analogues of the Cauchy-Riemann equations.
%
%
%
