%
\section{Dirac Operators in $\R^n$}
%
One important application of Clifford algebras and Spin groups comes from
physics. In the process of developing a relativistic equation for the electron,
Paul Dirac saw the need for a first order differential operator $D$ such
that $D$ squared to the Laplace operator\footnote{Our reasoning for choosing this sign
convention for $\Delta$ is twofold -- one reason is that the spectrum of $\Delta$
is positive with this choice of sign, and the second is that this definition
coincides with a generalized Laplace operator on a Riemannian manifold.}.
\[
\Delta = -\sum_i \frac{\partial^2}{\partial(x^i)^2}
\]
If $D$ were the be first order, it would have to be written as
\[
D = a^i \frac{\partial}{\partial x^i}
\]
for some coefficients $a^i$. However, it is clear that choosing scalar coefficients
for the $a^i$ will not suffice. For example, in $\R^2$ with the standard coordinates
$x$ and $y$, any first order operator $D = a^1\partial_x + a^2\partial_y$ satisfies
\[
D^2 = \left(a^1\frac{\partial}{\partial x} + a^2\frac{\partial}{\partial y}\right)^2
= (a^1)^2\frac{\partial^2}{\partial x^2} + a^1a^2\frac{\partial^2}{\partial x \partial y}
+ a^2a^1 \frac{\partial^2}{\partial y \partial x} + (a^2)^2 \frac{\partial^2}{\partial y^2}
\]
From this equation, we see that the $a^i$ must square to $-1$, and we must also
have that $a^1a^2 + a^2a^1 = 0$ in order for the mixed partial terms to vanish.
This is not possible if the $a^i$ are scalars (in either $\R$ or $\C$).
However, the required relations are exactly the relations between orthogonal
basis vectors in the Clifford algebra $\Cliff_{0,n}(\R)$!
%
\begin{defn}
Let $\set{e_i}$ be the standard basis for $\R^n$, $\set{e^i}$ its dual basis, and
$x^1, \ldots ,x^n$ the standard coordinates on $\R^n$.
The \ib{Dirac operator} on $\R^n$ is the first order differential operator
\[
D = e^i\frac{\partial}{\partial x^i}
\]
\end{defn}
%
The choice of using the dual basis elements $e^i$ instead of the $e_i$ is
so that $D$ behaves tensorially with respect to coordinate change.
It is not clear from the definition what function space $D$ should act on.
The partial derivative operators make sense for any vector valued function, but
multiplcation by $e^i$ does not make sense an arbitrary vector space -- it must
be a Clifford module. Therefore, $D$ is an operator
$D: C^\infty(\R^n, M) \to C^\infty(\R^n, M)$ on smooth functions from $\R^n$
to a Clifford module $M$. In fuller generality, functions from $\R^n \to M$
are equivalent to smooth sections of $\R^n \times M \to \R^n$, which is a bundle
of Clifford modules over $\R^n$. This viewpoint generalizes more naturally,
and the Dirac operator defined on a general Riemannian manifold is an operator
on sections of some bundle. \\

The Dirac operators in dimensions $1$, $2$, and $3$ exhibit some extremely
interesting behavior, corresponding to the appearances of $\C$ and $\H$
when $\R^n$ is given a negative definite bilinear form. In $1$ dimension,
the Clifford algebra $\Cliff_{0,1}(\R)$ admits the ordered basis $(1,e^1)$, which
then gives a basis for $C^\infty(\R, \Cliff_{0,1}(\R))$ as a
$\C^\infty(\R)$ module where for any $f : \R \to \Cliff_{0,1}(\R)$, we have the
decomposition $f(x) = u(x) + e^1 v(x)$ , giving a column vector representation
\[
f = \begin{pmatrix}
u \\
v
\end{pmatrix}
\]
The Dirac operator is $e^1\partial_x$, which is represented in matrix form as
\[
D = \begin{pmatrix}
0 & -\partial_x \\
\partial_x & 0
\end{pmatrix}
\]
Note that the matrix is block off-diagonal. Since we picked the ordered basis $(1,e^1)$.
for $\Cliff_{0,1}(\R)$, whose elements are even and odd respectively, this
shows that $D$ is an odd operator, as it reverses the grading. \\

In $2$ dimensions, $\Cliff_{0,2}(\R)$ has the ordered basis $(1, e^1e^2, e_1,e_2)$,
where we choose the ordering by the parity of the elements. Again, we can
write any function $f : \R^2 \to \Cliff_{0,2}(\R)$ in terms of this basis
as \[
f(x,y) = f_0(x,y) + e^1e^2f_{12}(x,y) + e^1f_1(x,y) + e^2f_2(x,y)
\]
where the component functions are all elements of $C^\infty(\R^2)$. In this
basis, the action of the dirac operator $D = e^1]\partial_x + e^2\partial_y$
is repreented by the matrix of differential operators
\[
D = \begin{pmatrix}
0 & 0 & -\partial_x & -\partial_y \\
0 & 0 & -\partial_y & \partial_x \\
\partial_x & \partial_y & 0 & 0 \\
\partial_y & -\partial_x & 0 & 0
\end{pmatrix}
\]
%
%TODO More stuff later, maybe in another section?
%