\chapter{Spin Structures on Manifolds}
%
\subsectionend $ $\\
%
\iffalse
\emph{Not all the geometrical structures are``equal". It would seem that the
Riemannian and complex structures, with their contacts with other fields of
mathematics and with their richness in results, should occupy a central position
in differential geometry. A unifying idea is the notion of a $G$-structure, which
is the modern version of an equivalence problem first emphasized and exploited
in its various special cases by \'Elie Cartan.} \\
%
\ib{-- ~ Shiing-Shen Chern}
%
\subsectionend
\fi
%
\section{Fiber Bundles}
%
\begin{defn}
Let $M$ and $F$ be smooth manifolds. Then a \ib{fiber bundle} over $M$ with
model fiber $F$ is a the data of a smooth manifold $E$ with a smooth map
$\pi : E \to M$ such that for every point $p \in M$, there is a neighborhood
$U \subset M$ containing $p$ such that there exists an diffeomorphism
$\varphi : \pi\inv(U) \to U \times F$ such that the diagram
\[\begin{tikzcd}
\pi\inv(U) \ar[rr, "\varphi"] \ar[dr, "\pi"']&& U \times F \ar[dl, "p_U"] \\
& U
\end{tikzcd}\]
commutes, where $P_U$ denotes projection onto the first factor. The map $\varphi$ is
called a \ib{local trivialization} of the fiber bundle $\pi : E \to M$.
\end{defn}
%
Given a fiber bundle $\pi : E \to M$ we often denote the fiber $\pi\inv(p)$
by $E_p$.
%
\begin{defn}
let $\pi : E \to M$ and $p : L \to M$ be fiber bundles with model fiber $F$
over $M$. A \ib{bundle homomorphism} is a smooth map $\varphi : E \to L$ such
that the following diagram commutes
\[\begin{tikzcd}
E \ar[dr, "\pi"']\ar[rr,"\varphi"] & & L \ar[dl, "p"] \\
& M
\end{tikzcd}\]
\end{defn}
%
An important property of fiber bundles is that they pull back.
\begin{defn}
Let $\pi : E \to M$ be a fiber bundle with model fiber $F$
and let $f : X \to M$ be a smooth map. Then the \ib{pullback} of $E$ by $f$ is
the data of a smooth manifold
\[
f^*E = \set{(x, p)~:~ x \in X, p \in \pi\inv(f(x)) }
\]
along with the projection $p : f^*E \to M$ given by $(x,p) \mapsto x$, giving
$f^*E \to X$ the structure of a fiber bundle over $X$ with model fiber $F$.
The bundle $f^*E$ also comes equipped with a natural map $\alpha : f^*E \to E$
where $\alpha(x,p) = p$. Pullbacks give rise to the diagram
\[\begin{tikzcd}
f^*E \ar[d, "p"']\ar[r, "\alpha"] & E \ar[d, "\pi"] \\
X \ar[r, "f"']& M
\end{tikzcd}\]
and are an instance of a general construction called the \ib{fibered product}.
\end{defn}
%
\begin{defn}
Let $\pi : E \to M$ be a fiber bundle. A \ib{local section} of $\pi : E \to M$
is a smooth map $\sigma : U \to E$ of an open set $U \subset M$ such that
$\pi \circ \sigma = \id_U$. If $U = M$, the section is called a \ib{global section}.
Equivalently, it is the smooth assignment of an element in $E_p$ to each point
$p \in U$. We denote the set of sections of $\pi : E \to M$ over a set $U$
as $\Gamma_U(E)$.
\end{defn}
%
A fiber bundle is a very general construction in which the fibers $F$ do
not necessarily admit extra structure. An important special case of a fiber bundle
is a vector bundle, where the fibers are vector spaces.
%
\begin{defn}
Let $M$ be a smooth manifold. A \ib{vector bundle} over $M$ is fiber bundle
$\pi : E \to M$ with model fiber $\R^n$ (or $\C^n)$ such that the local
trivializations $\varphi : \pi\inv(U) \to U \times \R^n$ (or $\C^n$)
restrict to linear isomorphisms on the fibers, i.e. for all $p \in U$, the restriction
$\varphi\vert_{\pi\inv(p)} : \pi\inv(p) \to \set{p} \times \R^n$
(or $\C^n$) is an isomorphism. The dimension $n$ of the fibers is called the
\ib{rank} of the vector bundle. A \ib{vector bundle homomorphism}
$\varphi : E \to L$ is a bundle homomorphism with the added stipulation that
the restrictions to the fibers $\varphi\vert_{E_x}$ are linear maps.
\end{defn}
%
\begin{exmp} \enumbreak
\begin{enumerate}
  \item Given a smooth manifold $M$, the tangent bundle $TM = \coprod_{p \in M} T_pM$
  is a vector bundle, where the rank is the dimension of $M$.
  \item The \ib{tautological bundle} over $\RP^n$ is the vector bundle that
  assigns to each subspace $\ell \in \RP^n$ itself as its fiber. An analgous
  construction defines the tautological bundle over the Grassmannian
  $\mathrm{Gr}_k(\R^n)$.
\end{enumerate}
\end{exmp}
%
\begin{defn}
A \ib{Lie group} is a smooth manifold $G$ with a group structure such that the
multiplication map $(g,h) \mapsto gh$ and the inversion map $g \mapsto g\inv$
are smooth.
\end{defn}
%
\begin{exmp} \enumbreak
\begin{enumerate}
  \item The group $GL_n\R$ of invertible $n \times n$ matrices is an open
  subset of $M_n\R$, and therefore a smooth $n^2$ dimensional manifold.
  \item The group $SL_n\R$ of $n\times n$ matrices with determinant $1$ is a
  closed submanifold of $GL_\R$.
  \item The orthogonal groups $O_n$ and special orthogonal groups $SO_n$ are Lie groups.
  \item The unitary groups $U_n$ and special unitary groups $SU_n$ are Lie groups.
\end{enumerate}
\end{exmp}
%
Another important class of fiber bundles are principal bundles, in which
the fibers have the structure of $G$-torsors.
%
\begin{defn}
Let $G$ be a Lie group, and $M$ a smooth manifold. A \ib{Principal } $G$\ib{-bundle}
over $M$ is the data of
%
\begin{enumerate}
  \item A smooth manifold $P$ with a map $\pi : P \to M$.
  \item A smooth right $G$-action on $P$ that is free and transitive on the
  fibers of $\pi$.
  \item For every point $p \in M$, a neighborhood $U \subset M$ containing $p$ and
  a $G$-equivariant diffeomorphism $\varphi: \pi\inv(U) \to U \times G$ (where
  the right action on $U \times G$ is right multiplication on the second factor)
  such that we get the commutative diagram
  \[\begin{tikzcd}
  \pi\inv(U) \ar[dr, "\pi"']\ar[rr, "\varphi"]&& U \times G \ar[dl, "p_U"]\\
  & U
  \end{tikzcd}\]
  where $p_U$ denotes projection onto the first factor.
\end{enumerate}
A \ib{principal bundle homomorphism} is a bundle homomorphism $\varphi : P \to Q$
that is $G$-equivariant.
\end{defn}
%
\begin{exmp}
Given a smooth manifold $M$ and a point $p \in M$, a basis of the
tangent space is a linear isomorphism $b : \R^n \to T_pM$. The group
$GL_n\R$ acts freely and transitively on the set of bases $\mathcal{B}_p$ on the right by
$b \cdot g = b \circ g$. Then the \ib{frame bundle} of $M$, denoted
$\mathcal{B}(M)$ is the disjoint union
\[
\mathcal{B}(M) = \coprod_{p \in M}\mathcal{B}_p
\]
where $\pi$ is the projection map $(p,b) \mapsto p$. Then $\mathcal{B}(M)$
is a principal $GL_n\R$ bundle over $M$.
\end{exmp}
%
\begin{exmp}
Given a smooth manifold $M$, a Riemannian metric $g$ induces an inner product
$g_p$ on each tangent space $T_pM$, where $g_p$ denotes the metric $g$ evaluated
at $p$. Then the set of orthonormal bases of $T_pM$ is the
set of all linear isometries $(\R^n, \langle\cdot,\cdot\rangle) \to (M, g_p)$
where $\langle\cdot,\cdot\rangle$ is the standard inner product on $\R^n$.
Then taking the disjoint union over all points $p\in M$ of orthonormal bases
for the tangent spaces $T_pM$ forms the \ib{orthonormal frame bundle}
$\mathcal{B}_O(M)$, which is a principal $O_n$ bundle.
\end{exmp}
%
\begin{defn}
Let $\pi : P \to M$ be a principal $G$-bundle over $M$, and let $F$ be a
smooth manifold with a smooth left $G$ action. Then the \ib{associated fiber
bundle}, denoted $P \times_G F$, is the set
\[
P \times_G F = P \times G / (p,g) \sim (p\cdot h, h\inv g)
\]
Since the group action on $P$ preserves the fibers, the projection
$p_1 : P \times F \to P$ composed with the projection $\pi : P \to M$
descends to the quotient, givng us a projection map
$\Phi : P \times_G F \to M$.
\end{defn}
%
The first thing to check is that $P \times_G F$ is a fiber bundle, justifying
the name.
%
\begin{prop}
Let $\pi : P \to M$ be a principal $G$-bundle and $F$ a manifold with a left $G$ action.
then the associated bundle $ \Phi : P \times_G F \to M$ is a fiber bundle with
model fiber $F$.
\end{prop}
%
\begin{proof}
We wish to provide local trivializations  $\Phi\inv(U) \to U \times F$ for the
associated bundle. Fix a local trivialization  $\psi : \pi\inv(U) \to U \times G$.
Then $\psi$ is of the form $\psi(p) = (\pi(p), \tilde{\psi}(p))$ for some
$\tilde{\psi} : \pi\inv(U) \to G$, satisfying $\tilde{\psi}(p\cdot g) = \tilde{\psi}(p)g$.
Define
\begin{align*}
\varphi : \Phi\inv(U) &\to U \times F \\
[p,f] &\mapsto (\pi(p), \tilde{\psi}(p)\cdot f)
\end{align*}
We first note that this is well defined on equivalence classes, since
\[
[p\cdot g, g\inv \cdot f] \mapsto (\pi(p \cdot g), \tilde{\psi}(p \cdot g)
\cdot g\inv \cdot f)
= (\pi(p), \psi(p)\cdot f)
\]
So $\varphi$ is well defined. We also note that $\Phi(\varphi[p,f]) = \pi(p)$
by how $\Phi$ was defined, so $\varphi$ is a local trivialization provided
it is a homeomorphism. To show this, we construct an inverse.
Define $\alpha : U \times F \to \Phi\inv(U)$
by $\alpha(u,f) = [\psi\inv(u,e), f]$, where $e$ denotes the identity element
of $G$. We then claim that $\alpha$ is the inverse. We compute
%
\begin{align*}
(\varphi \circ \alpha)(u,f) &= \varphi[\psi\inv(u,e),f] \\
&= (u, e\cdot f)\\
&= (u,f)
\end{align*}
%
In the other direction, we compute
%
\begin{align*}
(\alpha \circ \varphi)[p,f] &= \alpha(\pi(p), \tilde{\psi}(p)) \\
&= [\psi\inv(\pi(p), e), f] \\
&= [p, f]
\end{align*}
%
Therefore, $\varphi$ is a local trivialization, giving us that
$\Phi: P \times_G F \to M$ is a fiber bundle with model fiber $F$.
%
\end{proof}
%
There is a correspondence between sections of an associated bundle and
$G$-equivariant maps $P \to F$.
%
\begin{prop}
Let $\pi : P \to M$ be a principal $G$-bundle. Then there is a bijection
\[
\set{G\text{-equivariant maps } P \to F} \longleftrightarrow \Gamma_M(P \times_G F)
\]
\end{prop}
%
\begin{proof}
Since $F$ has a left $G$-action, we first convert this to a right $G$-action
by $f \cdot g = g\inv \cdot f$. Then what we mean by a $G$-equivariant map is
a map $\varphi : P \to F$ such that
\[
\varphi(p\cdot g) = g\inv\cdot \varphi(p)
\]
We then wish to use $G$-equivariant map $\varphi$ to produce a section
$\tilde{\varphi} : M \to P \times_G F$ of the associated bundle. For a
point $x \in M$, pick any $p \in \pi\inv(x)$ in the fiber. Then define
\[
\tilde{\varphi}(x) = [p, \varphi(p)]
\]
where $[p, \varphi(p)]$ denotes the equivalence class of $(p, \varphi(p))$ in
$P \times_G F$. We first claim that this map is well defined, i.e. it is
independent of our choice of point $p \in \pi\inv(x)$. We know that $G$ acts
freely and transitively on $\pi\inv(x)$, so all points in the fiber are of
the form $p \cdot g$ for a unique $g \in G$. Then we have that for any $g \in G$
\[
(p\cdot g, \varphi(p \cdot g)) = (p \cdot g, g\inv \cdot \varphi(p)) \sim (p, \varphi(p))
\]
where we use the $G$-equivariance of $\varphi$ and the definition of the equivalence
relation on the associated bundle. In addition, this induced map is
a section of $P \times_G F \to M$, since the image is represented by an
element of $P \times F$ with an element of the fiber $\pi\inv(x)$ in the first
factor. \\

Conversely, given a section $\sigma : M \to P \times_G F$, we wish to produce
a $G$-equivariant map $P \to F$. Given such a section, and a point $x \in M$,
we have that $\sigma(x) = [p, f]$ for some $p \in P$ and $f \in F$. Then
define $\tilde{\sigma} : P \to F$ such that $\tilde{\sigma}(p) = f$, and
$\tilde{\sigma}(p \cdot g) = g\inv \cdot f$. Since $G$ acts freely and transitively
on the fibers, this is well defined, and specifies the map on every
point of $P$. In addition, $\tilde{\sigma}$ is $G$-equivariant by construction.
Then the maps
\[
\set{G\text{-equivariant maps } P \to F} \longleftrightarrow \Gamma_M(P \times_G F)
\]
we provided are easily verified to be inverses to each other, giving us the
correspondence.
\end{proof}
%
In some sense, the geometry of the associated fiber bundle $P \times_G F$
is controlled by the group $G$, as $G$ determines a disinguished
group of symmetries of the fiber $F$.
%
\begin{exmp}[The tangent bundle as an associated bundle]
Given a manifold $M$, we can take the frame bundle $\pi : \B(M) \to M$, which is a
principal $GL_n\R$ bundle. The group $GL_n\R$ acts linearly on $\R^n$ in the
standard way, giving us an associated vector bundle $\B(M) \times_{GL_n\R} \R^n$.
We claim that this bundle is isomorphic to the tangent bundle $TM$, i.e. there
exists an diffeomorphism $\varphi : \B(M) \times_{GL_n\R} \R^n \to TM$ that
restricts to linear isomorphisms on the fibers and the diagram
\[\begin{tikzcd}
\B(M) \times_{GL_n\R} \R^n \ar[rr, "\varphi"] \ar[dr] && TM \ar[dl] \\
& M
\end{tikzcd}\]
commutes, where the maps to $M$ are the bundle projections. Recall that
elements of $\B(M) \times_{GL_n\R} R^n$ are represented by pairs $(b, v)$,
where $b: \R^n \to T_{\pi(b)}M$ is a linear isomorphism, and $v$ is a vector in $\R^n$.
Then define $\varphi$ by
\[
\varphi[b,v] = (\pi(b), b(v))
\]
This is well defined, since $\varphi[b\circ g, g\inv(v)] = (\pi(b \circ g),
(b\circ g)(g\inv(v))) = (\pi(b), b(v))$. This is an isomorphism, where the
inverse mapping maps $(p,v) \in TM$ to $(b, \tilde{v})$ where $b$ is any
basis of $T_pM$ and $\tilde{v}$ is the coordinate representation of $v$ in the
basis $b$. In this example, we see that the associated bundle identifies the same vector
under different coordinate transformations, which defines the symmetries of
$TM$.
\end{exmp}
%
In general, given a rank $n$ vector bundle $E \to M$, we can construct the
frame bundle  $\B(E)$ for $E$ and recover $E$ by taking the associated bundle
$\B(E) \times_{GL_n\R} \R^n$, so the process of taking frames and constructing
associated bundles are inverses.
%
\begin{defn}
Let $\pi : P \to M$ be a principal $G$-bundle, and $\rho : H \to G$ a group
homomorphism. The map $\rho$ gives $P$ a left $H$ action where
$h \cdot p = \rho(h) \cdot p$. A \ib{reduction of structure group} is the
data of a principal $H$ bundle $\varphi : Q \to M$ and an $H$-equivariant bundle
homomorphism $F : Q \to P$.
\end{defn}
%
The map $F : Q \to P$ induces a map $\tilde{F} : Q \times_H G \to P$, where we
map the equivalence class $[q,g]$ to $F(q) g$. This is well defined
on equivalences classes since
%TODO this next line might be unnecesary, since it is the same map
%constructed using the correspondence with sections -- double check this.
\[
(q \cdot h, \rho(h)\inv g) \mapsto F(q \cdot h)\rho(h)\inv g = F(q)\rho(h)\rho(h)\inv g
= F(q)g
\]
%
\begin{exmp}[Reduction from $GL_n\R$ to $O_n$]
Let $M$ be a smooth manifold, and $\pi : \B(M) \to M$ its bundle of frames.
The inclusion map $\iota: O_n \hookrightarrow GL_n\R$ gives an action of $O_n$ on
$\B(M)$, where given $b \in \B(M)$ and $T \in O_n$, $b \cdot T = b \cdot \iota(T)$.
We then take the quotient by this $O_n$ action, giving us a quotient map
$q : \B(M) / O_n$. Since the inclusion is injective, the $O_n$ action
is free on $\B(M)$, so this gives $q : \B(M) \to \B(M) / O_n$ the structure of a
$O_n$ bundle. In addition, the action of $O_n$ preserves the fibers of
$\pi : \B(M) \to M$, so $\pi$ descends to the quotient, so $\B(M) / O_n \to M$
is a fiber bundle with model fiber $GL_n\R / O_n$. Since $GL_n\R$ deformation
retracts onto $O_n$ via the Gram-Schmidit algorithm, $GL_n\R / O_n$ is
contractible, so the fiber bundle $\B(M)/ O_n \to M$ admits global sections.
Then given a section $\sigma : M \to \B(M) / O_n$, this gives an $O_n$ bundle
over $M$ via the pullback $\sigma^*\B(M)$. In addition, we get a $O_n$-equivariant
map $\sigma^*\B(M) \to \B(M)$ given by $(p, b) \mapsto (p, \iota(b))$. The
bundle $\sigma^*\B(M)$ can be thought of as the bundle of orthonormal frames
with respect to some Riemannian metric on $M$, and the fact that $\B(M) / O_n$
admits sections corresponds to the fact that every manifold admits a Riemannian
metric.
\end{exmp}
%
