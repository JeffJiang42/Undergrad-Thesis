%
\section{Clifford Modules}
%
\begin{defn}
A (left) \ib{Clifford module} for the Clifford algebra $\Cliff_{p,q}(\R)$ is a module
for $\Cliff_{p,q}(\R)$ in the usual sense i.e. a real vector space $V$ equipped
with an algebra action $\bullet: \Cliff_{p,q}(\R) \times V \to V$ satisfying
the following properties
\begin{enumerate}
  \item Every element of $\Cliff_{p,q}(\R)$ acts linearly on $V$.
  \item $(AB) \cdot v = A\cdot(B \cdot v)$ for all $v \in V$.
  \item $(A + B) \cdot v = A\cdot v + B\cdot V$ for all $v \in V$.
\end{enumerate}
Equivalently, it is the data of a real vector space $V$ and an algebra homomorphism
$\Cliff_{p,q}(\R) \to \End(V)$.
\end{defn}
%
\begin{defn}
A Clifford module is \ib{irreducible} if there exist no proper nontrivial submodules.
\end{defn}
%
From the classification of Clifford algebras, all the Clifford algebras are either
matrix algebras $M_n\F$, where $\F = \R$, $\C$, or $\H$, or products
$M_n\F \times M_n\F$ of two copies of the same matrix algebra. This is sufficient
to conclude that Clifford algebras are semisimple, so all Clifford modules
are direct sums of irreducible modules. Therefore, classifying all Clifford
modules reduces to classifying the irreducible Clifford modules.
%
\begin{thm}
Let $\F = \R$, $\C$, or $\H$. Then any nontrivial irreducible module for
$M_n\F$ is isomorphic to $\F^n$ with the standard action.
\end{thm}
%
\begin{proof}
We first note that $M_n\F$ acts transitively on $\F^n$, which implies that
it is irreducible. We then must show that $\F^n$ is, up to isomorphism, the only
irreducible $M_n\F$ module. The matrix algebra $M_n\F$ admits an increasing
chain of left ideals
\[
0 = I_0 \subset I_1 \subset \ldots \subset I_n = M_n\F
\]
where $I_k$ is the set of matrices where only the first $k$ columns are nonzero.
These ideals have the property that the quotient $I_k / I_{k-1}$ is isomorphic
to $\F^n$ as a left $M_n\F$ module. Then let $M$ be some nontrivial irreducible
$M_n\F$ module, and fix $m \in M$. Then the orbit $M_n\F \cdot m$ of $m$
under the algebra action is a nonzero submodule, so it must be all of $M$.
Then the map $\varphi : M_n\F \to M$ given by $A \mapsto A \cdot m$ is
a surjective map of left $M_n\F$ modules. Then there must exist some smallest
$k$ such that $\varphi(I_k)$ is nonzero, and by construction,
$\varphi\vert_{I_k}$ factors through the quotient $I_k / I_{k-1}$, which
is isomorphic to $\F^n$ with the standard action. Then since $\F^n$ is irreducible,
this gives us a nonzero map between irreducible modules, which is an isomorphism
by Schur's Lemma.
\end{proof}
%
\begin{thm}
Any nontrivial irreducible module for $M_n\F \times M_n\F$ is isomorphic to
either $\F^n$ where the left factor acts in the usual way, and the right factor
acts by $0$, or $\F^n$ where the left factor acts by $0$ and the right factor
acts in the usual way.
\end{thm}
%
\begin{proof}
Let $R$ denote $\F^n$ where the right factor acts nontrivially, and let $L$
denote $\F^n$ where the left factor acts nontrivially. Both $L$ and $R$ are
irreducible since $M_n\F \times M_n\F$ acts transitively on them. To show
that they are the only irreducible modules up to isomorphism, we use a similar
technique as above. Let $I_k$ denote the chain of increasing ideals in $M_n\F$,
as we used above. Then $M_n\F \times M_n\F$ admits a chain of increasing left
ideals $J_k$
\[
0 = J_0 \subset I_1 \times \set{0} \subset \ldots \subset M_n\F \times \set{0}
\subset M_n\F \times I_1 \subset \ldots \subset M_n\F \times M_n\F = J_{2n}
\]
We note that for $1 \leq k \leq n$, we have that $J_k / J_{k-1}$ is isomorphic to
$L$, and for $n+1 \leq k \leq 2n$, we have that $J_k / J_{k-1}$ is isomorphic to
$R$. Then given a nontrivial irreducible module $M$ and a nonzero element $m$,
we get a surjective map $\varphi : M_n\F \times M_n\F$ where $A \mapsto A\cdot m$.
Like before, there exists some smallest $k$ such that $\varphi(J_k)$ is nonzero,
which then factors through to an isomorphism $J_k / J_{k-1} \to M$, so $M$
is either isomorphic to $R$ or $L$.
\end{proof}
%
This then gives a full classification of the irreducible ungraded Clifford
modules.\\

Since the Clifford algebras are $\Z/2\Z$ graded, we have an analogous notion
of a graded Clifford module.
%
\begin{defn}
Let $A = A^0 \oplus A^1$ be a $\Z/2\Z$ graded algebra. Then a $\Z/2\Z$
\ib{graded module} for $A$ is a module $M = M^0 \oplus M^1$ for $A$ such that
$A^i \cdot M^j \subset M^{i + j \mod 2}$.
\end{defn}
%
A \ib{graded Clifford module} is then a $\Z/2\Z$ graded module for a Clifford
algebra $\Cliff_{p,q}(\R)$. We already have all the ingredients we need
to classify all of the graded Clifford modules : we have a complete
classification of the ungraded Clifford modules, and we have the relationship
between $\Cliff_{p,q}(\R)$ and its even subalgebra $\Cliff_{p,q}^0(\R)$.
%
\begin{thm}
There is a bijective correspondence
\[
\set{\text{Ungraded Clifford modules over } \Cliff_{p,q}^0(\R)} \longleftrightarrow
\set{\text{Graded Clifford modules over} \Cliff_{p,q}(\R)}
\]
\end{thm}
%
\begin{proof}
We provide maps in both directions. Let $M = M^0 \oplus M^1$ be a graded
module for $\Cliff_{p,q}(\R)$. The action of $\Cliff_{p,q}^0(\R)$ preserves
the even subspace $M^0$, which gives $M^0$ the structure of an ungraded
$\Cliff_{p,q}^0(\R)$ module. Let $r$ denote the mapping $M \mapsto M^0$. \\

In the other direction, let $V$ be an ungraded module over $\Cliff_{p,q}^0(\R)$.
Then let $t$ denote the mapping
$V \mapsto V' = \Cliff_{p,q}(\R) \otimes_{\Cliff_{p,q}^0(\R)} V$.
Where we treat $\Cliff_{p,q}(\R)$ as a left module under multiplication by
elements of $\Cliff_{p,q}^0(\R)$. We then endow $V'$ the structure of a
$\Cliff_{p,q}(\R)$ module, where $\Cliff_{p,q}(\R)$ acts on the $\Cliff_{p,q}(\R)$
factor by left multiplication. We further impose the grading on $V'$ where
\[
V' = ( \Cliff^0_{p,q}(\R) \otimes_{\Cliff^0_{p,q}(\R)} V) \oplus
(\Cliff^1_{p,q}(\R) \otimes_{\Cliff^1_{p,q}(\R)} V)
\]
We then claim that these two maps are inverses, which we will verify by showing
that the compositions $r \circ t$ and $t \circ r$ are identity. We first show
that $r \circ t$ is identity. Let $V$ be an ungraded module over $\Cliff_{p,q}^0(\R)$.
Then the even subalgebra of $t(V) = \Cliff_{p,q} \otimes_{\Cliff_{p,q}^0(\R)} V$
is $\Cliff^0_{p,q}(\R) \otimes_{\Cliff^0_{p,q}(\R)} V$, which is naturally isomorphic
to $V$ itself. In the other direction, let $M = M^0 \otimes M^1$ be a module for
$\Cliff_{p,q}(\R)$. Showing that $t \circ r$ is the identity then amounts to showing
that $\Cliff_{p,q}(\R) \otimes_{\Cliff_{p,q}^0(\R)} M^0$ is isomorphic to $M$.
The desired isomorphism is explicitly determined by the mapping
$a \otimes m \mapsto a \cdot m$.
\end{proof}
%
From this theorem, understanding graded Clifford modules is equivalent to understanding
the ungraded modules, since the even subalgebra of a Clifford algebra is isomorphic
as ungraded algebras with a smaller Clifford algebra. Putting everything together
gives us the complete classification of graded Clifford modules.\\

Given a ring homomorphism $\varphi : A \to B$, this induces a pullback map
${}_A\mathsf{Mod} \to {}_B\mathsf{Mod}$ of left modules. Given a $B$ module $M$,
we get an $A$ module via $\varphi$ by defining the ring action on the underlying
abelian group of $M$ to be $a \cdot m = \varphi(a)m$, where the right hand side
is the ring action from $B$. This pullback map reveals a beautiful periodicity
among the Clifford algebras and their modules, which is one of the many forms
of \ib{Bott periodicity}. The Clifford modules over a fixed Clifford algebra
$\Cliff_{p,q}(\R)$ form a commutative monoid under the direct sum, which
we denote $\M_{p,q}$. The inclusion $\R^{p,q} \hookrightarrow \R^{p,q+1}$ induces
an inclusion $\Cliff_{p,q}(\R) \hookrightarrow \Cliff_{p,q+1}(\R)$, which then
induces a monoid homomorphism $\M_{p,q+1} \to \M_{p,q}$. Likewise, the inclusion
$\R^{p,q}\hookrightarrow \R^{p+1,q}$ induces a monoid homomorphism
$\M_{p+1,q} \to \M_{q,p+1}$. Since these monoid maps are induced by
inclusions, they represent the restriction of Clifford modules. We can then
compute the cokernels of these restrictions, which computes the degree to
which Clifford modules of $\Cliff_{p,q}(\R)$ fail to extend to Clifford modules
over $\Cliff_{p+1,q}(\R)$ and $\Cliff_{p,q+1}(\R)$. To compute these cokernels,
it suffices to compute it for the table, since tensoring with $M_{16}\R$ will
result in the pattern in the chessboard to repeat. From the Clifford chessboard,
we observe that the inclusions of $\Cliff_{p,q}(\R)$ into $\Cliff_{p+1,q}(\R)$
and $\Cliff_{p,q+1}$ all fall into one of the following four cases.
%
\begin{enumerate}
  \item $M_n\F \hookrightarrow M_n\F'$ where $\F = \R$ or $\C$, and $\F' = \C$
  or $\H$ is the division algebra of twice the dimension of $\F$ as a
  vector space over $\R$.
  \item $M_n\F \times M_n\F \hookrightarrow M_{2n}\F$.
  \item $M_n\F' \hookrightarrow M_{2n}\F$, where $\F$ and $\F'$ are defined as above.
  \item $M_n\F \hookrightarrow M_n\F \times M_n\F$.
\end{enumerate}
%
Using semisimplicity of the Clifford algebras, along with out classification of
the ungraded Clifford modules, we then compute the cokernels for all of these
cases.
%
\begin{enumerate}
  \item In this case, the irreducible module for $M_n\F'$ is $(\F')^n$, which is
  twice the dimension of $\F^n$ as an $\R$ vector space. Then since both
  $M_n\F$ and $M_n\F'$ admit only a single irreducible module, the monoid homomorphism
  is a map $\Z^{\geq 0} \to \Z^{\geq 0}$ where $1 \mapsto 2$. The cokernel is then
  the group $\Z/2\Z$.
  \item The algebra $M_n\F \times M_n\F$ injects into $M_{2n}\F$ as block matrices
  of the form
  \[
  \begin{pmatrix}
  A & 0 \\
  0 & B
  \end{pmatrix}
  \]
  with $A,B \in M_n\F$. From this, we see that the irreducible module $\F^{2n}$
  decomposes into a direct sum of the two irreducible modules for $M_n\F \times M_n\F$.
  The monoid homomorphism $\Z^{\geq 0} \to \Z^{\geq 0} \times \Z^{\geq 0}$ is
  then given by $1 \mapsto (1,1)$, and the cokernel is the group $\Z$.
  \item In this case, the irreducible modules $(\F')^n$ and $\F^{2n}$ are the
  same dimension as vector spaces over $\R$, so the monoid homomorphism is given
  by $1\mapsto 1$, so the cokernel is the trivial group $0$.
  \item In this case, both irreducible modules for the product $M_n\F \times M_n\F$
  are the same dimension as the irreducible module for $M_n\F$, so the monoid homomorphism
  $\Z^{\geq 0 }\times \Z^{\geq 0} \to \Z^{\geq 0}$ is given by $(1,0) \mapsto 1$
  and $(0,1) \mapsto 1$, so the cokernel is again the trivial group $0$.
\end{enumerate}
%
With these cokernels in hand, we can fill in the table \\\\
%
\resizebox{.72\width}{!}{
\centering
\begin{tabular}{c |c|c|c|c|c|c|c|c|}
\hline
7 & \slashbox{$0$}{$\Z/2\Z$} & \slashbox{$0$}{$0$} & \slashbox{$\Z$}{$\Z$} &
\slashbox{$0$}{$0$} & \slashbox{$\Z/2\Z$}{$0$} & \slashbox{$\Z/2\Z$}{$0$} &
\slashbox{$\Z$}{$\Z$} & \slashbox{$0$}{$\Z/2\Z$}\\
\hline
6 & \slashbox{$0$}{$0$} & \slashbox{$\Z$}{$\Z$} & \slashbox{$0$}{$0$} &
\slashbox{$\Z/2\Z$}{$0$} & \slashbox{$\Z/2\Z$}{$0$} & \slashbox{$\Z$}{$\Z$} &
\slashbox{$0$}{$\Z/2\Z$} & \slashbox{$0$}{$\Z/2\Z$}\\
\hline
5 &  \slashbox{$\Z$}{$\Z$} & \slashbox{$0$}{$0$} & \slashbox{$\Z/2\Z$}{$0$} &
\slashbox{$\Z/2\Z$}{$0$} & \slashbox{$\Z$}{$\Z$} & \slashbox{$0$}{$\Z/2\Z$} &
\slashbox{$0$}{$\Z/2\Z$} & \slashbox{$0$}{$0$}\\
\hline
4 & \slashbox{$0$}{$0$} & \slashbox{$\Z/2\Z$}{$0$} & \slashbox{$\Z/2\Z$}{$0$} &
\slashbox{$\Z$}{$\Z$} & \slashbox{$0$}{$\Z/2\Z$} & \slashbox{$0$}{$\Z/2\Z$} &
\slashbox{$0$}{$0$} & \slashbox{$\Z$}{$\Z$}\\
\hline
3 &  \slashbox{$\Z/2\Z$}{$0$} & \slashbox{$\Z/2\Z$}{$0$} & \slashbox{$\Z$}{$\Z$} &
\slashbox{$0$}{$\Z/2\Z$} & \slashbox{$0$}{$\Z/2\Z$} & \slashbox{$0$}{$0$} &
\slashbox{$\Z$}{$\Z$} & \slashbox{$0$}{$0$}\\
\hline
2 & \slashbox{$\Z/2\Z$}{$0$} & \slashbox{$\Z$}{$\Z$} & \slashbox{$0$}{$\Z/2\Z$} &
\slashbox{$0$}{$\Z/2\Z$} & \slashbox{$0$}{$0$} & \slashbox{$\Z$}{$\Z$} &
\slashbox{$0$}{$0$} & \slashbox{$\Z/2\Z$}{$0$}\\
\hline
1 & \slashbox{$\Z$}{$\Z$} & \slashbox{$0$}{$\Z/2\Z$} & \slashbox{$0$}{$\Z/2\Z$} &
\slashbox{$0$}{$0$} & \slashbox{$\Z$}{$\Z$} & \slashbox{$0$}{$0$} &
\slashbox{$\Z/2\Z$}{$0$} & \slashbox{$\Z/2\Z$}{$0$}\\
\hline
0 & \slashbox{$0$}{$\Z/2\Z$} & \slashbox{$0$}{$\Z/2\Z$} & \slashbox{0}{0} &
\slashbox{$\Z$}{$\Z$} & \slashbox{$0$}{$0$} & \slashbox{$\Z/2\Z$}{$0$} &
\slashbox{$\Z/2\Z$}{$0$} & \slashbox{$\Z$}{$\Z$}\\
\hline
\slashbox{p}{q} & 0 & 1 & 2 & 3 & 4 & 5 & 6 & 7
\end{tabular}
} \\\\\\
where the top left corner of each box is the cokernel for the monoid homomorphism
induced by the inclusion $\Cliff_{p,q}(\R) \hookrightarrow \Cliff_{p+1,q}(\R)$,
and the bottom right  corner is the cokernel for the monoid homomorphism induced
by the inclusion $\Cliff_{p,q}(\R) \hookrightarrow \Cliff_{p,q+1}(\R)$. There
is an analogous table for the graded modules, where everything is shifted right
one square. The periodicity patterns in this table are one of the many
forms of \ib{Bott periodicity}, which has wide ranging implications in homotopy
theory and $K$-theory. The sequence
\[
\Z/2\Z \qquad \Z/2\Z \qquad 0 \qquad \Z \qquad 0 \qquad 0 \qquad 0 \qquad \Z
\]
is affectionately called the \ib{Bott song}.
%