%
\section{Clifford Modules}
%
\begin{defn}
A (left) \ib{Clifford module} for the Clifford algebra $\Cliff_{p,q}(\R)$ is a module
for $\Cliff_{p,q}(\R)$ in the usual sense i.e. a real vector space $V$ equipped
with an algebra action $\bullet: \Cliff_{p,q} \times V \to V$ satisfying
\begin{enumerate}
  \item Every element of $\Cliff_{p,q}(\R)$ acts linearly on $V$.
  \item $(AB) \cdot v = A\cdot(B \cdot v)$ for all $v \in V$.
  \item $(A + B) \cdot v = A\cdot v + B\cdot V$ for all $v \in V$.
\end{enumerate}
Equivalently, it is the data of a real vector space $V$ and a homomorphism
$\Cliff_{p,q}(\R) \to \End(V)$.
\end{defn}
%
\begin{defn}
A Clifford module is \ib{irreducible} if there exist no proper nontrivial submodules.
\end{defn}
%
From the classification of Clifford algebras, all the Clifford algebras are either
matrix algebras $M_n\F$, where $\F = \R$, $\C$, or $\H$, or products
$M_n\F \times M_n\F$ of two copies of the same matrix algebra. This is sufficient
to conclude that Clifford algebras are semisimple, so all Clifford modules
will be direct sums of irreducible modules. Therefore, classifying all Clifford
modules reduces to classifying the irreducible Clifford modules.
%
\begin{thm}
Let $\F = \R$, $\C$, or $\H$. Then any nontrivial irreducible module for
$M_n\F$ is isomorphic to $\F^n$ with the standard action.
\end{thm}
%
\begin{proof}
We first note that $M_n\F$ acts transitively on $\F^n$, which implies that
it is irreducible. We then must show that $\F^n$ is, up to isomorphism, the only
irreducible $M_n\F$ module. The matrix algebra $M_n\F$ admits an increasing
chain of left ideals
\[
0 = I_0 \subset I_1 \subset \ldots \subset I_n = M_n\F
\]
where $I_k$ is the set of matrices where only the first $k$ columns are nonzero.
These ideals have the property that the quotient $I_k / I_{k-1}$ is isomorphic
to $\F^n$ as a left $M_n\F$ module. Then let $M$ be some nontrivial irreducible
$M_n\F$ module, and fix $m \in M$. Then the orbit $M_n\F \cdot m$ of $m$
under the algebra action is a nonzero submodule, so it must be all of $M$.
Then the map $\varphi : M_n\F \to M$ given by $A \mapsto A \cdot m$ is
a surjective map of left $M_n\F$ modules. Then there must exist some smallest
$k$ such that $\varphi(I_k)$ is nonzero, and by construction,
$\varphi\vert_{I_k}$ factors through the quotient $I_k / I_{k-1}$, which
is isomorphic to $\F^n$ with the standard action. Then since $\F^n$ is irreducible,
this gives us a nontrivial map between irreducible modules, which is an isomorphism
by Schur's Lemma.
\end{proof}
%
\begin{thm}
Any nontrivial irreducible module for $M_n\F \times M_n\F$ is isomorphic to
either $\F^n$ where the left factor acts in the usual way, and the right factor
acts by $0$, or $\F^n$ where the left factor acts by $0$ and the right factor
acts in the usual way.
\end{thm}
%
\begin{proof}
Let $R$ denote $\F^n$ where the right factor acts nontrivially, and let $L$
denote $\F^n$ where the left factor acts nontrivially. Both $L$ and $R$ are
irreducible since $M_n\F \times M_n\F$ acts transitively on them. To show
that they are the only irreducible modules up to isomorphism, we use a similar
techinique as above. Let $I_k$ denote the chain of increasing ideals in $M_n\F$,
as we used above. Then $M_n\F \times M_n\F$ admits a chain of increasing left
ideals $J_k$
\[
0 = J_0 \subset I_1 \times \set{0} \subset \ldots \subset M_n\F \times \set{0}
\subset M_n\F \times I_1 \subset \ldots \subset M_n\F \times M_n\F = J_{2n}
\]
We note that for $1 \leq k \leq n$, we have that $J_k / J_{k-1}$ is isomorphic to
$L$, and for $n+1 \leq k \leq 2n$, we have that $J_k / J_{k-1}$ is isomorphic to
$R$. Then given a nontrivial irreducible module $M$ and a nonzero element $m$,
we get a surjective map $\varphi : M_n\F \times M_n\F$ where $A \mapsto A\cdot m$.
Like before, there exists some smallest $k$ such that $\varphi(J_k)$ is nonzero,
which then factors through to an isomorphism $J_k / J_{k-1} \to M$, so $M$
is either isomorphic to $R$ or $L$.
\end{proof}
%
This then gives a full classification of the irreducible ungraded Clifford
modules.
%
