%
\section{Spin Structures}
%
Every manifold $M$ admits a Riemannian metric, so there always exists a
reduction of structure group from $GL_n\R$ to $O_n$. If $M$ is orientable,
over each point $p \in M$, we have a preferred set of orthonormal bases of
$T_pM$, which correspond to a choice of component of the $O_n$-torsor of
orthonormal frames of $T_pM$. Doing this for each point $p \in M$, this
determines a subbundle $P$ of $\B_O(M)$. In addition, each fiber
has a free and transitive action of $SO_n$ via the inclusion
$SO_n \hookrightarrow O_n$, giving it the structure of a principal $SO_n$-bundle
over $M$. In addition, the inclusion $P \hookrightarrow \B_O(M)$ is
$SO_n$-equivariant, so it defines a reduction of structure group from $O_n$
to $SO_n$. We denote $P = \B_{SO}(M)$, and call it the \ib{bundle of oriented
orthonormal frames}.

A \ib{Spin structure} on $M$ is a further reduction of structure group to
$\Spin_n$ with respect to the double cover $\Spin_n \to SO_n$, i.e. the data of
a principal $\Spin_n$ bundle $P \to M$, along with a $\Spin_n$-equivariant map
$P \to \B_{SO}(M)$, where $\B_{SO}(M)$ denotes the bundle of positively oriented
orthonomal frames
and $\Spin_n$ acts on $\B_{SO}(M)$ through the double cover $\Spin_n \to SO_n$. In
particular, this map is a double cover, and fiber-wise is the double covering
$\Spin_n \to SO_n$.
Depending on the value $n$, there exist either one or two Spin representations
for $\Spin_n$. Let $\S$ denote the direct sum all of all the Spin representations
for $\Spin_n$, i.e. $\S$ is equal to the single Spin representation in the
case that $\Spin_n$ only admits one representation, and $\S = \S^+ \oplus \S^-$
in the case that $\Spin_n$ admits two inequivalent Spin representations $\S^+$
and $\S^-$. Then given a Spin structure on $M$ with a principal $\Spin_n$ bundle
$P$, the \ib{spinor bundle} over $M$ is the associated vector bundle
$P \times_{\Spin_n} \S$, which can be thought of as a Spin analogue to
the tangent bundle of $M$.
%
\begin{exmp}[Spin structures on $S^1$]
The group $\Spin_1$ is equal to $\set{\pm 1}$, so a Spin structure on $S^1$.
is a double cover $\pi : P \to S^1$ along with a $\Spin_1$-equivariant
map $P \to \B_{SO_1}(S^1)$. Since $SO_1 = 1$, $\B_{SO}(S^1)$ is the trivial
bundle $S^1 \times \set{1}$. Therefore, specifying a $\Spin_1$-equivariant map
$P \to S^1 \times \set{1}$ is no additional data, since we are forced to map
all of the fiber $\pi\inv(x)$ to $(x,1)$ for any $x \in S^1$. Consequently,
all double covers give rise to a reduction of structure group to $\Spin_1$.
There are only two double covers of $S^1$ up to isomorphism. One of them is the disconnected
double cover, which is the disjoint union $S^1 \coprod S^1$, which we denote
as $\pi_1 : P_1 \to S^1$. The other is the connected double cover, which the
circle covering itself via the map $z \mapsto z^2$, which we denote as
$\pi_2 : P_2 \to S^1$. For convenience, we parameterize $P_2$ with angles
$\theta \in [0, 4\pi)$, so the covering map is given by $\theta \mapsto e^{i\theta}$.
The Spin representation is the sign representation on $\R$, where $-1$
acts by multiplication by $-1$, and the complexifying gives us an action on
$\C$ where $-1$ acts by multiplication by $-1$, giving us two spinor bundles
$P_1 \times_{\Spin_1} \C$ and $P_2 \times_{\Spin_1} \C$.

In the first case, the associated bundle is a trivial bundle. Using the identification
with $\Spin_1$-equivariant maps $P_1 \to \C$ with sections of the associated
bundle, it suffices to find such a map to produce a global section of
$\pi_1 : P_1 \to S^1$. Write $P_1$ as the disjoint union $S_1 \coprod S_1$,
with the circles parameterized by angles $\theta,\varphi \in [0,2\pi)$.
The action of $-1$ is given by $\theta \mapsto \varphi$ and
$\varphi \mapsto \theta$. Then the mappings $\theta \mapsto e^{i\theta}$
and $\varphi \mapsto e^{i\varphi}$ define a $\Spin_1$-equivariant map
$P_1 \to \C$, giving us a trivialization of the associated bundle
$P_1 \times_{\Spin_1} \C$. In addition, we see that sections of $P_1\times_{\Spin_1}\C$
are equivalent data to maps $S^1 \to \C$, since once we map one of the components
of $P_1$ into $\C$, this entirely determines how we need to map the other component
in order to remain $\Spin_1$-equivariant. This further allows us to identify
sections of the spinor bundle with $2\pi$-periodic functions $\R \to \C$.

In the second case, the bundle is also trivial! We again contruct a trivialization
for the associated bundle by providing a $\Spin_1$-equivariant map
$\sigma : P_2 \to \C$. Using the parameterization of $P_2$ with angles
$\theta \in [0,4\pi)$, the $\Spin_1$ action on $P_2$ is given by
$-1 \cdot \theta = \theta + 2\pi \bmod 4\pi$. Then define $\sigma$ by
$\sigma(\theta) = e^{i\theta/2}$. This map is $\Spin_1$-equivariant, so
it produces a trivialization of the spinor bundle $P_2 \times_{\Spin_1} \C$.
In addition, we see that sections of the spinor bundle correspond
to $2\pi$-antiperiodic functions, i.e. functions $\psi : \R \to \C$ such that
$f(\theta) = -f(\theta + 2\pi)$, using the fact that we parameterized $P_2$
with angles from $[0, 4\pi)$. Another way to write a $2\pi$-antiperiodic map
$\psi$ is as a product $\psi(\theta) = e^{i\theta/2}f(\theta)$ for a $2\pi$-periodic
function $f$, which is the representation of the section $\psi$ with respect to the
trivialization $\sigma$ defined above.

The two different spin structures produced two isomorphic vector bundles,
but there is still a way to distinguish between the two -- their Dirac operators.
The Clifford algebra $\Cliff_{0,1}(\R)$ is isomorphic to $\C$ as an $\R$-algebra
via the mappings $1 \mapsto 1$, and $e_1 \mapsto i$, so the Dirac operator on
$\R$ can also be written as $i \frac{d}{dt}$. We then use our identifications
of sections of the spinor bundles with functions $\R \to \C$ to investigate the
Dirac operators on each bundle. In the case of the disconnected double cover $P_1$,
we have the identifications of sections of $P_1 \times{\Spin_1} \C$ with
$2\pi$-periodic functions $\R \to \C$. Then given a section $\psi$, we
can identify it as a function $\R \to \C$, and use the Dirac operator in $\R$,
which will again be a $2\pi$-periodic function, giving us another section,
giving us that the Dirac operator $D_1$ on sections of $P_1 \times_{\Spin_1} \C$
is just $i\frac{d}{d\theta}$. For the connected double cover, we used the
global section associated to $\sigma(\theta) = e^{i\theta /2}$ to identify
sections of $P_2 \times_{\Spin_1} \C$ as products
$\psi(\theta) = e^{i\theta/2}f(\theta)$ for a $2\pi$-periodic function $f : \R \to \C$.
Then applying the Dirac operator from $\R$ to this function, we get
%
\begin{align*}
D\psi(\theta) &= i\frac{d}{d\theta}\left( e^{i\theta/2}f(\theta) \right) \\
&= e^{i\theta/2}\frac{\partial f}{\partial \theta} - \frac{1}{2}e^{i\theta/2}
\end{align*}
%
So in the local trivialization $\sigma(\theta) = e^{i\theta/2}$, the operator $D_2$
operates on $2\pi$-periodic functions, just like $D_1$, and is given by
$D_2 = i\partial_\theta - \frac{1}{2}$. In particular, the first operator $D_1$ has
integer spectrum, and the spectrum of $D_2$ is the spectrum of $D_1$ shifted by
$\frac{1}{2}$, which allows us to distinguish to two Spin structures on $S^1$
by their Dirac operators.
%
\end{exmp}
%
Given an oriented manifold $M$, we get a reduction of structure group
of the frame bundle $\B(M)$ to $SO_n$, giving us a principal $SO_n$ bundle
$\B_{SO}(M)$. If $M$ has a nonempty boundary, this induces an orientation on
$\partial M$ in the following way: by first reducing the structure group to $O_n$,
we get a Riemannian metric $g$ on $M$. Then given a point $p \in \partial M$,
the tangent space of the boundary $T_p\partial M$ is a codimension $1$ subspace
of $T_pM$. The Riemannian metric $g$ allows us to pick a distinguished complementary
subspace to $T_p\partial M$ -- the orthogonal complement $(T_p \partial M)^\perp$.
From this subspace, we have a distinguished choice of vector -- the outward
normal vector. In appropriate coordinates, the inclusion of the tangent space
$T_p\partial M$ is given locally by the inclusion
%
\begin{align*}
\R^{n-1} &\hookrightarrow \R^n \\
(x^1, \ldots, x^{n-1}) &\mapsto (0, x^1, \ldots, x^{n-1})
\end{align*}
%
and the outward normal is the unit length vector with a positive first component.
This defines a vector field $N$ along $\partial M$, where the value $N_p$
of $N$ at the point $p$ is the outward normal vector in $T_pM$.
On the boundary $\partial M$, we restrict the frame bundle $\B_{SO}(M)$
to $\partial M$ by pulling back by the inclusion $\partial M \hookrightarrow M$,
giving us the restricted bundle $\B_{SO}(M)\vert_{\partial M}$, which is a principal
$SO_n$-bundle over $\partial M$. An element $b \in \B_{SO}(M)$ is an orientation
preserving linear isometry $(\R, \langle \cdot,\cdot\rangle) \to (T_pM, g_p)$, and
using the normal vector, we define a subbundle
$\B_{SO}(\partial M) \subset \B_{SO}(M)\vert_{\partial M}$ by
\[
\B_{SO}(\partial M) = \set{(p,b) \in \B_{SO}(M)\vert_{\partial M} ~:~
b(e_1) = N_p}
\]
where $e_1$ denotes the first standard basis vector of $\R^n$, and $\pi$ is
the bundle projection. We then get an $SO_{n-1}$ action on $\B_{SO}(\partial M)$,
where we include $SO_{n-1} \hookrightarrow SO_n$ as matrices of the form
\[
\begin{pmatrix}
1 & 0 \\
0 & A
\end{pmatrix}
\]
where $A \in SO_{n-1}$. This then acts on each fiber of $\B_{SO}(\partial M)$
by precomposition. This action is free and transitive, so this gives
$\B_{SO}(\partial M)$ the stucture of a principal $SO_{n-1}$-bundle over
$\partial M$. In addition, this bundle comes with a natural map to
$\B_O(\partial M)$, which is just the inclusion map, so it is a reduction
of structure group of $O_{n-1}$ to $SO_{n-1}$ In this way, we see that an
orientation on $M$ determines an orientation on the boundary. The preceding
discussion is summarized in the diagram
\[\begin{tikzcd}
\B_{SO}(\partial M) \ar[r, hookrightarrow] \ar[dr] &
\B_{SO}(M)\vert_{\partial M} \ar[d] \ar[r, hookrightarrow] & \B_{SO}(M) \ar[d] \\
& \partial M \ar[r, hookrightarrow] & M
\end{tikzcd}\]
In a similar fashion, a Spin structure on $M$ will also induce a Spin structure
on $\partial M$, though the process is slightly more involved. Given a Spin
manifold $M$, it comes equipped with a principal $\Spin_n$-bundle $\B_{\Spin}(M)$
along with a $\Spin_n$-equivariant map $\B_{\Spin}(M) \to \B_{SO}(M)$, where
the Spin action on $\B_{SO}(M)$ is induced by the double cover $\Spin_n \to SO_n$.
Just as before, we pullback both $\B_{SO}(M)$ and $\B_{\Spin}(M)$ along
the inclusion $\partial M \hookrightarrow M$. In addition, we construct the
$SO_{n-1}$-bundle $\B_{SO}(\partial M)$, which gives us the diagram
\[\begin{tikzcd}
&\B_{\Spin}(M)\vert_{\partial M} \ar[r, hookrightarrow]\ar[d] & \B_{\Spin}(M) \ar[d] \\
\B_{SO}(\partial M) \ar[r, hookrightarrow] \ar[dr] &\B_{SO}(M)\vert_{\partial M}
\ar[d] \ar[r, hookrightarrow] & \B_{SO}(M) \ar[d] \\
&\partial M \ar[r, hookrightarrow] & M
\end{tikzcd}\]
This diagram tells us the exact ingredients we need to construct the
$\Spin_{n-1}$-bundle over $\partial M$ -- it must be the pullback of
$\B_{\Spin}(M)\vert_{\partial M}$ along the inclusion
$\B_{SO}(\partial M) \hookrightarrow B_{SO}(M)\vert_{\partial M}$, so it fits
into the commutative square
\[\begin{tikzcd}
\B_{\Spin}(\partial M) \ar[d] \ar[r, hookrightarrow] &
\B_{\Spin}(M)\vert_{\partial M}\ar[d] \\
\B_{SO}(\partial M) \ar[r, hookrightarrow] & \B_{\Spin}(M)\vert_{\partial M}
\end{tikzcd}\]
In addition, it comes equipped with a map $p : \B_{\Spin}(\partial M) \to \partial M$
by composing the map $\B_{\Spin}(\partial M) \to \B_{SO}(\partial M)$ with the projection
$\pi : \B_{SO}(\partial M) \to \partial M$. However, it is not immediately clear
that the pullback bundle, (which we will suggestively denote by
$\B_{\Spin}(\partial M)$) is indeed a principal $\Spin_{n-1}$-bundle over
$\partial M$. We have a $SO_{n-1}$ action on
$\B_{SO}(\partial M) \subset \B_{SO}(M)\vert_{\partial M}$ via the subgroup of
matrices of the form
\[
\set{\begin{pmatrix}
1 & 0 \\
0 & A
\end{pmatrix} ~ : ~ A \in SO_{n-1}}
\]
The preimage of this subgroup under the double covering $\Spin_n \to SO_n$ is
a subgroup isomorphic to $\Spin_{n-1}$, giving us an action of $\Spin_{n-1}$ on
$\B_{\Spin}(M)\vert_{\partial M}$ by restriction. In addition, this action
preserves the preimage of $\B_{SO}(M)\vert_{\partial M}$ under the double
covering $\B_{\Spin}(M)\vert_{\partial M} \to \B_{SO}(M)\vert_{\partial M}$,
which is exactly the pullback $\B_{\Spin}(\partial M)$. Then since the
action of $\Spin_n$ is free, the restricted action of $\Spin_{n-1}$ on
$\B_{\Spin}(\partial M) \subset \B_{\Spin}(M)\vert_{\partial M}$ is
free, and is transitive by construction. Therefore, $\B_{\Spin}(\partial M)$
is the principal $\Spin_{n-1}$-bundle over $\partial M$ that we desire. All in
all, the construction is summarized by the diagram
\[\begin{tikzcd}
\B_{\Spin}(\partial M) \ar[d] \ar[r, hookrightarrow] & \B_{\Spin}(M)\vert_{\partial M}
\ar[r, hookrightarrow]\ar[d] & \B_{\Spin}(M) \ar[d] \\
\B_{SO}(\partial M) \ar[r, hookrightarrow] \ar[dr] &\B_{SO}(M)\vert_{\partial M}
\ar[d] \ar[r, hookrightarrow] & \B_{SO}(M) \ar[d] \\
&\partial M \ar[r, hookrightarrow] & M
\end{tikzcd}\]
%
\begin{exmp}[The induced Spin structure on $\partial D^2$]
Let $D^2 = \set{v \in \R^2 ~:~ |v| \leq 1}$ be the $2$-disk, equipped with the
Riemannian metric inherited from $\R^2$. We have that $\partial D^2 = S^1$.
Since $D^2$ is contractible, both the oriented orthonormal frame bundle
$\B_{SO}(D^2)$ and the spin bundle $\B_{\Spin}(D^2)$ inherited from $\R^2$ are
trivial bundles, so their restrictions onto $\partial D^2$ are also trivial.
In addition, we have that $\Spin_2 \cong SO_2$, and the covering map is given
by $g \mapsto g^2$.
Using the orientation on $D^2$ inherited from $\R^2$, we need to construct
the induced orientation on $\partial D^2$. To do so, we parameterize
$\partial D^2$ by angles $\theta \in [0,2\pi)$. Then the outward normal
at each point $\theta \in \partial D^2$ is the vector $(\cos\theta, \sin\theta)$,
where we use the canonical identification of $T_pD^2$ with $\R^2$. Then
the bundle $\B_{SO}(\partial D^2) \subset \B_{SO}(D^2)\vert_{\partial D^2}$
where the fiber over $\theta \in \partial D^2$ is the linear map $b_\theta$ given
by the matrix
\[
b_\theta = \begin{pmatrix}
\cos\theta & -\sin\theta \\
\sin\theta & \cos\theta
\end{pmatrix}
\]
Then pulling back $\B_{\Spin}(D^2)\vert_{\partial D^2}$ by the inclusion
$\B_{SO}(\partial D^2) \hookrightarrow \B_{SO}(D^2)\vert_{\partial D^2}$,
we have that
\[
\B_{\Spin}(\partial D^2) = \set{(\theta, g) ~:~ g^2 = b_\theta}
\]
Explictly, this means that the fiber of $\B_{SO}(\partial D^2) \to \partial D^2$
over a point $\theta$ is the two point set
\[
\set{\begin{pmatrix}
\cos\frac{\theta}{2} & -\sin\frac{\theta}{2} \\[5pt]
\sin\frac{\theta}{2} & \cos\frac{\theta}{2}
\end{pmatrix},
\begin{pmatrix}
-\cos\frac{\theta}{2} & \sin\frac{\theta}{2} \\[5pt]
-\sin\frac{\theta}{2} & -\cos\frac{\theta}{2}
\end{pmatrix}}
\]
which tells us that the $\Spin_1$-bundle $\B_{\Spin}(\partial D^2)$ is the
connected double cover given by $g \mapsto g^2$.
\end{exmp}
%
