%
\section{Spin Structures}
%
Every manifold $M$ admits a Riemannian metric, so there always exists a
reduction of structure group from $GL_n\R$ to $O_n$. If $M$ is orientable,
we can reduce the structure group further to $SO_n$, which corresponds to a
choice of orientation on $M$. A \ib{Spin structure} on $M$
is a further reduction of structure group to $\Spin_n$, with respect to the
double cover $\Spin_n \to SO_n$, i.e. the data of a principal $\Spin_n$ bundle
$P \to M$, along with a $\Spin_n$-equivariant map $P \to \B_{SO}(M)$, where
$\B_{SO}(M)$ denotes the bundle of positively oriented orthonomal frames.
Depending on the value $n$, there exist either one or two Spin representations
for $\Spin_n$. Let $\S$ denote the direct sum all of all the Spin representations
for $\Spin_n$, i.e. $\S$ is equal to the single Spin representation in the
case that $\Spin_n$ only admits one representation, and $\S = \S^+ \oplus \S^-$
in the case that $\Spin_n$ admits two inequivalent Spin representations $\S^+$
and $\S^-$. Then given a Spin structure on $M$ with a principal $\Spin_n$ bundle
$P$, the \ib{spinor bundle} over $M$ is the associated vector bundle
$P \times_{\Spin_n} \S$.
%TODO finish this
\begin{exmp}[Spin structures on $S^1$]
The group $\Spin_1$ is equal to $\set{\pm 1}$, so a principal $\Spin_1$-bundle
over $\S^1$ is a double cover $\pi : P \to \S^1$ along with a $\Spin_1$-equivariant
map $P \to \B_{SO_1}(S^1)$. Since $SO_1 = 1$, $\B_{SO}(S^1)$ is the trivial
bundle $S^1 \times \set{1}$. Therefore, specifying a $G$-equivariant map
$P \to \S^1 \times \set{1}$ is no additional data, since we are forced to map
all of the fiber $\pi\inv(x)$ to $(x,1)$ for any $x \in S^1$. Consequently,
all double covers give rise to a reduction of structure group to $\Spin_1$.
There are only two double covers of $\Spin_1$. One of them is the disconnected
double cover, which is the disjoint union $S^1 \coprod S^1$, which we denote
as $\pi_1 : P_1 \to S^1$. The other is the connected double cover, which the
circle covering itself via the map $z \mapsto z^2$, which we denote as
$\pi_2 : P_2 \to S^1$. For convenience, we parameterize $P_2$ with angles
$\theta \in [0, 4\pi)$, so the covering map is given by $\theta \mapsto e^{i\theta}$.
The Spin representation is the sign representation on $\R$, where $-1$
acts by multiplication by $-1$, and the complexifying gives us an action on
$\C$ where $-1$ acts by multiplication by $-1$, giving us two spinor bundles
$P_1 \times_{\Spin_1} \C$ and $P_2 \times_{\Spin_1} \C$. \\

In the first case,the associated bundle is a trivial bundle. Using the identification
with $\Spin_1$-equivariant maps $P_1 \to \C$ with sections of the associated
bundle, it suffices to find such a map to produce a global section of
$\pi_1 : P_1 \to S^1$. Write $P_1$ as the disjoint union $S_1 \coprod S_1$,
with the circles parameterized by angles $\theta,\varphi \in [0,2\pi)$.
The $\Spin_1$ action is then given by $\theta \mapsto -\varphi$ and
$\varphi \mapsto -\theta$. Then the mappings $\theta \mapsto e^{i\theta}$
and $\varphi \mapsto e^{-i\varphi}$ define a $\Spin_1$-equivariant map
$P_1 \to \C$, giving us a trivialization of the associated bundle
$P_1 \times_{\Spin_1} \C$. \\

In the second case, the bundle is also trivial!
%TODO finish this
The Clifford algebra $\Cliff_{0,1}(\R)$ is isomorphic to $\C$ as an
$\R$-algebra via the mappings $1\mapsto 1$, $e_1 \mapsto i$, so
the Dirac operator on $\R$ can also be written as $i \frac{d}{dt}$.
\end{exmp}
%
