%
\chapter*{Introduction}
%
\subsectionend $ $\\
\emph{Every morning in the early part of October 1843, on my coming down to
breakfast, your brother William Edwin and yourself used to ask me: "Well,
Papa, can you multiply triples?" Whereto I was always obliged to reply, with a
sad shake of the head, "No, I can only add and subtract them."}\\
%
\ib{-- William Rowan Hamilton}
%
\subsectionend $ $\\
%

On October 16, 1843, William Rowan Hamilton famously carved the equations
\[
i^2 = j^2 = k^2 = ijk = -1
\]
into Broom Bridge, marking his discovery of the quaternions with an act of
mathematical vandalism. It was well known at the time that the algebraic operations
on the complex numbers $\C$ encoded a great deal of the geometry of the plane
$\R^2$ -- addition corresponded to vector translation, and the multiplcation of
complex numbers corresponded to scalings and rotations. After fruitlessly attempting
to construct an analogous algebraic structure on $\R^3$, Hamilton came to the
realization that he needed an additional dimension to encode the rotations
of $\R^3$, therby constructing the quaternions $\H$.

Later in the $19^{th}$ century, William Clifford further generalized the algebras
$\C$ and $\H$ to higher dimensions when he constructed his geometric algebras,
which now bear his name. Clifford's general construction created an $\R$-algebra
that encoded the geometry of a vector space $V$ equipped with a nondegenerate
quadratic form $Q : V \to \R$. Clifford algebras would prove to play a central
part in the development of geometry, topology, and physics in the $20^{th}$
century.

Mathematicians in the early $20^{th}$ century were puzzled by an anomaly
regarding linear representations of the special orthgonal groups $SO_n$.
When studying the representations of the Lie algebras $\mathfrak{so}_n$,
they realized that some of the Lie algebra representations failed to
exponentiate to linear representations of the group $SO_n$, and instead
gave rise to projective representations. The root of the problem was the
fact that the special orthgonal groups were not simply connected --
for $n > 2$, the fundamental group $\pi_1(SO_n)$ is isomorphic to
$\Z/2\Z$\footnote{This can be illustrated in 3 dimensions with the well-known
belt trick, in which a 360 degree rotation of a belt buckle leaves the
belt twisted}, which indicated the existence of a simply connected double cover
and explained the source of the missing representations.
At this point, Clifford algebras came to the rescue, and these double
covers were realized as subgroups of the multiplicative groups of Clifford
algebras. These groups were dubbed the Spin groups, and the mysterious missing
represenatations became the Spin representations.

In $1928$, Paul Dirac was developing a relativistic theory of the electron
when he realized he needed a first order differential operator that squared
to the Laplacian. No first order operator with scalar coefficients was able
to satisfy this, but Dirac realized the the problem could be ameliorated with
the use of matrices, and constructed his gamma matrices. Remarkably, the
relations
\[
\gamma^\mu\gamma^\nu + \gamma^\nu\gamma^\mu = 2\eta^{\mu\nu}I
\]
satisfied by Dirac's gamma matrices are exactly the defining relations for the
Clifford algebra for Minkowski spacetime $\R^{1,3}$. Using these matrices, Dirac
derived his famous equation
\[
(i\gamma^\mu\partial_\mu -m)\psi = 0
\]
where $\psi$ was an element of the Spin representation of the group $\Spin_{1,3}$.
Spin groups and Clifford algebras would prove to be vital elements in the development
of quantum field theory.

Not to be outdone by the physicists, mathematicians also made extensive use
of Clifford algebras and Spin groups to produce astounding results in geometry
and topology. Through the work of many prominent mathematicians such as Michael
Atiyah, Isadore Singer, and Raoul Bott, Clifford algebras and Spin groups saw
remarkable applications in fields such as $K$-theory, homotopy theory, and
gauge theory, producing numerous seminal theorems like the Atiyah-Singer index
theorem and the Bott periodicity theorem. In particular, Atiyah and Singer
generalized Dirac's differential operator to manifolds equipped with a Spin
structure -- the Dirac operator. The methods and tools developed would prove
to be fundamental to the rapid advancement in the fields of geometry and topology
at the time, and remain important to this day. \\

In this thesis we aim to explore and develop some of the theory of
Clifford algebras, Spin and Pin groups, and Spin/Pin structures on manifolds.
In particular, we plan to construct and classify Clifford algebras and their
modules, and transport these constructions to the nonlinear world of manifolds.
We will develop some of the general theory for Spin and Pin structures on manifolds,
and will give several examples.
%