\chapter{Dirac Operators on Manifolds}
%
\subsectionend $ $\\
%
\emph{Mathematics is a part of physics. Physics is an experimental science, a part of natural science. Mathematics is the part of physics where experiments are cheap.} \\
%
\ib{-- ~ V.I. Arnold}
%
\subsectionend
%
\section{Connections}
%
We previously defined the Dirac operator $D$ on $\R^n$ and explored some
instances of $D$ in low dimensions. The constructions implicitly used the
Riemannian geometry on $\R^n$, along with its associated Spin structure.
To explore the Dirac operator in more generality on manifolds, we need to
develop some of the Riemannian geometry in the nonlinear world of manifolds.
%TODO Motivate connections, distributions, generalization of vector fields
%
\begin{defn}
Let $M$ be a smooth manifold. A \ib{distribution} on $X$ is a vector subbundle
$E \subset TX$.
\end{defn}
%
\begin{defn}
let $\pi : E \to M$ be a fiber bundle. The map $\pi$ determines a distinguished
subspace $\ker d\pi_e \subset T_eE$, called the \ib{vertical subspace}
which defines a distribution $V$ over $E$ called the \ib{vertical distribution}.
A \ib{connection} on $E$ is another distribution $H \subset TE$ such that
at every point $e \in E$, we have $V_e \oplus H_e = T_eE$. In other words,
it is a choice of splitting of the short exact sequence of vector bundles
\[\begin{tikzcd}
0 \ar[r] & V \ar[r] & TE \ar[r] & \pi^*TM \ar[r] & 0
\end{tikzcd}\]
If $E$ is a principal $G$-bundle, we ask for the horizontal distribution $H$
to be $G$-invariant, i.e. $H_{p \cdot g} = d(R_g)_p(H_p)$, where $R_g : E \to E$
denotes the right action of $g \in G$.
\end{defn}
%TODO, connections exist?
Having defined a connection, we should check that such an object actually exists.
To show this, we first need a few lemmas.
%
\begin{defn}
Let
\[\begin{tikzcd}
0 \ar[r] & A \ar[r, "\varphi"] & B \ar[r, "\psi"] & C \ar[r] & 0
\end{tikzcd}\]
be a short exact sequence of vector spaces. A \ib{splitting} of this
exact sequence is a map $i : C \to B$ such that $\varphi \circ i = \id_C$.
If a splitting exists, we say that the short exact sequence splits.
\end{defn}
%
A splitting $i : C \to B$ can be thought of as equivalent data to an isomorphism
$B \to A \oplus C$. Since the sequence is exact, the map $\varphi : A \to B$
is injective, so we can identify $A$ with its image $\varphi(A) \subset B$.
The splitting is also injective, giving an identifcation of $C$ with its
image $i(C)$. Since $i$ is a splitting, $\psi \circ i = \id_C$, which implies
that $i(C) \cap \ker\psi = \set{0}$, and since the sequence is exact,
$\ker\psi = \varphi(A)$. Therefore, the map
$\varphi \oplus i : A \oplus C \to B$ is injective, and by a dimension count,
it is an isomorphism.
%
\begin{lem}
Every short exact sequence of vector spaces
\[\begin{tikzcd}
0 \ar[r] & A \ar[r, "\varphi"] & B \ar[r, "\psi"] & C \ar[r] & 0
\end{tikzcd}\]
splits.
\end{lem}
%
\begin{proof}
We want to define a map $i : C \to B$ such that $\psi \circ i = \id_C$. Let
$\set{c_1, \ldots, c_n}$ be a basis for $C$. Since the sequence is exact,
the map $\psi : B \to C$ is surjective, so for each $c_i$, we can pick an element
$b_i \in B$ such that $\psi(b_i) = c_i$. Then define $i(c_i) = b_i$ and
extend $i$ to all of $i$, giving the desired splitting.
\end{proof}
%
\begin{lem}
The space of splittings of a short exact sequence
\[\begin{tikzcd}
0 \ar[r] & A \ar[r, "\varphi"] & B \ar[r, "\psi"] & C \ar[r] & 0
\end{tikzcd}\]
is an affine space over $\hom(C,B)$,
i.e. the difference $i - j$ can be canonically identified with an element of
$\hom(C,A)$.
\end{lem}
%
\begin{proof}
Let $i,j : C \to B$ be splittings of the exact sequence, and let
$c \in C$. Then since $i$ and $j$ are splittings,
\[
\psi((i-j)(c)) = \psi(i(c)) - \psi(j(c)) = c - c = 0
\]
Therefore, $(i-j)(c) \in \ker\psi$, so by exactness, there exists an $a \in A$
such that $\varphi(a) = (i-j)(c)$. By exactness, we know $\varphi$ is injective,
so this $a$ is unique. Therefore, the mapping $c \mapsto \varphi\inv((i-j)(c))$
defines the desired linear map $C \to A$.
\end{proof}
%
\begin{prop}
Given a fiber bundle $\pi : E \to F$, a connection $H \subset TE$ exists.
\end{prop}
%
\begin{proof}
It suffices to show that the short exact sequence
\[\begin{tikzcd}
0 \ar[r] & V \ar[r] & TE \ar[r] & \pi^*TM \ar[r] & 0
\end{tikzcd}\]
splits. Over any point $e \in E$, the space of splittings of
$0 \to V_e \to TE_e \to (\pi^*TM)_e$ is affine over the vector space
$\hom((\pi^*TM))_e \to V_e)$. Doing this fiberwise, we get a fiber bundle
$S \to E$ where the fiber over each $e \in E$ is the space of splittings
of the exact sequence.

\end{proof}
%TODO identification of vertical spaces with the Lie algebra.
%
