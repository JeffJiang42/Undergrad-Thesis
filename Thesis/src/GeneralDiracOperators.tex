\chapter{Dirac Operators on Manifolds}
%
\subsectionend $ $\\
%
\emph{Mathematics is a part of physics. Physics is an experimental science, a part of natural science. Mathematics is the part of physics where experiments are cheap.} \\
%
\ib{-- ~ V.I.Arnold}
%
\subsectionend
%
\section{Connections}
%
We previously defined the Dirac operator $D$ on $\R^n$ and explored some
instances of $D$ in low dimensions. The constructions implicitly used the
Riemannian geometry on $\R^n$, along with its associated Spin structure.
To explore the Dirac operator in more generality on manifolds, we need to
develop some of the Riemannian geometry in the nonlinear world of manifolds.
%
