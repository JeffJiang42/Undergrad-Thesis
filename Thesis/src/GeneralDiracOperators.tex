%
\section{Dirac Operators}
%
To discuss Dirac operators on manifolds, we restrict the previous discussion
to the case where the principal bundle we're concerned with is the principal
bundle of frames $\B(M)$ (or subbundles like $\B_O(M)$ and $\B_{SO}(M))$.
The covariant derivative has a very geometric interpretation
when applied to the $GL_n\R$ bundle of frames $\pi : \B(M) \to M$.
The natural action of $GL_n\R$ on $\R^n$ allows us to
define the associated vector bundle $\B(M) \times_{GL_n\R} \R^n$, which is
naturally isomorphic to the tangent bundle $TM$ as we showed before. We can then
define a natural $1$-form $\theta \in \Omega_{\B(M)}(\R^n)$ called the
\ib{soldering form}. Given $b \in \B(M)$, we know that $b$ is a linear
isomorphism $b : \R^n \to T_{\pi(b)}M$. Then given $v \in T_bP$, we
define the action of $\theta$ by
\[
 \theta_p(v) = b\inv(d\pi_p(v))
\]
The soldering form in some sense remembers that $\B(M)$ is the bundle of
frames. Then given a connection $\Theta \in \Omega_P^1(\gl_n\R)$ on $\B(M)$,
we use the $\Theta$ to define special vector fields on $P$, which we
denote $\partial_i$.\footnote{Note that these vector fields are not coordinate
 vector fields, though we have in the past used the notation $\partial_i$ to
 denote coordinate vector fields.} Given $b \in P$, we define $\partial_i\vert_b$
by $\partial_i\vert_b = \widetilde{b(e_1)}$, where $e_1$ denotes the first
standard basis vector of $\R^n$ and $\widetilde{b(e_1)}$ is the unique horizontal
lift of $b(e_1) \in T_{\pi(b)}M$ at $b$ with respect to the connection $\Theta$.
The vector fields $\partial_2, \ldots \partial_n$ are defined analogously,
and together the $\partial_i$ give a framing for the horizontal distribution on
$\B(M)$. The connection $\Theta$ on $\B(M)$ induces a covariant derivative $\nabla$
on any associated vector bundle $E \to M$, which has a very special relationship
with the vector fields $\partial_i$.
%
\begin{prop}
 Let $M$ be a smooth manifold and $\pi : \B(M) \to M$ its $GL_n\R$ bundle
 of frames. Equip $\B(M)$ with a connection $\Theta$ and define the vector fields
 $\partial_i$ as above. Then given a vector space $W$ with a representation
 $\rho : GL_n\R \to GL(W)$ let $E \to M$ denote the associated bundle, and
 let $\nabla$ denote the covariant derivative on $E$ induced by $\Theta$. Then
 given a section $\psi \in \Omega_M^0(E)$, it has a corresponding
 $GL_n\R$-equivariant map $\tilde{\psi} : P \to W$. Then the covariant derivative
 of $\tilde{\psi}$ is given by the formula
 \[
  \nabla\tilde{\psi} = \partial_i\tilde{\psi} \otimes e^i
 \]
 where $e^i \in (\R^n)^*$ denotes the dual basis to the standard basis $e_i$.
\end{prop}
%
\begin{proof}
 We first make sense of the formula. The covariant derivative $\nabla\psi$ is
 an element of $\Omega_M^1(E)$, which is a section of the bundle $E \otimes T^*M$.
 The bundle $T^*M$ is naturally isomorphic to the associated bundle
 $\B(M) \times_{GL_n\R} (\R^n)^*$ given by the dual to the standard representation
 of $GL_n\R$ on $\R^n$. Therefore, we get a representation of $GL_n\R$ on
 $W \otimes (\R^n)^*$, and $E \otimes T^*M$ is naturally isomorphic to the vector
 bundle $\B(M) \times_{GL_n\R} (W \otimes (\R^n)^*)$. Sections of this bundle
 are equivalent $GL_n\R$-equivariant maps $P \to W \otimes (\R^n)^*$. The action
 of the $\partial_i$ on $\tilde{\psi}$ produces $GL_n\R$-equivariant maps
 $\partial_i\psi : P \to W$, so the tensor product
 $\partial_i\tilde{\psi} \otimes e^i$ then defines a section of $E \otimes T^*M$. \\

 We then want to verify that this formula is true. Fix $x \in X$, $v \in T_xM$,
 and $b \in \pi\inv(x)$. The element $b$ determines a basis $b(e_1), \ldots b(e_n)$
 for $T_xM$. In this basis, $v$ has the representation $v^ib(e_i)$. Each
 of the $b(e_i)$ lift to $\partial_i\vert_b$, so the horizontal lift of $v$ to $b$
 is $v^i\partial_i\vert_b$. Then let
 $\tilde{\gamma} : (-\varepsilon, \varepsilon) \to P$ be an integral curve of
 $v^i\partial_i$. Then $\gamma = \pi \circ \tilde{\gamma}$ is a curve in $M$
 with $\gamma'(0) = v$, so the covariant derivative $(\nabla\tilde{\psi})_b(v)$ is
 given by
 \begin{align*}
  (\nabla\tilde{\psi})_b(v) & = \frac{d}{dt}\bigg\vert_{t=0}
  \tilde{\psi}(\tilde{\gamma}(t)) \\
                            & = v^i\partial_i\tilde{\psi}
 \end{align*}
 Therefore, $\nabla\psi = \partial_i \otimes e^i$.
\end{proof}
%
The vector fields $\partial_i$ on $\B(M)$ also give a recipe for constructing
differential operators between sections of associated vector bundles,
and we will use this recipe to construct a Dirac operator on a general Spin
manifold. Let $A$ and $B$ be linear representations of $GL_n\R$, and let
$E,F \to M$ be their associated vector bundles. The action of $GL_n\R$ on
$A$ and $B$ induces an action on $\hom(A,B)$, giving it the structure
of a $GL_n\R$ representation. Then given a $GL_n\R$-equivariant map
$\sigma : (\R^n)^* \to \hom(A,B)$, we can define a first order differential
operator
\begin{align*}
 D_\sigma : \Gamma_M(E) & \to \Gamma_M(E)                       \\
 \psi                   & \mapsto [\sigma(e^k)](\partial_k\psi)
\end{align*}
Where we interpret $\psi$ as a $GL_n\R$-equivariant map $P \to A$.
%
This method for constructing differential operator has already manifested itself
in several differential operators we have discussed.
%
\begin{exmp}
 Given a vector space $W$ with a $GL_n\R$ action, we construct the associated
 bundle $\B(M) \times_{GL_n\R} W$. A connection on $\B(M)$ induces a
 covariant derivative operator $\nabla$ on sections of the associated bundle.
 In terms of the horizontal vector
 fields $\partial_i$ defined on $\B(M)$, $\nabla$ is given by
 \[
  \nabla\psi = \partial_k\psi \otimes e^k
 \]
 Let $T : (R^n)^* \to \hom((\R^n)^*, W)$ be the map where $T(v)(w) = w \otimes v$.
 Then $\nabla = T(e^k)\partial_k$
\end{exmp}
%
\begin{exmp}
 Consider $\Lambda^k(\R^n)^*$ and $\Lambda^{k+1}(\R^n)^*$, which have a natural
 $GL_n\R$ action, and let
 $\varepsilon : (\R^n)^* \to \hom(\Lambda^k(\R^n)^*,\Lambda^{k+1}(\R^n)^*)$
 be the exterior multiplication map, i.e.
 \[
  \varepsilon(v)(w) = v \wedge w
 \]
 This map is easily seen to be $GL_n\R$-equivariant. Using our recipe, we get a
 differential operator
 \begin{align*}
  D_\varepsilon : \Omega^k_M & \to \Omega^{k+1}_M                         \\
  \psi                       & \mapsto [\varepsilon(e^k)](\partial_k\psi)
 \end{align*}
 which is the de Rham differential $d$.
 %TODO Prove this.
\end{exmp}
%
The preceding discussion was all done with $\B(M)$ and $G = GL_n\R$, but the
discussion applies to any principal bundle $P$ with a reduction of structure
group to $\B(M)$, like $\B_O(M)$, $\B_{SO}(M)$, and $\B_{\Spin}(M)$. \\
%

Recall that if $(M,g)$ is an oriented Riemannian manifold, we have a canonical
choice of connection on $TM$ -- the Levi-Civita connection. Furthermore, if $M$ is Spin,
we get a canonical choice of connection on $\B_{\Spin}(M)$.
%
\begin{defn}
 Let $(M,g)$ be a Spin manifold, and let $\Theta$ be the Levi-Civita connection
 on the oriented orthonormal frame bundle $\B_{SO}(M)$. The double cover
 $\pi : \B_{\Spin} \to \B_{SO}(M)$ gives a natural isomorphism
 \[
  T(\B_{\Spin}(M)) \to \pi^*(T\B_{SO}(M))
 \]
 So the pullback of the horizontal distribution $H$ determined by $\Theta$
 defines a connection on $\B_{\Spin}(M)$ called the \ib{Spin connection}.
\end{defn}
%
We saw in $\R^n$ that the Dirac operator $D$ acted on functions
$\psi \in C^\infty(\R^n, V)$ for a Clifford module $V$. Over a manifold $M$,
these Clifford modules might vary over each point of the manifold, so the
Dirac operator should act on sections $\Gamma_M(\S)$ for some bundle $\S \to M$
of Clifford modules over $M$. \\
%

While we have discussed Spin structures extensively, and Dirac operators are
of particular importance to Spin geometry, they are not dependent on a Spin structure.
Note that the Dirac operator we defined on $\R^n$ only required a nondegenerate
bilinear form, which we used to construct the Clifford algebra. This suggests
that the only structure necessary to define a Dirac operator is a Riemannian metric.
%
\begin{defn}
Let $(M,g)$ be an oriented Riemannian manifold, giving us a reduction of structure
group $\B_{SO}(M) \to \B(M)$. Under the standard representation, $SO_n$ acts on $\R^n$
with the standard inner product by isometries, so every element $A \in SO_n$ induces
an algebra automorphism  $\rho_A : \Cliff_{n,0}(\R) \to \Cliff_{n,0}(\R)$, giving
$\Cliff_{n,0}(\R)$ the structure of a $SO_n$ representation. The \ib{Clifford bundle}
of $M$ is the associated vector bundle
\[
\Cliff(M) = \B_{SO}(M) \times_{SO_n} \Cliff_{n,0}(\R)
\]
\end{defn}
%
In fact, the bundle $\Cliff(M) \to M$ is a bundle of Clifford algebras over $M$, where
over each point $p \in M$, the fiber $\Cliff(M)_p$ is $\Cliff(T_pM, g_p)$.
As a bundle of algebras, we can talk about bundles of modules over $\Cliff(M)$, which
are vector bundles $E \to M$ such that each fiber $E_x$ is a module over $\Cliff(M)_x$.
One example of such a bundle is $\Lambda^\bullet(TM)$. Working
fiberwise, the natural vector space isomorphism
$\Cliff(T_pM, g_p) \to \Lambda^\bullet(T_pM,)$ determines  a vector bundle isomorphism
$\Cliff(M) \to \Lambda^\bullet(TM)$. Using the Riemannian metric, we get an isomorphism
$TM \to T^*M$, and the composition of these isomorphisms determines a vector bundle
isomorphism $\Cliff(M) \to \Lambda^\bullet(T^*M)$.
%
\begin{prop}
Let $(M,g)$ be a Riemannian manifold, and $E \to M$ a bundle of left modules over
$\Cliff(M)$ equipped with a fiber metric under which Clifford multiplication is
orthogonal. Equip $E$ with a connection compatible with the fiber metric, and let
$R$ denote the curvature transformation associated to this connection.
\[
R_{V,W} = -\frac{1}{4}\sum_{i, j}\langle R_{V,W}e_i, e_j \rangle e_ie_j
\]
where $R$ denotes the Riemannian curvature transformation of $M$.
\end{prop}
%
In the case that the manifold $M$ admits a Spin structure, we can define further bundles.
%
\begin{defn}
Let $(M,g)$ be an oriented Riemannian manifold equipped with a Spin structure
$\B_{\Spin}(M) \to \B_{SO}(M)$, and let $\S$ be a left Clifford module over
$\Cliff_{n,0}(\R)$. The inclusion $\Spin_n \hookrightarrow \Cliff_{n,0}(\R)$ gives $\S$
the structure of a $\Spin_n$ representation. A \ib{spinor bundle} over $M$ is the
associated vector bundle
\[
\B_\Spin(M) \times_{\Spin_n} \S
\]
\end{defn}
%
The Levi-Civita and Spin connections induce canonical connections on $\Cliff(M)$ and
any spinor bundle $\S$. Since the Spin connection is derived from the Levi-Civita
connection, the curvature on a spinor bundle $\S$ is closely related to the
Riemannian curvature. \\
%

As one would expect, there is an intimate relationship between the Clifford bundle
and a spinor bundle over a Spin manifold. Namely, a spinor bundle $\S$ is a
bundle of left Clifford modules over $\Cliff(M)$, i.e. each fiber $\S_p$ is a left
Clifford module over the corresponding fiber $\Cliff(M)_p$.
In addition, any spinor bundle $\B_{\Spin}(M) \times_{\Spin_n}\S$ admits a fiber metric
in which the action of a unit vector in $\Cliff_{n,0}(\R)$ is orthogonal. One way
to construct such a metric is to fix an orthonormal basis of $\R^n$ and average
over the finite group generated by the basis in $\Cliff_{n,0}(\R)$. We now have the
ingredients to define a general Dirac operator. \\
%

Equip $\R^n$ with the standard inner product, and let
$b :(\R^n)^* \times (\R^n)^* \to \R$ be the induced inner product on $(\R^n)^*$.
Using the double cover $\Spin_n \to SO_n$, the group $\Spin_n$ acts on
$(\R^n)^*$ through the dual representation of $SO_n$ on $(\R^n)^*$. In addition,
$\Spin_n$ acts naturally on any Clifford module $\S$ by left multiplication, where
we think of $\Spin_n$ as a subgroup of the group of units
$\Cliff((\R^n)^*, b)^\times$. The \ib{Clifford multiplication map}
$c : (\R^n)^* \to \End(\S))$ given by $c(v) = va$ is easily
verified to be $\Spin_n$-equivariant, so we can apply our recipe for
constructing differential operators to construct the Dirac operator.
%
\begin{defn}
 Let $(M,g)$ be a oriented Riemannian manifold equipped with a Spin structure
 $\varphi : \B_{\Spin}(M) \to \B_{SO}(M)$ and spinor bundle $\S \to M$.
 Using the Spin connection on $\B_{\Spin}(M)$, define the horizontal
 vector fields $\partial_i$ on $\B_{\Spin}(M)$ by
 \[
  \partial_i\vert_p = \widetilde{\varphi(p)(e_i)}
 \]
 where $\widetilde{\varphi(p)(e_i)}$ denotes the horizontal lift of the
 image of $e_i$ under $\varphi(p) \in \B_{SO}(M)$ to $T(\B_{\Spin}(M))$.
 The \ib{Dirac operator} is the first order differential operator
 \begin{align*}
  D : \Gamma_M(\S) & \to \Gamma_M(\S)          \\
  \psi                & \mapsto c(e^k)\partial_k\psi
 \end{align*}
 where we interpret $\psi$ as a $\Spin_n$-equivariant map
 $\B_{\Spin}(M) \to \Cliff_{n,0}(\R)$.
\end{defn}
In terms of the covariant derivative $\nabla$ on the spinor bundle induced by the Spin
connection, the Dirac operator can be written in an orthonormal frame $e_i$ of $\S$
by
\[
D = \sum_i e_i \nabla_{e_i}
\]
However, this definition relies on a Spin structure. This construction works
for any bundle of modules over $\Cliff(M)$. For any such bundle $E \to M$, equipped
with a fiber metric in which the action of $TM \subset \Cliff(M)$ is orthogonal, the
Dirac operator on $E$ is defined by the same coordinate formula.
Note that this new definition agrees with our previous definitions of Dirac operators on
$\R^n$, as well as the ones we defined on $S^1$. In $\R^n$, the Dirac operator squares to
the Laplacian, but on a general manifold, this need not be true. Therefore, we
give a name to to operator $D^2$.
%
\begin{defn}
The \ib{Dirac Laplacian} is the map $D^2 : \Gamma_M(\S) \to \Gamma_M(\S)$.
\end{defn}

Given compact Riemannian manifold $(M,g)$, the Riemannian metric on $M$ induces an
inner product on all the associated tensor bundles of $TM$ and $T^*M$. In
particular, we get an inner product $\langle \cdot,\cdot \rangle $ on the exterior
powers $\Lambda^kTM$. Since $M$ is compact, this induces an inner product
$(\cdot, \cdot)$ on the space on $\Omega_M^k$ given by
\[
(\omega,\eta) = \int_M \langle \omega, \eta \rangle~dV_g
\]
where $dV_g$ is the Riemannian volume form of $M$. Under this inner product,
we get a linear map $d^* : \Omega^k_M \to \Omega^{k-1}_M$ that is the adjoint to the
exterior derivative $d$, which is constructed using the Hodge star operator
$\star : \Omega^k_M \to \Omega_M^{n-k}$. Again, we can express $d^*$ using our
recipe for differential operators. Fittingly, its formula reflects its relationship
with the exterior derivative $d$.
%
\begin{prop}
The operator $d^*$ is given by
\[
d^* = \iota(e^k)\partial_k
\]
where $\iota(e^k)(\omega) = \langle e^k, \omega \rangle$ is the inner product of
$e^k$ and $\omega$ induced by the Riemannian metric.
\end{prop}
%
Using $d$ and $d^*$, we construct the \ib{Hodge Laplacian}
\[
\Delta = dd^* + d^*d
\]
which is a vast generalization of the Laplacian on $\R^n$ and an important tool in
studying the geometry and topology of $M$. Thought of as a
map $\Omega^\bullet_M \to \Omega_M^\bullet$, the map $d + d^*$ is a square root
of $\Delta$, where we use the fact that $d^2 = (d^*)^2 = 0$. Since Dirac operators
were originally conceived to be square roots of the Laplacian, it's no surprise
that $d + d^*$ is a Dirac operator.
%
\begin{thm}
Under the isomorphism $\Cliff(M) \to \Lambda^\bullet(TM)$, the Dirac operator $D$
on $\Cliff(M)$ is given by $d + d^*$.
\end{thm}
%
\begin{proof}
Let $\varphi : \Cliff_{n,0}(\R) \to \Lambda^\bullet \R^n$ be the canonical isomorphism.
Recall that for $v \in \R^n$ and $\eta \in \Cliff_{n,0}(\R)$,
\[
\varphi(v\eta) = v \wedge \eta + \iota(v)(\eta)
\]
Therefore, under this isomorphism the Clifford multiplication map $c(e^k)$ becomes
the map $\varepsilon(e^k) + \iota(e^k)$, so the Dirac operator becomes
\[
D = (\varepsilon(e^k) + \iota(e^k))\partial_k = d + d^*
\]
\end{proof}
%
The Dirac operator has an intimate relationship with another Laplacian operator.
\begin{defn}
Let $(M,g)$ be a oriented Riemannian manifold, equipped with the Levi-Civita connection
$\Theta$ on $\B_{SO}(M)$, and let $\widetilde{\nabla}$ denote the covariant derivative
on $TM$. Let $E \to M$ be a vector bundle equipped with a fiber metric and compatible
connection $\nabla$. The \ib{connection Laplacian} is a map
$\nabla^*\nabla : \Gamma_M(E) \to \Gamma_M(E)$ defined by
\[
\nabla^*\nabla\psi = \trace(\nabla^2\psi)
\]
where $\nabla^2$ is the $2$-form defined on vector fields $V,W$ by
\[
\nabla^2_{V,W}\psi = \nabla_V\nabla_W\psi - \nabla_{\widetilde{\nabla}_VW}\psi
\]
for tangent vectors $V,W$ and a vector field $\psi$. In an orthonormal frame $\set{e_i}$
for $E$, $\nabla^*\nabla$ has the coordinate formula
\[
\nabla^*\nabla\psi = \sum_i \nabla^2_{e_i, e_i}\psi
\]
\end{defn}
%
\begin{lem}
For a Riemannian manifold $(M,g)$ and vector bundle $E \to M$, let $\nabla^2$  and
$\nabla$ be defined as above, and let $R$ denote the curvature transformation for $E$.
Then for vector fields $V,W$,
\[
\nabla^2_{V,W} - \nabla^2_{W,V} = R_{V,W}
\]
\end{lem}
%
\begin{proof}
We compute
\begin{align*}
\nabla^2_{V,W} - \nabla^2_{W,V} &= \nabla_V\nabla_W - \nabla_{\widetilde{\nabla}_VW}
- (\nabla_W\nabla_V - \nabla_{\nabla_W V}) \\
&= \nabla_V\nabla_W - \nabla_W\nabla_V - (\nabla_{\widetilde{\nabla}_VW}
- \nabla_{\widetilde{\nabla}_WV}) \\
&= \nabla_V\nabla_W - \nabla_W\nabla_V - \nabla_{\widetilde{\nabla}_VW
- \widetilde{\nabla}_WV} \\
&= \nabla_V\nabla_W - \nabla_W\nabla_V - \nabla_{[V,W]} \\
&= R_{V,W}
\end{align*}
where we use the fact that $\widetilde{\nabla}_VW - \widetilde{\nabla}_WV = [V,W]$ since
the Levi-Civita connection is torsion free.
\end{proof}
%
If $M$ is a compact manifold, a fiber metric $\langle\cdot,\cdot\rangle$ on
any vector bundle $E \to M$ induces an inner product $(\cdot,\cdot)$ on
$\Gamma_M(E)$ defined by
\[
(\psi, \varphi) = \int_M \langle \psi, \varphi \rangle dV_g
\]
where the inner product is taken fiberwise to obtain a smooth function
$\langle \psi,\varphi\rangle$, and $dV_g$ is the Riemannian volume form.
%
\begin{prop}
With respect to the inner product $(\cdot,\cdot)$ defined above, the connection
Laplacian on a vector bundle over a closed Riemannian manifold (i.e. compact and
boundaryless) is formally self adjoint i.e.
\[
(\nabla^*\nabla \psi, \varphi) = (\psi, \nabla^*\nabla\varphi)
\]
equivalently, $(\nabla\psi, \nabla\varphi) = (\psi, \varphi)$.
\end{prop}
%
\begin{proof} %TODO
We show the latter equality $(\nabla\psi, \nabla\varphi) = (\psi, \varphi)$. Fix
a point $p \in M$ and an orthonormal frame $\set{e_i}$ of $TM$ about $p$ such that all
the covariant derivatives $\nabla_{e_i}$ vanish at $p$ (e.g. geodesic normal
coordinates). Then in the fiber $E_p$, we
compute
\begin{align*}
\langle \nabla^*\nabla\psi, \varphi \rangle
&= \sum_j \langle \nabla_{e_j}\nabla_{e_j}\psi, \varphi \rangle \\
&= \sum_j e_j\langle \nabla_{e_j}\psi,\varphi \rangle
- \langle\nabla_{e_j}\psi,\nabla_{e_j}\varphi \rangle
\end{align*}
where we use compatibility of $\nabla$ with the metric. The first term can
be rewritten as $\dv X$ for a vector field $X$ on $M$ defined by the equation
\[
\langle X, Y \rangle = \langle \nabla_Y\psi,\varphi \rangle
\]
So we rewrite
\[
\langle \nabla^*\nabla\psi,\varphi\rangle=\dv X-\langle\nabla\psi,\nabla\varphi\rangle
\]
We then want to integrate both sides -- to do so we use integration by parts.
\begin{lem}[\ib{Integration by parts}]
Let $(M,g)$ be a compact Riemannian manifold with boundary, and let $f \in C^\infty(M)$.
Let $N$ denote the outward unit normal vector field on $\partial M$. Then for any
vector field $X$ on $M$, we have
\[
\int_M \langle \grad f, X\rangle dV_g
= \int_{\partial M} \langle X,N \rangle dV_{\tilde{g}}
+ \int_M (f\dv X) dV_g
\]
where $dV_{\tilde{g}}$ denotes the Riemannian volume form on $\partial M$ induced
by $M$.
\end{lem}
Then integrating on both sides, we find
\begin{align*}
(\nabla^*\nabla\psi,\varphi)
&= \int_M \dv X ~dV_g - \int_M \langle \nabla\psi,\nabla\varphi \rangle dV_g \\
&= 0 + \int_M \langle \nabla\psi,\nabla\varphi \rangle dV_g \\
&= (\nabla\psi,\nabla\varphi)
\end{align*}
where we use the fact that $\partial M = \emptyset$, and that the function $f$ in
the integration by parts formula is the constant function $1$.
\end{proof}
%
\begin{cor}
Let $E \to M$ be a vector bundle over a closed Riemannian manifold, equipped with
a fiber metric and compatible connection. Then $\nabla^*\nabla\psi = 0$ if and only
if $\psi = 0$.
\end{cor}
%
We two different ``Laplacian type" operators on a manifold -- the connection
Laplacian $\nabla^*\nabla$, and the Dirac Laplacian $D^2$. A natural question
to ask is how these operators differ, i.e. what is $D^2 - \nabla^*\nabla$? A priori,
one might expect the difference to be a second or first order differential operator.
The miracle is that the difference is a zeroth order operator, i.e. a tensor, which
only involves the curvature of the manifold.
%
\begin{thm}[\ib{The Weitzenb\"ock Formula}]
Let $\S \to M$ be a bundle of left modules over $\Cliff(M)$ equipped with a compatible
fiber metric and connection, and let $\set{e_i}$ be a orthonormal frame. Then
\[
D^2 = \nabla^*\nabla + \frac{1}{2}\sum_{i, j} e_i e_j R^\S_{e_i, e_j}
\]
where $R^\S$ denotes the curvature transformation of the bundle $\S$.
\end{thm}
%
\begin{proof}
It suffices to verify this identity pointwise. Fix a point $p \in M$ and an orthonormal
frame $\set{e_i}$ for $TM$ such that at $p$, the covariant derivatives
$\nabla_{e_i}$ vanish at $p$. In the orthonormal frame $\set{e_i}$, the Dirac operator is
expressed as $D = \sum_i e_i\nabla_{e_i}$, and the operator $\nabla^2$ is just given
by $\nabla^2_{e_i,e_j} = \nabla_{e_i}\nabla_{e_j}$ at the point $p$. We then compute at
$p$
\begin{align*}
D^2 &= \sum_{i,j}(e_i\nabla_{e_i})(e_j\nabla_{e_j}) \\
&= \sum_{i,j} e_ie_j\nabla_{e_i}\nabla_{e_j} \\
&= \sum_i e_i^2\nabla_{e_i}\nabla_{e_j} + \sum_{j < k} e_j e_k\nabla_{e_j}\nabla_{e_k} \\
&= \sum_i \nabla^2_{e_i}
+ \sum_{j < k} e_je_k (\nabla^2_{e_j, e_k} - \nabla^2_{e_k, e_j}) \\
&= \nabla^*\nabla + \frac{1}{2}\sum_{i, j} e_i e_j R^\S_{e_i, e_j}
\end{align*}
\end{proof}
%
The formula relating the Dirac Laplacian to the connection Laplacian and curvature
is a sort of ``proto-theorem." By applying it to Dirac operators over different
manifolds and spinor bundles, we can recover many theorems relating Laplacian-type
operators to curvature simply by computing the curvature term. For example,
as a corollary, we can express the Hodge Laplacian in terms of the connection Laplacian
and some curvature term. Remarkably the curvature term is a familiar friend from
Riemannian geometry -- the Ricci tensor.
%
\begin{cor}
Let $\Delta$ denote the Hodge Laplacian on a Riemannian manifold $M$. Then
\[
\Delta = \nabla^*\nabla + \mathrm{Ric}
\]
where $\mathrm{Ric}$ denotes the Ricci transformation
\[
\mathrm{Ric}(V) = \sum_{i} R_{e_i,V}(e_i)
\]
\end{cor}
%
\begin{proof}
Under the identification $\Cliff(M) \to \Lambda^\bullet(TM)$, we know that
$D^2 = \Delta$. Using the previous theorem, it then suffices to prove that the curvature
term in the Weitzenb\"ock formula under the identification is the Ricci tensor.
Let $\psi$ be any vector field. We then compute in an orthonormal frame $\set{e_j}$,
\begin{align*}
\frac{1}{2}\sum_{i,j}e_ie_jR_{e_i,e_j}(\psi)
&= \frac{1}{2}\sum_{i,j,k}e_ie_j\langle R_{e_i,e_j}(\psi), e_k\rangle e_k \\
&= \frac{1}{2}\sum_{i,j,k}\langle R_{e_i,e_j}(\psi), e_k\rangle e_ie_je_k
\end{align*}
We then apply the identity $\langle R_{V,W}X,Y\rangle = \langle R_{X,Y}V, W \rangle$
to each term of the summation, which yields
\begin{align*}
\frac{1}{2}\sum_{i,j,k} \langle R_{\psi,e_k}(e_i), e_j\rangle e_ie_je_k
&= \frac{1}{2}\left( \sum_{i = j = k} \langle R_{\psi,e_k}(e_i), e_j\rangle e_ie_je_k
~+ \sum_{\substack{i,j,k \\ \text{not all equal}}}
\langle R_{\psi,e_k}(e_i), e_j\rangle e_ie_je_k \right) \\[5pt]
&= \frac{1}{2}\left( 0 + \sum_{\substack{i,j,k \\ \text{not all equal}}}
\langle R_{\psi,e_k}(e_i), e_j\rangle e_ie_je_k \right)
\end{align*}
The first term vanishes since
$\langle R_{\psi,e_i}(e_i), e_i) \rangle e_i =\langle R_{e_i,e_i}(\psi),e_i\rangle e_i$,
which is zero, since $R_{V,V} = 0$. We then split up the remaining summation into four
separate summations, which correspond to the cases
\begin{enumerate}
  \item $i = j \neq k$
  \item $i \neq j = k$
  \item $i = k \neq j$
  \item $i \neq j \neq k \neq i$.
\end{enumerate}
We then compute for each case.
\begin{enumerate}
  \item In this case, we compute
  \begin{align*}
  \sum_{i =j \neq k}\langle R_{\psi,e_k}(e_i), e_j\rangle e_ie_je_k
  &= \sum_{i \neq k} \langle R_{\psi,e_k}(e_i), e_i\rangle e_k \\
  &= \sum_{i \neq k} \langle R_{e_i,e_i}(\psi), e_k\rangle e_k \\
  &= 0
  \end{align*}
  Since $R_{V,V} = 0$ for any $V$.
  \item We compute
  \begin{align*}
  \sum_{i \neq j = k}\langle R_{\psi,e_k}(e_i), e_j\rangle e_ie_je_k
  &= \sum_{i \neq j} \langle R_{\psi, e_j}(e_i), e_j\rangle e_i \\
  &= \sum_{i,j} \langle R_{\psi, e_j}(e_i), e_j \rangle e_i \\
  &= \sum_{i,j} \langle R_{e_j, \psi}(e_j),e_i \rangle e_i \\
  &= \sum_{j} R_{e_j, \psi}(e_j) \\
  &= \mathrm{Ric}(\psi)
  \end{align*}
  where we use the fact that the terms where $i = j$ are $0$, and the antisymmetry
  in the first pair and second pair of indices.
  \item In this case, we compute
  \begin{align*}
  \sum_{i = k \neq j}\langle R_{\psi,e_k}(e_i), e_j\rangle e_ie_je_k
  &= \sum_{i,j} \langle R_{\psi, e_i}(e_i), e_j \rangle e_ie_je_i \\
  &= -\sum_{i,j} \langle R_{\psi, e_i}(e_i), e_j \rangle e_j \\
  &= \sum_{i,j} \langle R_{e_i, \psi}(e_i) , e_j \rangle e_j \\
  &= \mathrm{Ric}(\psi)
  \end{align*}
  \item This term vanishes, where we use the fact that $e_ie_j = -e_je_i$.
\end{enumerate}
Therefore, the curvature term is just
\[
\frac{1}{2} \left( 2\mathrm{Ric}(\psi) \right) = \mathrm{Ric}(\psi)
\]
\end{proof}
%
Furthermore, if $M$ is a Spin manifold, we can go even further.
%
\begin{thm}[\ib{Lichnerowicz}]
Let $(M,g)$ be a Spin manifold, and let $\S$ be any spinor bundle associated to
$\B_{\Spin}(M$, and $D$ the Dirac operator associated to the Spin connection on $\S$.
Then
\[
D^2 = \nabla^*\nabla + \frac{1}{4}\kappa
\]
where $\kappa$ denotes the scalar curvature of $M$, which is obtained by taking
the trace of the Ricci tensor. In an orthonormal frame of $M$, $\kappa$ is given by
the expression
\[
\kappa = \trace(\mathrm{Ric}) = \sum_{i} \langle \mathrm{Ric}(e_i), e_i \rangle
\]
\end{thm}
%
\begin{proof} %TODO
Once more, we use the Weitzenb\"ock formula, and compute the curvature term, which
involves the curvature $R^\S$ of the spinor bundle. Recall that $R^\S$ is expressed
in terms of the Riemannian curvature by the formula
\[
R^\S_{V,W}(\psi) = \frac{1}{4}\sum_{i, j}\langle R_{V,W}e_i, e_j \rangle e_ie_j
\]
The curvature term in the Weitzenb\"ock formula then becomes
\begin{align*}
\frac{1}{2}\sum_{i,j}e_ie_jR^\S_{e_i,e_j}
&= -\frac{1}{8}\sum_{i,j}\sum_{k,\ell}e_ie_j\langle R_{e_i, e_j}(e_k), e_\ell \rangle
e_ke_\ell \\
&= -\frac{1}{8}\sum_{i,j,k,\ell} \langle R_{e_i, e_j}(e_k), e_\ell \rangle
e_ie_je_ke_\ell
\end{align*}
We note that since $R$ antisymmetric in the first pair of indices and in the
second pair of indices, any term where $i = j$ or $k = \ell$ vanishes.
Using this, we compute
\begin{align*}
-\frac{1}{8}\sum_{i,j,k,\ell} \langle R_{e_i, e_j}(e_k), e_\ell \rangle e_ie_je_ke_\ell
&= -\frac{1}{8} \sum_\ell \left( \sum_{i \neq j = k}
\langle R_{e_i, e_j}(e_k), e_\ell \rangle e_ie_je_k + \sum_{i = k \neq j}  \langle
R_{e_i, e_j}(e_k), e_\ell \rangle e_ie_je_k\right)e_\ell \\
&= -\frac{1}{8}\sum_\ell \left( \sum_{i,j} \langle R_{e_i, e_j}(e_j), e_\ell \rangle
e_ie_je_j + \sum_{i,j} \langle R_{e_i, e_j}(e_i), e_\ell \rangle e_ie_je_i\right)
e_\ell \\
&= -\frac{1}{8} \left(-\sum_{i,j,\ell} \langle R_{e_j,e_i}(e_j), e_\ell \rangle
e_ie_\ell - \sum_{i,j,\ell} \langle R_{e_i, e_j}(e_i), e_\ell \rangle e_je_\ell\right) \\
&= -\frac{1}{8} \left(-\sum_{i,j,\ell} \langle R_{e_i,e_j}(e_i), e_\ell \rangle
e_je_\ell - \sum_{i,j,\ell} \langle R_{e_i, e_j}(e_i), e_\ell \rangle e_je_\ell\right) \\
&= \frac{1}{4} \sum_{i,j,\ell} \langle R_{e_i, e_j}(e_i), e_\ell \rangle e_je_\ell \\
&= \frac{1}{4} \sum_{j, \ell} \langle \mathrm{Ric}(e_j), e_\ell \rangle e_je_\ell \\
&= \frac{1}{4} \kappa
\end{align*}
where the last equality comes from the fact that Clifford multiplication between
orthogonal unit vectors is anticommutative, leaving only the summation over $j = \ell$,
along with the fact that $e_j^2 = 1$.
\end{proof}
%
This relationship between curvature and the Dirac Laplacian gives striking results
regarding the topology of Riemannian and Spin manifolds.
%
\begin{thm}
Let $M$ be a closed Riemannian manifold (i.e. compact and $\partial M = \emptyset)$.
If the Ricci tensor is positive, i.e. for all vector fields $V$,
$\langle \mathrm{Ric}(V), V\rangle \geq 0$ with equality if and only if $V = 0$, the
first Betti number $b_1(M) = 0$.
\end{thm}
%
\begin{proof}
From Hodge theory, the space $\mathcal{H}^1(M)$ of harmonic $1$-forms on $M$ is
isomorphic to the the first de Rham cohomology group $H^1_{dR}(M)$. Let
$\psi \in \Omega^1_M$ be a harmonic form. (i.e. $\Delta\psi = 0$). Then
$0 = \nabla^*\nabla\psi + \mathrm{Ric}(\psi)$, so
$-\nabla^*\nabla\psi = \mathrm{Ric}(\psi)$. We then compute
\begin{align*}
\int_M \langle \mathrm{Ric}(\psi),\psi \rangle dV_g
&= -\int_M \langle \nabla^*\nabla\psi,\psi\rangle dV_g \\
&= -\int_M \langle \nabla\psi,\nabla\varphi\rangle dV_g \\
&= -\norm{\nabla\psi}^2
\end{align*}
Since $\mathrm{Ric}$ is positive, the integral
$\int_M \langle \mathrm{Ric}(\psi),\psi \rangle dV_g \geq 0$ with equality if and
only if $\psi = 0$. On the other hand, $-\norm{\psi} \leq 0$ so we conclude that
$\psi = 0$. Therefore, the only harmonic form on $M$ is $0$, so $b_1(M) = 0$.
\end{proof}
%
\begin{thm}
Let $M$ be a closed Spin manifold with positive scalar curvature. Then for any
spinor bundle $\S \to M$ with Dirac operator $D$, the kernel of $D$ is trivial.
\end{thm}
%
\begin{proof} %TODO
Let $\psi \in \Gamma_M(\S)$ where $D\psi = 0$. We then $0 = \nabla^*\nabla + \kappa/4$,
so $-\nabla^*\nabla\psi = \kappa\psi/4$. Integrating both sides, we find that
\begin{align*}
\int_M \kappa\norm{\psi}^2 dV_g
&= -\int_M \langle \nabla^*\nabla\psi, \psi \rangle dV_g \\
&= -\norm{\nabla\psi}^2
\end{align*}
Then since $\kappa$ and $\norm{\psi}^2$ are both nonnegative, we conclude
that $\psi = 0$.
\end{proof}
%