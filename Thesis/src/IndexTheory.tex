%
\section{Index Theory}
%
%TODO motivation?
A special spinor bundle to consider over a compact Spin manifold $M$ is when the Clifford
module is $\Cliff_{n,0}(\R)$ itself, giving us a spinor bundle
\[
\S(M) = \B_{\Spin}(M) \times_{\Spin_n} \Cliff_{n,0}(\R)
\]
This is a $\Z/2\Z$-graded spinor bundle, with subbundles
\[
\S^0(M) = \B_{\Spin}(M) \times_{\Spin_n} \Cliff_{n,0}^0(\R) \qquad
\S^1(M) = \B_{\Spin}(M) \times_{\Spin_n} \Cliff_{n,0}^1(\R)
\]
In addition, the space of sections $\Gamma_M(\S(M))$ of $\S(M)$ also inherits the
structure of a graded Clifford module, where the grading is given by
\[
\Gamma_M^0(\S(M)) = \Gamma_M(\S^0(M)) \qquad \Gamma_M^1(\S(M)) = \Gamma_M(\S^1(M))
\]
Since multiplication by a vector is odd, Clifford multiplication is odd, so
the Dirac operator is an odd operator with respect to this grading of $\Gamma_M(\S(M))$,
and has the block form
\[
\begin{pmatrix}
0 & D_1 \\
D_0 & 0
\end{pmatrix}
\]
where $D_i$ is the Dirac operator restricted to the subspace $\Gamma_M^i(\S(M))$.
The Clifford algebra $\Cliff_{n,0}$ is also a \emph{right} Clifford module over
itself, and this action commutes with the $\Spin_n$ action on the left. Therefore,
the right action of $\Cliff_{n,0}(\R)$ on $\B_{\Spin}(M) \times \Cliff_{n,0}(\R)$
descends to a right action on $\Cliff_{n,0}(\R)$ on $\S(M)$, giving each fiber
the structure of a right $\Cliff_{n,0}(\R)$ module. This induces a left action of the
opposite algebra, where for $a \in \Cliff_{n,0}^\mathrm{op}(\R)$ and a homogeneous
element $m \in \Cliff_{n,0}(\R)$, we have
\[
a \cdot m = (-1)^{|a||b|}m \cdot a
\]
Using the isomorphism of $\Cliff_{n,0}^{\mathrm{op}}(\R)$ with $\Cliff_{0,n}(\R)$, this
gives the fibers the structure of a left $\Cliff_{0,n}(\R)$ module.
%
\begin{prop}
The Dirac operator $D$ on $\S(M)$ commutes (in the graded sense) with the left
action of $\Cliff_{0,n}(\R)$, i.e. for a homogeneous element $a \in \Cliff_{0,n}(\R)$,
we have that $aD = (-1)^{|a|} Da$, since $D$ is an odd operator.
\end{prop}
%
\begin{proof}
It suffices to verify this on homogeneous elements of $\Gamma_M(\S(M))$ and
$\Cliff_{0,n}(\R)$ Let $\psi \in \Gamma_M(\S(M))$ and $a \in \Cliff_{0,n}(\R)$ be
homogeneous. We then compute
\begin{align*}
a D(\psi) &= a(e^j\partial_j\psi) \\
&= (-1)^{|a|(|\psi| + 1)} (e^j\partial_j) a
\end{align*}
where we use the fact that $D$ is an odd operator to conclude that the parity of
$D\psi$ is $|\psi| + 1 \mod 2$. Doing this in the other order, we compute
\begin{align*}
D(a\psi) = (-1)^{|a||\psi|}e^j\partial_j(\psi)a
\end{align*}
So $D$ and $a$ commute in the graded sense.
\end{proof}
%
To discuss the index of the Dirac operator, we need to make some assumptions from
elliptic theory and functional analysis. The first is the kernel of $D$ is finite
dimensional. Recall that $D$ commutes with the left $\Cliff_{0,n}(\R)$ action.
This then implies that the left action of $\Cliff_{0,n}(\R)$ preserves the kernel of
$D$, so it carries a left $\Cliff_{0,n}(\R)$ action. In addition, $\ker D$ has a
$\Z/2\Z$ grading obtained by intersection with $\S^0(M)$ and $\S^1(M)$, and the
$\Cliff_{0,n}(\R)$ action respects this grading, giving $\ker D$ the structure of a
graded Clifford module. This determines a $K$-Theory class
$[\ker D] \in KO^{-n}(\mathrm{pt})$ called the \ib{index} of $D$.
%
\begin{exmp}
Recall that we have two Spin structures on $S^1$, corresponding to the connected
and disconnected double covers of the circle. After making appropriate identifications,
sections of their corresponding spinor bundles are given by $2\pi$ periodic functions
$\R \to \C$, but the Dirac operators for the two Spin structures differ. We computed
earlier that the Dirac operator for the connected Spin structure is
\[
D = i\frac{d}{d\theta} + \frac{1}{2}
\]
and the Dirac operator for the disconnected Spin structure is $D = i\partial_\theta$.
We then made the observation that the former had no kernel, which implies that
the index determines the trivial element in $KO^{-1}(\mathrm{pt}) = \Z/2\Z$.
The Dirac operator corresponding to the disconnected Spin structure has a
$2$ dimensional kernel given by the constant functions $\R \to \C$, so its
index determines the nontrivial $K$-Theory class in $KO^{-1}(\mathrm{pt})$.
\end{exmp}
%
\begin{exmp}[\ib{Indices on the $2$-sphere}]
Recall that one of the corollaries of the Weitzenb\"ock formula is that on a closed
Spin manifold $M$ with positive scalar curvature, the kernel of the Dirac operator $D$
is trivial. This tells us that the $K$-Theory index of $D$ is the trivial element of
$KO^{-2}(\mathrm{pt}) \cong \Z/2\Z$.
\end{exmp}
%
\begin{exmp}[\ib{Indices on the torus}]
Identify torus $T^2$ as the square with opposite edges identified. Explicitly,
\[
T^2 = S / \left( (-1, x) \sim (1,x) \quad (x, -1) \sim (x, 1) \right)
\]
Then the oriented orthonormal frame bundle is the quotient
\[
S \times SO_2 / \left( (-1, A) \sim (1, A) \quad (A,-1) \sim (A, 1) \right)
\]
This simply glues the fibers over the identified edge via the identity map on $SO_2$.
Let $\rho : \Spin_2 \to SO_2$ be the double cover. Then Spin structures over $T^2$
are principal $\Spin_2$-bundles $P \to \B_{SO}(T^2)$ with a $\Spin_2$-equivariant map
$P \to \B_{SO}(T^2)$. Any such bundle will arise as a quotient of $S \times \Spin_2$
via identifications on the boundary. Since we glued together $\B_SO(T^2)$ using the
identity map on the boundaries, we require this gluing to be done via multiplication
by an element of $\rho\inv(1) = \set{\pm 1} \subset \Spin_2$. This gives us $4$
different spin structures, corresponding to a choice of $\pm 1$ on the two pairs of
edges. Name the bundles $P_{i,j}$, where $i$ denotes the choice on the top and
bottom edges, and $j$ denotes the choice on the left and right edges. In dimension $2$,
the Clifford algebra $\Cliff_{2,0}(\R)$ splits into two $\Spin_2$-representation,
which inherit a complex structure from the action of the element $e_1e_2$, and under
this complex structure, are conjugates with each other. Identifying
$\Spin_2 \cong S^1$, $\Cliff_{2,0}(\R) \cong \C \oplus \overline{\C}$ as a Spin
representation, where $e^{i\theta} \in S^1$ acts on $\C$ by $e^{i\theta}$ and on
$\overline{\C}$ by $e^{-i\theta}$. Sections of the spinor bundle
\[
\S(T^2) = P_{i,j} \times_{\Spin_2} (\C \oplus \overline{\C})
\]
then correspond to functions $\R^2 \to \C \oplus \overline{\C}$ that are
$\Spin_2$-equivariant, which corresponds to  either $2\pi$-periodicity or
antiperiodicity in the vertical and horizontal directions. These correspond with our
choices of $\pm 1$ for the gluing on the edges. For example, if we glued the left and
right edges by the identity, we get an periodic condition in the horizontal direction,
and if we glued by negative identity, we get an antiperiodic condition in the horizontal
direction. Therefore, under suitable trivializations, a section
$\psi \in \Gamma_{T^2}(\S(T^2))$ is a periodic function
$\C \to \C \oplus \overline{\C}$ using the standard identification $\R^2 \cong \C$.
We then can apply the Dirac operator $D$ we defined on $\R^2$. Using our sign
convention, by fixing a suitable basis for  $\Cliff_{0,2}(\R)$, the Dirac operator is
given by the matrix
\[
D = \begin{pmatrix}
0 & -2\partial_z \\
2\partial_{\overline{z}} & 0
\end{pmatrix}
\]
We then compute the index of the Dirac operator for each of the Spin structures.
Note that it suffices to compute the index of the operator $\partial_{\overline{z}}$
restricted to $\C$, where we use the equivalence of graded modules with ungraded modules
over the even subalgebra, and restrict the Dirac operator to the even subspace
of sections.
\begin{enumerate}
  \item For the bundle $P_{1,1} \to T^2$, sections correspond to $2\pi$-periodic
  functions $\C \to \C$ in both the horizontal and vertical directions. In this case,
  the kernel of $\partial_{\overline{z}}$ is the space constant functions $\C \to \C$,
  which implies that the index of the Dirac operator is the nontrivial element
  in $KO^{-2}(\mathrm{pt})$.
  \item For the bundle $P_{1,-1} \to T^2$, the sections correspond to functions
  $\C \to \C$ that are $2\pi$-periodic in the imaginary direction and
  $2\pi$-antiperiodic in the real direction. Therefore, like with the circle,
  we can write any such function as a product $e^{ix/2}\psi(z)$ using coordinates
  $z = x + iy$, and $\psi$ is periodic in both directions. Applying
  $\partial_{\overline{z}}$ to $e^{ix/2}\psi(z)$, we get
  \begin{align*}
  \frac{\partial}{\partial\overline{z}}(e^{ix/2}\psi(z))
  &= \frac{\partial}{\partial\overline{z}}(e^{ix/2})\psi(z)
  + e^{ix/2}\frac{\partial}{\partial\overline{z}}\psi(z) \\[5pt]
  &= \left[\frac{1}{2}\left(\frac{\partial}{\partial x}
  + i \frac{\partial}{\partial y}\right)(e^{ix/2}) \right] \psi(z)
  + e^{ix/2}\frac{\partial}{\partial\overline{z}}\psi(z) \\[5pt]
  &= \frac{i}{4}e^{ix/2}\psi(z) + e^{ix/2}\frac{\partial}{\partial\overline{z}}
  \end{align*}
  So using the identification with periodic functions, the Dirac operator is given by
  $\partial_{\overline{z}} + i/4$, which has trivial kernel, so the index is
  $0$.
  \item A similar computation holds in the case of $P_{-1,1}$, though in this case
  the Dirac operator is $\partial_{\overline{z}} - 1/4$.
  \item In the case of $P_{-1,-1} \to T^2$, sections are antiperiodic functions
  in both directions, which can be written as $e^{iz}\psi(z)$ where $\psi$ is
  $2\pi$-periodic in both directions. Applying $\partial_{\overline{z}}$, we compute
  \begin{align*}
  \frac{\partial}{\partial\overline{z}}e^{iz}\psi(z)
  &= \frac{\partial}{\partial\overline{z}}(e^{iz/2})\psi(z)
  + e^{iz/2}\frac{\partial}{\partial\overline{z}}\psi(z) \\[5pt]
  &= e^{iz/2}\frac{\partial}{\partial\overline{z}}\psi(z)
  \end{align*}
  where we use the fact that $e^{iz/2}$ is holomorphic to conclude that
  $\partial_{\overline{z}}e^{iz/2} = 0$. Therefore, the kernel again consists
  of constant functions, so the index is the nontrivial element of
  $KO^{-2}(\mathrm{pt})$.
\end{enumerate}
\end{exmp}

An interesting thing to study is the index of $-D^2$, which also has some very nice
properties that we will use, namely
\begin{enumerate}
  \item $-D^2$ is elliptic
  \item The eigenvalues of $-D^2$ are discrete and nonnegative
\end{enumerate}
%
The elliptic theory and functional analysis requires completing the space
$\Gamma_M(\S(M))$ into Sobelev spaces, and an important result called \ib{elliptic
regularity} ensures that the eigenvectors of the elliptic operator $-D^2$ are smooth
sections. Let $E_\lambda$ denote the eigenspace of $-D^2$ with eigenvalue $\lambda^2$
(which we can choose since $-D^2$ is positive). Then $E_\lambda$ is also a graded
$\Cliff_{0,n}(\R)$ module, where the grading is obtained by intersecting $E_\lambda$
with the even and odd subspaces of $\Gamma_M(\S(M))$. Then since $D$ commutes with
$-D^2$, it preserves and restricts to an odd operator on $E_\lambda$. In addition,
it commutes (in the graded sense) with the $\Cliff_{0,n}(\R)$ action on $E_\lambda$,
so the operator
\[
F_\lambda = \frac{D}{\lambda}\bigg\vert_{E_\lambda}
\]
squares to $-1$ on $E_\lambda$, so it defines an additional Clifford generator on
$E_\lambda$. This implies that the $K$-Theory class of $E_\lambda$ is trivial, since
we can extend $E_\lambda$ to a $\Cliff_{0, n+1}(\R)$ module via $F_\lambda$. This
tells us that all the information about the index is contained in the kernel, and is
unaffected by the eigenspaces of the operator $-D^2$.
%